\chapter{身體透明瑩澈的喇嘛}

我們這一次為了要掩蓋行蹤,所以決定繞道俄羅斯。雖然這樣的安排會花更多的時間和金錢,不過卻必須這麼做。火車從哥本哈根經由柏林,一路開往莫斯科。我們再從莫斯科搭飛機經由塔什肯(Tashkent)和喀布爾(Kabul)飛往德里。

東德和俄羅斯喚醒了我在從軍時的種種回憶。我發現那些狂妄自大的官員正是這世上最可笑的東西,現在逗人發笑的事情就更多了。然而我們更為那一些無法躲避內心裡沉重的壓迫的人們感到悲憫。他們的人生才是真的淒涼。我們悄悄將隨行多帶的尼龍襯衫、剃鬚刀和香菸分發出去。這是我們在如此倉促的時間內對他們所能盡的一點心意了。

在布列斯特\textbf{─}立陶夫斯克(Brest-Litovsk)(位於波蘭和俄羅斯之間的戰後邊境)通關時所花的時間很長。這時發生了一件趣事。我看見有一士兵連隊剛好路過,猜想他們應該是哥薩克人。他們當中有的長得牛高馬大,有的較為瘦小,我們的面孔倒是長得非常相像。我在丹麥不曾碰過像這樣的事,如今感覺有點兒像是身處於帶鏡的立櫃中一樣。如果我穿的不是白色襯衫,而是他們身上的褐色軍裝大衣,我大概能夠偷偷混入隊伍中而不被發現。這種經歷就猶如置身於夢境中一般,讓我看見我們對「我」的執著其實是如此地狹隘。

在莫斯科機場,有一群士兵想要了解捷克斯洛伐克(Czechoslovakia)的情況。他們的長官在出發之前說他們將會被派遣去擊退那些在蘇伊士運河高喊帝國主義的西方入侵者。可是抵達後才發現自己所對抗的竟是鄰國的斯拉夫(Slav)同胞。現在的情況讓他們感到非常迷惑、洩氣。由於早在半年前「布拉格之春」改革事件爆發時,我們曾經開車穿越該國境,於是便把所知的一切告訴他們。不一會兒,機場內就來了一群凶惡的警察強行把他們帶走,還試圖讓他們將我們所贈予的「皇家海軍」牌香菸歸還我們。我們當然無論如何也不肯收下。最後這些警察也收下一些香菸,還急忙地將香菸藏起來。整個場面就像一場鬧劇!

當時恰好有兩位美國人成功登上月球,成為首次登陸月球的人類。那些無處不在、無論對任何事都抱有「東方」和「西方」強烈分別心的「嚮導」,對此事表現得相當介懷。「這次換你們贏了。」雖然嘴裡是這麼說,實際上臉上的表情卻是酸溜溜的。我們還淡淡地回答說:「也不是甚麼大不了的事,他們早在去年就已登上月球,只不過沒公開而已。」當然此話一出並沒能讓我們變得受歡迎或討喜些。

為了要具備購用廉價航空券的資格,我們必須要在塔什肯住上五天。這些「嚮導」也給了我們有關從北部公路入藏的一些主意。如此一來我們不但能夠見識到在那裡定居下來的穆斯林部落,也能夠親身體驗一下那一個一直延伸至西藏北部羌塘(Changtang)的大沙漠。

在俄羅斯,去莫斯科大劇院(Bolshoi
Theatre)或穿上威風凜凜的制服者顯然都是白人,而棕色或黃皮膚者主要是建房子的勞工。在政府機構裡工作的人,尤其是女性都是看起來非常沮喪的模樣,所有東西看起來更是難以置信的粗造濫製和破落。我們一直希望能夠開啟這一個國家人民的潛能。一九八八年的秋天,我們的願望實現了。我們在聖彼得斯堡(Petersburg)和塔林(Tallinn)創立了首個地下佛教團體。十年後的今天,在這一個廣袤的大陸,從海參崴(Vladivostok)到聖彼得斯堡,我們一共建立了40所禪修中心,每一年我們都會在這一個國度待上兩個月的時間為大眾說法。

飛機飛越一座又一座的連綿山巒,它們看起來就像是被翻倒的巨大磚牆,喀布爾(阿富汗首都)出現在中亞晴朗的天空下。喀布爾比塔什肯更貧困,不過市內一些西方廣告讓整座城市看起來較為摩登。當地的天氣保持在大約攝氏零下十度的恆溫,空氣非常乾燥,所有的東西似乎閃爍著光芒。當時塔利班政權尚未執政,可是這一個國家的婦女和窮人也已經吃了不少苦頭。

在喀布爾機場,有一名男子銷售前往印度時所需的防疫注射證明書。我們是無論如何也不會向那些又粗又生銹的針管妥協的了。我們付數美元,他便在證書右方的虛線上簽下名字。

我們在德里取得了為期兩周的尼泊爾簽證。這是我們第二次搭火車北上。其實我們可以利用學生證買機票直接飛往加德滿都,價錢並不昂貴。只是陸路似乎成為了我們此次旅程的一部分。我們可不希望和一群遊客一起降陸在這一個對我們來說是希望亦是夢想的國度。此外,在冬天搭巴士一路北上會是非常美妙的經歷。這一次我們將能夠在碧空無雲也無雨的情況下,清楚地看見峰峰相連的喜馬拉雅山脈。

我們渴望能夠再次和那一位婦女見面。我們視她為那些神奇療癒能量之始源。這種感覺隨著旅途不斷地增長。由於警方已經開始緊盯著我們,所以我們才迫不得已要繞遠路,途經共產主義的俄羅斯。其實我們應該低調行事暫避風頭,可是由於答應了回程時會前往黎巴嫩幫忙帶一些大麻回來,因此才選擇不同的路線出入境。如今我們已離目的地不遠,覺得實在不需要過於周詳的計畫,再加上我們實在是迫不及待想要和她見面。巴士在傍晚時分抵達加德滿都,我們趕緊在郵局附近下車,免得到站後那些孩童蜂擁而至,到時想躲也來不及。這些孩童會纏著旅客不放,拉扯他們的行李和背包,熱心地為他們介紹下榻的旅店。這一種現象在過去六個月已不斷地「進化」,這種不光彩的改變,我們實在不想攪和其中。

我們下榻的旅店位於加德滿都老城區的中心。在前往旅店途中,我們貪婪地感受著周圍的氛圍。我們曾經在此地擁有太多值得回憶的冒險經歷了。這一次能夠舊地重遊真的很開心。抵達旅店後,我們才聽說莎瑪如今負責管理火警瞭望塔附近的一家新旅舍。莎瑪的消息靈通,是我們獲得各種情報的資源。這一次來到加德滿都,我們還一心冀望著他會認識那一名神秘的婦女。聽說莎瑪不在,我們又背起了行囊離開旅店。途經郵局時卻很巧地和他碰個正著。

莎瑪果然認識她。聽了我們的故事後,再瞧我們如此急切想要和對方見面,莎瑪笑了,還答應隔天早晨陪我們去見她。她就住在城外某個叫馬哈拉亞坤(Maharajgunj)的地方,靠近大街上流經數個主要領事館的供水系統。那一條也是加德滿都的主要道路,一路延續至一家中國鞋廠。當時,這條道路來到鞋廠後便逐漸退成一條供人們步行上山的狹窄小路,引向山谷盡頭一座稱為納吉貢巴(Nage
Gompa)的寺院。

這名婦女與夫家及其丈夫的小老婆一起住在街道附近的磚房裡。房子的後方是一個打理得井井有條的菜圃。我們聽說她的名字叫布塔拉錫米喇嘛(Buddha
Laximi
Lama),便擅自把它當作是一種崇高地位的標誌。殊不知在西藏的邊境地區,「喇嘛」一字已經成為了其中一個姓氏,完全失去了原來「慈悲的上師」的含意。她的家人招待我們,並且在他們家中一間低矮的房內喝茶。後來,她到房裡來了。她的體格矮小,整個人感覺十分平靜、從容,擁有一雙我們忘不了的「半斂細眸」(註:就像佛像垂視眾生的慈眼或修禪定時半開半闔的雙眼)。她的弟弟T.B.
喇嘛負責翻譯,我們有許多話要說。我們由衷地感謝她為我們帶來的一切美好事物,神奇的療癒能量,甚至是在走私和飆車時眷顧著我們的好運氣。當時喜歡飆車的我們曾撞毀不少摩托車和二手車,可是人卻總是平安無事。更玄妙的是,就連我們身邊的朋友也受到庇佑。我們將一隻金錶、一條金鍊子和其他禮物全都送給她,了表謝意。

聽完我們的故事後,布塔拉錫米就笑了,就連她的弟弟也忍不住捧腹大笑。不問也知道,我們交託給喀什米爾人的那二十美元,她當然是毫不知情。自從那一天在旅店的走廊上匆匆見過一面後,她便沒有將我們放在心上。她說:「我只是一個會做點小生意過日子的平凡婦女,並沒有甚麼特殊的能力。不過既然你們具有這方面的信心,你們應該要見一見我的喇嘛。他擁有無窮的力量。不過,他現在正好到外地去了,等他回來後,我會盡快安排你們和他見面。」

布塔拉錫米認為我們的生活方式太危險。她說她會請珠穆朗瑪峰附近某個西藏難民營的瑜伽士為我們製造某種「護身符」。這些藏文稱作為「松度」(Sungdue)的護身符具有保護的能量,而且會延伸到配戴者的身上,保佑他們遠離危難,予以利益。藏傳佛教(現又稱金剛乘佛教)中最常見的護身符便是金剛繩,或是以表徵五智的五種色線綁著的折起圖像,它們都非常有效。這位瑜伽士偶爾會來到加德滿都。許多農民會請他向天祈雨,這一次他恰巧也來到此地。布塔拉錫米說看在我們如此大方不計較的份上,她會以低價賣一些哈希什大麻和宗教物品給我們。帶著破滅的幻想與困惑,我們和她道別。即使無法相信她就如自己所說般平凡,至少我們也算是多交她這麼一個朋友,也為她的家人帶來不少歡樂。眼看像我們如此狂野的北歐人即將要和佛法善知識見面,他們亦感到非常高興。

在等待喇嘛歸來的同時,我們也逐漸投入和適應加德滿都的生活。除了一些舊有的朋友,我們也交了一些新朋友。然而,身邊的事物並不是那麼理想。當時我們天天嗑藥、吸大麻,人都變糊塗了,並未發現身邊的朋友們的狀況已日漸變差。六十年代初期與他們初次見面時,他們當中有許多人是非常有趣的理想主義者,希望創造一個更美好的世界,然而他們後來卻變得瘋瘋癲癲,其中有的人也死了。這一些原本快樂友善、心胸廣闊的「大麻幫」兄弟們,一個接一個漸漸變成了名副其實的行屍走肉。

尼泊爾的情況也不甚理想。我們當初在這一個被遺忘的小國裡發現的「完整性」,不僅是受到物質主義的威脅而已。北京方面不斷對這裡的政府施壓,堅持要在各處建造主幹道路。光是想到某個早晨一覺醒來,龐大的中國軍隊可能沿著這些主幹道路進入了這一片國土,突然出現在自家門前,大概沒有多少個尼泊爾人會喜歡。K.S喇嘛為人不錯。這一次我們向他購買了一些價格便宜,質量卻非常好的畫卷和塑像。他從一些能自由出入西藏的商家處捎來了不少情報,其中有不少是關於文化和一些熱愛自由的人民遭受壓迫的駭人故事。

在加德滿都的小店裡出現的物品,其實也訴說著他們的故事。有一天,市場上到處可見女性戴過的舊手鐲,這說明西藏的遊牧民族已經不被允許配戴這些飾物了。在西藏,沒有一個婦女會心甘情願放棄這些代代相傳的飾物。如此看來,實在不難想像他們的處境。像這種價格便宜,數量又多的藏族藝術珍品,都是博物館不允許運出國的收藏品。然而到了夜間,往往會有某個「友好」的官員騎著腳踏車出現,以低收費「批准」私運這些物品出國販售,然後又開開心心地騎車離去。

乘著開往拉薩送信的吉普車來到位於藏地邊境的溫泉時,河對岸的國度看起來就像是一個負傷的巨大動物。那些扛著一麻布袋又一麻布袋大麥粒渡橋來到尼泊爾邊境的藏民,是我見過唯一臉上是毫無笑容的西藏人了。他們將肩上的麻布袋卸下後,甚至連頭也沒抬起來就離開。這讓我們覺得非常奇怪。我們所認識的西藏人,是即使在患了肺結核吐血至死之前,仍然會幽默地開玩笑的人。

一名剛渡河逃出藏地的婦女說,當中共入藏時,他們必須趕緊將所有佛教典籍和圖像藏起來。大家都知道匿藏佛像等佛教文物的地方,只是沒有人敢去領取。她很開心自己能夠逃到尼泊爾,只是就像其他難民一樣,她不得已留下的,卻讓她痛心不已。

有不少人談論著在邊界發生過的種種小插曲(如果事不關己,這些小插曲聽起來其實挺有趣的),就像發生在四個法國人身上的事一樣。曾經有四個法國人在大霧之下不小心開車駛入沒有標記的邊境大橋,遭到中國方面逮捕,被拘留了三個星期才獲得釋放。在拘留期間,他們被被迫要背誦毛主席的小紅書。看見這些總是成群結隊或穿著制服的中國人一臉驚嘆地凝視著店裡的鋼筆和其他宣傳商品,感覺挺不可思議。這些都是在他們的國家所生產的東西,可是卻無法在當地購得。按照協議,中共方面在建路造橋時必須確保這些公路和橋梁的負荷量不超過一台滿載貨車的重量,可是他們所造建的公路和橋卻總是堅固得足以承載好幾輛坦克車。

印度政治對尼泊爾南部的人民具有更直接的衝擊。印度人會因為邊境一些愚蠢的小糾紛,貿然切斷尼泊爾油的供應。

等待多時,喇嘛終於回來了。在這段期間,他前往各地由他負責管理的寺院和禪修中心處理法務。喇嘛本身來自不丹,今天佛教在尼泊爾得以繼續保存和發展,有賴於他在佛法事業上不間歇的努力。在一個支撐信仰的社會結構逐漸瓦解的情況下,他必須付出很大的努力。洛本傑珠(Lopon
Chechoo)家喻戶曉,倍受尊崇。他曾經在喜馬拉雅山脈的洞穴裡跟隨上師們學法多年,掌握各種法教。如今,在布塔拉錫米的安排下,我們即將和他見面。

我們按約定先到布塔拉錫米的家,她還招待我們吃了一些特別辛辣的道地尼泊爾料理。我們後來也慢慢習慣了當地的口味。尼泊爾就像亞洲大部分地區,人們似乎一天兩餐都是吃一樣的食物。尼泊爾人的主食是白米飯,豆泥或小扁豆,更常見土豆片,或者加上一片肉,和大量的辣椒。用餐後,布塔拉錫米帶我們去見喇嘛傑珠,她的弟弟也隨行當我們的翻譯。我們在一片黑暗之中穿過一條條的窄巷與稻田間的阡陌小路,四周楊樹瑟瑟聲地搖曳響動。喇嘛傑珠的房子是一棟兩層樓高的長型紅磚建築,以尼泊爾風格建造,區分為兩個部分。只見大門敞開著,於是我們穿過一支支經幡飄揚的竹竿,進入屋裡。屋內一切從簡,乾淨整潔。雖然屋內到處都亮著燈泡微弱的褐黃色光芒,不過屋內極大部分看起來漆黑一片。屋內身著紅袍的男女眾帶領我們上樓(我們顯然是身處於寺院中!),然後沿著狹長的走廊來到喇嘛的房間。由於注意到有許多鞋子整齊地排列在房門外,因此我們也入鄉隨俗把鞋子脫下,尾隨布塔拉錫米進入房內。

我們終於和喇嘛見面了。喇嘛的教法在過去一直以不同的形式與我們同在,護佑著我們。他就像是一切美好的示現。他示意讓我們坐在面前,問我們從何而來,所做過的事。一如既往,我負責說話(漢娜則通常在一旁觀察),喇嘛非常和善地垂視著我們,在言談之中,我覺得我們之間完全沒有文化上的隔閡。

我感謝他通過手鐲所展現的療癒能量。我也告訴他有關我們的旅程和充滿刺激的人生。任誰也不難從我的語氣中聽出,我確實認為自己「酷」斃了;至於喇嘛,無論我說甚麼他都照單全收。有時他看起來像是在沉思,眼皮隱約在跳動,半闔著雙眼;雖然看似已經睡著,不過其實仍然保持覺知。當我們坐在那裡時就發生了一件讓我們感到驚嘆不已的事:喇嘛的身體漸漸融入於眼前的虛空之中。他的身體逐漸變得透明瑩徹,我們能夠穿透他的身體,清楚看見其身後牆紙上的圖案。當我們面向坐在右側的翻譯員,透過眼角去瞄喇嘛傑珠時,他看起來是實體。可是當我們直接望向他時,他的身體又再次呈現透明狀。

自從在德里遇見那位印度老人後,我便已經受夠了催眠。我突然想起隨身攜帶那一本有關西藏瑜伽的書籍,裡頭曾提及有關催眠的小把戲。我掏出從丹麥帶來的火柴盒,然後將之舉起在喇嘛的面前。若是在催眠狀態下,眼前的這一個火柴盒看起來應該也是透明的才對,可是它始終看起來是一個實心體。我的視線仍然能夠穿透喇嘛的身體,看見他身後的牆紙。我的內心突然生起非常強烈的信心和極大的喜悅。我非常感恩喇嘛能夠以如此讓人信服的方式向我們展現心性的力量。他能夠按照個人的意願自由地讓自己的身體溶入虛空。我將手腕上的錶脫下。這手錶是第一個登陸月球的歐美加(Omega)型號,曾陪伴我渡過了許多緊張刺激的旅程。我將手錶戴在喇嘛手上,真心地送給他當禮物。漢娜總是能夠和我心靈相通,當下她也感受到了這一股不可思議的力量。喇嘛稍微向前傾身,然後將雙手放在我們的頭頂上,給我們加持,噶瑪噶舉傳承的加持。只見眼前有一道光,那是無法言喻的一個境界。當我回過神來時發現自己已經站在喇嘛的家門前,腦袋裡一片空白。布塔拉錫米幫忙招了一輛三輪車,我們又搖搖晃晃地回到莎瑪的旅店。

當天晚上,我們只睡了一會兒。像我們這一些大量吸食大麻的人,從來就不是早起的鳥兒。那一天,我們凌晨三點就醒過來了。喜悅的感覺也散了。我們看著彼此,難以置信地問道:「你是不是也做同樣的夢了?」這真是一個糟糕的夜晚!一些令人難堪的舊時回憶紛紛浮現,主要都是童年時期發生過的一些事:像是曾說過的愚蠢謊言、毫無意義的偷竊、讓人惹上麻煩等等我們覺得不光彩的事,統統出現在夢裡了。突然內心裡有一種不好的感覺,覺得喇嘛可能已將我們的心仔細地檢視了一番,並從中發現了許多缺點。

接下來的幾天是痛苦的。那些夢境所留下的印記一直在腦海裡徘徊,當時候我們並無法理解箇中的意義,內心裡焦躁不安的感覺一直延續著。有一天,就在KS喇嘛的店裡,有一名印度裔的占卜者走了進來。他看了看我們的手掌,然後說我們很快會損失一筆金錢,而且在一個月內,我們將會面臨也許長達三至四個月的艱難時期。我們不怎麼理會他所說的話,只是給了他一些零錢,便禮貌地打發他走。

不久後,我們收到一封電報,說我們的朋友在黎巴嫩帶著一個手提箱的哈希什大麻被捕。現在他們一定急需我們的幫忙,而且我們也是時候該回國去了。只是這一次在離去之前,我們所做的事很有可能會應驗那個印度裔占卜者的預言。這一次我們非但沒有過往的謹慎,就連以往在行事前的良好預感,甚至是好運氣也都一併消失不見。當然,後果就可想而知了。

以前,我們總是會將貨品藏在背心裡,或直接綁在身上。不過由於每次都能夠僥倖過關也就導致我們漸漸鬆懈了下來。我們已經運走了一批大麻,現在打算將那一些黃銅製造的大佛首塑像以大麻填滿,運回丹麥。由於警方最近逮捕了我們的不少同伴,所以我們想了一個「妙計」:將十公斤上好品質的大麻分成一包一包五克裝,再附上說明書,然後選擇在繁忙時間空投哥本哈根市中心。我們希望能夠通過這樣的方式加速大麻合法化的程序。有關大麻合法化的事已經說了許久,這一次的行動要是成功了,我們所做的事就不再算是犯法,同時也能將朋友們從監獄裡弄出來。

供貨的這個男人讓我們感覺頗為不舒服,他看起來就像一個十足的罪犯。奈何市面上只有他能夠提供我們這些品質上好的大麻和所需的數量。他們從街角某個賣肉的店找來了稱肉的磅秤來稱貨。店裡的窗戶邊,蒼蠅因為DDT的荼毒正奄奄一息地作出垂死的掙扎。當時漢娜和我正怒火中燒,一心只是想著要報復。當我們往佛首塑像裡塞滿大麻時,心裡不斷想著該如何為已深陷麻煩中的朋友們辯護。我們也將現今主要用於宗教屠宰儀式上的尼泊爾彎刀放入包裹中。漢娜臨時想要把薰香一併放入包裹。薰香的香氣顯然很容易會引起海關人員的注意,我不斷強調這一點,可是不知何故,最後我們還是決定將薰香放入包裹中。我們似乎已將過去所累積的走私經驗全都拋到九霄雲外。我們彷彿已來到人生中的三叉路口,隨即為某一階段的生活畫下句點,然後準備迎接另一個全新的開始。

準備回國在際,我們亦仍惦念著喇嘛,心想即使赴湯蹈火也一定要在離開之前再見他一面。由於喇嘛總是非常忙碌,所以要見上一面也實在不易。不過,就在我們要離開的當天傍晚,布塔拉錫米替我們安排了一次和喇嘛見面的機會,讓我們至少可以向喇嘛道別。這一次我們收斂了先前狂妄的態度,將之前的夢境以及對夢境的詮釋,一一告訴喇嘛。他聽了之後只是哈哈大笑,然後分別給了我們一人一個將摺起的圖案以五色絲線結綁而成的護身符。這一個護身符比布塔拉錫米從祈雨的喇嘛處所獲得的更為精確。洛本傑珠仁波切說,我們必定會在一年之內再次回到尼泊爾,接下來我們也將會面臨各種不同的狀況,不過他會為我們祈禱,讓我們一定要生起信心,不可懷疑。離別之前,他對我們所說的一番話,爾後我們也有另一番體會。他說:「無論發生甚麼事,都將會是最好的安排。」最後,他將一小包經過特別加持的法藥放在我們的手心裡。在修法時,瑜伽士會以諸佛的能量加持這一些細小、形狀不一的草藥顆粒。我後來因為不小心吞下了不少法藥才發現它們的功效。當時身體內的能量突然變得非常強烈,讓我長達好幾個小時都無法彎腰,那種感覺就像是在身體的中央裝上鋼鐵彈簧一樣。

喇嘛再次給我們摸頂加持,我們帶著感動和感恩的心離開。布塔拉錫米也送了我們一份離別的禮物:由不同的音節串連而成的心咒,讓我們能時時憶念喇嘛。我們趕緊將咒語默記於心,然後開車前往機場。

