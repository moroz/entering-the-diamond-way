\chapter{陸路尼泊爾}

沿途路經氣候溫暖的國度,一切看起來並沒有多大的變化。土耳其人開車的技術和態度讓人不敢恭維,他們尤其喜歡在夜間熄了燈將卡車停泊在馬路中央,所以一路上每隔幾英哩就可看見一台被撞得稀巴爛的卡車。我們也以同樣的方式駕駛,甚至教會他們一件事:我們可不是省油的燈!土耳其人吃的食物很健康,擁有豐富的蛋白質。當我們越是往東部前進,所遇見的成年人就感覺越是壓抑,甚至擁有暴力的傾向。那些剃光頭、一臉老成的孩童們則不斷向駛過的車子丟石頭。只見男人們成群結黨,無所事事地聚在一起,頭戴著帽子,呷著一杯又一杯甜中帶澀的茶。他們看起來非常沮喪,而女人的蹤影則無處可見。

伊朗變得較以往更富裕了。我們第一次穿越此國境時,只記得中央公路顛簸難行。如今沿途經過里海(Caspian
Sea)的北方公路已將近建好。小巴行駛在柏油路上的感覺很棒,可是除此之外這一個國家便沒有其他能讓人感到振奮的事了,當地的居民也並不是那麼友善。我們常常不得已要強迫他們在某程度上以文明的方式對待彼此。

阿富汗這一個需要對人性恢復信心的國度仍然是旅途中的一片綠洲。當時並沒有所謂的塔利班政權,當地平靜又沉默寡言的人民感覺還不賴。他們屬於原本的單純的生活,不需要耍小把戲。他們依然將文化保存得完好無缺,滿足於傳統的價值觀,不會急著想要利用這些傳統的價值去換取現代社會無法確定的自由。他們的心胸猶如沙漠般浩瀚廣闊。只是當地越來越多的年輕人為了混一口飯吃當上了騙子。像「嘿,先生!」這一種拉攏人的叫喚聲更是不絕於耳。阿富汗讓我想起了六十年代初期的摩洛哥,我親眼看著這一個獨立又自給自足的國度一年比一年衰退,一年比一年變得更商業化。阿富汗還不至於發展到這種地步,當地的人民依然可敬。至少走在阿富汗的大街上並不會受到太大的干擾,他們仍然會把你視為是「賓客」,而非垂涎欲滴的「獵物」。不像在土耳其或伊朗,這裡的人不會死纏著人不放,或者老是盯著西方女性看。當地的婦女總是把自己裹得嚴嚴實實,活像是帳篷和木乃伊似的。

\genericFigure{./figures/walking-tent.jpg}{活像帳篷的阿富汗婦女}

我們在赫拉特(Herat)(阿富汗首個真正的城鎮)曾經擁有過許多有趣的回憶。對我們來說,赫拉特是我們真正踏入「東方」的一個起始點。一九六六年,「先知哈拉爾」、安德斯與我,我們三個丹麥人偷偷攀上了位於新舊城區之間的一座城堡牆壁,嚇壞了負責看守著這似乎是世界最古老的大砲的那一些士兵們。如今哈拉爾追隨一名印度教上師,而未曾斷絕過毒品的安德斯,不久之後死在卡拉奇某處的糞堆上。沒有人知道他的真正死因,只是猜測他是因為患上痢疾未治的原故。我們最後一次見面時,他整個人已經變得神經兮兮,老是幻想四處都是邪惡的吉普賽女郎。毒品真是害他不淺!

距離第一次向東自駕遊來到阿富汗迄今,這裡並沒有多大的改變。當時我們與摔傷了背骨的克里斯蒂安同行。早在六十年代初期,克里斯蒂安便已經去過阿富汗,更在較早之前幫忙打開前往摩洛哥的道路。阿富汗的某些部落有一種習俗,人們抽大麻向死者表示敬意。克里斯蒂安時常志願這麼做,因此在傍晚時分很常見他在大街上裸奔。正因為如此,他非常熟悉赫拉特和喀布爾的監獄。他曾經在卡拉奇的監獄裡待過一段時間,除了每天都遭到典獄官痛毆,他還不幸患上肝炎。有一天,克里斯蒂安在抽了大麻後因為迷幻作用以為自己的肝臟爆開了,因此為了要儘快了結痛苦,他竟然奪窗而出。這麼一個縱身一躍令他自此癱瘓。在坎達哈、喀布爾以及兩城之間的一些小鎮上,一些警察是克里斯蒂安的舊相識,他們曾經前來打聲招呼,還問了他數次:「哈囉,克里斯蒂安先生!現在腦袋還好吧?」

我們在赫拉特下榻的旅店不但有臭蟲,而且水龍頭打開了也不會有有半滴水流出來。旅店裡的臭蟲和我擁有相同的品味,就是特別喜歡漢娜。像這種小旅店,後室都是「阿富汗可樂廠」的秘密基地。「阿富汗可樂」是指某種有毒的綠色或紫色混合物,一般裝在用過的可樂瓶裡兜售。大喝幾口阿富汗可樂就足以在十分鐘內讓一個體格健壯的歐洲人狂奔洗手間,而且往往在這種情況下還老是碰到排長龍的慘況,事後還得跑向附近的藥房排隊等著買抗生素呢!當地的藥房能治好的不單止是腹瀉不止的問題。他們也兜售最好的德國嗎啡,而且不需要醫生開處方就可購得。當地的藥劑師一定覺得很詫異,怎麼會有那麼多年輕的老外旅者需要這種嗎啡。可是他們都極具生意頭腦,很快便學會將藥物的價格抬高。當地的大麻屬於「超強勁」的類型。柔軟、呈暗綠色,看起來和黎巴嫩、尼泊爾的大麻很不同。這些大麻來自北方,靠近俄羅斯邊境,經常讓吸食者陷入昏睡的狀態。在收割期間,農民會批上麂皮大衣在田中奔跑,此時所刮落的黏稠樹脂品質最好,當時可賣上一公斤十五美元的高價。

當地有一種形狀偏大的片麵包對胃最安全,而且無論上哪兒買的都好吃,迷你雞蛋也是。那些放掉水分後的酸奶也不錯。據說喝酸奶是當地人為了消滅結核病才延伸出來的文化,這種病可毀了當地大部分的牛隻。為了健康著想,我說還是吃麵包、喝紅茶最為保險。紅茶到處都有在賣,茶販用來裝茶的老舊茶壺上補丁處處,像馬賽克藝術。這些紅茶非常便宜,一杯也只售兩分錢,卻出奇有效地能消暑解渴。

赫拉特的西南公路是由俄羅斯人所建造,鮮少有人使用。偶爾會看見一台由美國人捐贈的老舊校車在這段公路上行駛,車內擠滿了人和動物,一不小心可能還會從窗邊掉下來;不然就是剛從英國打工賺了錢,如今衣錦還鄉的印度人或巴基斯坦人開著車子回鄉。有時候,我們開車開到一半必須停下來讓一群駱駝悠哉地過馬路。除此之外,在這條道路上的大多時候就只有一片無際的沙漠和唯美的風景與己相伴。

在坎達哈阿里的酒店,我們又碰到一些來自丹麥的朋友。我們在酒店裡連續狂歡了幾天,那兒可說是當時最潮的集合地點了。阿里如今坐擁另一家酒店,可是他的兄弟卻只能在一旁生悶氣。我們聽說他剛花了一千塊買了一個新老婆回家,可是對方卻拒絕和他行房。有人說他們曾見過艾克。艾克是一個充滿傳奇的歌手和作者。一年前,我們曾經一起派對。一九六七年十月,他將某一種稱為氫溴化物(Hydrobromide)[醫學上稱為氫溴酸右甲嗎喃(Romilar)]的毒品帶到丹麥,亦為丹麥人帶來了一個幻異迷離的秋天。如今他已經無法從巴基斯坦過境印度。早前他過量吸毒,隨後選擇結束了自己的性命。他在遺書裡寫說:「是我自己造成了這一切,錯就錯在內心裡負面消極的能量。」臨終之前,他最後所看見的應該是一張佛陀的照片,我們為此感到非常開心。我們知道這將有助於他投生到更好的地方去。

我們對路經喀布爾和開伯爾山口(Khyber
Pass)的路線已經非常熟悉。因此這一次有意嘗試經由奎達(Quetta)入境巴基斯坦的南部路線。當我們接近巴基斯坦邊境時,大夥突然建議要以我們和艾克最後一次相聚時的方式,也就是注射嗎啡去悼念他。其實這一個想法並沒有讓我感到太興奮。經過了六十年代初期和中期,分別在摩洛哥和倫敦患上肝炎的兩次經歷後,對於注射嗎啡這種事,我的心裡可說是五味雜陳。我曾試過有一次在注射嗎啡後(當時我身處在完全不可能發生地震的丹麥)突然覺得天旋地轉,接著牆上的書架就在違反了一切自然定律的情況下掉了下來,而且架角不偏不倚地擊中我的頭顱。雖然這一次心裡同樣覺得不妥,不過似乎也不可能說「不」,因為這樣會壞了大家的興致。這一次的注射並沒有讓我感覺像以往般亢奮,反而覺得無比的沉重。我想也許是護法們不怎麼高興,所以暗自決定要再謹慎一點兒。

抵達巴基斯坦邊境之前,比爾和亞倫想要攀爬一座怪狀林立的山脈。這些山脈看起來像是成堆的大石頭,我也只是曾經在羅德西亞(Rhodesia)看過類似的山形結構。這一次我沒有嘗試呈英雄要爬在前頭,只是尾隨在後。我老是覺得會有不好的事情發生。果然不出所料。攀爬在岩壁最前頭的比爾踩落了一顆大石頭。我讓身體儘量貼近山壁,試圖躲避滾落的石頭,只是落石還是擊中我的小腿。一直到德里,我幾乎無法走動。看來最好的教訓往往是伴隨著「痛苦」和「流血」。這也是我最後一次注射毒品。

這一條穿越巴基斯坦國境的南部路線相當顛簸難行。從土耳其開始,只見當地人都將路上圓形的坑窪弄成正方形,而且就那樣丟著不管。路上的卡車和巴士魯莽竄行,在那兒開車要使出渾身解數才不至於被擠出路外。當地的男性長得像歐洲人,高大魁梧得像座「城堡」;他們身上扛著國家制的來福槍、手槍、刀劍,掛著彈藥筒四處走動。偶而會有幾名婦女靜悄悄路過,她們仿若頂著帳棚在走路,僅僅透過層層的布紗詭秘地看世界。我們看見他們身上的武器都有標上「Made
in
Germany」(德國製造)的標誌,不過是反過來的字母「G」。儘管如此,它們看起來一樣的危險。

我們在拉哈爾(Lahore)花了好幾個小時辦理前往印度時會經過的最後一個路段的許可證。當地的官員煞是逗趣,他們都穿著制服短褲,頭上的帽子裝飾著又硬又挺的羽毛。他們就是在這樣的一副裝扮下來回奔忙。英國人令殖民地的人民穿上這麼誇張逗趣的服飾,不知道是不是在耍弄他們呢!

在印度邊境,海關人員安排了一位出了名擁有通靈能力的婦女在那兒把守,大家都非常怕她。甜美親切的她負責沒收人們身上所攜帶的違法盧比或其他走私貨物。她的魅力,讓我們男人難以招架,乖乖地就把身上的一小片大麻交出來。我們其實可以讓漢娜去應付她,只是想說這一小片大麻也許能夠轉移她的視線,保全我們攜帶入境的非法金錢。

印度並沒有多大的改變。儘管國家已經實施了緊急法案,可是當地的人民似乎不受影響,他們仍然一派的吵鬧、友善。先前途經的數個穆斯林國家,我們每到一站都必須把人推開,現在來到了印度的感覺真好。從伊斯坦堡開始,群眾一發生騷動,遭殃的就是人們的車輛。即使在印度,若沒必要(比方說:要運輸很多器材等)也建議別自己開車前往。

汽油在印度很昂貴。在那裡擁有一台轎車會讓你成為眾人的焦點,卻也同時將你和他人的距離拉遠,所以還是利用那些搖搖晃晃,永遠超載和遲到的巴士和火車較好。在這裡能夠看到各種最真實的「人性」的一面,這些都不容易在西方國家裡看得見。此外,若將車子停泊在一旁後離去,稍後當你再回到泊車地點時,車子往往就只剩下外殼而已。邊界地區經常發生這種事。當你一臉驚愕看著眼前遺留下來的車輛「殘骸」時,就會有幾個友好的陌生人(通常是警察)悄悄出現,然後對你說:「我們恰巧有一副可能適合你的引擎、輪胎和車頭燈喔\ldots{}。」之後,你便能花錢將車子的另一半再買回來了。

我們在德里很快便申請到了尼泊爾的簽證。不過由於比爾有事待辦,因此我們也多留了幾天。我們趁此機會去探望了一些以前在哥本哈根的老朋友。他們很早就戒了毒,現在旅居印度,當了某個斯瓦米(印度教宗師)的弟子。這位印度教宗師是最早在丹麥成名,而且還在丹麥建立了一所瑜伽學校的人。我們想前往拜訪,也順道看看他教些甚麼。就如先前所說的一樣,我和漢娜沒有對印度教徒太感興趣。他們友善得讓人無所適從,控制欲又強,也十分孤立,這一名宗教師尤其極端。他聲稱禁慾是求得解脫的唯一方法。對於漢娜和我來說,這簡直就是一派胡言!像這麼一個將眾生與喜悅、超凡和無限等美好體驗連接在一起的東西,怎麼可能會不好呢?我們知道藏傳的禪修法門具有空樂不二的雙修法。他們以性愛的喜悅狀態達到迅速的證悟,然而在這種主要的教法中,上師並不容易找到堪受法教者。

當這位斯瓦米進入房裡後,便開始向這兩組身著白衣的男女講示教義,然而他卻省略了禁慾主義不談。這一次的主題和對我們來說非常重要的療癒有關。當時我們對於過去所曾示現的種種療癒奇蹟仍然深深感恩;這些奇蹟為我們打開了一個全新的世界。然而斯瓦米卻有另一套想法。他說,治療他人反而會令自己發瘋,因此應該讓人們承擔自己的業力。他依據個人的經驗舉出了幾個例子作為論證,他還甚至說出了哥本哈根好幾個非常出名,如今卻都入住了精神病院的治療師的名字。

由於「業力」(Karma)一詞經常出現在此書中,西方國家也許不太熟悉這一個字眼,我在此簡單說明一下。對於證悟者而言,業力就猶如虛無幻化的遊戲而已;可是對凡夫俗子而言,它就像是必受因果制裁的桎梏。因為無明的緣故,業力束縛著眾生,統一了整個娑婆世界。一個人所體驗的業因果報是此生以及過去無數世的生命所累積的潛意識印記。如果沒有獲得清靜或轉化,這些印記將會依據各自的力量而成熟。結合了內、外各種因緣條件,它們造就了當下、臨終之際以及來生的種種際遇。就像是一場深層的夢境,當眾生犯下過失時,別業和共業將他們緊緊地束縛,當眾生感到「我」與眾生是一體的時候,因果業力就因此鬆綁了。然而,直到證悟之前——也就是認識自心本性乃為明空和無限的功德時,才能夠超越因果業力的法則。通過大乘佛法的種種方便,具大慈悲者將能夠攝收和改善他人的惡業。他們所消除的不僅是惡業所成熟的果報(痛苦),也包括使到眾生生起二元對立分別心的究竟成因「無明」。

\twocolumnFig{./figures/bodhgaya-stupa.jpg}{菩提迦耶的大佛塔}{./figures/bodhgaya-buddha.jpg}{在菩提迦耶的佛陀聖像}

當時我們並不太了解「業力」,因此斯瓦米所說的話讓我們感到非常不安。我們知道他也是一位智者。我突然站起來,說:「如果生病是那個人的業報,那麼我的業報就是去幫助他們。你這冷酷無情的人!」斯瓦米只是大笑,然後說:「現在坐下來靜坐。」在接下來的一個小時,我們非常專注地靜坐禪修。之後我們再次碰面,氣氛也不錯,只是我們都會禮貌性地避重就輕,避免交換任何意見。我們之間的觀點差異太大了!對於療癒,我們需要一個肯定,因此請示了喇嘛。喇嘛告訴我們說:「盡自己的能力去利益他人,別管自己。」這樣的教誨比較容易明白,也相對地比較適合我們。

最令我們感到驚訝的是,我們當中對靈性最不感興致的比爾,竟然堅持要前往菩提迦耶。我們不太樂意,因為從地圖上看來,菩提迦耶距離尼泊爾非常遠,而且路況很差。我和漢娜一心想著要見喇嘛傑珠,而亞倫則暫時受夠了印度。不過,車子是比爾的,更何況他是我們的朋友,單憑這兩點一切便已有了定奪。我們幾乎不怎麼聽說過有關菩提迦耶這一個地方,抵步後,我們開始對她具有了更深入的了解。

菩提迦耶位於比哈省(Bihar)迦耶市近郊。這一個坐落於印度北部的小鎮就是當年佛陀悟道成佛的地方。據說這一個遍滿了殊勝加持力的地方,是諸佛達到最圓滿證悟之處,也就是諸佛開展漸次增上悟境,達證不二境界,並圓滿了無上正等正覺的聖地。兩千五百五十年前,釋迦牟尼佛降世並說法四十五年,早在祂之前便已經有三尊佛降世。直到劫數完結之前,一共會有一千尊佛降世,救度無量有情眾生。

在菩提迦耶可見佛教各個宗派的寺院,雖然我們一心掛念著喇嘛傑珠和尼泊爾,卻仍對菩提迦耶豐富的面貌感到嘆為觀止。即使身在菩提迦耶,我們依舊被西藏人所吸引。我們非常幸運能夠親見度巴喇嘛(Dukpa
Lama)的轉世,並且領受他的加持。上一世度巴喇嘛正是喇嘛傑珠的老師。此外,我們也非常幸運能夠親見拓殿耶喜仁波切(Thubten
Yeshe Rinpoche)與拓殿左巴仁波切(Thubten Zopa
Rinpoche)。兩位仁波切之後非常積極地在西方國家推動弘法事業。聽聞拓殿耶喜仁波切的上一世乃為女尼,因此非常了解女人。當我們來到藏傳寺院時,恰好碰到令仁波切(Ling
Rinpoche
)正在傳授灌頂。令仁波切是第十四世達賴喇嘛的兩位主要上師之一。當時參加灌頂的人都分獲一盧比,由於這些金錢來自於生活也不算富裕的人們,因此讓我們感到既感動又尷尬。當時我們也分獲一塊肉,可是作為素食主義者一年了,這一塊肉讓我們有點反感。當時的我們因為無知,所以連忙將那一盧比給了乞丐,又將那一塊肉給了一條狗。一直到今天我們才知道,在灌頂時無論分獲甚麼都至少要吃一點,然後精神上接受它。儘管如此,我們還是感覺到自己領受了上師的加持。引頸長盼了三天後,我們終於驅車前往尼泊爾。

