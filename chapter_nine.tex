\chapter{皈依我佛}

就像印度北部其他的鐵路旅程一樣,錫金之行仿若將我們帶到另一個世界。在此鐵道系統上穿梭幾個星期後,若是一個敘事風格多變的作者,必定已經擁有足夠的題材去寫一本精彩絕倫的書了。北印度鐵路不僅是一個交通系統,其實它更像是一個人間大熔爐,各式各樣的印度人生就在此坦蕩地上演。

若要前往喜馬拉雅東部地區,首先必須從尼泊爾乘搭當地的普通列車一路顛簸南下,之後再轉搭臭名昭著的「阿薩姆號」(Assam
Mail)。沿途經過的小村莊通常也只是零落的幾戶人家,住在由茅草和竹子所搭建的簡陋小屋裡。只要親眼看見山麓下那一座瘧疾肆虐的森林就不難明白當初三支欲遠征尼泊爾的隊伍為何會在途中消聲匿跡。這一座森林以東是一片人口過密的低地平原。西方人為這裡的居民送來高效的青霉素時,忘了留下避孕的藥具。這一片低地從阿富汗邊境的開伯爾山口一直延伸至緬甸,如果當地人口只剩目前的十分之一,那裡會是一片天堂。在這一片廣袤的平川上,菩提迦耶和靈鷲山(佛陀曾說法的地方)那幾座挺拔而立的小山峰,宛如寬廣的胸膛上突起的乳頭,劃破了四周的單調與靜謐。

鐵路上沿途的風景不斷在改變,火車上的乘客亦然。火車廂裡先是擠滿了皮膚黝黑,對瘧疾擁有頑強的抵抗力,體格較瘦小的「雅利安入侵前期」的土著印度人。當時,從來不會有人在這一段路程擔心車票的問題,因為沒有列車員敢上車來查票。當火車漸漸駛入佔地廣闊的北部城鎮時,車廂裡的土著印度人逐漸由皮膚白皙、體格高大,擁有「歐洲血統」的印度人取而代之。大約三千年前,這些人種從今天的烏克蘭遷徙南下,將黑皮膚的原住民推向了邊緣。他們與當地土著的偶然交合,繁衍出比哈爾人(Biharis)等族群。

這些「烏克蘭人」利用吠陀(Vedas)和奧義書(Upanishads)保護基因的純正性,建立了基本上算是獨特的印度教文化,也就是當今世上所普遍體驗的典型「印度」文化。種性制度為社會階層間設下了嚴密苛細的藩籬,各種行為舉止上的規則在特定的情況下對於各階層的控制亦相當管用。在這一種制度下,佛陀本身是屬於戰士種性。儘管今天大部分人會想像佛陀擁有非常「亞洲人」的五官特徵,然而經典裡所描繪的佛陀身材高大挺拔,而且擁有一雙藍色的眼睛。

火車以蜗牛般的速度緩慢行駛著,這一趟長達好幾千公里的路程很快就會讓人感到沉悶。你大概會慶幸自己手上有一本可讀的書或能持誦咒語,這樣才不至於無所事事。窗邊掠過的風景連續兩天都是一片片的平地、茅草屋、棕櫚樹、一棵一棵垂懸著氣根的大樹,以及腐蝕破落的寺院建築。沿途也看見一些瘦骨如柴,牛角寬大的白牛拉著輕便(偶爾木製)的犁或那些不會卡在泥沼裡的大輪推車。印度人把牛視為神物,因此非常善待它們。只見四處擠滿人潮。他們大部分身著白色衣物,當中有許多人正在乞討。這些林林總總的人與風景完整了整個畫面。

我總是感覺有一半的印度人民其實就在這趟火車上或在這四周圍。旅者若不表現得落落寡交,反而是神態自然地四處觀望,那麼一天之內難免會被問上百次:「你是哪個國家的人?」這裡的人民似乎真的都很想和人溝通、說說話。我們當然也希望給與回應與支持,只是他們的英文水準就只侷限在同一個問題:「你是哪個國家的人?」他們大概也只從朋友那裡學過這麼一句話而已。我們可以隨便回答:從下斯洛波維亞(Lower
Slobovia),月球或隨便說個甚麼地方都好,他們都會以一副「我知道」的模樣點點頭,年紀稍大的會附和說:「噢,英國啊!」年輕一輩的就會說:「噢,美國啊!」有一個毛髮凌亂的印度人突然跳進我們的車廂裡,他搞不好是一個偽裝自己的「存在主義」大師也說不定。他盯著我們,大喊:「你在哪裡?」,接著又消失在人群裡。

我們在西里古里(Siliguri),距離加德滿都以東兩天路程的小鎮下火車。在這一個小鎮上,我們又再次被體格嬌小、圓面孔,總是笑容滿面,精力充沛的山區居民所包圍。這裡的居民操著一口我們都聽得懂的尼泊爾話,四周又再次恢復平靜,沒有刺耳的嘶喊或尖叫聲。我們知道自己很快又可以和噶瑪巴見面,喜悅又再次在心裡滋長。

在火車站的二樓,我們跨過不少在報紙或薄紙皮上就地而睡的人們,才領到了所謂的「入境」許可證。旅者必須持有這一張許可證才能進入喜馬拉雅東部山麓地區的大吉嶺(Darjeeling)。他們通常不會在上山時檢查,不過若想要延長逗留的時間,或進入像是錫金等限制區時就必須向當局人員展示此證件。由於負責發許可證的警察先生睡著了,我們必須先把他搖醒,所以整個過程多花了一點兒時間。幸好他們幫我們延緩最後一輛的吉普車。這一趟車程倒是出乎意料的舒適。

吉普車裡只坐了六或七個人。在尼泊爾,一台福斯小巴擠廿三個大人、幾個小孩,再加上幾個麻袋的大米是很平常的事。我們也沒料到車身會那麼新。我們曾經有過一個很奇怪的想法,以為人們是故意讓這些吉普車看起來「老舊」的。不過,我相信沒多久這些吉普車就會像古董陸虎(Land
Rover)一樣,每行駛十公里就要停下來作各種不同的修理。目前,司機先生至少不需要用力轉動方向盤兩次才能驅動車子。

上山的道路是單一與雙車道交替,偶而路段上還會有上世紀末期在蘇格蘭建造的迷你鐵軌穿過。造路的工程師與建鐵道的人關係應該不怎麼好。我們的吉普車必須停下來幾次,等待玩具般大小的火車駛過,而且路上也無任何鐵路道口的標記,相當危險難行。然而,每一次的顛簸就意味著距離噶瑪巴更近一步,所以我們也沒有任何怨言。

從西里古里開始,我們駛經一條很長的直路,穿越沿途茶園林立的一片低地,再蜿蜒上坡越過一片亞熱帶雨林。這一個區域與阿薩姆邦(Assam)接壤,阿薩姆邦是全世界最潮濕的地方,每年在短短的幾個月內降雨平均13至17米。經歷了兩個小時沿途只有幾隻木馬,彎路又多的車程後,我們的司機將車子停在可宋(Kurseong)火車站對街。那裡位於海拔約一千米處,是西里古里之後的第一個城鎮。此中途休息站還不錯,我們嚐了些茶點,也欣賞了優美的平原風景。從那里前往索納達(Sonada)也是差不多兩個小時的車程。後來,索納達也成為一個對我們來說非常重要的地方。沿著曲折蜿蜒的道路行駛了約八公里的路程後,車子漸漸駛入山口。沿著上坡路行駛約2.5公里處有一座稱為古木(Ghoom)的小鎮。由於這一座小鎮常常籠罩著厚厚的雲霧,因此許多當地人就索性仿英語單字把它稱為「葛隆木」(Gloom),即「陰鬱」的意思。

道路將會在山口的不遠處分出一條岔路。司機說我們很快會駛入一條非常陡峭的下坡路,然後會經過茶園來到堤斯塔(Tista)的大橋。道路將會在那裡分岔,一條前往卡林邦(Kalimpong),另一條道路則前往錫金。在通往錫金的路上,吉普車靠慣性滑行了約6或7公里後,我們終於抵達目的地。大吉嶺是此區域的主要城鎮與行政中心。其實大吉嶺原本被稱為「多傑嶺」(Dorje-Ling),意即「堅不可摧的覺悟之地」,多傑喇嘛的主要道場也在此地。原來的錫金-\/-西藏殖民地[如今已稱為「布提雅布斯提」(Bhutia
Basty)]就在山腳下。而主城鎮則延續了英國殖民文化的烙印。每年雨季,人們會來到此地以避開肆虐印度東部的瘧疾。

大吉嶺以一個非常戲劇性的方式迎接我們到來。我們看見有一名婦女撲倒在昏暗的街頭,並且瘋狂地嘶喊,頭上的傷口還流著血。一開始時我們以為她被人打搶,於是想幫忙扶她起來,殊不知我們一靠近她,她便力竭聲嘶地大叫。後來我們發現原來她只是喝得爛醉,而且她不是因為恐懼或痛苦才力竭聲嘶地大叫,反而更像是憤怒。我們猜想等她酒醒後應該就會沒事,所以便由她去。不過,她倒是很像一名天生的現代芭蕾舞者或歌劇演員,讓人留下了深刻的印象。

大吉嶺的空氣清新,即使是到了夜晚,整個城鎮看起來比加德滿都更顯明亮與莊嚴。只是這裡缺少了美麗的佛塔與令人印象深刻的文化。我們下榻的旅館雖欠缺了尼泊爾旅館的那種魅力,不過亦少了灰塵和汙垢。在這裡幾桶溫水是非常奢侈的享受。我們就在大吉嶺冰涼的空氣裡,伴著深沉的美夢酣然入睡。

翌日早晨,巍然屹立的干城章嘉山脈與其世界第三高峰在陽光下閃耀著光芒。大吉嶺充滿了誘人的魅力。我們從「下市集」(鎮上較為貧困的一區,也是下榻旅館的所在地)發現了通往郵政總局上方較高尚區域的一條路徑。看見這裡的大別墅、商店和餐廳完美地保存了英式風情,頗讓我感覺不可思議。一般上,當殖民地脫離了殖民母國後,殖民母國的文化、語言在當地的發展都會面對停滯不前的局面。然而,這個地方卻比今天的英國更充滿傳統的「英式」氛圍。只是這裡缺少了願意消費的遊客,對當地的經濟確實造成危機。像我們這種凡事先看價格的節約型旅客對此地的經濟發展更是毫無幫助。

最近,隨著富有的印度人一波接著一波地來到此地遊玩,這裡的服務業又開始蓬勃發展起來。這裡的居民擁有像天使般的耐心,然而要滿足外來的殖民者似乎比自己的同胞更容易。那些來自孟買和德里的有錢人可不是一般的傲慢與難伺候。

在大吉嶺呼吸著清新、冰涼的空氣,不得不讚同這裡確實是一個讓人能好生休養之地。我們在郵局裡找到成堆來自家人和朋友的信件而高興不已。在尼泊爾,如果信件堆疊太多無人認領,他們會乾脆把信件都燒掉。大吉嶺一直是印度的一片綠洲。儘管昔日的風光不在,卻依然不失其獨有格調。

法院大樓要十點才開門。錫金的簽證都是在這裡簽發。我們很早就抵達,所以就在法院大樓外邊打坐邊等待,同時祈請噶瑪巴的加持,讓我們順利取得簽證。由於在亞洲待了很長的時間,我們知道在喜馬拉雅中部收到消息說簽證獲批了待領,來到東部地區後並不保證簽證真的批下來了。幸好我們還是如願取得了簽證,內心裡感到雀躍不已。

接下來我們將會需要一台吉普車。若要搭乘前往錫金的吉普車,我們必須回到那天下車的地點。候車站就在下市集的屠宰場旁,那裡也是苦力、計程車司機和樵夫居住的地方。我們的第一站是古木的山口。由於這一次是在白天出發,所以沿途看見舍利佛塔就很興奮。雖然這裡的佛塔在造型結構上略顯稚嫩,無法與尼泊爾的繁複與精細匹比,不過我們仍很開心能在山口和山丘上看見它們的蹤跡。隨著雲霧被風吹散,我們還看見一座巍然聳立的藏傳佛教寺院。

我們在古木左轉,不一會兒便行駛在陡峭的下坡路上。我們驟然從海拔2500米的地方落到海拔數百米處。這裡有一座大橋,橫渡從西藏奔騰而至的堤斯塔大河。大吉嶺的高原氣候瞬即由熱帶氣候所取代。空曠的山坡上不再是一片片的茶園,取而代之是較短粗、果肉厚實的香蕉樹,以及總是空空的橘子樹。這裡的香蕉要比歐洲那些低劣的東西好吃多了。尤其在山上吃太多油膩的食物,現在正好能吃這些便宜又鮮甜的水果解膩。

我們必須在大橋處填寫個人資料,通常這一個過程將會花半個小時以上的時間。因此我們的司機就擁有了充足的時間為他那已很擁擠的吉普車再多找幾名乘客。就當我們仍在疑惑他們到底是如何保護這座大橋而不受中國軍隊或當地共產黨的炸彈所威脅時,吉普車已經卡嗒卡嗒地行駛在陡峭的道路上,取右道前往卡林邦。車道左側是一條河流,沿途經過一排又一排的軍營。這些軍營都藏在森林裡,不過一如既往地沒甚麼了不起。畢竟當年英國只需五千名英軍就征服了印度!我們的證件在橋頭和橋尾兩端一共被複印和檢查了四次。眼見我們為這麼多人提供了這麼多那麼「具有意義」的工作,真是讓人感到「欣慰」啊!

我們的吉普車先經過孟加拉邦(West
Bengal),這裡「酒不多見」,然而來到錫金後卻處處可見推銷當地威士忌酒的廣告牌,標榜著這些威士忌酒是錫金最偉大的貢獻(我想是掀起化學戰的最大貢獻吧?!)。西方人把這種威士忌酒稱為「Monkey
Whiskey」,據說若不小心將酒滴落桌面,它會腐蝕桌上的清漆和色漆面。當地甚至有一家賣酒的店家叫作「噶瑪巴」,不過我們說服店主將標誌撤了。雖然西藏人看不懂,不過在西方人的眼裡看來卻非常滑稽與諷刺。

當我們看見第一個以藏文標示的路牌時,心情非常雀躍。我們知道很快便能與噶瑪巴見面。吉普車裡的其他乘客,他們平時的話題不外是久下不停的雨或農作物的收成。如今我們這兩個總是轉動著念珠持誦咒語,對那些他們覺得再也平常不過的事物展現出如此強烈熱忱的白人巨人,應該給了他們不少茶餘飯後的話題。

吉普車再行駛約八公里路便抵達岡托克(Gangtok),然而司機先生卻突然在左車道分叉出的一條垮了的小路上把車子停下。他指著前方蜿蜒向上的道路,說:「再往上走就是隆德(Rumtek)了,大喇嘛噶瑪巴就住在那裡。」由於我們帶著很多行李(主要是他人交託的禮物),天空又下著倾盆大雨,我們建議付他雙倍的價錢,把我們直接載至隆德。不過對方卻說自己沒有該路段的通行證,因此不敢鋌而走險。他建議我們先到岡托克住一晚,等隔天早晨天氣好轉後再回來。我們對他的話置若罔聞。噶瑪巴就近在呎尺,沒人能夠拉遠我們與他的距離。我們將所有的行李和禮物揹在身上,活像一棵會走路的聖誕樹。就在此時,有一輛沒載客的吉普車從逆向的車道駛過來。要麼這名司機的膽子比較大,不然就是因為他沒有一個經營旅館的叔叔,所以他願意僅以三十盧比的價格冒險開上這一條壞損的公路,把我們送到寺院去。雨水就像瀑布般直瀉而下。這場豪雨著實非比尋常,吉普車的遮篷或塑料布都無法防止雨水入侵。窗外的可見度非常低,司機更像是憑記憶沿著蜿蜒的道路緩緩向上行駛了11公里。我們的一些行囊,即使擁有層層保護也全都濕透。

忽然之間,吉普車的車燈照射在一道灰白色的牆壁上。牆上是許多鎖孔形狀的深色窗戶。車燈接著照在了敞開著的大門上。寺院的台階上未見任何人的蹤影。就連我們在某個類似休息室的地方卸下行李時也不見有人。我們陸續敲了好幾道房門,終於有人前來應門。每個人見到我們都詫異不已,其中包括一些我們在尼泊爾認識的喇嘛。他們聚集在一起,七言八語地討論起來,有的大概忙著去找會說英文的人過來幫忙翻譯。後來,Jigmela(-la:尊稱)來了,他是噶瑪巴的侄子,也是今天噶瑪巴駐歐洲的官方代表。Jigmela很委婉地說服我們說時間太晚,不方便覲見「滿願如意寶」。西藏人都稱呼噶瑪巴為「滿願如意寶」。後來,我們也贊同說明早再來覲見噶瑪巴會比較殊勝,幸好喇嘛們在寺院對街的旅館裡幫我們找到了可下榻的地方。

經過半年深切的盼望後,我們終於在翌日早晨見到了噶瑪巴。我們興奮地衝向噶瑪巴,他慈悲地給予我們大力的加持,然後說:「你們來了就好。你們可以把我視為導師。現在告訴我,你們想要甚麼?」我們想要的東西很難用三言兩語就說清楚。其實我們也未曾真正思考過這一個問題。我們告訴噶瑪巴,我們希望能追隨他為他服務,噶瑪巴知道怎麼做才是最好。「好吧!」噶瑪巴說:「明天十五,是一個特別的日子,也是結夏安居日的最後一天。明天我會授與你們在家居士戒(Genyen)。」我們不知道在家居士戒究竟是甚麼,不過只要是噶瑪巴給的就一定是好東西。我們把頭髮剃掉,只在靠近頭頂漩渦處留幾根頭髮。我們不知道只有和尚或女尼才會把頭髮剃光。女尼帕嫫幫忙我們準備在明天的儀式中所需的傳統禮品。這名老派的老女尼原籍英國,給人的印象非常深刻。她之所以能長年留在隆德,是因為擁有印度國籍的原故。早前她曾跟隨達賴喇嘛,爾後因為想要專注修行而非辯論,所以轉而跟隨噶瑪巴學法。她為了西藏的難民與文化,比任何人都付出了更多努力,只可惜不是每一件事都取得成功。一九六七年,她將自己在喜馬拉雅西部所培育的四名西藏轉世活佛送往蘇格蘭,其中三位爾後因為鬧出醜聞而令整個藏傳佛教蒙羞。無論是過去或甚至到今天,他們雖以西方人透明化的作風為前提,卻嚴重地阻礙了金剛乘佛教在西方國家(尤其是英國和北美)自力的發展。

漢娜的氣質,無法言喻。皮膚白皙的她,頂著光禿禿的頭,散發著睿智的光芒。我們帶著雀躍的心情等待明天降臨的同時,也不忘把握時間探索四周。我們也儘量爭取時間靜坐,對噶瑪巴的禮物保持著開明的態度。

夜幕很快降臨,寺院裡的燈火也一盞接著一盞地熄滅。遠處傳來一陣陣的長號角聲,聲嘯長空,迴盪在山谷之中,似乎向居民們保證,護法常在,居民常獲加被。在萬籟俱寂的夜裡,有一扇窗戶,總是亮著燈。那是噶瑪巴的房間,就座落於寺院的最頂樓。這一盞燈照亮了整片山谷。我在半夜醒過來,匆匆辦點事,只見那一盞燈依然亮著。後來我因為心情太興奮實在無法睡著,一大早就爬了起來,窗戶裡的那盞燈依然亮著。清晨的寺院一片寂靜。後來我們才知道,噶瑪巴雖然在晚間休息,可是卻沒有真的入睡。他那當下現前、無所不在的明覺性恆常是了了分明,無所障礙的。

樓上的禪修室布置得華麗而莊嚴,噶瑪巴身處其中,依然散發著一種攝人的光芒與氣韻。噶瑪巴席地而坐,背向著窗戶,面前是一張小桌子。他微笑著接過我們準備的禮品。進入禪修室時,幾位喇嘛向我們展示傳統的頂禮方式。首先雙掌合十,依次放在頭、喉嚨和心間上,接著跪地磕頭。這一個步驟依次重複三次。那一刻之前,深受西方國家開放式教育所影響的我們未曾親身體驗過這一種象徵「敞開與放棄對自我的執著」的動作。我們認為那只是一些熱帶國家的原住民才會做的事兒,不過卻未曾想像自己有一天會向任何人事物彎腰頂禮。無論如何,如今的我們真的非常喜歡身邊的這些出家人,那一天是我們第一次在噶瑪巴面前彎腰頂禮。一開始時感覺有點兒彆扭,不過這種感覺很快便因為我們對噶瑪巴的信心一掃而空。其實一直到九十年代中期,西方國家也有類似對喇嘛或祭壇行禮的文化,只是後來他們儘量避免進行任何看起來與回教相似的儀式。

噶瑪巴向前傾身,將我們頭上僅剩的頭髮剪掉。我們被納入「僧伽」(Sangha)作為世尊的追隨者,修行佛法的一員。接著,噶瑪巴還授與我們法名。我們滿心期待他會授與我們一些特殊法門的灌頂,騰空而飛的神通力,我們亦期待會有其他特殊的事發生,可是接下來的事卻和想像中的截然不同。在這一個似無止盡的儀軌中,我們蹲俯在地板上,跟隨噶瑪巴念誦一大段經文。不一會兒,我們便開始覺得骨頭隱隱作疼,然而在非凡莊嚴的噶瑪巴面前,我們對此冗長的儀式並無任何質疑。後來終於聽到噶瑪巴說:「當我打一個響指時,皈依境自然會現前。注意集中精神。」其實當下皈依的境界便已歷歷在目,非常清楚明白。就在僧眾們繼續念誦經文的當兒,噶瑪巴朝我們這兒撒了一些米粒,顯然重要的一刻已發生了。最後,噶瑪巴說:「你們必須像我一樣,對佛陀具足信心。」我們心裡卻想:「我們對你比較有信心。」

就在寺院屋頂的陽台上,吉梅醫生替我們翻譯剛領受的法名。漢娜的法名是「巨力荷花」,而我則是「智慧之洋」,我們當然沒有理由抱怨了。每一個人的法名的第一個字都是「噶瑪」(Karma),意即金剛乘佛教噶瑪噶舉傳承的子弟。同時它具有為了廣度眾生而開展證悟事業的願力。其實,皈依的意義是無量無盡的,直至證悟佛果為止,其證量是不可思議的。以利他的發心接受皈依將使一切變得具有意義起來,那也是當下最自然的修行。之後,法的新滋味便會不斷現前。這種皈依是內外悲智合一的境界,不只是一般所謂的「佛教修行的簡介」而已。

在過程中,人們展呈對「三個珍貴稀有之寶」(又稱三寶)的信心。三寶乃為佛、其教法(佛法)與修行上的善友,即諸菩薩(僧)。「佛」超越諸天神與人的境界。此名相表徵圓滿證悟的境地。人類歷史上的第一尊佛,乃為兩千五百五十年前的悉達多太子。佛境地乃每個人堅不可摧、永恆的本質,只是煩惱阻礙了人們對佛性的覺知。世尊的教法展示了何謂世俗與勝義,並且結合了其仍然活躍的加持力與清楚的指示,教導人們該如何達至證悟。此外,在修行道上的夥伴與助手亦很重要,否則大部分人都難以有所進展。目標、方便與靈性上的善友因此對於佛教裡的各個宗派都很重要。他們為成千上萬的人創造許多機緣,讓具有意義與獨立的靈性生活不再遙不可及。

若只是要單純的「走路」的話,這三個就足夠了,可是若想「騰空飛翔」的話就必須導師的指引。上師的覺性就是弟子與佛聯繫的橋梁。他傳授有利的教法,他的事業行持將會是弟子利他的楷模。證道上最直接、快捷的金剛乘佛教中,第四個不可或缺的皈依境,藏文稱為「喇嘛」(Lama,意即上師),表徵三寶的總集體。「喇嘛」在此意指具有傳承證量的上師們,代表諸佛的加持力不斷延續至今。許多人對此的體驗是溫暖和喜悅的,是人們可依賴的正面能量。「本尊」(Yidam)是指心性與證悟的結合。祂帶來世間與出世間的特殊功德,代表諸佛證悟境界的各種形貌狀相,譬如說是寂靜的、忿怒的、男的、女的、單身或雙運身相。這一種超越凡夫一般的概念思維,在灌頂或禪修時被引介入我們的潛意識中,使之漸次增長,與諸佛的智慧合而為一。

其實這些在藝術書籍和博物館裡常見的本尊形相與人的心識一樣真實。當心能清明開放如虛空時,純淨的本尊相和咒音將會自然顯現。祂們源自於諸佛菩薩悲智合一的本性,為喚醒眾生本有的覺性而出現。雖然在藏傳佛教中擁有各種不同的本尊形相,然而若因為本尊的數量與種類眾多而放棄修行是錯誤的。各種本尊相純粹只是心識的表現形式,祂們其實都擁有相同的本質。這好比一家好的餐廳會備有一個豐富多樣的餐牌供食客選擇一般。就修行而言,一本尊即等於一切,只要證悟一個本尊就等於證悟了一切。

護法讓一切佛法事業變得可能。不同的護法擁有不同程度的證量,噶瑪噶舉傳承所祈請的護法都是已達至圓滿證悟者。其佛力能為利益眾生而當下立刻顯現。祂們的忿怒相巧妙地消除了眾生內在的阻塞和外在的障礙。噶瑪噶舉傳承的主要護法乃大黑袍金剛護法,二臂瑪哈嘎拉,祂代表忿怒的威猛力量。這種忿怒威猛的力量和方便二者密不可分。將內心的狂亂及外在有害的障礙當下轉化,金剛乘的法門是全天候無休的修法。它喚醒個人最深層的潛意識,讓多生多世的記憶重新活化起來。因此,這些一般只在生老病死或中陰階段的關鍵時刻才會出現的記憶便獲得釋放。如今在最短的時間以最快的速度將之激活起來,這一種淨化業力的過程是痛苦的。雖然這之後喜悅和自由將會增長,但是人們總是脆弱而善忘的,稍有障礙便會氣餒,生起退心,所以這種大威猛的智慧力量不可或缺,以便打破人們慣有的模式,並且吸收其黑暗的力量,進而讓體驗任運生起。

因此,無論此第四皈依境被稱為「上師」或「上師---本尊---護法」,意思都是一樣的。後者使皈依的意義更能多層次的展現出來。

即如先前所說,皈依的圓滿意義只有在證悟的當下方能完全展現。其正面的力量卻能在當下便能感受得到。它能勾攝眾生,使之與佛結緣。受之啟發的禪修能夠清除眾生阿賴耶識中的負面業習,把惡業化解。皈依使心更加全面完整,因此更能幫助其他眾生。

九月十五對隆德眾人而言是一個很特別的日子。這一天是為期六周的「耶涅」(Yerne)或「結夏安居日」的最後一天。在這段閉關期間,眾出家人將會集結在一起修行、打坐、誦經。這一天也會舉行祭典舞蹈。從大清早開始,長號角與雙黃管的奏鳴聲便一直在耳邊迴盪。我們連走帶跑地從下榻的旅館趕往寺院湊熱鬧。旅館距離寺院也只不過是幾步路的路程。只見僧眾正在院子裡繞著寺院主樓周匝走三圈。他們的頭上頂戴著與密勒日巴尊者相關的傳統金黃與紅色的寬大法帽。雖然身上穿著最好的僧袍,然而他們的自我意識卻沒那麼強烈,不像西方人的遊行般拘謹。反之,他們笑著,不時開開小玩笑,和從各地遠道而來看他們的親人與朋友點頭示意。

在上師面前時的寧靜感覺是前所未有的強烈,眾人更是深受其所感動。燃燒檜樹時所產生的濃煙裊裊升空,與降雨雲交織在一起,因此無法看清是否就快要下雨。雨季仍未結束,如果在舉行祭典舞蹈時又開始降雨的話必定會非常掃興。寺院外的一座小山丘的平地上有一偌大的白色帳篷,篷頂上貼著八吉祥的圖案。帳篷的另一邊設有一個較小的帳篷,帳篷的一面敞開。從帳篷上可看見整個場地的概況,寺院的屋頂,山谷的遠處,西藏的岩峰山巒。當僧眾完成了三圈的繞行後,整個隊伍沿著台階拾級而上來到帳篷處。大喇嘛們在小帳篷裡坐下,其餘的人則圍繞著祭典舞蹈的場地坐下。我們在一群大人物中找到一個靠近噶瑪巴的位子坐下。從那裡除了能夠看見噶瑪巴的身影,也經常能夠聽見他那爽朗豪邁的笑聲。噶瑪巴厚實的笑聲發自丹田深處,總是輕易地便能感染身邊的人。在眾多喇嘛之中,噶瑪巴在展現其證悟的大樂方面有著其獨特的一套方式。這種證悟的大樂與他是無二無別的;這種喜樂的爆發力更是無與倫比,像大自然的力量般震攝人心。

儘管天空中烏雲密布,看起來快降下滂沱大雨,可是不知何故卻一直沒有動靜。此時,祭典舞蹈開始了,以生動的方式詮釋了佛陀的教誨。第一場演出由年輕的夏瑪仁波切所呈現。他是眾年輕一輩的轉世活佛中年紀較長的一位,乃為無量光佛的化身。無量光佛,藏文「Opame」,梵文「Amitabha」,日本人稱之為「阿彌達佛」,中國人則稱之為「阿彌陀佛」。「非個人化」的大智慧能讓人掌控自己的生理基因,像夏瑪仁波切便是最好的例子。他的面貌表情就和佛教文化中世代傳頌的阿彌陀佛一模一樣。

此舞蹈表演和一個未經過訓練的心識在中陰期間的種種經歷有關。中陰(Bardo)是指死亡(神識離開肉體)與再生之間的一個中間狀態。直到獲得真正的解脫之前,眾生的神識都是由投射所制約,因此它會認為所體驗的過程都是真實的,有關這一點在著名的《西藏生死書》中便有詳細的解釋。以各種標誌性的動作和甚具張力的面具,夏瑪仁波切具體地展現了中陰身的種種心識活動與狀態。就如先前所說,當新的感知能力在死亡之際遭到切割時,未受調訓的心不會比平常更具智慧。對於不斷改變的種種投射,它仍會把它們視為是真實的,因此產生嗔心或慢心,貪心或妒心,執著或愚痴等無可避免的煩惱。在不超過七周的這段期間,所強烈偏向的煩惱將會決定神識在六道中的投生處所。因此無論是地獄或天堂,作為鬼魂或天人,投生人道或畜牲道,這種種的體驗不外是眾生所造作和儲存的善惡業行,或稱「業力」(Karma)。眾生所經歷的生與死是一組不斷變幻的夢境,生生世世的延續。簡單來說,這一段舞蹈所要展呈的重點是,在心識了悟其開明、澄淨與無限的本質之前,眾生便會不斷在輪迴中流轉。唯達至證悟後才能無止盡的無畏無苦、喜樂與慈悲,這種力量更是不可思議地強大。

在場的觀眾都能感受到昆吉夏瑪巴的表演所欲傳達的信息。就實際層面而言,他將勝義諦與世俗諦結合在一起,把道和果統合起來。物質主義者強調無常跟變幻的真理,而對於虛無主義者來講,一切只不過是因果的呈現而已。死後的善業以明淨的白色狀相顯現,惡業則以黑色的恐懼或驚慌的狀相顯現。這一個表演對於固執和膚淺的人是不錯的教誨。

接下來的節目是一個非常古老的舞蹈。那是一個有關某個願意將財產、妻子,甚至是雙眼拱手讓人的國王的故事。那個國王的確非常慷慨,只是冗長浮誇的演說讓整齣戲變得十分沉悶。不久,我們便開始覺得那可憐的王后能從他身邊逃脫實在是一件可喜可賀的事。其實在場無論正在進行任何表演,我們的焦點永遠落在噶瑪巴及其周圍所發生的事上。對我們而言,他就是比甚麼都重要。

將近中午時分,我們巧遇一群參加豪華遊的義大利遊客,於是便開始向他們講解有關藏傳佛教的種種。儘管南歐民族在西藏和尼泊爾人身邊看起來是如此高大魁梧,可是同時亦顯得我們是如此的不成熟和分化。這是我第一次正式向西方人講解有關西藏人的宗教和文化,這一個經驗給與了我一個很大的衝擊。爾後,這亦成為了生命中很重要的一部分。我們的講解引起了他們的興趣,他們之後還一起進入帳篷內領受噶瑪巴的加持。現場沒有人說話,可是噶瑪巴和喇嘛們的臉上卻洋溢著無比的喜悅。我們知道他們一直希望能夠將教法傳播至西方國家去。這一個覺知給了我們一個很大的衝擊,我們永遠無法忘記:往後這亦成為了我們生命中非常重要的一個任務。

當所有表演結束後,陰鬱的天空仍未下雨,在場的觀眾很即興地上前展現他們的技巧,娛樂大眾:有的表演摔角、唱歌等等;也有人叫我秀出幾招,於是我便拿出看家本領,向他們展現各種高難度的彈跳。西藏人鐵定都沒有看過這些表演。從尼泊爾到隆德,眾人因為我看起來像藏東的戰士,因此都稱我為「康巴」。那天之後,我的大力氣很難不成為眾人矚目的焦點,一些天真無邪小喇嘛更不時會把我的身體當成樹木般攀爬嬉戲。我那一身結實的肌肉大概是最受人們所贊揚的特質了。

在表演的間歇期間,某人拿了一台收音機前來。那一台收音機是不丹國王送給噶瑪巴的禮物。一九七Ο年,這一台收音機在喜馬拉雅區域看起來可是件奇怪的東西。那台收音機似乎是壞了,他們希望能把它修好。大部分的西方人認為西藏人都擁有不可思議的神通力(當然這不是真的!);反之,西藏人估計我們都擁有甚麼都能修理好的「技術神通」。看來我的瑞士刀又再次派上用場了。當我將刀子攤開來時,他們都驚嘆不已。雖然我比較偏愛摩托車等會動的東西,這台收音機乃屬芝麻綠豆的小問題,未幾便將它修好了。收音機還即時播放了有關當地暴風雨的報導。在場的人都感到十分詫異。漢娜與我亦然。那天整片山谷幾乎被洪水淹沒,可是我們這兒卻一滴雨也沒落下。一直到表演結束了,眾人進入了有瓦遮頭的地方後,雨水才開始嘩啦啦地落下,一直持續了好幾個小時。接下來兩天的祭典亦然:祭典開始前停雨,傍晚活動結束後才又開始下起雨來。就在第三天活動的那個傍晚,我藉故隨口問一名會說英語的僧人為何會出現像這樣的情況。對方簡單回答說:「滿願如意珍寶(噶瑪巴)不希望在表演期間下雨,你看那兒的僧侶們都正在忙著燒香念咒。」每一次和噶瑪巴在一起,我們都知道自己必須要學習的事物很多,在學校學過的事物太少,心裡總是恨不得向學校討回以前繳過的學費。我們所接受的正規教育著實應具有噶瑪巴這種能帶來解脫的實際教授。
