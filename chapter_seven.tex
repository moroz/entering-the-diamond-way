\chapter{雪巴人的國度}

當我們回到加德滿都時,洛本傑珠也剛巧回來了。我們很高興能夠再次和仁波切見面,他聽了我們的歷險經歷後被逗得哈哈大笑。可是他很快又要動身遠行。這一次,他將前往不丹。仁波切若不在加德滿都,我們留在鎮裡也沒什麼意思,於是想要再次回到喜馬拉雅的山區裡去。不過這次不只是爬山而已。山區裡的雪巴國度仍然保有完整的西藏文化,我們深受其所吸引。這一次我們若想要趕在春天起程就必須要加緊腳步了,否則當雨季來臨時,我們整個月就只能受困於風雨之中。

喇嘛傑珠在離開之前已告訴我們應該前往拜訪哪個喇嘛和寺院,給予我們他的加持和護佑。泰瑞和里察在上一趟旅程中教會了我們許多事,如今我們必須要靠自己了。從加德滿都到「雪巴人的國度」雪村布(Shercumbu)最好的方法,就是搭上清晨送郵件的吉普車前往西藏邊境,然後在蘭桑戈(Lamsango)或巴拉拔希(Barabesi)下車,再從那裡往東北的方向徒步前進。

我們在中午時分抵達。河邊簡陋的鋅版屋住滿了來造路的中國工人,我們沒有多作停留便繼續登上那看似遙無止盡的山頭。一路上幾乎不見有樹,太陽閃耀著刺眼的光芒,隨身攜帶的水壺裡也只裝有白開水。當太陽消失在山後,這裡很快便入夜了。就在快天黑之前,我們找到了一個有瓦遮頭的地方。他們這裡除了米飯和茶,便沒有其他食物。

隔天一早,我們又繼續上路。我們在中午時分抵達山脊,當天的天氣清爽宜人。我們稍作休息一會兒後又開始依著蜿蜒的山路走下山。這一周下來就只是不斷重複一樣的模式:上坡來到山口、下坡抵達山谷,接著又再爬坡上山,如此周而復始。其中有一段路會經過佛塔、佛像石雕及其他的文化標誌。幾個世紀以來,西藏人一直是沿著這一條路線將商品從加德滿都運往印度。離開此路段不遠,眼前僅是一片遭侵蝕的枯黃山丘,以及那些生活貧困、棲身於簡陋小屋裡的山區居民。我們偶爾在路上會遇見前來乞討的印度苦行僧,有時則會難得遇見幾個年輕體壯的西方人。那些年,歐洲面孔(尤其是獨自在尼泊爾旅行者)並不常見,他們通常都是參加那些安排妥當,隨行還會有許多挑夫幫忙搬運行李的團體行程。每當途經農場,我們會問:「有牛奶嗎?」「雞蛋?」,農家若沒有這些食品的時候,就會問:「那有大豆嗎?」「扁豆呢?」然而,他們通常只有一些去糠的大米和大量的辣椒而已。如果不幸巧遇有一支探險隊剛路過,那麼他們的食物便所剩無幾。漸漸地就連我們自己也開始猶豫起來。這些山區的農民只有幾平方米的梯田可供耕種,若他們把食物全賣了,我們所付的金錢根本無法讓他們充飢吧?然而這些純樸的山區人民都喜歡鈔票,他們還會把錢花在那些不但會破,而且電池還很昂貴的手電筒上,甚至也會去買那些他們根本用不著的圓珠筆。

若在喜馬拉雅一帶旅行,記得要注意城鎮以外的居民都會偏愛新鈔,而且若用大鈔付錢,他們通常沒有零錢可找。一九七Ο年,十盧比(一美元)或以上都算是大鈔。因此在離開加德滿都之前,我們先將一些小鈔和新鈔留了起來傍身,以備不時之需。此外,有些山谷裡的居民對硬幣也沒什麼信心,尤其是半盧比的硬幣。如果路經某些小山谷村莊,當地人民若以許多硬幣當作零錢找還你時可要當心,因為接下來要擺脫這些硬幣也不是簡易之事。

山區裡的人民一般上都很熱心,他們會耐心地告訴我們眼前那些沒有標示的道路,哪一條才是前往目的地的正確方向。我們也很高興自己能為他們做一些事當作回報。每抵達一個村莊,村民們都是帶著各種疑難雜症,一個接著一個地來迎接我們這些白人訪客。他們通常都是患有甲狀腺腫大、流感、傷口感染或頭痛等問題症狀,一般上幾滴碘酒和創可貼就能解決所有的問題。其實假如有人教導他們給鬧腹瀉的孩童補充足夠的水分,便能毫不費力地救了多條人命。在尼泊爾活不過五歲的兒童比率,大概高達百分之六十五。有一名醫生曾經告訴我們說,因為父母親錯誤的觀念,認為孩子在腹瀉時不應該喝任何東西下肚,所以他們大部分是脫水致死。

我們經常在路上遇見一些體型巨大的流動「障礙」,底部還會露出一雙小腳在走動。這些會走動的「樹叢」其實只是一群扛著一捆又一捆樹葉的婦女,這些樹葉是男人們才剛從樹上砍下的。這種砍自草樹的葉子,將會被儲存起來以備過冬或饑荒時所用,同時也用作為餵養牛羊的糧食。儘管草樹的葉子都被剝得清光,只剩下光禿禿的樹幹,可是一年後樹葉又會再次長出來了。如今這些光禿禿,看起來張牙舞爪的枝幹,讓整個地區看似戈雅畫作裡的風景。每一片山谷都擁有各自獨特的特徵和其獨有的氛圍。有的友善親和,有的則咄咄逼人、傲慢,甚至非常商業化。這一路走來,接下來究竟會是甚麼樣的一番風景,要應付甚麼樣的人,只要來到山口便已能略知一二。有些山谷大部分居民的長相很相像;在有的山谷又會遇見幾個婦女共事一夫,閉門一家親的情況。

翌日,我們踏上一段延自西藏南部的古道。這一條古道就如先前所說的一樣,一路上佛塔、佛像、刻石等數不勝數,洋溢著濃郁的傳統文化氣息。當道路向東分岔出另一條路時,沿路可見的風景又再次變得非常沉悶。接下來的日子,我們仍是在印度教徒的家門口留宿(按習俗外人不能入屋),一路上也沒有甚麼讓人感到振奮的事。我們一直不斷向前,期待著遠處正在等待我們的另一番風景。

吉里(Giri)是座落在那一片廣袤山谷中的一個美麗的村落。有一個瑞士人在這裡建立了一座農場,雜交培育當地和歐洲的牛隻品種,然後生產出尼泊爾品質最好的乳酪,運往加德滿都販售。村裡甚至有一條可供小型飛機降落的跑道。我們下榻的地方提供瑞士式的住宿,設備文明,但價格並不便宜。經歷過先前那一段沉悶的旅途後,我們特喜歡他們那烤得香噴噴的麵包。

吉里之後,我們又開始了登山的旅程,整個地區也顯然起了明顯的變化。而今,我們身處在一個古老的佛教區域。在那裡所發生的事深深地觸動了我們的心靈。這一切似乎蘊藏著一種說不出的意義。我們感到安心、自在、歡喜。

我們才離開小村莊沒多遠,就在一條狹窄的小徑上遇上一隻欲向我們衝來的成年公牛。它似乎是想自個兒獨霸整條小徑,不讓任何人擋路。我將漢娜推向身後的那塊大石,然後趕緊從刀鞘中拔出那一把尼泊爾軍用刀。這一把長兩英尺的刀具,其刃鋒利。這隻畜生噴著鼻息,瞄準了我和漢娜後便發威猛朝我們這兒衝過來,幸虧我們都沒被牛角撞著。這隻公牛像發了狂似的,還好我不必被迫將它刺死。這一場偶發的小騷動卻意外提高了我們對此壯麗的環境的感受度。這裡四周圍是一片對稱而立的松樹林,和早些日子的一片枯黃截然不同。

近中午時分,我們仍在前往山口的坡路上,此時有一個透明、發光,擁有四隻手臂的幻影在眼前豁然顯現。此人形幻影由光與能量所組成,呈現月光石的顏色,懸浮在空中。我無法相信眼前所見,揉一揉眼睛再抬頭看,幻影似乎在我們徒步登山的同時,一直懸浮於空中長達數小時。當時人們還不知道全息圖(Hologram)。換作今天,我會把當初浮立於空中的影像形容成全息圖,只是它更為清晰,而且閃耀著光芒,有一種無以言喻的美麗。所見幻影是觀世音菩薩[梵文:「阿瓦洛吉蝶濕伐羅」(Avalokiteshvara);藏文:「千瑞吉」(Chenrezig)]所示現的主要法相。觀音菩薩乃為一切諸佛菩薩慈悲的總聚。這是我生平第一次如此有意識地親見菩薩,而且還是這麼長的一段時間。不過當時的我並沒有全然去接受祂,反而因為高興過了頭,對於如此殊妙的示現不知道該作如是想。當我和漢娜討論數次後,一致認為那必定是我們進入了佛教地區的一種徵象。

當夜幕低垂,我們來到一個偌大的雪巴人木屋。這裡所提供的住宿和膳食是平常收費的一倍,也就是兩個人差不多五角錢左右。由於天色已暗,我們又挺喜歡這戶人家,因此便決定留下來,也不討價還價。換作平常,討價還價還似乎真是一個免不了的正規作業。無論是此戶人家還是房子本身都有一種說不出的純淨,我們實在沒有心思要去壓榨他們。後來當他們發現我們也是噶瑪巴的弟子時,更是主動降價,我們也很高興地分享他們的食物。

這是一個溫馨的夜晚。這些雪巴人興致勃勃地想知道一切有關我們那身在加德滿都的喇嘛的事。後來他們還送了一串禪修用的念珠給漢娜。這一串念珠本來屬於住在該地區一名證量很高的老瑜伽女行者。漢娜將念珠纏繞在手腕上,當下便接受了其加持。漢娜頓時感到有一股熱能從手臂向上流向心臟,漸而遍遞全身,十分不可思議。稍後,我習慣性地點燃了一根大麻菸想要為此美好的夜晚加分,卻突然感覺渾身不對勁。這是我第一次感到這麼不適,心跳狂亂。我老是忘記了這裡是千佛的國度,祂們顯然並不喜歡我們以大麻菸當香薰「作供」。

\fullPageFigure{./figures/chenrezig.jpg}{佛陀成道日當天在斯瓦彥布,其身色轉變為粉紅色並向我們微笑的觀世音菩薩}

翌日,我們必須翻越好幾個山口。沿途一片廣袤無垠,曾有好幾個小時的路程,路徑的右側一直有一條河流與我們作伴。當天的天氣頗為潮濕,一路狂風暴雨,偶爾還夾雜著大冰雹從天而降。然而亦因為如此惡劣的天氣,我們獲得某個熱心的雪巴人收留,到他們家去避風躲雨。這些雪巴人十分有趣,而且他們家中的供養壇亦非常有看頭。供壇上的佛像或唐卡中所描繪的諸佛聖像都擁有一雙巨型的眼睛和不合乎比例的手,與其粗糙樸素的造型相映成趣。無論是寂靜相、忿怒護法相、女相、男相、單身相或呈雙運身相,這些粗糙的佛像處處可見,護佑加持著那裡的道路和房屋。我們在這裡不需要把人當作小孩般看待,彼此間能夠作出真正的溝通與交流。即使彼此間能說的話或能做的事情不多,當下能相伴於身邊似乎就已經足夠。每當他們發現我們與噶瑪巴的關係後就不願意收下食物或茶水的錢。幸好我們也學聰明了,懂得在他們家的供養壇上留下一些盧比。這樣不但能省下口舌,對方也無法拒絕。

那一天由於我們要拜訪多個地方,因此並沒能走遠。不過,我想我們也是來對地方了。隔天早晨,他們帶了一個瘦得只剩下皮包骨的孩童前來。他快死了。我把那裝有噶瑪巴頭髮的容器放在他的頭頂上,他睜開眼睛笑了。無論那之後發生了甚麼事,那孩子的生命氣息已逐漸增強,他一定會好起來。

我們隔天在十點左右抵達班德(Bander)。班德位於海拔600米處,是一個處於兩山溪流交匯處的營地。從這裡開始一直往上走便會進入雪巴人的心臟地區。人們曾經很仔細地描述過此趟路程幾次:首先,上坡走約三公里路抵達山口,之後再沿著山脊走幾個小時;最後會穿過一片樹林走約幾百碼的下坡路程,之後再向前走差不多一個小時便能抵達雪巴人居住的地方。

山上已有降雪的痕跡。雖然時間不早了,可是我們依然想趕在天黑之前抵達目的地。人們都勸告我們不如等到翌日早晨再出發,不過我們這兩個歐洲人可沒有這等閒情逸致和耐心等待,加上我們對諸佛的加持力亦抱有極大的信心,所以選擇繼續趕路。一路上曲折蜿蜒,我們沿著那看似無盡頭的荒蕪陡坡不斷前進,走過一片無與倫比的美麗景緻。途中不時傳來啾啾鳥兒鳴聲,唯遇巨鷹在頂上蒼穹間盤旋周匝時方才嘎然停止。每走到轉彎處,又見另一片更廣闊的風景。我們靠隨行攜帶的罐裝煉乳來補充足夠的體力,馬不停蹄地趕路,沒有歇息。

在半山腰處有一座寺院,我們決定前往拜訪。寺院裡有一隻巨大如牛的藏獒在看守,一見我們兩個陌生人便狂吠不止。前來應門的喇嘛倒是很有學問,而且非常友善。他很開心地帶我們參觀寺院,而且對於我們就那些精美的佛像和畫卷所提出的問題皆能對答如流。可惜我們無法逗留太久,否則當天就很難趕在天黑之前找到一個有瓦遮頭的地方落腳。

一路上的積雪把鞋子都浸濕了。我們在近山口不遠處巧遇一支小型的馬幫商隊,他們還熱心地邀請我們加入他們的行列。他們知道在山脊的另一方有可下榻的地方,而且還說沒人能夠趕在天黑之前抵達雪巴都城尊貝喜(Jumbesi)。從這一座山口,我們必須繼續走約三個小時才能抵達目的地,那是相當累人的一段路程。我們未曾試過與馬幫同行,喬叟(Chaucer)與其他作家在著作中所描繪的種種情境,如今活現眼前。這些自由的人民無論是文化還是脾性似乎都是恆古不變,而且非常有趣。他們對動物所表現出的仁慈,尤其當我們在回教國家親眼看過那麼多動物遭到虐待的情形後,更是我們未曾預想過的驚喜。這些雪巴人風趣幽默,有一種打自內心散發出的恬淡閒適與自在。看到有人前來請求我們的加持,對他們來說似乎是再自然不過的事情,反而是我們自己覺得不甚自在。我們第一次碰到這種情況。我想這也許是和我較早前在空中看見觀世音菩薩的示現有關。這畢竟也不是一件尋常的事。我將一個印有噶瑪巴肖像的別針和那個裝有噶瑪巴頭髮的容器放在他們的頭頂上,傳承的加持力帶來了極大的滿足感與覺受。沿著山脊而行的這一條道路引領我們穿越了景色超凡的杜鵑木林。從山脊遠眺,兩側的山谷景緻幽絕;越是深入,沿途的風景就越是更勝一籌。當太陽消失在地平線上,我們的馬幫商隊終於進入了雪巴人的山谷。雖然我們每個人都已濕到膝蓋上,卻掩不住內心裡的喜悅。

尼泊爾擁有超過十三種不同的民族部落。各個部落族群之間鮮少互動,各有各的生活。大部分的印度教徒生活在茅草屋裡,其他族群的房屋就活像是博物館似的,像尼瓦族佛教徒的房子就是一個很好的例子。他們的房子總是給加德滿都山谷市中心的那些遊客們帶來很大的驚喜。雪巴人堅強、活躍,他們無論在哪兒所建造的房子都很好看。雪巴人的木屋寬敞,其建造年代之久遠可追溯至西藏人開始進入尼泊爾經商的時期。雪巴人就是在那個時候開始從事商業貿易和經營馬幫。後來當這一種職業因為中國對西藏的破壞而消失之際,他們找到了另一條出路,轉而成為許多喜馬拉雅遠征隊伍和加德滿都商賈的嚮導。他們會願意為了打拼和機會而不惜遷徙他鄉,亞洲社會向來頗受傳統所束縛,所以雪巴人這種特質算是相當難得,再加上天生的巨肺,於是成為了高山嚮導的理想人選。然而他們如今也不得已要面對年輕人外流的問題。雪巴族群就像西藏人和不丹人一樣,喜歡組織小家庭,希望家中成員都有機會受教育。不像同地區的其他族群的婦女們,肚皮從來沒有閒下來過,好像是要為真主阿拉組織一支軍隊方善罷甘休似的,不然就是希望多生兒子,好讓他們長大後能賺錢養,為自己能安享晚年作保障。這一種自我主義不但污染了原本美麗的國度,更令其變得雜亂無章。

轉眼間天色已暗,若要繼續趕路會相當不便。就在此時,我們看見前方左側有燈亮著,那是我們在旅程中所遇見的第一個農場。我們整隊人馬受到農家所款待。農場的主樓後方是一個圍起的三角形草地,草地上豎著一支長竿,長竿上懸掛著勝利幢正隨風飄揚。我們躺在剛割下的乾草上,伴著馬鳴聲與經幡隨風拍打的聲音酣然入睡,然後做了一場深沉的美夢。

翌日早晨,馬幫的那戶雪巴人邀請我們到禪修室去。他們之前看過我們在靜坐,所以現在想要給我們看一些東西。他們將鞍囊打開,然後將裡頭的東西一一陳列在桌上。昨天晚上甫抵達農場,他們拿進屋裡的第一件東西便是這一個鞍囊。我們平時難有機會看到手工如此精細的佛像和大修行者的聖像,不過聖像上的臉孔都已被搗毀,顯然是遭到中國人所破壞。它們都是非常寶貴和嚴謹的藝術作品。聖像底座完好無缺,這意味著這些聖像仍然具有加持力,這大大地提升了其作為禪修工具的功效。聖像內部空心的部分主要由經卷及大瑜伽行者與具足加持力的聖地之舍利所填實,全然接受聖像者將能感受其中殊妙的能量。雪巴家族說:「今後它們會有新的面孔,然後我們會把它們帶到一個人們能夠了解和懂得珍惜它們的地方去。」

我們很感動。他們大老遠將這些具足加持力的諸佛聖像運離西藏,好讓它們能真正發揮其作用,這是很了不起的。可是我們心裡也清楚,這些聖像將無法安然地在寺廟裡被供奉很久。有一些無恥的商賈付錢請幫派到全國各地去竊取佛像,之後再轉手把它們賣到西方國家去。這些佛像來到西方國家後,內部的裝藏物品會被海關人員(為查看裡頭是否暗藏毒品)或好奇的收藏家所掏空;不然就是淪落到一些人的家中充當一件充滿異國風情的裝飾品或是一項投資而已,這些人根本就不懂得這些佛像對心性發展的真正意義。這一個地方給了我們一個很寶貴的機會去明白業力運轉的方式,也看眾生如何種下了來世苦樂果報的種子。有一些果報未必會在來世方才顯現。業力在助緣下也許會更快成熟也說不定!

此時,整個加德滿都都在談論著欽尼喇嘛的幾個兒子。他們持械打搶被捕,如今已押庭受審。持械搶劫這等事在尼泊爾可說是前所未聞的境況。他們就在噶瑪巴給了欽尼喇嘛那一件白袍後的第二天落網。我們心裡都明白,那件白袍象徵噶瑪巴對欽尼喇嘛滌罪的支持。在審訊過程中,欽尼喇嘛的家族涉及不同領域違法犯罪行為的事隨即也昭然若揭。這些事在尼泊爾似乎老早便已是一個公開的秘密。

前往雪巴國度都城尊貝喜(Jumbesi)的路況雖好,也有明確的標示,可是並不好走。一路上不知何故總有農民不斷從田地裡拋石頭,所以我們一直小心翼翼地避開這些飛來橫「石」。我們沿途經過許多田園風的小木屋和數不勝數的舍利佛塔。這些佛塔很難讓人忽視。我們最近聽說其實舍利佛塔內安奉著許多法物和舍利,給整個環境非常殊勝的加持。人在加德滿都很快便學會沿著佛塔的中心以順時鐘的方向周匝繞行。

舍利佛塔[「Stupa」窣堵坡]是佛教文化中的一種表現形式。佛塔的造型是以內、外、密各層面對宇宙的解釋而建造,經典中亦不斷強調佛塔對人的心識會產生不斷深化的影響。它們幫助眾生將當下的娑婆穢土轉化為原本的佛淨土。它們在眾生的心裡種下了覺悟的種子,使之能明心見性,透視萬法本質,而且超越一切期望或恐懼,將眾生引領至一個無論任何境遇都是真實和圓滿的境地。因此,眾生的心識將漸漸能夠辨識其本性中的喜樂與自由。眾生的心識,將能從有緣境走向無緣境,從迷亂走向菩提。

佛塔最底部「稜角分明」的方形基座象徵宇宙的堅固性或「土」元素,以透明黃色作為表徵。塔座與諸煩惱中的「我慢心」以及「南方」相應。到證悟時,煩惱即轉化為平等性智,明瞭萬法平等,萬法本自不有,裡裡外外皆因種種因緣條件和合所成。塔座上方呈圓形、水滴狀的部分象徵水元素,宇宙的「流動性」,以透明藍色作為表徵,與「忿怒」煩惱及「東方」相應。當忿怒轉化為清淨的大圓鏡智時,它會顯現出萬物的真如本性,即不增不減。其上方長方形的結構象徵「熱能」、火元素和「西方」,以透明紅色作為表徵,象徵「自私的欲望」,一個人通過修行將之轉化為妙觀察智。這一種頓悟和樂天派的人最為相應,他們在觀察到事物之個別差異性時,也能悟出萬事萬物之整體性和統一性。在西藏與尼泊爾,一般上這部分(如同在博德納和斯瓦彥布的佛塔一樣)四面繪有巨大的「慧眼」,觀望四方。其上方有環狀物的錐形結構,一般象徵「風」的元素和「移動性」,以透明綠色,表徵「北方」和諸煩惱中的「嫉妒」與「嫉羨」二心。通過見(見地)、修(禪修)、行(行持)將此二種煩惱轉化為成所作智。塔頂乃為新月造型,新月懷裡抱有一個太陽,或被理解為「空」元素中烈火熾盛的水珠與盛滿生命甘露之碗的一種表徵。它透明、發光或像月光石。在其不淨的狀態,它表徵「無明」與「昏沉」。這種煩惱鮮少以直接的方式對治之,通常會在一個人對治前四種煩惱時間接被轉化為超越時空侷限之法界體性智。這五種智慧成就圓滿的覺悟。這五種智慧以無畏的洞察力、任運的喜樂與大慈悲心示現,並以豐富、不可思議、護佑的方式給予眾生寧靜,給予眾生加持。西藏畫軸上以金剛坐姿盤腿而席,交叉雙臂,身呈藍色的金剛總持[藏文「多傑羌」(Dorje
Chang)],在噶瑪噶舉傳承被視為是圓滿覺性的表徵,如金剛般堅固不壞,通過喇嘛,尤其是歷代噶瑪巴所示現。又稱為大手印(英文:Great
Seal;藏文:Changchen;梵文:Mahamudra),乃為一切眾生皆具有的究竟本質,無論何時何地都保持這種正知便是圓滿的證悟。

在雪巴人的國度裡,一簇一簇齊肩高的瑪尼石牆都是作為原始的「路標」而存在,為行人標明行走的路線之外,也和舍利佛塔擁有同樣的象徵意義。人們以五種顏色在石牆上刻上「嗡嘛呢唄彌吽」與其他經文,也刻有各種家喻戶曉的佛形相與吉祥圖案的浮雕。因此沿著石牆行走,也成為了能打開心性的一種修行。

儘管此地區落後純樸,佛教的影響卻尤見深遠。此地區萬事萬物都是大整體的一部分,對於心態正面的觀察者而言,這裡充滿說服力地表彰了各種趨入心意識各大層面內義的法門。漢娜與我非常享受這種氛圍,感覺自己正被所信任和喜愛的力量所包圍、啟發。我們激動也開心,不時慫恿著同行者策馬奔騰。

我們一行人路過一塊巨大的懸岩,懸岩上紋刻著巨大的咒語,似乎是才剛漆上顏料不久。只見眼前的風景變得開闊起來,高地下是一片廣袤的山谷。山谷裡約有二十來間房屋緊緊靠在一起,像一個小村寨。那裡就是雪巴國度的都城尊貝喜。當我們靠近尊貝喜時,看見在山谷左側的草坪上坐落著一棟獨立的高房。有一名義大利籍的年輕人出現在高房門口,然後朝我們跑過來。這名年輕人住在高房裡,正在學習唐卡繪畫,而且對於我們一路上碰到甚麼人感到非常好奇。他看起來非常孤獨和迷惘。他一句藏文也不懂,我們猜想他會突然跑過來搭訕,應該也只是想確定自己是否還活著而已。就在這群房屋後方的數公里處,半隱藏在山后,矗立著一座宏偉莊嚴的寺院,我們深受吸引。由於先前答應了丹麥的朋友在擬定任何計畫前會先到鎮裡與他們會面,因此我們還是先到鎮裡去了。我們在鎮裡的兩家吃店裡都不見他們的蹤影,不過卻發現他們在鎮裡無人不曉。了不起的他們似乎喝光了整個尊貝喜的「唱酒」(Chang)。唱酒是雪巴人自製的一種啤酒。我們聽說他們出遠門去了,打算要把周圍地區的唱酒都喝光。後來我們也打聽到想要前往的那座寺院正是度齊仁波切(Tuchi
Rinpoche)的寺院。我們遵守了和朋友的約定先到鎮裡會面,現在既然他們出遠門了,那麼我們便留下紙條和大部分行李,迫不及待想要登山去拜訪度齊仁波切的寺院。傑珠仁波切曾向我們大力推薦這位度齊仁波切,因此心裡非常期待能與他見面。

我們走在青山綠水之中,一路上伴隨著瞬息萬變的天氣與英式公園般的優美景緻,頗饒有詩意。寺院是典型的藏式建築,也許不久前有人慷慨地捐了些錢給寺院,只見沿途刻有的咒語圖紋,全閃耀著新漆獨有的亮麗。寺院主樓依山而建,主樓前是一座中央庭園,四周圍起了一道牆。主樓周圍建有單層的排房,樓下則是喇嘛、出家人和女尼的寮房。寺院後山上的石岩處(就在主樓右側要走好一段距離方能抵達的高地),岩壁上如蜂窩般散落的石窟是禪修室。瑜伽行者們在這些建在岩壁裡狹小的禪修室內閉關修行數月,甚至是數年的時間。這些禪修房一般上小得只能供人坐著。在外行人眼裡看來,這些小禪修室似乎是可怕又折磨人的地方。可是對於修行者而言卻是一個能生起大樂之處。像這樣的一個處境能夠讓人避開外在環境的惑亂,心往內觀,瞭解真如本性。唯心才能體悟,才能有所成就;而且通過適當的教誨,大樂便能任運而生。這是所有人永恆的本質精髓。

有一隻很吵的巨型大狗負責看守這座寺院。不過就在我們瞄準向它丟小石頭幾次後,它總算很識相地知難而退。可是我們就是沒法子讓那一群叫聲粗劣嘶啞的烏鴉閉嘴,因此敲門敲了大半天才有人前來打開一條細細的門縫,瞧瞧外邊來者何人。

前來應門的僧人聽說我們是噶瑪巴的學生,而且是喇嘛傑珠派來的人之後,他才把門打開。他看起來非常高興,一路領著我們到廚房去招待我們喝茶。在西藏人的家裡或寺院裡,廚房是主要的活動場所。倘若你受得了火爐裡的冒煙,西藏人的廚房倒是個好地方,只是對我們西方人來說,這煙有點兒太刺眼了。那位僧人是個天才。他非常擅於模仿人,就連所有旅客的噩夢——在加德滿都那個討人厭的簽證官員,他也能模仿得唯妙唯俏,把我們逗得捧腹大笑。我們完全沒有預想過在如此嚴肅莊嚴的地方會有這麼一個逗趣且擅於模仿的人。他亦和我們分享了他們的故事。他跟隨度齊仁波切與其他僧侶,從珠峰北部地區逃到這裡來,逃離中國的壓迫。他向我們展示舊寺院裡的一幅壁畫(如今寺院已毀)和某張全新的相片:照片裡噶瑪巴正施賜福加持手勢,腳板和掌心彼此直線相對。

幸好我們留了一些阿斯匹林、創可貼和碘酒在身邊,這時候就可以用來當作小禮物送人。那位僧侶把我們帶到一間裝飾精美的房間裡,說我們可以在那兒留宿。他說:「仁波切剛進入了為期六個月的閉關。他就在樓上禪修,沒人能見他。歡迎你們在這兒留宿。當是在家好了,別拘束!」

就在我們踏上雪巴國之旅前,我們收到了人生中第一本有關金剛乘修法的英文書籍。送書人是喜馬拉雅東部小鎮卡林邦的一名華裔瑜伽行者。他曾要求我們從丹麥給他找來一些書籍,那些書籍的類型可真是有點兒讓我們不知所措。在那一個純真的年代,這些描繪「女性美」的書籍使到我們的國家名聲大噪。只是我們不明白一個女人的裸體究竟對他會有何作用。難不成西方人和東方人的內部脈輪存有微妙的差異?好吧!姑且先不管究竟是甚麼原因,他確實給了我們一份很棒的回禮。如今,這一本綠救度佛母[英文:Liberatrice;又藏文:多瑪(Dolma);梵名:多羅(Tara),乃諸佛菩薩慈悲的化現,以女相度化眾生]修法的英文翻譯本已經安然地躺在我的背包裡。那個年代,許多禪修方面的書籍都是藏文版本,因此這英文譯本就顯得特別珍貴。

當時我們並無意識到自己缺少了成為一個正信佛教徒所需的「印證」。噶瑪巴在尼泊爾的兩個月,我們參加了一次又一次的黑寶冠儀軌和灌頂。噶瑪巴的威力讓我們為之暈眩,不過周圍也沒人主動向我們提起有關皈依的事。即使尚未「正式」成為佛教徒,(當然我們依然是佛教徒),在觀想度母莊嚴圓滿的形相,念誦其心咒「嗡 塔列 度塔列 度列 梭哈」時,感覺仍舊是相當美好的。我們只是是很難去辨識散亂的自心其本來面目其實是與度母不變的覺悟本質無二無別。無論如何,度母就如慈母般愛護著我們,心門一開,便能感受到祂如潮湧般的大愛與加持。

由於翌日早晨醒來後證悟未至,我們心想也許應該要藉由隨行攜帶的那些高純毒品來提升美好的經驗。漢娜一次攝取了平日雙倍的份量,而我則依平日的份量攝取了數次後便出門去了。我們此行的目的地是光禿禿山坡上的一塊小平地,就在寺院往上走幾百米處。那個地方看起來非常適合靜坐禪修。LSD(俗稱「弗得」的一種致幻劑)的藥效很快,我們在上坡途中便已開始發揮迷幻的作用。那一條陡峭的小徑沙塵滾滾,不時還有小石頭滾落。小徑仿若剛從沉睡中甦醒過來,整個大地也像會呼吸般律動、搖晃起來。我們來到山坡上的小平地。在和煦的陽光下,放眼就能遠眺整個雪巴國度的美景,我們能在這一個距離寺院只有數百米的山坡上靜心打坐,心裡除了喜悅還是喜悅。,我們的氣息與群山大地一次又一次地融為一體。紫銅和黃銅的元素能量通過灰色的岩石溶入我們體內。它的意義不斷見次彰顯,身體的的各個部分,每一個細胞和原子都處於大樂之中。忽然,我注意到周圍有馬蠅在飛舞,也許它們從剛才開始就一直在那裡。我讓第一隻馬蠅吸飽了血,心想這隻馬蠅若因為吸了我的血而莫名其妙地踏上迷幻之旅的話不就很好笑麼?後來當它飛走時,我有注意到它很踉蹌地轉了幾個圈。當另一隻馬蠅停落在我身上想要大快朵頤之際,我沒多想便將它一掌拍落——扁了。就在我的手臂上還躺著一隻被拍扁卻仍然微動著身子掙扎的馬蠅之際,原先那種美好的感覺乍地消失不見了。我雖然擁有讓世界生起又幻滅的能力,但卻無法對一隻小動物生起慈悲心。這讓我感到相當困擾。

傍晚時分,我們沿著下坡路回到寺院,藥效仍持續著。把馬蠅打死後,我的內心一直非常不安,似乎有甚麼正努力想要掙脫出來。我們在下坡途中巧遇幾個正在建房子的雪巴人;在我眼裡看來,他們又矮又醜。我深知一個人外在的際遇是其內心的一種投射,這實在不是一個好的跡象。現在問題是我該如何從當下的迷幻異境中抽離,重新恢復正常的意識狀態,回到那個友善、正面又積極也饒有衝勁的自己。就在我們快要抵達寺院之際,我們注意到一棵小樹,樹上繫了多卷法本。這些法本已十分破舊,無法在法會儀軌中使用了。人們將這些法本掛在樹上,想必也是希望將它們的正面能量傳播給眾生。這一個舉動背後的慈悲心才是關鍵,其餘的都已不重要。當我了悟了這一點後,緊緊揪著的內心終於能釋懷。我的雙眼泛著感恩的淚水,內心裡一遍又一遍重複承諾著要致力於推廣這些寶貴的教法。漢娜站在我的身旁,我們誓言要將大智、大悲、大力統合的教法傳播出去。

進入寺院後,我們在諸佛面前再次堅定地許下誓願,使之穩固、加深,並且一直維持著此願力。我們認為這必定是過去的一種因緣,未來生生世世也必定會延續下去,其實這不外是常理:如果我們老是為了自己,問題就會生起。如果我們一心為他人著想,則自然能如意滿願。

我們的丹麥同胞終於回到尊貝喜。他們已將周邊地區所能找到的啤酒全都喝光,現在回到尊貝喜,心裡盼望著這裡的人已釀好新酒等他們歸來。能再次見到他們真好!我們都有很多故事想要和彼此分享。他們很快便融入了雪巴國度,成為了這裡的一分子。他們那維京人式的行事作風廣為當地人所接受,和他們完全沒有隔閡,很是親近。最近,他們才剛和某個歷史學家談論過雪巴人的歷史,簡要說一下大概是這樣的:

從十四世紀開始,雪巴人便分成三波南下,每一波的遷徙相隔了數百年的時間,分散至目前他們所居住的各個地方。他們來自藏東地區(如今的康巴省),顯然是遭到好戰且擁有部分歐洲人血統的外來部落所驅逐。新一代的康巴是強悍的藏民。直到今天,他們仍然堅毅地抵抗著無情的中國軍隊。有人說,康巴族一半是強盜,一半是聖人。在這一片受到保護、美麗的喜馬拉雅山谷中,雪巴人仍然保留著一些在西藏已近乎失傳的古樸習俗。在這些習俗當中,最廣為人知的就是人們不分晝夜,飲酒歌舞長達一周的宗教慶典。

這一種盛宴不僅是單純的狂歡活動。它們有助於消除人與人之間的侵略性與隔閡。在場的喇嘛引導其證悟的能量,使到這些雪巴人能感受彼此間的同一無二,達至一體的境界,因此這些活躍的人們的心境才會如此平和,整個民族如此團結的緣故。這些喇嘛有的來自「舊派」的法教傳承,也有的來自噶瑪巴的傳承。他們當中許多人亦擁有明妃。作為具有高證量的上師,他們能夠同時受用且又超越一切世間的欲樂。

尊貝喜的每一個人都知道度齊仁波切已進入長達半年的閉關,通過禪修轉化此險惡世界中的種種苦厄。這裡的人們在生活上總是不離皈依。人們確信圓滿的證悟乃為人生中最究竟的目標;引導他們走向此目標的方法就是佛陀的教法,也就是佛法;在道上,他們所能信賴的朋友,就是僧伽。若要以最有效的方便取得最快速的成果亦需皈依喇嘛,即上師。他(她)將會是信心、精神上的力量與佛法事業的根源。

這一種對於究竟皈依的確信不但啟發了雪巴人,同時也啟發了貧困的西藏難民。如今越來越多西方的修行者也能彰顯他們心靈圓滿的特質,當心靈不再執取於希望與恐懼時,一種溫暖圓滿的狀況便能任運無礙地顯現。這一種任運的能量能夠超越二元對立與主客二體的分別,而且是任運自然地現起的。

在這裡房子都是以灰色的粗石和全木板所建造,堅固、扎實,也不難進入。每一間屋子都會有一間供奉佛的房間,有的會用以修行或置空。後者通常會有一個供宗教舞蹈的院子。這些房子以手工粗糙的木柵欄圍起,和加德滿都山谷裡頭那些雕工精細的房子有著很大的差別。供壇上有許多已具上百年歷史的佛像,就如沿途所見的許多畫卷一樣,它們都擁有非常明顯的標記:一雙巨大無比的手和眼睛,和一個極長的腹部。這些獨特的標記特徵表示它們都是由農村(而非城鎮)裡的工匠所製造。

在尊貝喜放眼一望總是不離佛塔的風景,古魯仁波切(Guru
Rinpoche)的加持力更是遍蓋了整個區域。古魯仁波切乃為一個證量圓滿的瑜伽行者,也是他將佛教傳播至藏地。古魯仁波切「珍貴的上師」,也稱為「貝瑪桑巴哇」(Padmasambhava)或「貝瑪炯聶」(Pema
Jungne),意即「蓮花生」,乃為密乘佛教最高法教之持有者。大約在西元741年,蓮花生大士(簡稱「蓮師」)將一個究竟、超越自我的證悟的視野,和眾多能讓行者達至證悟的金剛乘修持方便,從如今已隸屬阿富汗的某個地區帶到了西藏。他曾將眾多法教傳給了其中一位明妃,後來外道入侵,佛法受到慘重的破壞,蓮師的教法以伏藏的方式被保留了下來,而且只能在因緣具足的時空下才會通過瑜伽行者取藏。他的教法受到舊譯寧瑪派所認可,但卻不太受到統治西藏的黃派或「善規」格魯派所認同。

數天後,我們一行約七或八人,展開了一趟旅程。我們在平均海拔3000米的地方遊走,計畫環繞雪巴國度的心臟地帶一圈,拜訪一些重點景點和寺院。此趟行程要避開的地方就只有著名的天波齊(Tangboche)和龐波齊(Pangboche),這兩個地方是挑戰珠峰者的基地營。澳洲和日本的兩大遠征隊伍將一路上的食物都掃光了,致使沿途的食物價格高漲得離譜!

我們從地勢較低的某條路徑出發,取道某條路經全鎮最大佛塔的道路,再走過濕滑的木板塊,左轉渡河,然後沿著某座山腰順勢而上。蜿蜒的小路帶領我們穿越猶如無人到過的一片樹林,沿途的景色,美麗得讓人窒息。每到轉角處就是一片全然不同的風景,無論遠近皆美不勝收。

途中倘若路經雪巴人的房子,他們大部分都會邀請我們到屋內做客,招待我們食物。他們會向我們探消息,對於我們隨身攜帶的物品都會觸摸和討論一番。幾乎每到一處,我們隨行攜帶的泡沫膠墊都會被用來交易五次以上,其他的物品則至少一次。他們也會迫不及待將噶瑪巴贈與我們的舍利放在頭頂上,領受他的加持。大部分的人會問噶瑪巴目前住在甚麼地方,然後也會很自豪地說自己的哪個叔叔或阿姨曾在加德滿都見過黑寶冠,領受過噶瑪巴的加持。

雪巴人的食物選擇極為有限。他們主要吃馬鈴薯,都是直接將馬鈴薯扔進壁爐裡烤,然後用手扒開來吃。他們也有烘過的青稞粉,叫做「糌粑」。他們將青稞粉和茶摻拌在一塊兒,和勻的麵糰也叫做糌粑。如今在大部分情形下,他們不得不將就著用小麥、玉米,甚至是大豆。這種傳統的主食是此民族的文化中最著名的東西之一,有說只要一個地方有人對喇嘛抱著極大的信心,且以糌粑為食,那裡就是西藏。雪巴人的餐盤中有時亦會有酸奶,如果周圍恰好住著會屠宰動物的穆斯林,又或剛巧碰到死了的動物,那麼他們便有肉可吃。蔬菜會把雪巴人給嚇跑!許多西藏人是在流亡他鄉後才知道「蔬菜」的存在。即使到了今天,他們當中仍有許多人會以為這些綠色的東西只是給牛吃的而已。

製作糌粑的方法有幾種。一般上,人們會先將穀物倒進一大鍋的熱砂子裡,然後開始拌炒鍋裡的砂穀,直至穀物爆裂開來。之後利用篩子過砂,剩下的穀物將會被放入一個袋子裡搖晃,以去掉外殼。最後,剩下的穀粒就會被細磨成粉。雪巴人大量喝的「茶」,西方人實在不敢恭維。原因是西方人對「茶」一字的定義與他們大不相同之故。對西藏人而言,「茶」是一種以牛奶、鹽巴和牛油,再加上一點兒低檔茶調味混合而成的「湯」。對於居住在高山上的人而言,這種茶的脂肪含量高,喝了有助於保暖禦寒,抵擋刺骨的大風和寒意。如果要將它理解為我們所熟悉的「茶」,確實是挺可怕的;不過作為「湯」倒是不賴。此外,旅遊書籍裡常寫說要用變味的牛油來煮茶也不是事實。他們當然也肯定會喜歡以新鮮的牛油來煮茶,只是對於生活貧瘠的他們來說,也就只能物極其用。數天後,我們也嚐了一口茶,箇中的滋味,恐怕畢生難忘。

雪巴人的家都設有煙囪,這種基本設施也許讓他們的生活變得輕鬆一點,然而無論我們進入哪戶雪巴人的家,不到幾分鐘便會開始眼淚直流,接著便會火速踉蹌逃離屋內那一個開放式的壁爐。壁爐裡的排煙理應是會通過屋頂上的洞口(只要把木瓦板蓋推開便可)被排出屋外。木瓦板蓋的位置取決於風的強度和方向,不過這一個系統顯然無法有效地發揮其功能。再說,他們所焚燒的木材內似乎含有過量的酸性物質。因此即便所排出的煙不多也仍會感覺刺眼,不一會兒工夫眼淚便開始嘩啦啦地流。

我們這幾個西方人因為刺眼的排煙不斷在揉眼睛,雪巴人們似乎沒有受到任何影響。大體上,他們都不是太嬌氣的人。他們能空手將燒得火紅的碳塊從火堆裡拉出,也能在沒有手柄的情況下,將一個裝有沸水剛燒開的大銅鍋扛起。對我們來說,無論是生活方式、身材或五官,他們每個人看起來都差不多並不特別吸引,不過他們在的日常生活上所經歷的事,可能遠比我們這一些生活在一個消費主義世界裡的人還要來得更多姿多采。雪巴人有一種說不出的古肅,像從那些灰色的岩石裡蹦出來似的,所以那幾個較「摩登」的(就像模仿簽證官員的那個出家人)就會顯得特別出眾。

就像在尼泊爾時一樣,無論我去到哪兒,人們都想要將我繫在腰帶上的那把廓爾喀彎刀買下來,我只好每次都禮貌地藉機轉移話題。大部分在尼泊爾所購得的刀,質量都很好,它們都是利用裝在貨車裡的那種彈簧所鑄造,像我這一把廓爾喀彎刀就是一把軍用的刀具,刀身的平衡性很好,甚至能砍斷一棵小樹。無數次的冶煉鍛造使它成為了一把完美的刀具。按照傳統,「彎刀出鞘必見血」,因此廓爾喀士兵們都會在指頭上留一個開放性的小傷口。如今我已將這把廓爾喀彎刀交由弟弟去保管,就在他位於瑞典南部樹林裡的家中。

旅程在一開始時的步調很緩慢。我們的朋友約翰是某個印度教斯瓦米的信徒,他堅持要穿著他那礙手礙腳的長袍上路。他甚至希望比當地人更道地,竟然打著赤腳走在那些即使是當地人也會很樂意穿著鞋子走的路面上。我們為了遷就他而把腳步放慢,沿途不知看了多少左右兩側的花草樹木。後來我們索性不管悠哉的他,直直地向前走。我們穿著舒適的網球鞋,帶著輕便的行李上路,那種感覺還真不賴!

我們沿途一直思憶著西藏最著名的瑜伽行者密勒日巴尊者,大地似乎也甦醒了過來。九百多年前,密勒日巴尊者曾經在此地區靜坐禪修。自從我在那被遺忘的山谷中修持拙火定開始,便一直能夠感覺到尊者的加持。當地人向我們指出數個尊者曾經在那裡禪修的洞穴。尊者的圓滿證量及所傳唱的道歌一直利樂無數眾生。關於尊者的書籍,像《西藏偉大的瑜伽行者密勒日巴》(Tibet's
Great Yogi Milarepa)、《飲入山泉》(Drinking from the Mountain
Stream),以及《密勒日巴尊者的十萬首證道歌》(The Hundred Thousand Songs
of
Milarepa)都是藏傳佛教中非常顯著的代表。他在人生中各個階段的發展,迄今仍然啟發了許多人。尊者以生命實踐佛法,他的故事比任何學術性的論文更能讓我們了解金剛乘佛教所追求的目標和修行方式。

這整個地區似乎仍和九百多年前一樣並沒有多大改變,就連當地人民的生活亦然。除了多了一些塑料製品和些許技術上的改進,他們的生活方式依然保有了舊時的傳統。前往曲旺(Tjuang)的道路有兩條。曲旺是山谷上方一個非常老舊、宏偉的寺院綜合建築。我們選擇路程較短卻相對較陡峭的登山者路線。不一會兒工夫,我們便被四周林立的寺院所包圍。

寺院主樓的周圍依然建有許多小型的禪修房。一些受過特別訓練的瑜伽行者會在秋收前到高原上作法,將可怕的冰雹趕走。在防雹儀式中,行者吹響長號角和海螺施法,將雹雲引走。山上的冰雹可不是鬧著玩的,所降下的冰粒可達至雞蛋般的大小,輕易就能取人和牲畜的性命,將農民辛苦勞作的莊稼破壞殆盡。這裡的寺院已顯得非常破敗和荒涼。如今只剩下幾名僧侶和女尼住在寺院裡,就連寺院大堂內極其精美的藝術創作也已失修多年。這裡不像印度的拉達克(Ladakh)或西藏的中西部,在高海拔地區氣候乾燥,只要沒有外來入侵或受到像文革運動這種事的干擾,無論是建築物或藝術品都能保存長達世紀之久。這裡的氣候潮濕,屋頂一旦破洞,霉菌很快便會出現,那些沒受到良好保護的物品便會很容易因此壞掉。總之,寺院周圍的地區給人一種陰鬱頹喪的感覺。當地甚具影響力的喇嘛們都已經離開,已經沒有人將人們凝聚在一起,就連許多農民也漸漸遷徙至其他地區去生活。

美國人理察在曲旺上課已有一段時間。他是最早開始修持金剛乘法門且深感受益的西方人之一。理察希望能夠過修行的生活,卻因為遇到我們這幾個飢渴的丹麥人而頻頻受阻。他也很容易受人慫恿而動搖!當我們剛抵達曲旺時,「將軍」伊恩便已經前去拜訪他。「將軍」可是一個難纏的客人,說一不二。他有一個了不起的本事,就是總是非常清楚地知道哪戶人家正在釀酒,而且他也擁有堅毅不拔的精神和耐力,能扛起兩大桶裝滿了全發酵的青稞「湯」沿著漫長又蜿蜒的道路上山。有人曾經開玩笑調侃他說,如果他能夠將這些良好的品質都運用在修行上的話,這個世界便會多一個偉大的上師。

\genericFigure{./figures/dorje-bell.jpg}{金剛杵和鈴,乃為象徵男女雙運、悲智雙運與空樂雙運的法器。}

我們只在曲旺待了一晚。那天早晨,漢娜將某個已往生的女尼在生前所使用的小金剛杵和鈴買了下來。我們付了七美元,就當時的行情來說是太貴了。當時的我們天真懵懂,不曉得許多穿著僧袍的人就像大部分的亞洲人一樣,都是天生的生意人。我們總是不殺價就按對方所提出的價格乖乖付錢,甚至認為這也是一種加持。即使是今天,我們對他們的了解多了,若要和他們交易也仍然會覺得很奇怪。然而有許多特殊的事情已聚合了在一起,如今這些殊妙的禪定加持物賜予了世界各地的金剛乘禪修中心一種不共的加持能量。

放眼遠眺脚下的山谷,只見蜿蜒的小徑一路引領至塔信都(Taxindu)寺院。從珠穆朗瑪峰吹襲而來的雪霧與風籠罩了一切,當霧散去時,一座巨大的山峰冰壁就在視野裡的左方。就在長長的山坡上那一條小徑旁,矗立著一間小木屋。這裡住著一名年老的女瑜伽行者,身上總是穿著一襲襤褸的紅色長袍。大家說她的家總是不缺牛奶和酸奶。我無法將視線從她的身上挪開。她擁有一種優雅卻又帶點不安的女性能量。這一種女性能量自幼便深深地吸引著我,總是能激發起我的保護欲。

就在毗鄰珠穆朗瑪峰的這一個地方住著像這樣的一名女性,如果她是生活在哥本哈根這種現代城市,必定會成為那種婦女組織領導人或在各種領域都能創造出偉大藝術的傑出女性。我看著她,內心裡泛起了難得的思鄉之情,也讓我想起了母親,想起了業力的運轉,想起了導致我們投生各處的種種業因行持。塔信都位於南池市集(Namzhe
Bazaar)和基地營之前的山口上。雪峰出現在雲霧瀰漫之間,彷彿觸手可及,周遭的景觀亦瞬息萬變。打從我們踏入那片土地開始,心裡便一直希望有朝一日能再次回到那個地方,逗留較長一點的時間,然後在山脈與雲霧之間靜心地閉關修行。

塔信都寺院的住持喇嘛是一名年輕快樂的轉世活佛,他要麼已證入世間與出世間法的不二境界,不然就是對教授佛法毫無興趣。不管怎樣,他老是在逃避我們的問題,而且手持法器念誦了祈請文後,便開始向我們兜售各種法器供物。我們一度很武斷,近乎以自以為是的道德意識去判斷他。當時我們並沒有意會到也許他的寺院需要經費,而且過去他也想必曾試圖向一些對佛法興致缺缺的西方人講解過佛法。對於他所兜售的這些珍貴的物品及所提出的合理價格,我們非但沒有心存感激,反而立即就將他歸類為「商業喇嘛」,絲毫沒有察覺到他那細微、開放的能量。後來,我們向他購買了一些畫工非常精細,表徵佛的淨化能量的金剛界曼陀羅畫像和一些莊嚴的法器供物。

我們的最後一個行程是位於札爾沙(Jalsa)的西藏難民營。我們在途中經過一座小寺院,那是到加德滿都去為人民祈雨的那個喇嘛的寺院。布塔拉錫米就是向這位喇嘛請得我們生平第一個護身符。喇嘛並不在,寺院裡只有他的家人。我們留下了供養物,感謝喇嘛護佑我們遠離一切麻煩。直到今天,在他的加持下,我們不但躲過超速監視,那一輛又一輛雖殘舊,速度卻很快的德國轎車和摩托車也依然耐用。儘管難民營的外觀看來相當貧瘠,營內的感覺卻截然不同。直到西藏人涉及政治之前,在他們身上都能普遍感受到這種很殊妙的能量與特質。營內家家戶戶都設有安奉佛像和喇嘛聖像的供壇,樹木與木桿之間也總是掛滿了彩色的經幡,隨風飄揚。我們途經一些房子,由於大門敞開著所以能看見一群婦女正在邊聊天邊編織著色彩絢麗的藏式毯子。這些婦女們的心情似乎不錯。不久前,人口控制局的人員曾到訪,分派了一些避孕套給她們。現在她們可以安心行房卻又不必擔心懷孕的問題。我們似乎每走到一處都會遇到一些在噶瑪巴蒞訪加德滿都時認識的老朋友。

和這些老朋友一同坐在難民營裡的食堂,我有那麼一瞬間浮現出「似曾相識」的幻覺記憶。有人認為那是早期生活中的一種記憶重現,有的人則認為那只是內心裡對某些感官印象所產生的雙重反應。直到獲得客觀的證實之前,我想還是不要冒然對它們強加太多固定的觀念較為明智。當目光停留在門外的風景時,三名康巴戰士的剪影在布滿紅霞的天空下顯得分外出眾。我知道自己曾經見過完全相同的情景。而且那是只有在雪域西藏才會看見的景觀。

我們從札爾沙抄捷徑回到尊貝喜去拿其他的行李。雨季快要來臨,我們必須立即動身回到加德滿都。我們連續趕了四天路,漢娜仍然保持著輕盈的步伐,而我在沉重的背囊下只能拖著緩慢的腳步前進。當我們趕到蘭桑戈(Lamsango)時,巴士正準備駛離。當天傍晚,我們抵達加德滿都。那一趟登山行之前種種未知的可能性,似乎正漸漸開始具體地彰顯。

