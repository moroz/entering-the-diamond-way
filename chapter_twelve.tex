\chapter{噶瑪巴確定了我們的方向}

我們在克里斯大宅的美籍鄰居蘇和里察,每天都會花好幾個小時進行某種奇怪的體操。只見他們雙手合十,依次先高舉於頭頂,然後置於喉嚨和心間,就如我們在皈依儀式中所學的一樣。不過與其是雙膝跪地叩頭,他們是一叩掌心貼地,身體向前傾滑,展五體投地之姿,瞬即又再站直身子,不斷重複著此一步驟無有歇息。加拿大人阿健曾在寺院大殿裡進行法會時很「突兀」地重複了做這些動作,其他人則在屋裡重複做著這些動作。也許我們對噶瑪巴死心塌地,再加上先前曾向卡盧仁波切提出刁鑽的問題而不經意得罪了一些虔誠的弟子,又或者是我們提問的方式不對;不管怎樣,大家似乎都在忙著做自己的修行功課,我們很難問出個所以然。即使內心裡非常好奇,可是不久後我們也忘了這麼一回事兒。直到卡盧仁波切快要結束有關「痛苦」的開示時,他開始對這種五體投地式的運動作出了講解。

原來這是四加行中的第一加行:「大禮拜」。藏傳佛教的眾具德上師曾說過,像是「止」(梵:Shamatha,藏:Shiney)和「觀」(梵:Vipashyana,藏:Lhagtong)等簡單的修持,以及其他較高階的「密續」修法是否能圓滿修持,很多時候是賴於加行的功夫。行者需要通過這種準備工作集聚無量功德與智慧後,在修法上才會有真正的成果。大禮拜除了能有效地消除我慢,對於身語意也具有極大的影響。它能清理身體中心的能量脈輪,使心轉向菩提。卡盧仁波切時常說:「十萬次大禮拜是菩提道上關鍵性的第一步。」儘管人們稱此第一加行為「十萬次大禮拜」(藏:Bum),不過行者所必須完成的實際數量其實是111,111次。

漢娜已準備好要馬上展開大禮拜的修持,然而我的驕慢心卻察覺到事情不妙,於是趕緊找了個藉口推搪,我說:「如果噶瑪巴要我們做大禮拜,他應該早就跟我們說了。而且如果真要開始的話,應該一氣呵成把它完成。所以我想直到和噶瑪巴再次見面之前,我們最好還是先專心打坐。」我自以為這是避開大禮拜的絕妙之計。即使我們一天做一千次,我們也需要花超過三個月的時間才能完成十萬次大禮拜。我們肯定不想在無聊的索納達待這麼久。儘管我們很開心能擁有像卡盧仁波切般的導師,感恩之心也每日不斷增上,可是我們的心始終只向著噶瑪巴──我們心目中那偉大的笑佛。

我總是看見他人在做大禮拜,自己卻不曾親身嘗試,心裡難免不是味兒。於是某天出自於好奇心,我清理了房內地板上的一些木板,然後鋪上睡袋作為護膝之用,再在腹部會接觸地板的位置鋪一個棉墊,雙手穿上一雙穿了洞的襪子以便利於滑行,準備親身體驗一下大禮拜。我就只是想嘗試看看,其他人沒必要知道。大禮拜帶來了很特別的感覺,其實那種感覺相當振奮人心。十七歲那一年,我參加了AFS學生交換計畫前往美國,每當在練習美式足球時,凌空搶球都會讓我感到極大的快樂。如今做大禮拜也讓我嚐到了相似的亢奮與快感。在這過程中專注的進行一連串動作,似乎會讓人上癮,我一口氣做了大概五至六百次大禮拜。我只是想要知道大禮拜究竟是怎麼一回事,除了腹部肌肉出乎意料的痠痛,我後來也只是將它歸類為一種極具「異國風情」的運動而已。之後當我們去參加卡魯仁波切的開示時,仁波切微笑著說道:「你已經開始做大禮拜,很好很好!它們對消除身、語、意方面的障礙非常有幫助。」他對我做大禮拜一事極力地讚揚,在場的一些西藏人也不斷附議,熱切地說著十萬次大禮拜的種種好處。我不能當眾說出自己其實只是試驗性地做大禮拜,而且並無意願想要繼續。這樣不但會對他人的修持造成負面的影響,也會破壞仁波切的弘法工作。如今我陷於一個騎虎難下的處境,實在是非常尷尬。這一次我們著著實實地感覺自己被擺了一道。我和漢娜就這樣開始了大禮拜,一天三遍,每遍兩百次。即便如此,我們每天仍具有充足的時間去做其他事,像去探索那些位於錫金邊境的山麓地區等等。

我們聽說在大吉嶺並不只是只有一個像卡盧仁波切般的大瑜伽士。就在鎮外山坡上的那一片樹林裡,住著一名具德上師堪珠仁波切(Kanjur
Rinpoche)。堪珠仁波切是噶瑪巴在西藏的弟子,亦是寧瑪傳承地位最高的其中一名上師。儘管我和漢娜一直認為最好只是緊隨一個傳承,不將各傳承的法教參雜一起(換言之,不明白一件事,也總比搞混幾件事來得好),可是我們卻對堪珠仁波切感到非常好奇,同時也很渴望能領受他的加持。於是某一天,我們動身穿越那一片美麗的樹林,踏上了拜訪仁波切之路。

寧瑪與噶舉兩大派在一些名相上存有許多差異;然而多世紀以來,寧瑪和噶瑪噶舉傳承的法教和加持力曾相結合過數次。如今寧瑪傳承的法教也是噶舉傳承重要法教的一部分。這兩大傳承經常同時由一個喇嘛所持有,第三世噶瑪巴朗炯多傑(Rangjung
Dorje)就曾將寧瑪傳承的大圓滿教法(藏:Dzogchen)與噶舉傳承的大手印(藏:Chagchen)融合一體。一般而言,大圓滿法教對憤怒心重者非常有效用,而貪執心重者則適合修持大手印。對於「自心本具佛性」具信者,這兩大教法將會是最快成就的方便法門。堪珠仁波切與其明妃和身為轉世活佛的兒子就住在檜樹參天而立的山腰處。他們美麗的石灰房屋正在建造中,身邊不時圍繞著一群富有的法國籍弟子。若要進入仁波切的居所必須先通過大柵門,那裡由兩隻德國狼犬守著。這兩隻狼犬兇惡極了,聽說前幾天才咬死了人。坐在那裡的幾個西方人絲毫沒有要前來幫忙的意思。就像和卡盧仁波切較親近的那幾個西方弟子一樣,他們都只是專注於個人的發展,視無旁人。我們對自己承諾,往後若在西方國家建立藏傳佛教的禪修中心,任何挑起事端導致不和睦者一定會被攆走。後來當我真的如願開始創立禪修中心時,我並沒有忘記當年許下的承諾,而且相當樂於執行它。甚麼官僚制度,傲慢乖戾的態度,自我中心─這些通通皆與金剛乘的法教不符,它們完全背離了噶瑪巴的精神。

當我們成功打開大柵門時,這兩隻德國狼犬並沒有給我們帶來任何困擾,堪珠仁波切亦及時出現。仁波切踩著涼鞋,穿著長睡衣直直走進屋裡。他的妻子讓人感覺慈愛和藹,我們立刻就對他們產生好感。仁波切坐在其法座上,問了我們幾個問題,然後便進入禪定中,我們亦跟隨他一起靜坐。這比平時更具意義。此時我們就安住於自心本性中,超越一切概念思維,了了分明。當我們回歸日常生活中的分別和造作時,幾個小時已經過去了。我們搭晚班巴士回到索納達後,依然能感覺到仁波切的加持力。後來當我們再次前往拜訪仁波切,亦出現了類似的情況。仁波切在我們面前的椅子上坐下,開始持誦「嗡啊吽邊札咕嚕貝瑪悉地吽」。忽然之間,體內一股洶湧澎湃的能量往上直衝,我整個人都在顫抖,感覺到無限的可能性突然展現。我的內心充滿感恩與歡喜,以額頭碰觸仁波切的膝蓋俯身頂禮。一九七七年,當我們與隆德寺的喇嘛們在日內瓦時,傳來了仁波切圓寂的消息,當時噶瑪巴承諾說仁波切很快便會乘願歸來。一九八一年十一月,噶瑪巴在芝加哥圓寂之前,他認證了堪珠仁波切以及其餘廿名很重要的喇嘛的轉世。堪珠仁波切曾是噶瑪巴在西藏的僧侶之一。他擁有很高的禪修境界,智慧空行母出現與之雙修,達證了不二的境界。

恰札仁波切(Chagdral
Rinpoche)也是寧瑪傳承的大喇嘛。仁波切位於古木附近的寺院建有許多莊嚴的舊卡當巴式的舍利佛塔。這一座寺院也因為這些美麗的佛塔而成為了當地的地標。恰札仁波切並非一名轉世活佛,可是他已在此生證得了圓滿的證量。仁波切是卡盧仁波切的好友,他因其非常準確的占卜和嚴格實際的教法而著名。有一個太妹型的西方女生,仁波切給她傳法之前曾令她清理馬房長達半年之久,後來她還患上了肺結核病才下山。當仁波切獲得金錢時,他總是會下山前往西里古里,將能買的魚隻都買下,對它們念咒後再帶到河裡放生。仁波切是素食主義者,就像古魯仁波切所有的弟子一樣,除了供香之外,其餘無論任何形式的煙對他們來說都是有害的。他們說:「煙會使諸佛不願靠近。」至於酒精,他們的態度則較為開放。適量地飲酒甚至有延年益壽的作用。仁波切在西藏時曾經拜訪過許多古魯仁波切和密勒日巴尊者閉關禪修的洞穴。他有兩隻驢子,一隻負責搬書,另一隻則負責搬運他的食物,他則按照個人的心意隨時隨地靜坐禪修。不久後,慷慨的不丹皇室協助他在尼泊爾創立了一所閉關中心。一九八一年的秋天,那是我們最後一次見到仁波切,他看起來仍非常健康。

當時寧瑪傳承的持有者乃為敦珠仁波切(Dudjom
Rinpoche)。仁波切於一九八七年在法國圓寂。就像噶瑪巴,仁波切身邊總是圍繞著許多追隨他的人和眾多前來請求幫助的信眾與弟子。仁波切往返卡林邦與尼泊爾兩地,在七十、八十年代期間,他曾前往西方國家數次弘法,更在倫敦、巴黎和紐約成立了許多禪修團體。我們在卡林邦時曾嘗試拜訪仁波切,不過當時仁波切的居所已被丟空多時無人居住。如今在大吉嶺,我們恰好趕在仁波切離開的前一天見到他。敦珠仁波切患有嚴重的哮喘病。他的弟子認為那是他承擔了許多眾生的惡業的緣故,而瑜伽士陳氏卻按照道教的一套說法,認為仁波切的哮喘病是因為性事過於頻密所致!不管人們怎麼想,仁波切虛弱的身體倒是讓很多人積下侍奉上師的功德。仁波切讓我們免於頂禮,我們問候他過得如何,他的回答簡潔有力。他說:「我很痛苦。」後來仁波切還給了我們倆一包包的喇嘛法藥(瑜伽士和喇嘛在修法時加持過的不規則顆粒草藥),然後輕輕地給予我們摸頂加持。一九八一年,我們第三次前往朝聖,當時仁波切仍能給替我們整百個朋友加持。

位於大吉嶺與古木之間的山丘上(從汽油泵再往上的地方)住著竹千仁波切(Drukchen
Rinpoche)與德澤仁波切(Thugtse
Rinpoche)。如今竹千仁波切也已轉世。這兩位仁波切似乎不怎麼喜歡卡盧仁波切,原因是卡盧仁波切在索納達的地位較他們更為德高望重。這兩位仁波切對此地區具有重大的貢獻,他們經常辦大型的課程和誦經法會。他們的努力使得當地的佛教徒擺脫過往「消費型」的被動角色,進而投入較積極的實修行列。他們的寺院建築相當現代化,法會上的樂音律動亦相當震撼人心,只是看到一個受認證的活佛失心瘋讓我們覺得非常不安。當時的我們並無法明白這種事。

後來我們經常拜訪座落於大吉嶺城外的布提亞布斯地提(Bhutia
Basty)。這裡非常有趣,它是一個在家人的團體,然而在運作方面卻仍有很大的進展和學習的空間!與我們西方人的團體比較起來的話,它顯然不夠活躍和透明。

雖然我們會去拜訪鄰近地區的其他喇嘛,可是我們並沒有忘記卡盧仁波切才是我們學法的導師。噶瑪巴經常會直接把學生送到卡盧仁波切這裡來學習佛法的基礎,而我和漢娜則是在較間接的情況下接觸了仁波切。一直到今天,噶瑪巴在西藏古老的寺院所流傳下來的傳統法教與僧團制度也成為了他在法國、加拿大、挪威和美國所建立的寺院的基礎,所有的修法都是以藏文進行。直到ㄧ九九八年,我們和朋友所創立的超過兩百間在家與瑜伽禪修中心都是直接為噶瑪巴而立,尤其在昆吉夏瑪巴的拜訪之後,更為我們增添了不少創新和重要的主張與理念。我們都是法友,大家都有權力決定團體內的事務。這一種開放的風格,大手印的教法,以及噶瑪巴上師禪修法,使不少人更加成熟起來。後來創立的團體屬於在家人團體,行事上亦更為透明化,人們能將生活和佛法結合起來,讓這些聰明和獨立的心智擁有更開明和平等的平台充分地展現自己的才能。

我們幾乎每天都跟隨著卡盧仁波切學法,亦在他的指導下修持大禮拜,可是我們卻從來不從仁波切那裡領受任何灌頂。由於灌頂將會以不同層次觸動受灌者的心靈,因此除了噶瑪巴,我們不希望讓其他任何人觸碰到這一塊。有ㄧ次,我們在仁波切開始傳授灌頂之前離開了他的房間,所有人都覺得匪夷所思。卡盧仁波切注意到我們的壓力,也知道我們很猶豫;有一天他告訴我們噶瑪巴即將從不丹東部前往西部,途中必須繞道印度而行,因此勸誡我們應該藉此機會向噶瑪巴請益有關灌頂的事。這是我們聽他說過最動聽的話了!我們的心情頓時雀躍萬分,激動得無法言語。當我們知道自己即將再見噶瑪巴後,所有事物似乎在散發著光芒。在接下來的幾天,我們不斷收集有關他的行程資料,以及該地區警員的分布狀況。儘管我們從來都不知道消息從哪兒傳來,可是那些無所不在的康巴人似乎總是對噶瑪巴的一舉一動瞭如指掌。他們傳來了消息說噶瑪巴數日後入境印度時將會經過的某個路段。由於連結不丹東西兩端的道路尚未完成,因此必須繞道印度方能抵達目的地。由於路只有ㄧ條,所以我們會在上次登上豪華旅巴的那個地方與噶瑪巴會合。這ㄧ次我們輕而易舉便領到了通往卡林邦的所需文件,也很慶幸自己非常熟悉那ㄧ條軍用公路。當那條主要的道路被關閉時,所有車輛就只能取道此軍用公路,而且在這條路上一直到抵達不丹邊境之前,都不會有任何檢查站。

這ㄧ次我們在巴士抵達關口前便下車,然後步行至邊境檢查站問看我們是否可以在那裡等待噶瑪巴。雖然值班的官員並不是上次那ㄧ班「天才」,不過他們同樣是緊張兮兮的,而且還威脅我們若不趕緊離開就將我們逮捕。他們大概是擔心會丟了工作才這樣吧?在印度打破飯碗可不是鬧著玩的事。我們告訴他們的長官說我們是噶瑪巴的弟子,不過他怎麼也不相信。他看來是曾接受過古典印度教的教育,所以才會不斷重複說:「那是需要幾世修來的緣份。」無可否認,兩個金髮的歐洲人說自己與德高望重的噶瑪巴關係如家人般親密(噶瑪巴可是西藏眾多偉大的瑜伽士和日益漸增的印度信徒的上師呢!)這聽起來確實是挺奇怪的。可是對方卻遺漏了ㄧ點:許多來到東方的大塊頭白人,過去世也許曾經在此居住過也說不定。就像那些深受基督教文化所吸引的亞洲人一樣,他們不只是為了金錢或表示自己高人一等,他們只是曾與基督教有過一段淵源,重拾過去世所遺留下來的印記而已。

其實ㄧ個偉大的上師若要通過弟子的個性而認出他們並非難事,只是還是避免談及前世會比較好。因為人們會不斷推測,而且亦無法真的可以肯定,加上每個人要應付今生的自我意識已經足夠疲累了,多一事還不如少一事,因此為人師者還是專心教導弟子,繼續提升他們的心性潛力就好。

這名邊境檢查站的官員對我們試圖改進他的ㄧ套說法似乎很不以為然。不過由於這只是在浪費時間,而且一不小心還會惹出大麻煩(他們已經近乎迫切),所以我們乖乖地揹起背包打算轉身離開。在ㄧ排松樹直立的沙地上走了ㄧ小段路後,我們登上了ㄧ輛來自不丹的吉普車。吉普車把我們送到毗鄰的村莊,那裡仍屬限制區的範圍。由於這ㄧ條公路沒有其他分岔路,噶瑪巴的車輛必定會經過這裡,所以只要守在這兒就一定會萬無一失。我們趕緊離開大街避開警察的耳目(管他是便衣還是穿著制服的警察,我們一定不可被發現),同時亦開始四處打聽當晚可下榻的地方。除了可以避開那些惱人的官員,這ㄧ個小鎮大概是我們到訪過喜馬拉雅南部最美妙的ㄧ個地方了。小鎮上相思樹參天而立,綠蔭如蓋,鎮上的居民性情溫和自然。與他們相處時我不必說教,真是讓人鬆了一口氣。當我們問他們哪裡有旅館時,他們直接把我們帶到寺廟裡去,鋪好蓆子邀請我們留下。我彷彿又感覺到了在嬉皮式湧入這些東方國家前的那種熱情。這ㄧ個天堂般的地方居然還有ㄧ個仍然管用的電話,當我們聽到從彭措林傳來的消息說噶瑪巴要到隔天才會抵達時,內心裡不禁讚嘆他們的殷勤招待來得真是合時。

蓆子就鋪在主殿旁,每隔一段時間就會有當地的居民進入主殿,敲響銅鑼把諸神喚醒,供養食物或硬幣,再許個願後方才離開。他們有的人也會為我們帶來食物。過了一會兒,有幾個塔芒族人前來邀請我們為他們的寺廟加持。塔芒族主要是佛教徒。此刻情景,一個小動作便遠勝千言萬語。我們究竟是應該帶上背囊,抑或是相信他們好呢?在印度這麼久,這是我們第一次冒險將背囊留下,然後跟隨他們走開。總之,我們能感覺到他們很開心。

他們帶領我們走在一條剛種了樹的坡路上,沿路樹影成蔭,我們心想:「印度在人口爆炸之前(當年白人把盤尼西林帶到印度卻忘記留下避孕方法嘛),應該就是如此的美麗。」這些塔芒族的寺廟離村莊不遠。只見寺廟建在木柱上,這種高腳建築的結構是為了防止蛇鼠的侵害,具有很明顯的塔芒族風格。就像在其他山麓地區處處所見的唐卡一樣,唐卡上是擁有一雙超大的手和眼睛的四臂觀音聖像。唐卡旁還有一些金剛杵和金剛鈴,看得出其手工生疏粗糙,就連表面上的雕刻和輪廓線條也顯得模糊不俐落。

他們顯然並不常使用該寺廟,而且他們大部分的習俗都涉及了大量的酒精。不過只要想到諸佛菩薩,基本上還是能感覺到正面的能量。我們邊撒米邊持誦著所修持的咒語為他們加持寺廟。

翌日早晨,我們選擇了ㄧ個能看清楚路況又不會被發現的「風水位置」等待噶瑪巴的到來。這裡是村莊的入口,而且路上設有低坎,所有路過的車輛都必須減速,因此我們一定不會錯過噶瑪巴。就在我們等了幾個小時,吃了許多香蕉下肚後,噶瑪巴的路虎終於出現,行駛在其護航車隊的最前頭。我們跑向噶瑪巴的吉普車前,他的司機嚇得連忙剎車。噶瑪巴見到我們後開懷地大笑,並與我們寒暄了一會兒。

稍後,他吩咐人把我們藏在卡車上的行李堆內,之後我們便跟著大隊浩浩蕩蕩地進入了不丹境內。

在彭措林,車子就停在靠近以前曾下榻的那ㄧ家賓館的某個旅館前。我們躲開了那些想要和我們聊天的隆德寺的朋友,直奔噶瑪巴身邊。我們並沒有注意到那些正在排著隊等待著要覲見噶瑪巴的貴賓,直接就跑進了噶瑪巴的休息室內。我們將護照交給噶瑪巴,告訴他說我們希望能領取不丹護照,那麼以後便不會再與他分離。當時的我們完全沒有想要回到歐洲去的想法,一心只想親近他。結果噶瑪巴握著我們的護照,語重心長地對我們說道:「你要為自己擁有一個國籍和明確的身份而不是身為難民而感到幸運。你們應該要明白這ㄧ點。不要放棄你們的護照,日後會有用的。」噶瑪巴答應會儘他所能幫助我們,同時邀請我們和他一起坐在竹椅上。

四周正在醞釀著一種很奇怪的氛圍,我們也說不出個所以然。此時,有一個健壯魁梧的男人進入室內,來到噶瑪巴面前席地而坐,我們認出他的身份,靜悄悄地溜了出去。人們領口上的別針與四處張貼的照片上都是他的肖像:不丹的ㄧ國之君。

夜已深,我們恰好有機會再次覲見噶瑪巴。一踏入房內,便被眼前的景象震撼得說不出話來。噶瑪巴整個人散發著金色的光芒。我們回過神後,便趕緊向噶瑪巴請益有關灌頂的問題。噶瑪巴微笑著說道:「真是傻瓜!難道你們看不出來卡盧仁波切代表著我嗎?你們當然應該從他那兒領受灌頂。回去索納達,好好地修持加行,有機會便來錫金看我。」

後來我們有幸覲見了不丹的王室成員。一九七九與一九八七年,他們邀請我們前往不丹作客。不丹國王如今迎娶了傑珠仁波切的四位姪女為妃。此外,我們亦見到噶瑪巴的侄子Jigmela,他不時強調閉關中的行者很快便能完成大禮拜的修持。完成了最後一場的黑寶冠儀式後,我們在凌晨三點離開。這次我們和老友吉梅醫生一起坐在某輛卡車舒適的座位上。由於噶瑪巴必須作多站停留,給不同的人們和地方加持,因此我們的車子較早抵達堤斯塔大橋。我們捧著許多要要寄回家鄉送給朋友們的照片和聖物法器,等待噶瑪巴到來時請他加持。怎知一回過神來,噶瑪巴的車子已經悄悄駛過,到了橋的另一端。

