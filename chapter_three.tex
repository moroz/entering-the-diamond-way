\chapter{牢獄裡的自由}

我們原本要在黎巴嫩停留的計畫已不再可行。我們幾乎可以肯定如今警方必定已掌握了我們的名字。我們除了必須儘快趕回家讓父母親安心,也得想辦法幫助朋友。當我們抵達最後一個中轉站法蘭克福(Frankfurt)時,我們撥了一通電話,得知第一箱的佛首像已經抵達丹麥。如今,眼前的問題就只剩下我們該如何安全地入境丹麥。

我們匆匆忙忙地趕上直飛哥本哈根的班機,深信好運氣將會一如既往地眷顧我們。在機場,我們不斷重複念誦著咒語,在過海關檢查護照時,恰好跟隨在兩名警察和他們所扣押的兩個格陵蘭人身後,海關人員以為我們也是警察,所以沒檢查護照就直接讓我們過關。

我們的父母親很不開心。他們對我們那自以為高尚的出發點難以苟同,無法相信我們是追求心靈自由的戰士,志於幫助他人擺脫攻擊性和酗酒的行為。在他們眼裡,孩子們是大難臨頭了。有的報章也已經報導警方懷疑我們所涉及的勾當,只是尚未獲得確鑿的證據能將我們逮捕歸案而已。

就在我們抵達哥本哈根後不久,第二箱的佛首塑像也抵達了,而且還被海關所查獲。包裹中的薰香的氣味讓他們起了疑心。他們打開佛首塑像,發現我們原本想要空投撒在哥本哈根的哈希什大麻。由於包裹的收件人不是我們的名字,也不是以我們的名義寄出,因此我們能暫時洗脫嫌疑。只是外頭風聲很緊,周圍的防護牆似乎正在逐漸崩塌。我們身邊的朋友越來越多人遭到圍捕,身上一直配戴著的護身符也突然不見了,怎麼也找不著。此時,另一個不詳的預兆緊接著出現,我們竟然接到一張超速的罰單!這是前所未有的事。我們還未能來得及收拾細軟到朋友家中暫避風頭,警察便已聞風而至。

其實我們原本可以先在南歐避一避風頭,直到風聲沒那麼緊才露面的。然而,接二連三所發生的事似乎都在告訴我們,那是我們必須從中學習的課題。我們事先已經被告知會有事情發生,而且喇嘛也曾經允諾無論發生甚麼事,他都會保護我們。我們覺得這一連串的事情想必也是受到他所影響,因此決定順其自然,讓因果去解決。

警察方面半點也不懈怠。他們蒐集了不少線索,足以構成有力的證據將我們扣留。當時,丹麥政府正在計畫將走私販毒的控訴,從輕罪改為重罪。因此在正式落實此項條例之前,他們需要一個有力的個案例證和一個很好的宣傳。像丹麥這種賦稅高,海岸線又長的北歐小國,走私就是一種傳統,我們把它視為是一項緊張刺激又「紳士」的活動。然而當時有不少外國人也開始插上一腳,他們在丹麥的走私活動所施的手段更是談不上高雅。再說他們亦將像是鴉片等等非常危險的毒品大量引入丹麥。如今濫用這些毒品者大多數來自於社會的較低階層。通過毒品來拓展心靈的崇高時期顯然已經結束,我們的政府誓言要制止濫用毒品,必定是擁有很好的理由才會這麼做。

我和漢娜向來都是盡量以最便宜的方式過生活,認為節約就是自由。然而,即便這樣警方還是找到證據證明我們花在車子和旅行(我們當時的主要花費)上的金錢,遠比我的教學工作所賺取的金錢來得更多。現在問題主要是面對眼前的多項控訴,我們該承認多少而已。某個程度上,我們不得已要和警方合作,否則第二箱佛首塑像的收件人──一名臨盆在即的孕婦大概得在牢獄裡把孩子生下了。此外,我也得盡快將漢娜從拘留室裡弄出來,這一些沒有門把的狹小房子實在不合她所好。因此我暗自決定,只要是警方掌握足夠證據的控訴便認罪。我告訴警方,漢娜對我的交易不甚了解,甚至可說是毫不知情。至於其他人,我只是承認自己認識某個名為喬的阿拉伯人。每次接受盤問時,我都是不斷重複同樣的故事,藉機告訴警方,無論如何我也不會出賣朋友。負責盤問我的兩名警員並不欣賞我的做法,不過基於他們都曾參加二次大戰時的抵抗運動,他們應該非常了解我的立場。另一方面,我深知自己必須儘快讓此案件上庭開審,否則當警方從中東方面獲取更多資料時,我們就真的是大禍臨頭了。我一定要想辦法讓我們的案件受到緊急處理。

就在某個晴朗的早晨,我有意識地將刀子緩緩刺入胸口。我沿著肋骨刺入,這樣才不會造成太大的傷害。痛楚是還好,只是覺得肌肉從肋骨上被撕裂的聲音有那麼一點兒噁心。刀柄露在胸口上,看起來極富戲劇性。有關當局真的以為我試圖自殺。隨後我被送進監獄醫院,這亦加快了案件送審的速度。

在此期間,漢娜和我只是見過一次面,不過我們一直都有保持聯絡。我們會在每天寫給對方的信件裡特別註明心裡生起某些想法時的時間,而有趣的是我們的想法總是彼此相應。負責讀信的警察很快就放棄挑撥我們之間的關係。多年後,當我們把這些信件交給某個著名的心理學家時,對方也表示未曾見過兩個人會如此有默契。

我第一次靜坐是在指導某人如何靜坐的時候。醫院裡同房的病友是一個可憐的傢伙,精神明顯失常,早前卻不明原因地成為漏網之魚,逃過政府的法眼而沒遭到逮捕。他過去是一名業餘的竊賊,擁有非常純熟的幹案技巧。只是他如今已失去了人生的方向。眼看著困惑和一臉沮喪的他,心中不禁為他感到憐憫。

有一天,我對他說:「你需要靜坐。」雖然我也只是從所讀的那本西藏書籍中認識何為「靜坐」,可是我想至少向他展示一下。我模仿圖片和佛像中所展示的靜坐姿勢,坐直了身子。結果我從此次的經驗中,獲得了比預想中更大的收穫。當我將身體和雙腳擺好位置後,不可思議的事情發生了。霎時四周圍的一切似乎都在發光,我感覺自己仿若在空中漂浮了起來,就像完全從軀體中脫離似的。額頭後方逐漸形成一種柔和的力道,猶如一股清涼的風在頭顱下流動。喇嘛、漢娜、我的父母親、朋友這一切美好的人事物似乎就在當下與我同在,內心裡生起了大樂的感受。與我心靈相通的漢娜也強烈地感受到了,於是在信中問道:「當時候你做了甚麼?我想一定是一件很美好的事?」自此,我們每天都會花幾個小時靜坐禪修。

當我再次回到個人的牢房裡後,更是覺得自己似乎是受邀到牢裡學習靜坐的。自從那時候開始,我們並沒有做其他事,只是專注地禪修靜坐。整個過程充滿喜悅,流動於額頭與頭顱頂端之間的一股力道,感覺亦非常美妙。靜坐所產生的能量一度讓我懷疑自己在尼泊爾染上了疾病,不過感覺又出奇地「對味」,我不希望任何人知道這件事。反正這應該是和喇嘛有關,像這般的內在體驗,我們又能對醫生說些甚麼呢?我寧願是自己染上某種疾病,也不願意失去這種美妙的「萬物合一」的體驗。

與此同時,警方不知道從何處找到了我們的護身符。就在某天夜晚,我親身體驗了它們的力量。

我的人生一直都是充滿刺激與挑戰。年幼時的我總覺得樹長得不夠高,長大後又覺得摩托車的速度不夠快。我從來不懂得何為焦慮,更不懂得畏懼。當其他人都在回避危險,允許恐懼和分化在生命中顯現,我卻會縱身躍入其中,並能夠從中感受極大的喜悅。然而,某一天夜晚在牢房裡,我卻因為恐懼(一定是!)的壓迫而醒了過來。當時,我覺得這一定是和脖子上的護身符有關,因此便將它摘下來放到床鋪上方的架子上。心間的壓迫感頓時消散了,而我又沉沉睡去。夜裡的牢房漆黑一片,就連開關燈等那麼簡單的事,我們也有專人代勞。翌日早晨,我發現繞綁著護身符的其中一條絲線鬆脫了,改變了它原來的形狀。我將它還原。那次之後我都會特別留意護身符的外狀。我總是覺得應該將它們好好收進小袋子裡才對。

此時,在較早一次的旅途中一直隨身攜帶的《西藏瑜伽與秘密的教義》一書也被送到牢裡來,這一次我們決定要用心鑽研書中的內容。漢娜和我每天都會互通信件,一頁一頁地探討書中更高的禪修,並且和彼此分享各自在靜坐禪修上的美妙經驗。我們成功地運用這些更高的禪修方便與技巧。通常若想要學習這些技巧,事前都必須先獲得具德上師的灌頂和多年的準備作為基礎。當時我們能夠成功學習這些技巧並不是因為開智慧,只是純粹幸運而已。而且當時我們總是覺得這些修法似曾相識。似乎早已懂得。

先前我假裝自殺加快了事情的發展。我們的案件在六周內便交由一名法官去審理。法官名為「懷特」(White,中文有「白色」之意),我們認為這是一個好兆頭。當我在庭上為自己辯解時,仿若處於一種出神的狀態。我聽見自己說話的聲音,卻也同時覺得說話的人並不是我自己。我告訴他們說,我們相信意識的拓展,大麻使到我們不那麼具有攻擊性,對他人也能夠抱著更開明的態度。我也告訴他們有關療癒以及在尼泊爾的喇嘛。對當時的我而言,毒品意味著療癒的能量,以及心靈的解脫!我天真地將二者混為一談,認為「果甜樹則良」,殊不知並非如此。結果,整個局勢突然變得對我們非常有利,就連地區檢察官也為我們辯護。結案時,我們獲判最輕的刑罰:漢娜當庭釋放,而我則需服刑四個月。若將拘留的時間加算在內,估計夏至時我便自由了。

然而,警方高層對於這樣的審判結果感到相當不悅。他們對我們所幹的好事,一直都是了然於心。不管是單獨還是和朋友聯手,我們總是妨礙他們辦公,壞了他們不少好事,而且還幫助/坦護那一些惹了麻煩的人。我們幫忙掩蓋他人行蹤,幫忙保護同伴們和貨物的安全,甚至報復那些太過火的警員。我們獲得法庭輕判引起了許多人的疾憤,他們也立即提出了上訴。其實他們是希望能夠殺一儆百,以儆效尤。此時他們也應該掌握了不少來自中東更有利的新證據。眼看這下真的是要大禍臨頭了。然而高庭法院的檔期已滿,直到九月之前都不可能提出上訴。現在正值四月初旬,他們甚至有意將我扣押至那個時候。

漢娜聽到有關消息後感到極為厭惡,而我則怒不可遏。若我們想要脫身,就只有兩種可能性:喇嘛幫助我們度過難關(實際上他是如何辦到的,我們實在是毫無頭緒);或者是我們試試書中最後一種修法,也就是藏文稱作為「波瓦」(Phowa)的遷識法。通過波瓦遷識法,行者的心識將能夠有意識的離開肉體。

這是金剛乘的其中一種修持法門。遷識法中的「心風奪舍」(Dronjug)修法,也是西藏偉大的瑜伽士瑪爾巴最喜歡的一種法門。這一種修法能夠讓心識從一個軀體遷往另一個軀體。許多人把此修法當作玩笑,因此導致修法逐漸失傳。一開始時,我們只是粗略地嘗試書中的禪修方法,並不了解這些修法所蘊含的真正意義。我們以為這一種修法能夠讓我們在扣押期間以冬眠的形式使意識離開身體,自由地四處漂移。然而,此法教真正是為了往生「淨土」而設。它和眾生在死亡之際的意識轉移有關。

讀者可能會問,我們曾經在優秀的大學裡受過多年教育,甚至也有教導一些非常實際的科目如哲學和語言,像我們這樣的人怎麼會相信這些事呢?鑒於我們過往的經驗,漢娜和我都抱有「嬉皮」一代相同的想法。我們深信心識的唯一界限就是無明和習氣,心識的真如本性是無限的。後來我們對這一點更是具有一份不可動搖的確信。當時的純理論在爾後亦成為了實際的修持。

就在我們感到進退兩難之際,我們給喇嘛傑珠寫了一封信,告訴他所發生的事。三周後,上訴不知明因地被取消。這大概也是丹麥司法院史前無例的一件事。

總檢察長和輯毒隊成員非常生氣,說司法部裡一定是有人瘋了,才會意外地撤回上訴。事實上,身在尼泊爾的喇嘛傑珠在這五天裡一直把自己關在房間裡閉關修持夢瑜伽。通過此修法,心識將以某個所選擇的佛本尊形相離開軀體,四處漂移。這也難怪我們那麼愛他呀!

在單人牢房內度過了三個月的監禁後,我被轉移至大牢房和其他囚友一起度過最後數周的刑期,日子單調又平淡。然而這也是觀察老愛打架滋事者、騙子和酒鬼們的生活的一個大好時機,內心裡很自然地就會對他們感到悲憫。他們對周圍的事物非常執著,認為這一切都是真實的,殊不知正是這一份執著令痛苦變得更為真實。

刑滿獲釋當天正值夏至,家人朋友為我重獲自由而舉杯歡慶。我非常開心能夠和漢娜,以及摯愛的父母和朋友團聚。我們決定以後都不再分開,這亦意味著我們該為走私販毒的日子畫下句點。

我從前幾個月的經驗中得到一個結論。朋友們來探監時經常會偷偷塞一些哈希什大麻給我。當我在囚室裡偷偷吸食這些大麻時才忽然發現它們對靜坐禪修一點益處也沒有,反而會使到禪修的體驗變得膚淺和微弱。那一種恆常的寧靜、專注和喜悅,似乎在吸了一口大麻後便消散不見,反之擁攏而至的是混亂的思緒。那一種感覺就像是蓬勃的朝氣突然消失不見般。只是在我為了獲釋而歡慶的數周後,這種省悟在與友人聚在一起集體抽大麻時又忘失得一乾二淨了。我們又開始吸毒,繼續作迷幻異境裡的嚮導。

好在我們與喇嘛傑珠的連繫無論如何也不會退失。禪修時所感受到的那一股柔和的力道仍在,似乎是應許著要在任何時候把我們更深入的內在一層一層地開放起來。每件事物都有著多層次,無盡延伸的意涵。我們的世界是完整的,我們能夠感覺到成長。幾個月後,久等的指示終於出現了。我們以前登山時的哥兒們(來自蘇格蘭的紅髮亞倫)來到丹麥。與他同行的美國友人比爾要找一台二手的福斯小巴,他想要開著小巴前往尼泊爾。之後我們買了小巴,收拾好行囊,浩浩蕩蕩隨著大夥兒一起出發。
