\textbf{當藏傳佛教與西方相遇}

                        喇嘛歐雷‧尼達爾 著

第一章 我們的蜜月之旅

第二章 身體透明瑩澈的喇嘛

第三章 牢獄裡的自由

第四章 陸路尼泊爾

第五章 黑寶冠喇嘛

第六章 被遺忘的山谷

第七章 雪巴人的國度

第八章 最後的迷幻之旅

第九章 皈依我佛

第十章 不丹之路

第十一章 向卡盧仁波切學法

第十二章 噶瑪巴確定了我們的方向

第十三章 家在索納達

第十四章 修持

第十五章 菩薩境地

第十六章 第一次回國

第十七章 印度南部的西藏難民營

第十八章 人生如夢

第十九章 正式開展弘法利生之事業

第一章

\textbf{我們的蜜月之旅}

一九六八年的夏天,我們選擇到尼泊爾度蜜月,這顯然是一個還算不錯的主意。早期仍十分理想化的歐洲嬉皮文化與古老的藏傳佛教教派正好在此時相遇。這兩種迴然不同的文化的相遇,正好給予了前者一個指引的方向,也讓後者不至於淪落為在博物館內展覽的文化遺產,而得以作為實際的修持繼續被保存下來。爾後多年,我們不斷往返於青蔥翠綠的北歐低地平原與喜馬拉雅的冰峰雪脈之間,致力於推動在家或瑜伽式金剛乘佛教的修法,使之不斷繼續茁壯的成長。這也是西方唯心主義與亞洲唯物主義有史以來第一次相遇,開拓人類心性潛力廣大且全新的一面。

後來,我與我那漂亮的妻子漢娜,甚至是身邊的許多好朋友這才發現,其實我們所遇見的是一種能讓我們掌握生命的工具。金剛乘佛教將我們那狂亂不安的心識狀態,以善巧的方式轉化成一個澄淨與喜樂之境。它以一種非常實際與具體的方式,讓生活上的各個方面都成為是一種修行,使到解脫與證悟不再遙不可及。

那些年,我總是與其他走私販及一群敢作敢為、喜歡刺激冒險的朋友為伍。那個時候若想要待在一些來自西藏或不丹的得道高僧身邊承事學習也不是不可能的事情。這些高僧大德所展現的典範以及其善巧殊勝的教誨,皆深深地啟發了我們。爾後,經過慎密的觀察,他們也選擇我們作為法侶,將妙廣精深的傳法和禪修引介到西方國家。

我寫這一本書的目的是希望回顧當年當這兩種如此截然不同卻同樣珍貴的文化相遇時所發生的點點滴滴。這一本書的延續──《Riding
the
Tiger》(暫無中譯)則著重描述之後在法務上所取得的發展與成果。這也許是這兩個高度文明的人有史以來第一次如此有意識地嘗試從彼此身上學習各自的優點。在過去數年間,由於有許多位持有完整內、密傳法的老喇嘛相繼圓寂,因此我想如今重新印刷此書,仔細回顧咀嚼,並與你分享當初相遇時的種種,也是一種難得的美好。

第一次遇見金剛乘佛教時,漢娜二十二歲,而我二十七歲。我們的父母是哥本哈根以北居民區學院裡的教師。他們都是難得可貴之人,我們在一個對「人性本善」充滿信心的世界裡成長。儘管自幼便在一個完全人性化的環境下被撫育長大,可是我總是不斷夢見某個不曾相識的山區裡戰亂流離的情境。夢境裡的我奮勇地將那些圓臉的士兵們一一擊退,拼了命去保護那一些當時的我僅能理解成是「裹著婦女紅色布段的那些男人們」。直到二十五年後第一次在尼泊爾看見藏族僧侶,我才開始明白夢境裡的這些人究竟是誰。又在隔了四十五年後,因為秘密前往已被中國所佔領的西藏東部旅行,才親眼目睹夢境裡那一些我拼了命想要保護的山區與村莊。

我這一輩子對於任何會限制我的自由的人事物(不管對方是否是甚麼大人物或是體制)很自然便會對抗到底。漢娜也是一個不受約束的人,只是她較為內斂,不像我會對外作出猛烈的反抗。

一九六一年的秋天,我服滿兵役退伍(我想當時所屬的部隊也是巴不得能儘快擺脫我)。我算是早期開始抽大麻的少數丹麥人之一。大麻(pot),當時候是這麼說的,後來相繼出現了不同的名稱,從大麻菸草葉(grass)到笑不停煙草(laughing
weed)都有。當時,我的哲學課成績很好,任何與心性有關的課題都能夠燃起我的強烈興趣。我從吸大麻的過程中獲得許多全新的見解,後來更藉由嗑迷幻藥踏入全新的體驗,接下來有好幾年都讓毒品主導了自己的生活。由於我認為它們為眾生帶來顯著的利益,因此決定要以奧爾德司‧赫胥黎(Aldous
Huxley)及其著作《知覺之門》(Doors of
Perception)作為撰寫博士論文的主題。

接下來幾年,當我在丹麥和德國求學時,除了任由毒品侵蝕我的心靈之外,我也相當熱衷於拳擊和賽摩托車,而且還曾釀出不少意外。而漢娜則努力完成最後幾年的高中教育。

漢娜與我就在哥本哈根大學裡的「食人族」(The
Cannibal)食堂再次相遇。我十歲的那年,漢娜只有五歲,我曾經在北方郊區的樹林裡教她利用樹枝造小木屋。因為女生都不擅長爬樹的關係,我對她們的評價頗差,認為她們煞是無趣,可是當時年幼的我卻漫步送漢娜回家。我想,那顯然是我這輩子第一次愛上了一個人。然而不久之後,漢娜舉家遷往更北部的地區生活,我們自此便失去了聯絡。如今事隔十五年,她又再次站在我的面前,不可思議的美麗。我甚至幾乎把身旁那位火辣的紅髮女郎給忘了。當時漢娜已訂婚四年,可是不久後,我們便開始一起生活。

一九六六年三月,二十五歲生日的那一天,我第一次嚐試了LSD(註:麥角二乙酰胺,香港人俗稱「弗得」的一種強烈致幻劑)。當我走在中世紀哥本哈根式的老街道上(如今其面貌大概已經改變了不少),返回派對途中,某道窄門引起了我的注意。我穿越這一道窄門,庭院深深只有一盞燈的光亮。我知道有一些很特別的體驗正在等待著我。我駐足於某一棟房子前,其窗空廣而深邃。我聽見自己對著窗戶說:「就是現在了,向我展示一切吧!」就在此時,我能感覺到四周圍彷彿在迅速竄動著,有一股迴旋的白色光芒籠罩著我。這一種感覺彷彿就像是整個宇宙的能量猛然湧入了體內,然後在空中爆發似的。一股永恆、明亮的力量遍滿四周。回到派對上後,我試圖回想先前的種種,也只能說:「我無法確定那是甚麼,不過感覺真好。」

不久後,漢娜也加入我的行列,繼續探索意識裡的其他層次。她的經驗總是深邃而又溫暖。從二十名朋友開始,直到我們的小群體增至三十人(大概也是北歐首個內聚群體),我們嗑遍了當年所推出的迷幻藥,不過卻似乎沒有從中汲取任何教訓,天真地認為這些藥物能夠啟蒙我們,對我們有所助益,殊不知它們已經悄悄地開始損壞身邊好友們的身心。現在的我願意付出一切去交換他們,可惜他們當中大部分的人已經死了,再也回不來了。當初他們都是以非常痛苦的方式離開這一個世界。其他仍活著的人也相繼迷失在自己的私密小宇宙裡,此生恐怕再也無法回到現實世界中。今天,對於毒品我們只有一句忠告,那就是:遠離它們!毒品會侵害我們的潛意識,使我們的靈魂變得平板空洞。而且毒品具有潛伏性,對我們所造成的破壞並不會瞬即顯現,更不容易修復。若想要取得證悟,只要方法正確就可以回歸我們的自心本性。

當年,我就在ㄧ所夜間學院裡教英文。我們常趁著假期前往北非、黎巴嫩和阿富汗作短途旅行,順便給朋友們帶回一些哈希什大麻(Hash)。ㄧ九六七年,當我從孟買北部搭飛機飛往印尼的途中(當時因為不久之前肝炎發作,加上身上緊得讓人窒息的西裝背心裡裝帶著三十四公斤的黃金而感到反胃不適),我透過窗戶看見來自喜馬拉雅山的夜光雲。雲層美麗得讓人心動。我知道我一定要前往這些雲朵所來自的地方。那裡有甚麼東西正等待著我們。某種很重要的東西。

ㄧ九六八年五月,我和漢娜結婚了。這算是目前為止我們所做過最明智的ㄧ件事。當年漢娜二十二歲,我二十七歲,我們前往尼泊爾度蜜月。早在ㄧ九六六年,我曾經想要通過陸路的方式前往尼泊爾,不過由於當時印度和巴基斯坦爆發了ㄧ場戰爭,所有人滯留在阿富汗差不多長達三個星期(時間長得足以染上亞洲最可怕的痢疾)。當時我的體重大概掉了五十五磅,在返回歐洲的途中,因為身無分文所以也只能沿路搭便車或徒步行走,我也曾為了掙路費到醫院賣血。現在回想起來,這一切也可說是ㄧ個相當有趣的經歷。

這一次我們並不想作任何中途停留。就在學生們考完試的當天,我們便鑽進了ㄧ台老舊的福斯小巴,一路開往漢堡市。有一台二手的福斯金龜車正在漢堡市等待著我們。取車後,我們開著小巴直驅南下。克雷斯、克里斯蒂安(我人生中的第一個大麻便是他提供的)及其太太也同行。當時,克里斯蒂安已經不良於行。他是在某次經歷「迷幻之旅」時忽地跳出窗外摔斷了背脊才導致要坐輪椅,因此這一趟旅程只剩漢娜、克雷斯和我負責開車。然而,克雷斯很快便受不了伊朗攝氏55度的高溫、境內漫天的沙塵飛揚以及穿越沙漠那一條未經鋪徹的破爛坡路。而我們則不惜一切代價要儘快趕往尼泊爾。打破了連續趕路六天六夜的紀錄後,我們終於入境阿富汗。克里斯蒂安和太太為了強勁的哈希什大麻想要留在喀布爾,因此在喀布爾放下他們後,我們便趕緊賣了小巴,將所賣得的價錢用作我們的旅費。才剛賣了小巴不久,漢娜便重蹈我兩年前的覆轍,染上痢疾病倒了,幸虧當時下榻的旅館設有洗手間。等到第二天,漢娜已經服食了許多抗生素,我帶著她搭巴士繼續往東前進。巴士將會一路開下開伯爾山口,進入印度和巴基斯坦以北的平原地區。

在喀布爾,我們巧遇正在修復世界最大佛像的幾位丹麥人。此尊佛像高五十三米,矗立於喀布爾以北約一個小時巴士車程的巴米揚(Bamian)。這是ㄧ九九六年穆斯林塔利班假不符合伊斯蘭教之名誓要摧毀的文化瑰寶之一。若讓塔利班政權統治北阿富汗,同樣的情況仍然會發生。這幾個丹麥人是修復這些佛教文化遺跡的專家。雖然一千多年前,穆斯林為了剝奪佛教的影響力而將大佛的鼻子擊落,導致莊嚴的大佛遭到了相當程度的損壞,可是早在六十年代後期,可觀看的東西仍不少。約公元前300年,就在亞歷山大大帝的時代結束後,犍陀羅文化在此地取得了興盛的發展。其實要辨識出犍陀羅文化的藝術並不難,佛像上的鬍子便展現了希臘與印度文化元素的混合,緊緊地追隨所征服之地區的文化元素。座落於巴基斯坦的白沙瓦博物館也擁有相當令人感到嘆為觀止的文物收藏。博物館中擁有眾多經過詳細考察和研究(甚至是可追溯至英格蘭時期)的收藏品,可是眼看他們將近期所奪得的印度槍械也以如此傲人的姿態展示於博物館中,便顯得極為突兀和格格不入。

來到印度後的感覺還不錯。我們在邊界時便已經感受到其中的差異了。我們將所有緊繃的情緒,回教國家性壓抑的氛圍都統統留在背後。身處回教國家時,我時常得把人從漢娜的身邊推開。現在終於可以鬆一口氣,如釋重負。費羅茲布爾(Ferozepur)是我們下一個即將前往的城鎮,我們會在那裡搭乘接續開往新德里的電車。在那個年代,若想要辦理尼泊爾簽證就必須前往新德里。現在他們在邊境就有發為期一周的簽證了。新德里是人們為了辦簽證才會去的地方,沒有多少遊客會想要在那裡多作逗留。在舊德里鄰近地區有那麼一丁點美麗的印度洋景象,也許有的西方人會喜歡,只是如今那裡也只是剩下一片喧囂。除非身上擁有多餘的金錢能夠自由地繼續移動,否則無可奈何要留在該地區的人,在面對那裡老盯著人看的居民,還真是少一點耐心也不行。

我們生平第一次接觸印度教大師是在新德里。那一次會面收穫豐富,真想向學校討回過去曾繳過的學費,只是他始終無法觸動我們的心靈。我們感受不到與他們之間的聯繫,心裡也完全沒有想要和他們產生任何聯繫的想法。他們有點過份熱情,讓我們無法適從。只是,他們所展現出的力量也頗讓人印象深刻。此事就發生在康諾特廣場最外圍的圓環區裡。這一個行政地區是英國人留給印度人最後離別的禮物。據說這使到印度人更糊塗了。

在這一個餐廳和紀念品商店林立的中心點,一條條的街道向外擴展,四周圍大使館和別墅林立。在新德里所發生的事也同樣在這裡上演。我們走在拱廊下,忽然有一名裹著頭巾的印度老人走到面前,嘴裡咕噥著說甚麼「幸運的額頭」,便往我手心裡塞了三張捲起的紙張。他不斷在我面前比手畫腳,讓我把注意力放在他的雙手上。他同時也直視我的雙眼,讓我說出一種水果的名稱。我在心裡想著各式的水果,當中有的或許他連聽也未曾聽過。當我還在心裡盤算著是否應該戲弄像他這樣一個老人家時,我已經衝口而出,說了「蘋果」。他看起來高興極了,從我手中拿出其中一張紙,打開一看,紙上寫著的正是「蘋果」。「現在請說出一種花的名字。」他堅持要我說,接著又發生了同樣的情況。當我心裡在想著木槿和其他的異國植物時,就在那種半恍惚的催眠狀態下,腦海裡突然冒出了玫瑰花,因此很自然地便說了「玫瑰」。

第二張紙裡寫的當然就是「玫瑰」,他老人家開心得直跳腳。在接下來的第三個測試中,我必須說出一到十之間的其中一個數字,我的自尊心只允許我默想著數字「1」,我堵住一切外來的干擾,堅持自己的想法。然而,他老人家還算是挺走運的。他試圖向我暗示數字「7」,然而由於最後的那一張紙上寫著的字看起來非常像是數字「1」,因此成為了他挽回局勢的籌碼,我們之間的遊戲才得以繼續。我們跟隨他來到某個不會被警察發現的角落,然後他打開大多數占卜師都會隨身攜帶的小本子,對我們說:「把錢放這兒吧!我給你占卜未來。」

在北歐,教會都是由稅收所支持,因此身為北歐人民的我們並不喜歡看見金錢和靈性如此公然的存在任何掛勾。不過正因為當天尼泊爾大使館關門了,也看在這老頭子如此賣力的份上,我們心裡生起了一絲悲憫,便在小本子夾了兩美元。漢娜在一旁不斷催促說我們必須趕緊離開。老人知道自己已經無法掌控整個局勢,或留住我們,於是也給我們贈送了一份臨別之禮。這似乎是正在向我們展示他的能耐不僅是能對人作出心理暗示而已。他說有一個名為簡森的女孩向老家那邊的警察提起了有關我們的事。當時,老人的一番話對我們來說一點兒意義也沒有,因為朋友們的姓氏,我們本來就不懂得多少。不過,後來我們是有聽說某個同名的女人向警方告密舉報我們。這一個老人原來還真的是具有某種程度的洞察力,能夠未卜先知啊!臨別之際,他還語重深長的叮囑我(正如其他智者經常會做的事一樣)要繼續留在漢娜身邊。

打從我們離開哥本哈根那天開始,那本綠色的書本便一直是我們在路上的良伴。書本的封面是魁偉莊嚴的佛像。我們一有時間就會努力鑽研書本上的內容。這一本書就是由喇嘛卡孜‧達瓦桑珠(Lama
Kazi Dawa
Samdup)所譯,英國學者艾文斯‧溫玆(Evans-Wentz)所編輯的《西藏瑜伽與秘密的教義》(Tibetan
Yoga and Secret
Doctrines)。書中包含了藏傳佛教噶舉傳承廣妙精深、極為有效的修法。當年,佛陀也只是將這些法門傳授給利根的及門弟子。每當我們翻開書本仔細閱讀,就會生起一種非常強烈的內在體驗,那是ㄧ種我們不曾認識的境界。內心裡是ㄧ陣特別的暖意、喜樂,一種心馳神往。當這種相當愉悅、強烈,又某種程度上刺麻疼痛的能量從體內中央向上移升時,內在響起了一把聲音,更趨強烈地不斷重複對我說著:「我們正在途中,現在真的在來著的途中了。」從漢堡市開始,這一把聲音便一直反覆在心中縈繞。如今我們與目標已如此接近,更是感到無比心焦如焚。於是就在一領到簽證後,我們便迫不及待地登上火車,直奔尼泊爾。

從德里至羅克索(Raxaul)的旅程是我們搭乘印度東北線火車的初體驗。羅克索是人們進入尼泊爾邊境城市比爾根傑(Birgunj),再從那裡搭巴士前往加德滿都的一個城鎮。不曾親身體驗過的人大概也無法想像。火車每開到一站,便會有數以百計身穿白衣的印度同胞高聲呼叫,然後奮力讓自己擠進車廂裡。車廂裡擁擠得人們都近乎跌落窗外,就連厢頂也同樣擠滿了人。然而神奇的是,在如此酷熱的天氣,人潮間不斷衝撞的情況下,竟然鮮少發生任何暴力行為。一開始時人們總是在高聲呼喊叫囂(這是標準程序的其中一部分),可是一旦火車開始行駛,人們的心情似乎又會緩和下來,直至火車抵達下一站,另一波的人潮推擠開始為止。

在印度眾多省份當中,比哈省(Bihar)算是受到自然災害的打擊最為嚴重的一個地方。年復一年,當地的莊稼都深受洪水或旱災所摧毀。就連當地居民的精神層面,其實也相當匱乏。他們甚至相信已過世的母親的靈魂也會處處和他們作對,因此他們擁有特別的護身符來抗衡這些「加害」他們的靈體。

我們的旅程漸漸從這一番喧囂的景象躍遷至尼泊爾美麗的山麓丘陵。我們彷彿跨入了ㄧ個居住著無數山民的「另一個」世界。早在巴特那(Patna)乘船橫渡恆河的時候,我們就看見了第一個體格瘦小卻扎實的尼泊爾人,他們通常是正在回家路上的廓爾喀軍人。他們就像是處於ㄧ片喧囂和騷擾的海洋中一座寧靜的島嶼。在他們身上,我們能夠感覺到那迷人的一體性。當印度平原與其困惑的人民逐漸消失在背後,四周圍開始由綠意盎然的小山所取代時,有一種寧靜的感覺漸漸滋長,隨即將我們緊緊包圍。這一個地方的人民讓我們感到安心、自在。即便是女性也可以自然又自由地獨自四處走動,不必擔心會陷入危險。這裡的女性會直視你的臉龐說笑嘻鬧。和之前我們途經穆斯林國家時所遇見那些裹著圍巾、矇著臉龐,活像是個「帳篷」般的婦女,又或者老是躲避任何形式上的接觸的印度婦女相比,這裡的女性多了ㄧ種難得的開放。儘管曾經有許多人告訴我們,與尼泊爾人進行交易的話要當心受騙,一開始時我們倒也不以為意。畢竟也只是幾毛錢的問題而已。不過說實在的,我們也不懂得如何去分辨那些小銅錢和鋁硬幣的價值。這些硬幣的兩面只刻有幸運符號,沒有數字。後來,當我們發現在當地售賣的火柴盒都是密封包裝以防遭盜時,才真的提醒我們應該更為審慎地處理身上的錢財。隨著我們漸漸懂得如何更聰明地和他們進行交易或打交道,我們和當地的人民才取得了更密切的接觸。這也是他們第一次如此認真地對待我們,不把我們當一般遊客耍得團團轉。

入境印度時,雨季已經開始了。從比爾根傑到加德滿都,中間有好幾個地方的道路都已經被雨水沖走了。在喜馬拉雅山區一帶,這些道路的年度修復工作就是當地人僅可確定的工作來源,所以這也是為甚麼他們不願意把道路的修補工作做得太好的原因。為保萬無一失,他們還會將大石留在山腰上的關鍵地點,然後選在「合適的時機」讓它們滾落下山,確保更多的收入。他們一天可賺取的最高收入只有半美元,因此對於像奢華的悠長假期這種想法並不太能夠接受。

我們所乘坐的巴士非常老舊,還會發出許多噪音。就像全世界任何一種交通工具一樣,這台老巴士也是滿員超載。我們乘著巴士,用心感受著旅途中的一切,顛簸搖晃地緩緩上山。有一件事我們挺欣賞,並且很快就適應,那就是適逢面對任何險境時(這一路上暗藏的危機還不少),無論是司機先生或者是乘客本身都會豁然大笑(或會心一笑)。我們都明白這是試圖消除內心裡的恐懼,卻又讓自己不失顏面的做法。這種智慧,確實讓我們留下了非常深刻的印象。就在黃昏降臨暮色蒼茫之際,四周圍的景色更顯超凡脫俗。巴士正緩緩駛入這一個受到諸佛菩薩所加被的美麗國度。

舊時的加德滿都非常美好。尼泊爾只是早幾年前才開始對外開放,因此遊客並不多見。嬉皮士潮流也尚未登陸此地。若在印度區不小心撞倒一頭牛,可會為自己招來危險。如果無法「證明」(一般是通過適地的捐錢)這一隻對印度教徒來說等同是神明的動物是自己決定在車前自殺的話,那麼肇事者就可要倒大霉了。我們在加德滿都看見兩、三輛車身非常堅固的老舊計程車,其中有一輛還是不必擔心維修問題的舊款沃爾沃(Volvo)。現在加德滿都街上那些車身被撞凹損毀的計程車都是日產的達特桑(Datsun)和豐田(Toyota)。

當地司機為了省汽油,所以常常會以高速檔開車,而且不時鳴響車笛,在狹窄的巷弄裡竄行。這種情形確實也展現了我們在文化上的差異。像是當地的苦力、手推車、騎著自行車或路上的行人都會非常有耐心的讓路,可是幾乎所有的遊客都會被這些噪音所激怒。即便到了今天仍然沒有改變的是,如果不想成為無數外來微生物的寄主,在當地還是要注意只喝煮過的水,只吃自己剝皮的水果。大多數的傍晚,文化豐富的尼瓦利人(Newari)都會聚集在寺廟裡,念誦著佛教古老的禪修法本。他們也以手搖手風琴的音聲,以及聲音尖細的敲擊鼓作為配樂。這一種音樂很直接地便能觸動人的心靈,隨即留下巨大的衝擊。走在一道一道具有歷史的街道上,人人的心靈似乎都能夠獲得寧靜,心裡所思所想都必然將會瞬即實現。就好比當我們想要去探望朋友時,總會在路上和他們碰個正著,然後聽他們說出像:「我正要去找你」這樣的話。有時甚至未有時間想清楚該如何明確地許下願望,所祈求的事物便已經掌握在手中。在這一個地方,心想即成的喜悅不曾間斷。這裡建有許多具有上百年歷史的佛教寺院,流傳著佛教瑜伽士們不間斷的傳承,這一切大概就是讓人們心想即成的箇中原因。祂們所創造的神聖能量場遍蓋加德滿都谷地,縮小了人們二元對立的習慣性與分別心,因此只要心想便靈。然而,品質最純、效果最強勁的哈希什大麻,也可在如此神聖的佛教國度,從授權商家處合法購得。我們悠閒地在店家樓上試貨,商家的女兒還會給我們奉上茶水招待,順便和我們聊聊對於下一次收成的期望。今天,人們所購得的大麻多為低檔貨,混雜了其他物質。再加上標榜著道德主義的美國人用錢打發尼泊爾政府,如今兜售大麻在尼泊爾可是受到禁止的。至少官方說法上是如此。

六十年代後期,加德滿都出現了許多有趣的怪人始祖,就像「八指艾迪」,那一個在小屋茶館創造了許多神秘語言的亞美尼亞人。他對加德滿都許多蓄著長髮,希望「有所作為」的人具有很大的影響力,而且不用多久屬於他的怪胎党便如雨後春筍般紛紛湧現,只要憑著他們的小鬍子和肢體動作便能夠輕易將他們認出來。當時也有一位吸大麻吸得有點亢奮,被稱為「Uncle」(安哥)的印度人。安哥四處分送加入了牽牛花提取物所改良過的哈希什大麻餅。許多「饞嘴的」在嚐過了之後都會飄然個好幾天。當然,就在拐角處還有一名醫生非常樂意提供應急注射的服務。每當他在尋找靜脈要進行注射時,他都會贊揚自己所使用的器具絕對是經過妥善的消毒,而且還不時建議其客戶在嚐試過中國的海洛因後,下次應該要試試他的緬甸可卡因。據說這一種可卡因的品質極為特殊。他從未給任何人在一天之內多過一次的注射,宣稱是為了不讓人上癮。然而,每每在清晨遇見他的客戶,他們總是看起來行色匆匆。我們試圖停下腳步和他們聊幾句話,可是他們卻總是迫不及待想要離開,看起來就是十足癮君子的模樣。也許醫生以為自己只是為他們提供了一點日常的小樂子,像去一趟電影院消遣般,殊不知他們都已經上了癮。要麼市場上提供這種服務的人不只是醫生一人,否則就是這些人其實遠比他所想像的更為脆弱。

回想當年,彼得‧賽樂(Peter
Seller)的電影在尼泊爾是受到禁止的,原因是他和尼泊爾國王長得太相像的關係。這裡時常舉辦各種各樣的宗教節日和遊行。有一次,整座城鎮發起罷工,某個德士司機遭到殺害的事件震驚了所有人。當地的居民對於日常生活上你瞞我詐的事情頗習以為常,甚至政治殺害事件也不是甚麼新鮮事兒,可是當地卻從未發生過任何持械搶劫(為了金錢公然殺人!)的事件。在大多數的日子裡,這裡的人們總是擁有不少歡慶的理由,在街上的某處總能夠發現有一個人人戴著花環,伴奏著音樂的遊行隊伍,在某個街角和另一群抬著擔架將死者送往河邊焚化廠的人擦肩而過。焚化廠就在屠場隔壁,在屠場裡工作的工人主要是穆斯林。他們在亞洲許多地方也是從事屠宰的工作,按照他們的宗教信仰,屠殺動物並沒錯。

加德滿都的房子都很小,身材中等的北歐人如漢娜和我來到尼泊爾後就成了巨人。我們在老城區大多數的房子內都無法站直身子,很多時候就連伸展一下身體也不可。在這裡房子的窗戶上都沒有裝上玻璃,只以漂亮的木雕裝飾;牆壁和天花板通常會被開放式壁爐的煙燻得一片漆黑,就只有泥土地板永遠看起來像是嶄新的一樣。那是因為地板一旦變髒或壞損,他們就會直接在舊泥土地板上重新鋪上一層薄薄的新地板。

在加德滿都舊區的每一處,甚至是毗鄰的城鎮如巴丹(Patan)和巴德崗(Bhatgaon),我們被許多美麗的景色和歷史所包圍,感覺就像是生活在一個令人感到震撼的藝術展覽之中一樣。當地幾乎處處可見的舍利佛塔更是出眾奪目。這些依據特別比例所建造的佛教舍利塔,如同寺廟裡所見種種圖像一般,都是為了開啟一個人內在的佛性而建。如果不畏懼街上撲鼻難聞的氣味、乞丐和病人,那麼這些舍利佛塔留在心靈裡的烙印,必定能夠在未來將你引領至證悟與喜樂的境界。

早在我們之前已有許多丹麥人來到了極具色彩的加德滿都,那些年有很多有趣的人和他們的事蹟都被寫進一些人的博士論文裡。我在這裡為大家舉一個例子。這一件事和毒販老手尼爾斯有關。奉行印度教的尼爾斯擁有和苦行僧無異的裝束打扮。他身穿淺褐色僧袍,帶著高過頭部的三叉鐵戟。有一天,就在帕斯帕提那(Pashupathinath)──尼泊爾最神聖的印度教聖地,有一群苦行僧發現了他這一個「舶來」的「同類」。他們大喊了一聲:「美國人!」,便衝過去將他大打一頓,然後就把他攆走了。隔天在加德滿都的市場內,出現了一套以低廉的價格出售的苦行僧袍、三叉戟和托缽。同時,有人瞧見尼爾斯穿著球鞋和牛仔褲,正準備著要體驗另一趟的「迷幻之旅」。

讀過《西藏瑜伽與秘密的教義》這一本書後,我們當然很渴望能夠親近喇嘛。而我們在加德滿都所見到的第一個「喇嘛」,倒是有一點兒讓我們大吃一驚。這位喇嘛叫做欽尼喇嘛(「欽尼」在當地方言即是「華人」的意思),就住在一棟粉紅色的房子裡,若從陽台上眺望,可見博德納大佛塔的入口處。由於他所提供的美元兌換率是銀行的兩倍,因此朋友老早就建議我們去找他換錢。我們注意到在他家門前停放著好幾輛賓士豪華轎車,不過當時對此倒也沒有多在意。直到後來,我們才聽說原來他的「喇嘛」頭銜是繼承而得,而他的職責就是作為博德納大佛塔的管理員。看來這一份工作他似乎是頗為得心應手,而且絕對是有賺沒虧。

欽尼喇嘛的個子不高,身材倒是頗為粗壯魁偉。他頂著光頭,五官面貌像極了華人。當我們進入房裡時,他看來似乎是剛睡醒。他慢慢坐直了身子,而我們詫異地發現他的兩隻手臂上各自戴了至少五隻手錶,從手腕倒手肘上部一字排開。雖然看起來很可笑,可是我們卻天真地在想,也許他是利用這些手錶來進行非常複雜的調息練習。他開口第一句話,就是問我們是否販賣槍械,我們也天真地心想,也許眼前這一個慈悲的男人正為康巴人(西藏的自由戰士)伸張正義呢!我們支持正義。結果我只能悔恨地告訴他,除了隨身攜帶的刀子以外,我們並沒有其他武器。他之後又想要知道我們是否擁有磁帶錄音機。那倒是有一台,而且是菲利浦的第一款型號。當時盒式磁帶錄音機在市面上才剛新鮮出爐,難得千里迢迢帶著這台錄音機踏上此趟向東之旅,而且迄今仍然運作正常,所以告訴欽尼喇嘛我們要先用它錄下藏傳寺院裡的一些音樂。當時我們相當肯定自己對佛教的這一份心意將會令他感到無比歡喜,不過若真為如此,他倒是隱藏得很好。

當他發現沒有甚麼東西可向我們購買後,反而想要賣鴉片給我們。我們對眼前這位「聖人」的單純感到十分訝異,執意向他解釋說鴉片會讓人上癮,十分不健康,他不應該隨便向人兜售。鴉片正慢慢地奪取了身邊許多朋友的性命。他此時所說的話卻給了我們一個錯覺,誤以為東方國度裡所有「聖人」都有服食某些毒品。這一個錯覺更是一直跟隨了我們許多年。欽尼喇嘛最後以特貴的價格把哈希什大麻賣給我們,不過也以極好的匯率兌換了我們的美金。

當天傍晚,我們和朋友們在城中某家茶館想試試新進的大麻,卻差點兒沒有笑破肚皮。欽尼喇嘛以三美元的高價賣給我們的「上等」哈希什,其實只是抹了黑鞋油的木炭。這一個教訓告訴我們,即便是最有名的商家,購買前也得先驗貨。

在加德滿都有很多可看、可做和可以體驗的事物。可是,城鎮外圍某座小山上的斯瓦彥布寺院(Swayambhu)卻尤其吸引我們。若要繞最短的捷徑而行,抵達斯瓦彥布之前會先經過一堆剛遭屠宰的動物的頭骨,以及一群看得出來吃得好可是卻又很兇猛的野狗。這一路上所途經的地區相當齷齪,而且沒有人會停在附近的茶館歇息(這些茶館大概是從附近的河流打水的吧?)。其實,不管是這些令人不快的因素,抑或是這一個城鎮本身的超凡魅力,對我們來說都不重要。我們每天只是一心想要往山上走去,那裡似乎有甚麼不斷地牽引我們前往。

從加德滿都放眼遠眺,斯瓦彥布就矗立在一座金字塔狀的山丘上。通往寺院的路有三條,都是一些陡峭的階梯,沿途有許多大佛像和無數的舍利佛塔。舍利佛塔的形相表徵成佛時所證得的五種智慧,這五種智慧乃一切有情眾生本俱的智慧。山上的中央圓塔傳說源自於飲光佛的時代(梵文:Mahakasha,迦葉)。祂是在釋迦牟尼佛之前降世的第三尊佛。這一棟外觀宏偉的建築是偉大的西藏瑜伽士嘉華噶瑪巴與其主要的傳承持有者昆吉夏瑪巴在尼泊爾的主要道場。後來噶瑪巴與夏瑪巴更成為了我們非常重要的兩位上師。

圓錐形的大佛塔周圍是ㄧ座環狀的院子。熙熙攘攘的訪客總是順時針繞塔而行,如果不嫌棄轉經筒油孳孳的人也可以伸手推動轉經筒。經過數座磚房後只見有一座灰白色的藏傳寺院,其莊嚴宏偉的中央入口直接通往佛殿裡的供養壇。

尼泊爾語「斯瓦彥布」意即「自生」之處。對於這一個地區的佛教徒而言,它就好比印度的菩提迦耶(當初釋迦牟尼佛就是在此地證得圓滿菩提)般神聖。由於在印度,當地人老是纏著遊客不放,因此大多數的人在尼泊爾會感到較為自在。此外,這一個地區的能量場非常殊勝和強烈,不僅利於禪修,亦能廣澤大眾,迅速滿一切願。

我們對此地的第一印象其實並不太好。我們既不喜歡此地的灰塵僕僕(就像在尼泊爾大多數地方一樣,想要找一個可以坐下來的地方也難),也不喜歡那一些假穿僧袍混進人群中,想要拉客兜售物品的騙子。四周也不時傳來猴子憤怒的尖叫聲,它們總是爭吵不休。眼看那些長得又大又壯的猴子欺侮較弱小的猴子,咬它們或搶食物時,心裡很不是滋味。生病的老狗會自行來到佛塔前嚥下最後一口氣,想必是受到此地的能量所牽引,這些可憐的小東西啊!儘管眼睛、耳朵、鼻子不斷遭到這種種景象所衝擊,我們卻感覺自己似乎非常了解這一個地方。雖然我們在尼泊爾擁有不少有趣的朋友,可是我們鮮少在天黑以前就離開斯瓦彥布。

寺院內的莊嚴報身佛像散發著一種巨大的能量。我們經常感覺佛像似乎會呼吸似的,而身穿紅袍的僧人們所唱誦的經文和法器交織而成的樂音,也會不知覺讓人著迷。直到很久以後,我們才了解箇中靈性與秘密的含意,不過即便不懂,也無礙於我們感受它的力量。

第一次來到斯瓦彥布時,有兩位僧人讓我們留下了深刻的印象。其中一位就是薩曲仁波切(Sabchu
Rinpoche),他後來成為了我們「寂靜」和「忿怒」出入息法的導師。當時,他是寺院裡的住持喇嘛,若干年後當仁波切圓寂時,他展示了其證量。據說仁波切圓寂後一個星期,他仍然保持著禪定的坐姿,他的體內及周圍的能量凝結成無數像珍珠和寶石般彩色的舍利子。

當時,我們對仁波切的成就一無所知,可是只要見到他,便會感到很振奮和高興。有一次,我很隨興地跟隨他繞著佛塔周匝而行,他突然轉過身,凝視著我的雙眼,然後用藏文對我說了一些話。當然,我是完全沒有聽懂他的意思。我能夠感受到他的慈愛,可是卻同時當著他的面,突然爆笑起來。就近距離來看,他和丹麥某個很有名氣卻非常搞笑的體育記者長得真像!

有一個叫彭厝的年輕出家人送了我們一串「瑪拉」。「瑪拉」即是串有一百零八顆珠子的念珠,表徵證悟之前的八識心法,以及證悟後所顯現的如來百德。他也送了我們一小張照片,照片上是ㄧ個魁梧莊嚴的男性,手持著黑寶冠頂戴在頭上。這位年輕的出家人彭厝還不斷重複地說:「噶瑪巴、噶瑪巴」,彷彿正試圖告訴我們照片中的人究竟有多重要似的。

在那個時候,若要前往尼泊爾第二大城市博克拉(Pokhara),就只能依賴那一台老舊的轟炸機。博克拉位於加德滿都以西的山谷地帶,地方機場就是ㄧ片草地。當地最常見的交通工具是租用的馬匹。由於我們無論如何也無法像他們那樣抽打這些馬匹(可是若不打,它們就真的是ㄧ動也不動),結果到最後我們無論走到哪兒都得硬拉著它們一塊走。瑞士人在這裡建立了ㄧ個規模頗大的西藏難民營。我們很開心見到他們是真的在幫助這些西藏難民學習不同的維生技能,像編織等,讓他們能夠憑著自己的力量養家餬口,而不是向他們兜售西方文化或基督教。然而,這些瑞士人最後還是因為受到中共的壓力而被迫離開。今天博克拉的西藏難民營變成主要靠旅遊業維生。

在尼泊爾待了幾個星期後,我們不只是能夠認出ㄧ些熟悉的面孔,就連他們當中幾種不同的民族和文化也能夠分辨出來。雖然對於ㄧ開始時所遇到的人們,他們親切的笑臉在我們的眼裡看起來都一樣,可是來自群山另一方的西藏民族卻擁有一種說不出的吸引力。我們總是會熱心地購買他們所兜售的物品,並且會試圖以肢體語言和他們溝通。

其實西藏人非常聰明,雖然這並無法從他們管理國家的方式中看出來。儘管世界已邁入二十世紀,他們卻仍然過著像中世紀時期歐洲般的生活:在中部和南部,他們擁有一個實行階級制度,並且沉湎於排外政策的中央政府;在北部和西部,他們是獨立自主,又溫馴善良的遊牧民族;國土內到處都有寺院林立,各自擁有不同的制度。

大家都害怕東部的康巴劫匪。康巴人樂天爽朗,數世紀以來一直竭力驅逐中國的入侵,是素以驍勇好戰聞名的部落。對康巴人來說,每個人都是自己的國王。儘管各地區的民俗風情、制度不同,他們也不奉行民主、透明制,也不講究人權,然而當一個人遇上了西藏人,清楚看見他們在人際間的交流,以及超乎於異常的彬彬有禮後,這一切種種的差異都顯得不重要了。除了佛教給與了他們深遠的影響之外,他們的幽默感往往也能夠讓所有事情都變得相當合理,讓人感覺自己似乎是置身於一個至善至美的地方似的。中部和西部的藏民性情溫和,擁有像愛斯基摩人般的臉龐;東部的部落則是剛毅堅韌,相貌酷似納瓦霍民族。他們看起來總是輕盈自在,流露出說不出的內在平靜。如此的氣韻,想必是和他們對於整個宇宙所持有的獨特見解有關。

就物質層面而言,他們當中大多數人仍屬世界上最貧困的難民之一。他們沒有聲稱自己遭到種族迫害,也沒有威脅要發起戰爭或革命,因此一直以來都沒有激起太多的國際注意或援助。今天,有許多西藏孩童死於痢疾,而大多數的成年人則患有肺結核。

莎瑪,是我們投宿的那一家旅館的主人。旅館坐落於城鎮的中心,一晚才五十美分。莎瑪說得一口流利的英語,常常提供我們當地活動的最新消息。他也是我們的乳麋供應商,也常為我們提供大量一般哈希什愛好者都饞嘴的甜奶茶。我們的隔壁房住了一個來自喀什米爾的商人,擁有該地區典型的英迪拉‧甘地型五官。有時候他看見我們喂他的狗兒吃焦糖會感到很吃驚。在這一個大部分父母親沒能力給孩子買糖果的國度,我們的動作確實是頗不尋常。和我們一起一路開車到喀布爾的克萊斯和他的關係特別要好。克萊斯在我們抵達數星期後也飛到了加德滿都。他花了一些時間才把他的甲蟲車賣掉,而且途經巴基斯坦時,有人幾乎把他的耳朵咬斷了一半,所以才姍姍來遲。當時他被迫要到當地的醫院求醫,住院期間也差點兒要了他的命。現在康復了,才能再次動身來到這兒。也許這一個喀什米爾人曾從克萊斯那兒聽說過我有多麼敬愛我的父母和弟弟。反正有一天他來到我們的房間,當時我們正在嘗試某些非常強勁的LSD,他不斷在我們的面前說起了有關我父親的好話而贏得了我的信心。在迷幻藥的作用下,我們聽他敘述那些從西藏來到喀什米爾的喇嘛們的本領,以及許多未曾聽說過的神秘故事,包括喇嘛們如何戰勝印度教祭司的精采故事。我們聽得出神,甚至因為太興奮差點兒從椅子上掉了下來。這些對於西方國家的人來說神秘不已的事蹟,現在能夠親耳聽說,感覺太美妙了!後來,他提及自己隔天會和某個擁有這種神秘力量的婦女見面,說如果我們能夠給她一些錢,她會在山區裡的聖地替我們點燃酥油燈祈福,保佑我們遠離一切障礙和險難。他也建議我們戴上銀製手鐲,說是對肝臟好。我剛經歷了兩次肝炎復發,漢娜則是首次受到肝炎的折磨。他的一番話正好戳中我們的死穴。不管怎樣,我們再也不願意受到肝炎的威脅。

隔天晚上,我們按照原定計畫飛返歐洲。為了躲避印度的海關,我們選擇了一條較迂迴可是卻相對便宜,途經巴基斯坦東、西部的路線。印度機場內的官員膽敢視尼泊爾為較落後的「內陸」地區,我們擔心他們會因為我們在尼泊爾所購買的東西而對我們諸多為難。當時,我們還以為自己所購得的西藏古畫卷和佛像很值錢,殊不知其實我們已經犯下了新手常犯的錯誤,也就是只會傻傻地去注重物品上的那一層銅綠(其實只要將物品埋在煙囪或地底下幾個星期就會產生銅綠的效果),反而沒去注意看手工是否精細。今天,我們擺放在禪修中心裡的各種飾品,沒有一件是第一次遠赴尼泊爾時所帶回來的物品。當時,這一雙因為時常吸大麻而通紅的眼睛,大概也很難挑出甚麼好東西。其實這些畫軸或銅像,非常講究比例和顏色的使用是否正確,否則無法帶來任何具有啟發性的效果。若想要對心性有所助益,正確的比例和顏色非常重要。關於這一點,經書裡都有非常精確的記載。我們唯一擁有上好品質的物品,就是「多噶」(Dolkar)這位女菩薩的畫卷和銅像。「多噶」,也就是打從一開始就以不同的形相顯現在我們面前的白度母。祂曾經在南非的某座山腰上救了我一命。現在,祂又以「度噶」(Dukar),又稱大白傘蓋佛母的形相顯現,護佑著遍布世界各地的我的學生們和愛飆車的我。

在尼泊爾的三個星期是一場美妙的經歷,所有的事情都比所願的來得圓滿。離開之前,我們見到了喀什米爾人所提及的那位婦女,那天她來到我們的旅館。我們和她交談了幾句,她的ㄧ雙「佛眼」,和身上所散發著的深沉、平靜的氣質,讓我們留下了一個非常深刻的印象。我們將一個裝有廿十美元的信封交給那位喀什米爾朋友,請他轉交該名婦女。反倒是克萊斯對這個喀什米爾人有所保留。他以為我瘋了,認為我們應該把錢留著到阿富汗多買一些哈希什大麻。那裡的哈希什至少要比尼泊爾的價格便宜一半。我們想趁離開之前購買那一個可用以對抗肝炎的銀製手鐲。我們幾乎搜遍了加德滿都的小商店,可是仍然一無所獲。後來,有人建議我們到西藏村裡去找。雖然時間有限,我們心想在前往機場途中,應該可以繞道去找找看。不到十分鐘的時間,這一個看起來頗具威嚴的西藏婦女拿出了兩隻重達170克,手工精美的純銀手鐲。她當時開價要\$10一隻。當我們用手輕輕觸碰手鐲時,似乎能夠感覺到裡頭一陣陣脈搏式的跳動,就像一種鮮活的能量。類似的經驗,在這麼多年來我們也只是碰過一次。離開尼泊爾之前,我們已經決定了,一定要再回來。

在坎達哈省(Kandahar)短暫的停留期間,我們前往阿里酒店(我們的毒品秘密供應處)欲領等候多時的大麻。之後,我們經由德黑蘭(Teheran)飛往羅馬,並且在當地找到了一輛要遞送至哥本哈根的可租用汽車。約二十個小時後,我們便來到了丹麥邊境。我們先試著空手過境探一探風聲,看看是否有人留訊給我們。儘管有個女孩曾經向警察提起過我們的事,慶幸的是整個過程仍然相當順利。

雖然這是我們第一次前往尼泊爾,而且逗留的時間並不超過一個月,可是內心裡的某個部分卻自此未曾離開。這是一次非常深刻的經歷,也令我們對生命的看法增添了不同的色彩。雖然從外在看來並沒有多大改變,可是我們清楚知道,內心裡已經經歷了一場重大的變化。

就在短短幾天之內,我們的鄰居喬真突然病得非常嚴重。他和我們這一群「毒友」一起住在充滿中世紀風情,風景如畫的哥本哈根克里斯欽港(Christianshavn)堤岸對面的ㄧ棟老房子內。喬真在樓梯間昏倒了,臉色黯沉臘黃,雙眼充血呈褐紅色澤,看起來可怕極了。當時他患上了嚴重的肝炎。我把他抬回寓所,在他尿尿的時候扶了他一把。他的ㄧ泡尿看起來不但呈棕褐色,而且很濃濁。其實他相當可憐,這是他第二次病情復發。回想起多年以前,我們可說是不打不相識。雖然打成平手,可是後來卻發展出一段深厚的情誼。摩托車和毒品是我們的共同興趣。他也練拳擊,而且在四十二場賽事中都幾乎是大獲全勝。由於相較於其他吸大麻的人,他會需要更多的體力,因此有時候他會為自己注射安非他命來增強體魄。他顯然是曾在某個地方和生了病的人共享了一支針筒,所以現在才會患上肝炎。我想起了喀什米爾人曾經說過,我們的銀製手鐲會對肝臟有利,因此我把自己手上的鐲子戴在了他的手腕上。曾有那麼一瞬間,我停駐在掛在門上的剃鬚鏡前,出神地凝視了一會兒。忽然之間,我感受到有一股巨大的能量向我襲侵而至,就像接通了一股高壓電流似的。我像柱子般佇立著,似乎有一道光芒吸收了所有的想法和體驗。我不確定像這樣的情況究竟持續了多久,不過當我回過神來之後,我只是覺得整個人疲憊不堪,茫然失措。這一切是在我沒有吸食任何毒品的情況下發生,可是它卻比我吞了大劑量的迷幻藥時的經驗還要來得強勁。我只是曾在在後院第一次吸大麻時,有過類似的強烈體驗。我下樓來到街角的某個公共澡堂,在流水下站了好長一段時間,思度著:「那究竟是甚麼?那一道光究竟是從何而來?」

翌日早晨,我和漢娜到喬真家去探望。只見他的臉色和眼睛已經恢復了原來的色澤,他說連排尿也是清澈的。他還告訴我們晚上睡覺時作了許多很深沉的夢,告訴他要遠離針管。這真的是奇蹟啊!一般上受到感染的肝臟需要數月的時間再生,根本不可能以這樣的方式就被治癒好。我們真的感到非常開心!也許能夠解釋喬真突然好轉的理由只有一個,就是那些銀製手鐲真的具有療癒的力量。

正當我們在房內試圖給整件事想出一個合理的解釋時,耳邊忽然傳來了一陣敲門聲。傑特(她大概是我們所認識最不快樂的ㄧ個人了)就站在門外。她一看見我們,就只是說了這一句話:「我得肝炎了!」

我們再次覺得不可思議。在這之前我們已經有好幾個月未曾見過傑特。她不可能會知道這裡究竟發生了甚麼事,而且她似乎是受到某種力量的牽引才來到我們這兒。雖然她的情況頗糟,可是就在她戴上了漢娜的手鐲一個星期後,肝炎就被治好了。後來,她還向我們描述自己所看見的「恩典之光」。傑特來自丹麥西岸的漁人區。這裡也是國內仍然保持強烈基督教文化的區域,至於「恩典之光」,當地的人都是這麼說的。

漢娜和我感到非常幸福。通過這種方式去幫助別人,真的為我們帶來了非常不可思議的快樂。之後陸陸續續也出現了幾個療癒的個案。一九六八年的秋天,我們身邊有超過二十名的毒友被完全治好。我們總是感覺到治療的能量已經被激活,偶爾也會浮現出一些莫名的想法,大概是吸收了那些戴過手鐲的人的念流所致。現在沒有甚麼事能夠和這些療癒的事件相提並論了,就連我們時常去的黎巴嫩也開始黯然失色。基於習慣,我們不斷重複做著這一件事,朋友們也對我們抱有不少期望。所有事情似乎不斷把我們推向尼泊爾。就在開始放寒假的第一天(整整一個月都不會有學生來上課!),我們收拾好行囊再次動身往東出發。

第二章

\textbf{身體透明瑩澈的喇嘛}

我們這一次為了要掩蓋行蹤,所以決定繞道俄羅斯。雖然這樣的安排會花更多的時間和金錢,不過卻必須這麼做。火車從哥本哈根經由柏林,一路開往莫斯科。我們再從莫斯科搭飛機經由塔什肯(Tashkent)和喀布爾(Kabul)飛往德里。

東德和俄羅斯喚醒了我在從軍時的種種回憶。我發現那些狂妄自大的官員正是這世上最可笑的東西,現在逗人發笑的事情就更多了。然而我們更為那一些無法躲避內心裡沉重的壓迫的人們感到悲憫。他們的人生才是真的淒涼。我們悄悄將隨行多帶的尼龍襯衫、剃鬚刀和香菸分發出去。這是我們在如此倉促的時間內對他們所能盡的一點心意了。

在布列斯特\textbf{─}立陶夫斯克(Brest-Litovsk)(位於波蘭和俄羅斯之間的戰後邊境)通關時所花的時間很長。這時發生了一件趣事。我看見有一士兵連隊剛好路過,猜想他們應該是哥薩克人。他們當中有的長得牛高馬大,有的較為瘦小,我們的面孔倒是長得非常相像。我在丹麥不曾碰過像這樣的事,如今感覺有點兒像是身處於帶鏡的立櫃中一樣。如果我穿的不是白色襯衫,而是他們身上的褐色軍裝大衣,我大概能夠偷偷混入隊伍中而不被發現。這種經歷就猶如置身於夢境中一般,讓我看見我們對「我」的執著其實是如此地狹隘。

在莫斯科機場,有一群士兵想要了解捷克斯洛伐克(Czechoslovakia)的情況。他們的長官在出發之前說他們將會被派遣去擊退那些在蘇伊士運河高喊帝國主義的西方入侵者。可是抵達後才發現自己所對抗的竟是鄰國的斯拉夫(Slav)同胞。現在的情況讓他們感到非常迷惑、洩氣。由於早在半年前「布拉格之春」改革事件爆發時,我們曾經開車穿越該國境,於是便把所知的一切告訴他們。不一會兒,機場內就來了一群凶惡的警察強行把他們帶走,還試圖讓他們將我們所贈予的「皇家海軍」牌香菸歸還我們。我們當然無論如何也不肯收下。最後這些警察也收下一些香菸,還急忙地將香菸藏起來。整個場面就像一場鬧劇!

當時恰好有兩位美國人成功登上月球,成為首次登陸月球的人類。那些無處不在、無論對任何事都抱有「東方」和「西方」強烈分別心的「嚮導」,對此事表現得相當介懷。「這次換你們贏了。」雖然嘴裡是這麼說,實際上臉上的表情卻是酸溜溜的。我們還淡淡地回答說:「也不是甚麼大不了的事,他們早在去年就已登上月球,只不過沒公開而已。」當然此話一出並沒能讓我們變得受歡迎或討喜些。

為了要具備購用廉價航空券的資格,我們必須要在塔什肯住上五天。這些「嚮導」也給了我們有關從北部公路入藏的一些主意。如此一來我們不但能夠見識到在那裡定居下來的穆斯林部落,也能夠親身體驗一下那一個一直延伸至西藏北部羌塘(Changtang)的大沙漠。

在俄羅斯,去莫斯科大劇院(Bolshoi
Theatre)或穿上威風凜凜的制服者顯然都是白人,而棕色或黃皮膚者主要是建房子的勞工。在政府機構裡工作的人,尤其是女性都是看起來非常沮喪的模樣,所有東西看起來更是難以置信的粗造濫製和破落。我們一直希望能夠開啟這一個國家人民的潛能。一九八八年的秋天,我們的願望實現了。我們在聖彼得斯堡(Petersburg)和塔林(Tallinn)創立了首個地下佛教團體。十年後的今天,在這一個廣袤的大陸,從海參崴(Vladivostok)到聖彼得斯堡,我們一共建立了40所禪修中心,每一年我們都會在這一個國度待上兩個月的時間為大眾說法。

飛機飛越一座又一座的連綿山巒,它們看起來就像是被翻倒的巨大磚牆,喀布爾(阿富汗首都)出現在中亞晴朗的天空下。喀布爾比塔什肯更貧困,不過市內一些西方廣告讓整座城市看起來較為摩登。當地的天氣保持在大約攝氏零下十度的恆溫,空氣非常乾燥,所有的東西似乎閃爍著光芒。當時塔利班政權尚未執政,可是這一個國家的婦女和窮人也已經吃了不少苦頭。

在喀布爾機場,有一名男子銷售前往印度時所需的防疫注射證明書。我們是無論如何也不會向那些又粗又生銹的針管妥協的了。我們付數美元,他便在證書右方的虛線上簽下名字。

我們在德里取得了為期兩周的尼泊爾簽證。這是我們第二次搭火車北上。其實我們可以利用學生證買機票直接飛往加德滿都,價錢並不昂貴。只是陸路似乎成為了我們此次旅程的一部分。我們可不希望和一群遊客一起降陸在這一個對我們來說是希望亦是夢想的國度。此外,在冬天搭巴士一路北上會是非常美妙的經歷。這一次我們將能夠在碧空無雲也無雨的情況下,清楚地看見峰峰相連的喜馬拉雅山脈。

我們渴望能夠再次和那一位婦女見面。我們視她為那些神奇療癒能量之始源。這種感覺隨著旅途不斷地增長。由於警方已經開始緊盯著我們,所以我們才迫不得已要繞遠路,途經共產主義的俄羅斯。其實我們應該低調行事暫避風頭,可是由於答應了回程時會前往黎巴嫩幫忙帶一些大麻回來,因此才選擇不同的路線出入境。如今我們已離目的地不遠,覺得實在不需要過於周詳的計畫,再加上我們實在是迫不及待想要和她見面。巴士在傍晚時分抵達加德滿都,我們趕緊在郵局附近下車,免得到站後那些孩童蜂擁而至,到時想躲也來不及。這些孩童會纏著旅客不放,拉扯他們的行李和背包,熱心地為他們介紹下榻的旅店。這一種現象在過去六個月已不斷地「進化」,這種不光彩的改變,我們實在不想攪和其中。

我們下榻的旅店位於加德滿都老城區的中心。在前往旅店途中,我們貪婪地感受著周圍的氛圍。我們曾經在此地擁有太多值得回憶的冒險經歷了。這一次能夠舊地重遊真的很開心。抵達旅店後,我們才聽說莎瑪如今負責管理火警瞭望塔附近的一家新旅舍。莎瑪的消息靈通,是我們獲得各種情報的資源。這一次來到加德滿都,我們還一心冀望著他會認識那一名神秘的婦女。聽說莎瑪不在,我們又背起了行囊離開旅店。途經郵局時卻很巧地和他碰個正著。

莎瑪果然認識她。聽了我們的故事後,再瞧我們如此急切想要和對方見面,莎瑪笑了,還答應隔天早晨陪我們去見她。她就住在城外某個叫馬哈拉亞坤(Maharajgunj)的地方,靠近大街上流經數個主要領事館的供水系統。那一條也是加德滿都的主要道路,一路延續至一家中國鞋廠。當時,這條道路來到鞋廠後便逐漸退成一條供人們步行上山的狹窄小路,引向山谷盡頭一座稱為納吉貢巴(Nage
Gompa)的寺院。

這名婦女與夫家及其丈夫的小老婆一起住在街道附近的磚房裡。房子的後方是一個打理得井井有條的菜圃。我們聽說她的名字叫布塔拉錫米喇嘛(Buddha
Laximi
Lama),便擅自把它當作是一種崇高地位的標誌。殊不知在西藏的邊境地區,「喇嘛」一字已經成為了其中一個姓氏,完全失去了原來「慈悲的上師」的含意。她的家人招待我們,並且在他們家中一間低矮的房內喝茶。後來,她到房裡來了。她的體格矮小,整個人感覺十分平靜、從容,擁有一雙我們忘不了的「半斂細眸」(註:就像佛像垂視眾生的慈眼或修禪定時半開半闔的雙眼)。她的弟弟T.B.
喇嘛負責翻譯,我們有許多話要說。我們由衷地感謝她為我們帶來的一切美好事物,神奇的療癒能量,甚至是在走私和飆車時眷顧著我們的好運氣。當時喜歡飆車的我們曾撞毀不少摩托車和二手車,可是人卻總是平安無事。更玄妙的是,就連我們身邊的朋友也受到庇佑。我們將一隻金錶、一條金鍊子和其他禮物全都送給她,了表謝意。

聽完我們的故事後,布塔拉錫米就笑了,就連她的弟弟也忍不住捧腹大笑。不問也知道,我們交託給喀什米爾人的那二十美元,她當然是毫不知情。自從那一天在旅店的走廊上匆匆見過一面後,她便沒有將我們放在心上。她說:「我只是一個會做點小生意過日子的平凡婦女,並沒有甚麼特殊的能力。不過既然你們具有這方面的信心,你們應該要見一見我的喇嘛。他擁有無窮的力量。不過,他現在正好到外地去了,等他回來後,我會盡快安排你們和他見面。」

布塔拉錫米認為我們的生活方式太危險。她說她會請珠穆朗瑪峰附近某個西藏難民營的瑜伽士為我們製造某種「護身符」。這些藏文稱作為「松度」(Sungdue)的護身符具有保護的能量,而且會延伸到配戴者的身上,保佑他們遠離危難,予以利益。藏傳佛教(現又稱金剛乘佛教)中最常見的護身符便是金剛繩,或是以表徵五智的五種色線綁著的折起圖像,它們都非常有效。這位瑜伽士偶爾會來到加德滿都。許多農民會請他向天祈雨,這一次他恰巧也來到此地。布塔拉錫米說看在我們如此大方不計較的份上,她會以低價賣一些哈希什大麻和宗教物品給我們。帶著破滅的幻想與困惑,我們和她道別。即使無法相信她就如自己所說般平凡,至少我們也算是多交她這麼一個朋友,也為她的家人帶來不少歡樂。眼看像我們如此狂野的北歐人即將要和佛法善知識見面,他們亦感到非常高興。

在等待喇嘛歸來的同時,我們也逐漸投入和適應加德滿都的生活。除了一些舊有的朋友,我們也交了一些新朋友。然而,身邊的事物並不是那麼理想。當時我們天天嗑藥、吸大麻,人都變糊塗了,並未發現身邊的朋友們的狀況已日漸變差。六十年代初期與他們初次見面時,他們當中有許多人是非常有趣的理想主義者,希望創造一個更美好的世界,然而他們後來卻變得瘋瘋癲癲,其中有的人也死了。這一些原本快樂友善、心胸廣闊的「大麻幫」兄弟們,一個接一個漸漸變成了名副其實的行屍走肉。

尼泊爾的情況也不甚理想。我們當初在這一個被遺忘的小國裡發現的「完整性」,不僅是受到物質主義的威脅而已。北京方面不斷對這裡的政府施壓,堅持要在各處建造主幹道路。光是想到某個早晨一覺醒來,龐大的中國軍隊可能沿著這些主幹道路進入了這一片國土,突然出現在自家門前,大概沒有多少個尼泊爾人會喜歡。K.S喇嘛為人不錯。這一次我們向他購買了一些價格便宜,質量卻非常好的畫卷和塑像。他從一些能自由出入西藏的商家處捎來了不少情報,其中有不少是關於文化和一些熱愛自由的人民遭受壓迫的駭人故事。

在加德滿都的小店裡出現的物品,其實也訴說著他們的故事。有一天,市場上到處可見女性戴過的舊手鐲,這說明西藏的遊牧民族已經不被允許配戴這些飾物了。在西藏,沒有一個婦女會心甘情願放棄這些代代相傳的飾物。如此看來,實在不難想像他們的處境。像這種價格便宜,數量又多的藏族藝術珍品,都是博物館不允許運出國的收藏品。然而到了夜間,往往會有某個「友好」的官員騎著腳踏車出現,以低收費「批准」私運這些物品出國販售,然後又開開心心地騎車離去。

乘著開往拉薩送信的吉普車來到位於藏地邊境的溫泉時,河對岸的國度看起來就像是一個負傷的巨大動物。那些扛著一麻布袋又一麻布袋大麥粒渡橋來到尼泊爾邊境的藏民,是我見過唯一臉上是毫無笑容的西藏人了。他們將肩上的麻布袋卸下後,甚至連頭也沒抬起來就離開。這讓我們覺得非常奇怪。我們所認識的西藏人,是即使在患了肺結核吐血至死之前,仍然會幽默地開玩笑的人。

一名剛渡河逃出藏地的婦女說,當中共入藏時,他們必須趕緊將所有佛教典籍和圖像藏起來。大家都知道匿藏佛像等佛教文物的地方,只是沒有人敢去領取。她很開心自己能夠逃到尼泊爾,只是就像其他難民一樣,她不得已留下的,卻讓她痛心不已。

有不少人談論著在邊界發生過的種種小插曲(如果事不關己,這些小插曲聽起來其實挺有趣的),就像發生在四個法國人身上的事一樣。曾經有四個法國人在大霧之下不小心開車駛入沒有標記的邊境大橋,遭到中國方面逮捕,被拘留了三個星期才獲得釋放。在拘留期間,他們被被迫要背誦毛主席的小紅書。看見這些總是成群結隊或穿著制服的中國人一臉驚嘆地凝視著店裡的鋼筆和其他宣傳商品,感覺挺不可思議。這些都是在他們的國家所生產的東西,可是卻無法在當地購得。按照協議,中共方面在建路造橋時必須確保這些公路和橋梁的負荷量不超過一台滿載貨車的重量,可是他們所造建的公路和橋卻總是堅固得足以承載好幾輛坦克車。

印度政治對尼泊爾南部的人民具有更直接的衝擊。印度人會因為邊境一些愚蠢的小糾紛,貿然切斷尼泊爾油的供應。

等待多時,喇嘛終於回來了。在這段期間,他前往各地由他負責管理的寺院和禪修中心處理法務。喇嘛本身來自不丹,今天佛教在尼泊爾得以繼續保存和發展,有賴於他在佛法事業上不間歇的努力。在一個支撐信仰的社會結構逐漸瓦解的情況下,他必須付出很大的努力。洛本傑珠(Lopon
Chechoo)家喻戶曉,倍受尊崇。他曾經在喜馬拉雅山脈的洞穴裡跟隨上師們學法多年,掌握各種法教。如今,在布塔拉錫米的安排下,我們即將和他見面。

我們按約定先到布塔拉錫米的家,她還招待我們吃了一些特別辛辣的道地尼泊爾料理。我們後來也慢慢習慣了當地的口味。尼泊爾就像亞洲大部分地區,人們似乎一天兩餐都是吃一樣的食物。尼泊爾人的主食是白米飯,豆泥或小扁豆,更常見土豆片,或者加上一片肉,和大量的辣椒。用餐後,布塔拉錫米帶我們去見喇嘛傑珠,她的弟弟也隨行當我們的翻譯。我們在一片黑暗之中穿過一條條的窄巷與稻田間的阡陌小路,四周楊樹瑟瑟聲地搖曳響動。喇嘛傑珠的房子是一棟兩層樓高的長型紅磚建築,以尼泊爾風格建造,區分為兩個部分。只見大門敞開著,於是我們穿過一支支經幡飄揚的竹竿,進入屋裡。屋內一切從簡,乾淨整潔。雖然屋內到處都亮著燈泡微弱的褐黃色光芒,不過屋內極大部分看起來漆黑一片。屋內身著紅袍的男女眾帶領我們上樓(我們顯然是身處於寺院中!),然後沿著狹長的走廊來到喇嘛的房間。由於注意到有許多鞋子整齊地排列在房門外,因此我們也入鄉隨俗把鞋子脫下,尾隨布塔拉錫米進入房內。

我們終於和喇嘛見面了。喇嘛的教法在過去一直以不同的形式與我們同在,護佑著我們。他就像是一切美好的示現。他示意讓我們坐在面前,問我們從何而來,所做過的事。一如既往,我負責說話(漢娜則通常在一旁觀察),喇嘛非常和善地垂視著我們,在言談之中,我覺得我們之間完全沒有文化上的隔閡。

我感謝他通過手鐲所展現的療癒能量。我也告訴他有關我們的旅程和充滿刺激的人生。任誰也不難從我的語氣中聽出,我確實認為自己「酷」斃了;至於喇嘛,無論我說甚麼他都照單全收。有時他看起來像是在沉思,眼皮隱約在跳動,半闔著雙眼;雖然看似已經睡著,不過其實仍然保持覺知。當我們坐在那裡時就發生了一件讓我們感到驚嘆不已的事:喇嘛的身體漸漸融入於眼前的虛空之中。他的身體逐漸變得透明瑩徹,我們能夠穿透他的身體,清楚看見其身後牆紙上的圖案。當我們面向坐在右側的翻譯員,透過眼角去瞄喇嘛傑珠時,他看起來是實體。可是當我們直接望向他時,他的身體又再次呈現透明狀。

自從在德里遇見那位印度老人後,我便已經受夠了催眠。我突然想起隨身攜帶那一本有關西藏瑜伽的書籍,裡頭曾提及有關催眠的小把戲。我掏出從丹麥帶來的火柴盒,然後將之舉起在喇嘛的面前。若是在催眠狀態下,眼前的這一個火柴盒看起來應該也是透明的才對,可是它始終看起來是一個實心體。我的視線仍然能夠穿透喇嘛的身體,看見他身後的牆紙。我的內心突然生起非常強烈的信心和極大的喜悅。我非常感恩喇嘛能夠以如此讓人信服的方式向我們展現心性的力量。他能夠按照個人的意願自由地讓自己的身體溶入虛空。我將手腕上的錶脫下。這手錶是第一個登陸月球的歐美加(Omega)型號,曾陪伴我渡過了許多緊張刺激的旅程。我將手錶戴在喇嘛手上,真心地送給他當禮物。漢娜總是能夠和我心靈相通,當下她也感受到了這一股不可思議的力量。喇嘛稍微向前傾身,然後將雙手放在我們的頭頂上,給我們加持,噶瑪噶舉傳承的加持。只見眼前有一道光,那是無法言喻的一個境界。當我回過神來時發現自己已經站在喇嘛的家門前,腦袋裡一片空白。布塔拉錫米幫忙招了一輛三輪車,我們又搖搖晃晃地回到莎瑪的旅店。

當天晚上,我們只睡了一會兒。像我們這一些大量吸食大麻的人,從來就不是早起的鳥兒。那一天,我們凌晨三點就醒過來了。喜悅的感覺也散了。我們看著彼此,難以置信地問道:「你是不是也做同樣的夢了?」這真是一個糟糕的夜晚!一些令人難堪的舊時回憶紛紛浮現,主要都是童年時期發生過的一些事:像是曾說過的愚蠢謊言、毫無意義的偷竊、讓人惹上麻煩等等我們覺得不光彩的事,統統出現在夢裡了。突然內心裡有一種不好的感覺,覺得喇嘛可能已將我們的心仔細地檢視了一番,並從中發現了許多缺點。

接下來的幾天是痛苦的。那些夢境所留下的印記一直在腦海裡徘徊,當時候我們並無法理解箇中的意義,內心裡焦躁不安的感覺一直延續著。有一天,就在KS喇嘛的店裡,有一名印度裔的占卜者走了進來。他看了看我們的手掌,然後說我們很快會損失一筆金錢,而且在一個月內,我們將會面臨也許長達三至四個月的艱難時期。我們不怎麼理會他所說的話,只是給了他一些零錢,便禮貌地打發他走。

不久後,我們收到一封電報,說我們的朋友在黎巴嫩帶著一個手提箱的哈希什大麻被捕。現在他們一定急需我們的幫忙,而且我們也是時候該回國去了。只是這一次在離去之前,我們所做的事很有可能會應驗那個印度裔占卜者的預言。這一次我們非但沒有過往的謹慎,就連以往在行事前的良好預感,甚至是好運氣也都一併消失不見。當然,後果就可想而知了。

以前,我們總是會將貨品藏在背心裡,或直接綁在身上。不過由於每次都能夠僥倖過關也就導致我們漸漸鬆懈了下來。我們已經運走了一批大麻,現在打算將那一些黃銅製造的大佛首塑像以大麻填滿,運回丹麥。由於警方最近逮捕了我們的不少同伴,所以我們想了一個「妙計」:將十公斤上好品質的大麻分成一包一包五克裝,再附上說明書,然後選擇在繁忙時間空投哥本哈根市中心。我們希望能夠通過這樣的方式加速大麻合法化的程序。有關大麻合法化的事已經說了許久,這一次的行動要是成功了,我們所做的事就不再算是犯法,同時也能將朋友們從監獄裡弄出來。

供貨的這個男人讓我們感覺頗為不舒服,他看起來就像一個十足的罪犯。奈何市面上只有他能夠提供我們這些品質上好的大麻和所需的數量。他們從街角某個賣肉的店找來了稱肉的磅秤來稱貨。店裡的窗戶邊,蒼蠅因為DDT的荼毒正奄奄一息地作出垂死的掙扎。當時漢娜和我正怒火中燒,一心只是想著要報復。當我們往佛首塑像裡塞滿大麻時,心裡不斷想著該如何為已深陷麻煩中的朋友們辯護。我們也將現今主要用於宗教屠宰儀式上的尼泊爾彎刀放入包裹中。漢娜臨時想要把薰香一併放入包裹。薰香的香氣顯然很容易會引起海關人員的注意,我不斷強調這一點,可是不知何故,最後我們還是決定將薰香放入包裹中。我們似乎已將過去所累積的走私經驗全都拋到九霄雲外。我們彷彿已來到人生中的三叉路口,隨即為某一階段的生活畫下句點,然後準備迎接另一個全新的開始。

準備回國在際,我們亦仍惦念著喇嘛,心想即使赴湯蹈火也一定要在離開之前再見他一面。由於喇嘛總是非常忙碌,所以要見上一面也實在不易。不過,就在我們要離開的當天傍晚,布塔拉錫米替我們安排了一次和喇嘛見面的機會,讓我們至少可以向喇嘛道別。這一次我們收斂了先前狂妄的態度,將之前的夢境以及對夢境的詮釋,一一告訴喇嘛。他聽了之後只是哈哈大笑,然後分別給了我們一人一個將摺起的圖案以五色絲線結綁而成的護身符。這一個護身符比布塔拉錫米從祈雨的喇嘛處所獲得的更為精確。洛本傑珠仁波切說,我們必定會在一年之內再次回到尼泊爾,接下來我們也將會面臨各種不同的狀況,不過他會為我們祈禱,讓我們一定要生起信心,不可懷疑。離別之前,他對我們所說的一番話,爾後我們也有另一番體會。他說:「無論發生甚麼事,都將會是最好的安排。」最後,他將一小包經過特別加持的法藥放在我們的手心裡。在修法時,瑜伽士會以諸佛的能量加持這一些細小、形狀不一的草藥顆粒。我後來因為不小心吞下了不少法藥才發現它們的功效。當時身體內的能量突然變得非常強烈,讓我長達好幾個小時都無法彎腰,那種感覺就像是在身體的中央裝上鋼鐵彈簧一樣。

喇嘛再次給我們摸頂加持,我們帶著感動和感恩的心離開。布塔拉錫米也送了我們一份離別的禮物:由不同的音節串連而成的心咒,讓我們能時時憶念喇嘛。我們趕緊將咒語默記於心,然後開車前往機場。

\textbf{第三章}

\textbf{牢獄裡的自由}

我們原本要在黎巴嫩停留的計畫已不再可行。我們幾乎可以肯定如今警方必定已掌握了我們的名字。我們除了必須儘快趕回家讓父母親安心,也得想辦法幫助朋友。當我們抵達最後一個中轉站法蘭克福(Frankfurt)時,我們撥了一通電話,得知第一箱的佛首像已經抵達丹麥。如今,眼前的問題就只剩下我們該如何安全地入境丹麥。

我們匆匆忙忙地趕上直飛哥本哈根的班機,深信好運氣將會一如既往地眷顧我們。在機場,我們不斷重複念誦著咒語,在過海關檢查護照時,恰好跟隨在兩名警察和他們所扣押的兩個格陵蘭人身後,海關人員以為我們也是警察,所以沒檢查護照就直接讓我們過關。

我們的父母親很不開心。他們對我們那自以為高尚的出發點難以苟同,無法相信我們是追求心靈自由的戰士,志於幫助他人擺脫攻擊性和酗酒的行為。在他們眼裡,孩子們是大難臨頭了。有的報章也已經報導警方懷疑我們所涉及的勾當,只是尚未獲得確鑿的證據能將我們逮捕歸案而已。

就在我們抵達哥本哈根後不久,第二箱的佛首塑像也抵達了,而且還被海關所查獲。包裹中的薰香的氣味讓他們起了疑心。他們打開佛首塑像,發現我們原本想要空投撒在哥本哈根的哈希什大麻。由於包裹的收件人不是我們的名字,也不是以我們的名義寄出,因此我們能暫時洗脫嫌疑。只是外頭風聲很緊,周圍的防護牆似乎正在逐漸崩塌。我們身邊的朋友越來越多人遭到圍捕,身上一直配戴著的護身符也突然不見了,怎麼也找不著。此時,另一個不詳的預兆緊接著出現,我們竟然接到一張超速的罰單!這是前所未有的事。我們還未能來得及收拾細軟到朋友家中暫避風頭,警察便已聞風而至。

其實我們原本可以先在南歐避一避風頭,直到風聲沒那麼緊才露面的。然而,接二連三所發生的事似乎都在告訴我們,那是我們必須從中學習的課題。我們事先已經被告知會有事情發生,而且喇嘛也曾經允諾無論發生甚麼事,他都會保護我們。我們覺得這一連串的事情想必也是受到他所影響,因此決定順其自然,讓因果去解決。

警察方面半點也不懈怠。他們蒐集了不少線索,足以構成有力的證據將我們扣留。當時,丹麥政府正在計畫將走私販毒的控訴,從輕罪改為重罪。因此在正式落實此項條例之前,他們需要一個有力的個案例證和一個很好的宣傳。像丹麥這種賦稅高,海岸線又長的北歐小國,走私就是一種傳統,我們把它視為是一項緊張刺激又「紳士」的活動。然而當時有不少外國人也開始插上一腳,他們在丹麥的走私活動所施的手段更是談不上高雅。再說他們亦將像是鴉片等等非常危險的毒品大量引入丹麥。如今濫用這些毒品者大多數來自於社會的較低階層。通過毒品來拓展心靈的崇高時期顯然已經結束,我們的政府誓言要制止濫用毒品,必定是擁有很好的理由才會這麼做。

我和漢娜向來都是盡量以最便宜的方式過生活,認為節約就是自由。然而,即便這樣警方還是找到證據證明我們花在車子和旅行(我們當時的主要花費)上的金錢,遠比我的教學工作所賺取的金錢來得更多。現在問題主要是面對眼前的多項控訴,我們該承認多少而已。某個程度上,我們不得已要和警方合作,否則第二箱佛首塑像的收件人──一名臨盆在即的孕婦大概得在牢獄裡把孩子生下了。此外,我也得盡快將漢娜從拘留室裡弄出來,這一些沒有門把的狹小房子實在不合她所好。因此我暗自決定,只要是警方掌握足夠證據的控訴便認罪。我告訴警方,漢娜對我的交易不甚了解,甚至可說是毫不知情。至於其他人,我只是承認自己認識某個名為喬的阿拉伯人。每次接受盤問時,我都是不斷重複同樣的故事,藉機告訴警方,無論如何我也不會出賣朋友。負責盤問我的兩名警員並不欣賞我的做法,不過基於他們都曾參加二次大戰時的抵抗運動,他們應該非常了解我的立場。另一方面,我深知自己必須儘快讓此案件上庭開審,否則當警方從中東方面獲取更多資料時,我們就真的是大禍臨頭了。我一定要想辦法讓我們的案件受到緊急處理。

就在某個晴朗的早晨,我有意識地將刀子緩緩刺入胸口。我沿著肋骨刺入,這樣才不會造成太大的傷害。痛楚是還好,只是覺得肌肉從肋骨上被撕裂的聲音有那麼一點兒噁心。刀柄露在胸口上,看起來極富戲劇性。有關當局真的以為我試圖自殺。隨後我被送進監獄醫院,這亦加快了案件送審的速度。

在此期間,漢娜和我只是見過一次面,不過我們一直都有保持聯絡。我們會在每天寫給對方的信件裡特別註明心裡生起某些想法時的時間,而有趣的是我們的想法總是彼此相應。負責讀信的警察很快就放棄挑撥我們之間的關係。多年後,當我們把這些信件交給某個著名的心理學家時,對方也表示未曾見過兩個人會如此有默契。

我第一次靜坐是在指導某人如何靜坐的時候。醫院裡同房的病友是一個可憐的傢伙,精神明顯失常,早前卻不明原因地成為漏網之魚,逃過政府的法眼而沒遭到逮捕。他過去是一名業餘的竊賊,擁有非常純熟的幹案技巧。只是他如今已失去了人生的方向。眼看著困惑和一臉沮喪的他,心中不禁為他感到憐憫。

有一天,我對他說:「你需要靜坐。」雖然我也只是從所讀的那本西藏書籍中認識何為「靜坐」,可是我想至少向他展示一下。我模仿圖片和佛像中所展示的靜坐姿勢,坐直了身子。結果我從此次的經驗中,獲得了比預想中更大的收穫。當我將身體和雙腳擺好位置後,不可思議的事情發生了。霎時四周圍的一切似乎都在發光,我感覺自己仿若在空中漂浮了起來,就像完全從軀體中脫離似的。額頭後方逐漸形成一種柔和的力道,猶如一股清涼的風在頭顱下流動。喇嘛、漢娜、我的父母親、朋友這一切美好的人事物似乎就在當下與我同在,內心裡生起了大樂的感受。與我心靈相通的漢娜也強烈地感受到了,於是在信中問道:「當時候你做了甚麼?我想一定是一件很美好的事?」自此,我們每天都會花幾個小時靜坐禪修。

當我再次回到個人的牢房裡後,更是覺得自己似乎是受邀到牢裡學習靜坐的。自從那時候開始,我們並沒有做其他事,只是專注地禪修靜坐。整個過程充滿喜悅,流動於額頭與頭顱頂端之間的一股力道,感覺亦非常美妙。靜坐所產生的能量一度讓我懷疑自己在尼泊爾染上了疾病,不過感覺又出奇地「對味」,我不希望任何人知道這件事。反正這應該是和喇嘛有關,像這般的內在體驗,我們又能對醫生說些甚麼呢?我寧願是自己染上某種疾病,也不願意失去這種美妙的「萬物合一」的體驗。

與此同時,警方不知道從何處找到了我們的護身符。就在某天夜晚,我親身體驗了它們的力量。

我的人生一直都是充滿刺激與挑戰。年幼時的我總覺得樹長得不夠高,長大後又覺得摩托車的速度不夠快。我從來不懂得何為焦慮,更不懂得畏懼。當其他人都在回避危險,允許恐懼和分化在生命中顯現,我卻會縱身躍入其中,並能夠從中感受極大的喜悅。然而,某一天夜晚在牢房裡,我卻因為恐懼(一定是!)的壓迫而醒了過來。當時,我覺得這一定是和脖子上的護身符有關,因此便將它摘下來放到床鋪上方的架子上。心間的壓迫感頓時消散了,而我又沉沉睡去。夜裡的牢房漆黑一片,就連開關燈等那麼簡單的事,我們也有專人代勞。翌日早晨,我發現繞綁著護身符的其中一條絲線鬆脫了,改變了它原來的形狀。我將它還原。那次之後我都會特別留意護身符的外狀。我總是覺得應該將它們好好收進小袋子裡才對。

此時,在較早一次的旅途中一直隨身攜帶的《西藏瑜伽與秘密的教義》一書也被送到牢裡來,這一次我們決定要用心鑽研書中的內容。漢娜和我每天都會互通信件,一頁一頁地探討書中更高的禪修,並且和彼此分享各自在靜坐禪修上的美妙經驗。我們成功地運用這些更高的禪修方便與技巧。通常若想要學習這些技巧,事前都必須先獲得具德上師的灌頂和多年的準備作為基礎。當時我們能夠成功學習這些技巧並不是因為開智慧,只是純粹幸運而已。而且當時我們總是覺得這些修法似曾相識。似乎早已懂得。

先前我假裝自殺加快了事情的發展。我們的案件在六周內便交由一名法官去審理。法官名為「懷特」(White,中文有「白色」之意),我們認為這是一個好兆頭。當我在庭上為自己辯解時,仿若處於一種出神的狀態。我聽見自己說話的聲音,卻也同時覺得說話的人並不是我自己。我告訴他們說,我們相信意識的拓展,大麻使到我們不那麼具有攻擊性,對他人也能夠抱著更開明的態度。我也告訴他們有關療癒以及在尼泊爾的喇嘛。對當時的我而言,毒品意味著療癒的能量,以及心靈的解脫!我天真地將二者混為一談,認為「果甜樹則良」,殊不知並非如此。結果,整個局勢突然變得對我們非常有利,就連地區檢察官也為我們辯護。結案時,我們獲判最輕的刑罰:漢娜當庭釋放,而我則需服刑四個月。若將拘留的時間加算在內,估計夏至時我便自由了。

然而,警方高層對於這樣的審判結果感到相當不悅。他們對我們所幹的好事,一直都是了然於心。不管是單獨還是和朋友聯手,我們總是妨礙他們辦公,壞了他們不少好事,而且還幫助/坦護那一些惹了麻煩的人。我們幫忙掩蓋他人行蹤,幫忙保護同伴們和貨物的安全,甚至報復那些太過火的警員。我們獲得法庭輕判引起了許多人的疾憤,他們也立即提出了上訴。其實他們是希望能夠殺一儆百,以儆效尤。此時他們也應該掌握了不少來自中東更有利的新證據。眼看這下真的是要大禍臨頭了。然而高庭法院的檔期已滿,直到九月之前都不可能提出上訴。現在正值四月初旬,他們甚至有意將我扣押至那個時候。

漢娜聽到有關消息後感到極為厭惡,而我則怒不可遏。若我們想要脫身,就只有兩種可能性:喇嘛幫助我們度過難關(實際上他是如何辦到的,我們實在是毫無頭緒);或者是我們試試書中最後一種修法,也就是藏文稱作為「波瓦」(Phowa)的遷識法。通過波瓦遷識法,行者的心識將能夠有意識的離開肉體。

這是金剛乘的其中一種修持法門。遷識法中的「心風奪舍」(Dronjug)修法,也是西藏偉大的瑜伽士瑪爾巴最喜歡的一種法門。這一種修法能夠讓心識從一個軀體遷往另一個軀體。許多人把此修法當作玩笑,因此導致修法逐漸失傳。一開始時,我們只是粗略地嘗試書中的禪修方法,並不了解這些修法所蘊含的真正意義。我們以為這一種修法能夠讓我們在扣押期間以冬眠的形式使意識離開身體,自由地四處漂移。然而,此法教真正是為了往生「淨土」而設。它和眾生在死亡之際的意識轉移有關。

讀者可能會問,我們曾經在優秀的大學裡受過多年教育,甚至也有教導一些非常實際的科目如哲學和語言,像我們這樣的人怎麼會相信這些事呢?鑒於我們過往的經驗,漢娜和我都抱有「嬉皮」一代相同的想法。我們深信心識的唯一界限就是無明和習氣,心識的真如本性是無限的。後來我們對這一點更是具有一份不可動搖的確信。當時的純理論在爾後亦成為了實際的修持。

就在我們感到進退兩難之際,我們給喇嘛傑珠寫了一封信,告訴他所發生的事。三周後,上訴不知明因地被取消。這大概也是丹麥司法院史前無例的一件事。

總檢察長和輯毒隊成員非常生氣,說司法部裡一定是有人瘋了,才會意外地撤回上訴。事實上,身在尼泊爾的喇嘛傑珠在這五天裡一直把自己關在房間裡閉關修持夢瑜伽。通過此修法,心識將以某個所選擇的佛本尊形相離開軀體,四處漂移。這也難怪我們那麼愛他呀!

在單人牢房內度過了三個月的監禁後,我被轉移至大牢房和其他囚友一起度過最後數周的刑期,日子單調又平淡。然而這也是觀察老愛打架滋事者、騙子和酒鬼們的生活的一個大好時機,內心裡很自然地就會對他們感到悲憫。他們對周圍的事物非常執著,認為這一切都是真實的,殊不知正是這一份執著令痛苦變得更為真實。

刑滿獲釋當天正值夏至,家人朋友為我重獲自由而舉杯歡慶。我非常開心能夠和漢娜,以及摯愛的父母和朋友團聚。我們決定以後都不再分開,這亦意味著我們該為走私販毒的日子畫下句點。

我從前幾個月的經驗中得到一個結論。朋友們來探監時經常會偷偷塞一些哈希什大麻給我。當我在囚室裡偷偷吸食這些大麻時才忽然發現它們對靜坐禪修一點益處也沒有,反而會使到禪修的體驗變得膚淺和微弱。那一種恆常的寧靜、專注和喜悅,似乎在吸了一口大麻後便消散不見,反之擁攏而至的是混亂的思緒。那一種感覺就像是蓬勃的朝氣突然消失不見般。只是在我為了獲釋而歡慶的數周後,這種省悟在與友人聚在一起集體抽大麻時又忘失得一乾二淨了。我們又開始吸毒,繼續作迷幻異境裡的嚮導。

好在我們與喇嘛傑珠的連繫無論如何也不會退失。禪修時所感受到的那一股柔和的力道仍在,似乎是應許著要在任何時候把我們更深入的內在一層一層地開放起來。每件事物都有著多層次,無盡延伸的意涵。我們的世界是完整的,我們能夠感覺到成長。幾個月後,久等的指示終於出現了。我們以前登山時的哥兒們(來自蘇格蘭的紅髮亞倫)來到丹麥。與他同行的美國友人比爾要找一台二手的福斯小巴,他想要開著小巴前往尼泊爾。之後我們買了小巴,收拾好行囊,浩浩蕩蕩隨著大夥兒一起出發。

\textbf{第四章}

\textbf{陸路尼泊爾}

沿途路經氣候溫暖的國度,一切看起來並沒有多大的變化。土耳其人開車的技術和態度讓人不敢恭維,他們尤其喜歡在夜間熄了燈將卡車停泊在馬路中央,所以一路上每隔幾英哩就可看見一台被撞得稀巴爛的卡車。我們也以同樣的方式駕駛,甚至教會他們一件事:我們可不是省油的燈!土耳其人吃的食物很健康,擁有豐富的蛋白質。當我們越是往東部前進,所遇見的成年人就感覺越是壓抑,甚至擁有暴力的傾向。那些剃光頭、一臉老成的孩童們則不斷向駛過的車子丟石頭。只見男人們成群結黨,無所事事地聚在一起,頭戴著帽子,呷著一杯又一杯甜中帶澀的茶。他們看起來非常沮喪,而女人的蹤影則無處可見。

伊朗變得較以往更富裕了。我們第一次穿越此國境時,只記得中央公路顛簸難行。如今沿途經過里海(Caspian
Sea)的北方公路已將近建好。小巴行駛在柏油路上的感覺很棒,可是除此之外這一個國家便沒有其他能讓人感到振奮的事了,當地的居民也並不是那麼友善。我們常常不得已要強迫他們在某程度上以文明的方式對待彼此。

阿富汗這一個需要對人性恢復信心的國度仍然是旅途中的一片綠洲。當時並沒有所謂的塔利班政權,當地平靜又沉默寡言的人民感覺還不賴。他們屬於原本的單純的生活,不需要耍小把戲。他們依然將文化保存得完好無缺,滿足於傳統的價值觀,不會急著想要利用這些傳統的價值去換取現代社會無法確定的自由。他們的心胸猶如沙漠般浩瀚廣闊。只是當地越來越多的年輕人為了混一口飯吃當上了騙子。像「嘿,先生!」這一種拉攏人的叫喚聲更是不絕於耳。阿富汗讓我想起了六十年代初期的摩洛哥,我親眼看著這一個獨立又自給自足的國度一年比一年衰退,一年比一年變得更商業化。阿富汗還不至於發展到這種地步,當地的人民依然可敬。至少走在阿富汗的大街上並不會受到太大的干擾,他們仍然會把你視為是「賓客」,而非垂涎欲滴的「獵物」。不像在土耳其或伊朗,這裡的人不會死纏著人不放,或者老是盯著西方女性看。當地的婦女總是把自己裹得嚴嚴實實,活像是帳篷和木乃伊似的。

我們在赫拉特(Herat)(阿富汗首個真正的城鎮)曾經擁有過許多有趣的回憶。對我們來說,赫拉特是我們真正踏入「東方」的一個起始點。一九六六年,「先知哈拉爾」、安德斯與我,我們三個丹麥人偷偷攀上了位於新舊城區之間的一座城堡牆壁,嚇壞了負責看守著這似乎是世界最古老的大砲的那一些士兵們。如今哈拉爾追隨一名印度教上師,而未曾斷絕過毒品的安德斯,不久之後死在卡拉奇某處的糞堆上。沒有人知道他的真正死因,只是猜測他是因為患上痢疾未治的原故。我們最後一次見面時,他整個人已經變得神經兮兮,老是幻想四處都是邪惡的吉普賽女郎。毒品真是害他不淺!

距離第一次向東自駕遊來到阿富汗迄今,這裡並沒有多大的改變。當時我們與摔傷了背骨的克里斯蒂安同行。早在六十年代初期,克里斯蒂安便已經去過阿富汗,更在較早之前幫忙打開前往摩洛哥的道路。阿富汗的某些部落有一種習俗,人們抽大麻向死者表示敬意。克里斯蒂安時常志願這麼做,因此在傍晚時分很常見他在大街上裸奔。正因為如此,他非常熟悉赫拉特和喀布爾的監獄。他曾經在卡拉奇的監獄裡待過一段時間,除了每天都遭到典獄官痛毆,他還不幸患上肝炎。有一天,克里斯蒂安在抽了大麻後因為迷幻作用以為自己的肝臟爆開了,因此為了要儘快了結痛苦,他竟然奪窗而出。這麼一個縱身一躍令他自此癱瘓。在坎達哈、喀布爾以及兩城之間的一些小鎮上,一些警察是克里斯蒂安的舊相識,他們曾經前來打聲招呼,還問了他數次:「哈囉,克里斯蒂安先生!現在腦袋還好吧?」

我們在赫拉特下榻的旅店不但有臭蟲,而且水龍頭打開了也不會有有半滴水流出來。旅店裡的臭蟲和我擁有相同的品味,就是特別喜歡漢娜。像這種小旅店,後室都是「阿富汗可樂廠」的秘密基地。「阿富汗可樂」是指某種有毒的綠色或紫色混合物,一般裝在用過的可樂瓶裡兜售。大喝幾口阿富汗可樂就足以在十分鐘內讓一個體格健壯的歐洲人狂奔洗手間,而且往往在這種情況下還老是碰到排長龍的慘況,事後還得跑向附近的藥房排隊等著買抗生素呢!當地的藥房能治好的不單止是腹瀉不止的問題。他們也兜售最好的德國嗎啡,而且不需要醫生開處方就可購得。當地的藥劑師一定覺得很詫異,怎麼會有那麼多年輕的老外旅者需要這種嗎啡。可是他們都極具生意頭腦,很快便學會將藥物的價格抬高。當地的大麻屬於「超強勁」的類型。柔軟、呈暗綠色,看起來和黎巴嫩、尼泊爾的大麻很不同。這些大麻來自北方,靠近俄羅斯邊境,經常讓吸食者陷入昏睡的狀態。在收割期間,農民會批上麂皮大衣在田中奔跑,此時所刮落的黏稠樹脂品質最好,當時可賣上一公斤十五美元的高價。

當地有一種形狀偏大的片麵包對胃最安全,而且無論上哪兒買的都好吃,迷你雞蛋也是。那些放掉水分後的酸奶也不錯。據說喝酸奶是當地人為了消滅結核病才延伸出來的文化,這種病可毀了當地大部分的牛隻。為了健康著想,我說還是吃麵包、喝紅茶最為保險。紅茶到處都有在賣,茶販用來裝茶的老舊茶壺上補丁處處,像馬賽克藝術。這些紅茶非常便宜,一杯也只售兩分錢,卻出奇有效地能消暑解渴。

赫拉特的西南公路是由俄羅斯人所建造,鮮少有人使用。偶爾會看見一台由美國人捐贈的老舊校車在這段公路上行駛,車內擠滿了人和動物,一不小心可能還會從窗邊掉下來;不然就是剛從英國打工賺了錢,如今衣錦還鄉的印度人或巴基斯坦人開著車子回鄉。有時候,我們開車開到一半必須停下來讓一群駱駝悠哉地過馬路。除此之外,在這條道路上的大多時候就只有一片無際的沙漠和唯美的風景與己相伴。

在坎達哈阿里的酒店,我們又碰到一些來自丹麥的朋友。我們在酒店裡連續狂歡了幾天,那兒可說是當時最潮的集合地點了。阿里如今坐擁另一家酒店,可是他的兄弟卻只能在一旁生悶氣。我們聽說他剛花了一千塊買了一個新老婆回家,可是對方卻拒絕和他行房。有人說他們曾見過艾克。艾克是一個充滿傳奇的歌手和作者。一年前,我們曾經一起派對。一九六七年十月,他將某一種稱為氫溴化物(Hydrobromide)[醫學上稱為氫溴酸右甲嗎喃(Romilar)]的毒品帶到丹麥,亦為丹麥人帶來了一個幻異迷離的秋天。如今他已經無法從巴基斯坦過境印度。早前他過量吸毒,隨後選擇結束了自己的性命。他在遺書裡寫說:「是我自己造成了這一切,錯就錯在內心裡負面消極的能量。」臨終之前,他最後所看見的應該是一張佛陀的照片,我們為此感到非常開心。我們知道這將有助於他投生到更好的地方去。

我們對路經喀布爾和開伯爾山口(Khyber
Pass)的路線已經非常熟悉。因此這一次有意嘗試經由奎達(Quetta)入境巴基斯坦的南部路線。當我們接近巴基斯坦邊境時,大夥突然建議要以我們和艾克最後一次相聚時的方式,也就是注射嗎啡去悼念他。其實這一個想法並沒有讓我感到太興奮。經過了六十年代初期和中期,分別在摩洛哥和倫敦患上肝炎的兩次經歷後,對於注射嗎啡這種事,我的心裡可說是五味雜陳。我曾試過有一次在注射嗎啡後(當時我身處在完全不可能發生地震的丹麥)突然覺得天旋地轉,接著牆上的書架就在違反了一切自然定律的情況下掉了下來,而且架角不偏不倚地擊中我的頭顱。雖然這一次心裡同樣覺得不妥,不過似乎也不可能說「不」,因為這樣會壞了大家的興致。這一次的注射並沒有讓我感覺像以往般亢奮,反而覺得無比的沉重。我想也許是護法們不怎麼高興,所以暗自決定要再謹慎一點兒。

抵達巴基斯坦邊境之前,比爾和亞倫想要攀爬一座怪狀林立的山脈。這些山脈看起來像是成堆的大石頭,我也只是曾經在羅德西亞(Rhodesia)看過類似的山形結構。這一次我沒有嘗試呈英雄要爬在前頭,只是尾隨在後。我老是覺得會有不好的事情發生。果然不出所料。攀爬在岩壁最前頭的比爾踩落了一顆大石頭。我讓身體儘量貼近山壁,試圖躲避滾落的石頭,只是落石還是擊中我的小腿。一直到德里,我幾乎無法走動。看來最好的教訓往往是伴隨著「痛苦」和「流血」。這也是我最後一次注射毒品。

這一條穿越巴基斯坦國境的南部路線相當顛簸難行。從土耳其開始,只見當地人都將路上圓形的坑窪弄成正方形,而且就那樣丟著不管。路上的卡車和巴士魯莽竄行,在那兒開車要使出渾身解數才不至於被擠出路外。當地的男性長得像歐洲人,高大魁梧得像座「城堡」;他們身上扛著國家制的來福槍、手槍、刀劍,掛著彈藥筒四處走動。偶而會有幾名婦女靜悄悄路過,她們仿若頂著帳棚在走路,僅僅透過層層的布紗詭秘地看世界。我們看見他們身上的武器都有標上「Made
in
Germany」(德國製造)的標誌,不過是反過來的字母「G」。儘管如此,它們看起來一樣的危險。

我們在拉哈爾(Lahore)花了好幾個小時辦理前往印度時會經過的最後一個路段的許可證。當地的官員煞是逗趣,他們都穿著制服短褲,頭上的帽子裝飾著又硬又挺的羽毛。他們就是在這樣的一副裝扮下來回奔忙。英國人令殖民地的人民穿上這麼誇張逗趣的服飾,不知道是不是在耍弄他們呢!

在印度邊境,海關人員安排了一位出了名擁有通靈能力的婦女在那兒把守,大家都非常怕她。甜美親切的她負責沒收人們身上所攜帶的違法盧比或其他走私貨物。她的魅力,讓我們男人難以招架,乖乖地就把身上的一小片大麻交出來。我們其實可以讓漢娜去應付她,只是想說這一小片大麻也許能夠轉移她的視線,保全我們攜帶入境的非法金錢。

印度並沒有多大的改變。儘管國家已經實施了緊急法案,可是當地的人民似乎不受影響,他們仍然一派的吵鬧、友善。先前途經的數個穆斯林國家,我們每到一站都必須把人推開,現在來到了印度的感覺真好。從伊斯坦堡開始,群眾一發生騷動,遭殃的就是人們的車輛。即使在印度,若沒必要(比方說:要運輸很多器材等)也建議別自己開車前往。

汽油在印度很昂貴。在那裡擁有一台轎車會讓你成為眾人的焦點,卻也同時將你和他人的距離拉遠,所以還是利用那些搖搖晃晃,永遠超載和遲到的巴士和火車較好。在這裡能夠看到各種最真實的「人性」的一面,這些都不容易在西方國家裡看得見。此外,若將車子停泊在一旁後離去,稍後當你再回到泊車地點時,車子往往就只剩下外殼而已。邊界地區經常發生這種事。當你一臉驚愕看著眼前遺留下來的車輛「殘骸」時,就會有幾個友好的陌生人(通常是警察)悄悄出現,然後對你說:「我們恰巧有一副可能適合你的引擎、輪胎和車頭燈喔\ldots{}。」之後,你便能花錢將車子的另一半再買回來了。

我們在德里很快便申請到了尼泊爾的簽證。不過由於比爾有事待辦,因此我們也多留了幾天。我們趁此機會去探望了一些以前在哥本哈根的老朋友。他們很早就戒了毒,現在旅居印度,當了某個斯瓦米(印度教宗師)的弟子。這位印度教宗師是最早在丹麥成名,而且還在丹麥建立了一所瑜伽學校的人。我們想前往拜訪,也順道看看他教些甚麼。就如先前所說的一樣,我和漢娜沒有對印度教徒太感興趣。他們友善得讓人無所適從,控制欲又強,也十分孤立,這一名宗教師尤其極端。他聲稱禁慾是求得解脫的唯一方法。對於漢娜和我來說,這簡直就是一派胡言!像這麼一個將眾生與喜悅、超凡和無限等美好體驗連接在一起的東西,怎麼可能會不好呢?我們知道藏傳的禪修法門具有空樂不二的雙修法。他們以性愛的喜悅狀態達到迅速的證悟,然而在這種主要的教法中,上師並不容易找到堪受法教者。

當這位斯瓦米進入房裡後,便開始向這兩組身著白衣的男女講示教義,然而他卻省略了禁慾主義不談。這一次的主題和對我們來說非常重要的療癒有關。當時我們對於過去所曾示現的種種療癒奇蹟仍然深深感恩;這些奇蹟為我們打開了一個全新的世界。然而斯瓦米卻有另一套想法。他說,治療他人反而會令自己發瘋,因此應該讓人們承擔自己的業力。他依據個人的經驗舉出了幾個例子作為論證,他還甚至說出了哥本哈根好幾個非常出名,如今卻都入住了精神病院的治療師的名字。

由於「業力」(Karma)一詞經常出現在此書中,西方國家也許不太熟悉這一個字眼,我在此簡單說明一下。對於證悟者而言,業力就猶如虛無幻化的遊戲而已;可是對凡夫俗子而言,它就像是必受因果制裁的桎梏。因為無明的緣故,業力束縛著眾生,統一了整個娑婆世界。一個人所體驗的業因果報是此生以及過去無數世的生命所累積的潛意識印記。如果沒有獲得清靜或轉化,這些印記將會依據各自的力量而成熟。結合了內、外各種因緣條件,它們造就了當下、臨終之際以及來生的種種際遇。就像是一場深層的夢境,當眾生犯下過失時,別業和共業將他們緊緊地束縛,當眾生感到「我」與眾生是一體的時候,因果業力就因此鬆綁了。然而,直到證悟之前──也就是認識自心本性乃為明空和無限的功德時,才能夠超越因果業力的法則。通過大乘佛法的種種方便,具大慈悲者將能夠攝收和改善他人的惡業。他們所消除的不僅是惡業所成熟的果報(痛苦),也包括使到眾生生起二元對立分別心的究竟成因「無明」。

當時我們並不太了解「業力」,因此斯瓦米所說的話讓我們感到非常不安。我們知道他也是一位智者。我突然站起來,說:「如果生病是那個人的業報,那麼我的業報就是去幫助他們。你這冷酷無情的人!」斯瓦米只是大笑,然後說:「現在坐下來靜坐。」在接下來的一個小時,我們非常專注地靜坐禪修。之後我們再次碰面,氣氛也不錯,只是我們都會禮貌性地避重就輕,避免交換任何意見。我們之間的觀點差異太大了!對於療癒,我們需要一個肯定,因此請示了喇嘛。喇嘛告訴我們說:「盡自己的能力去利益他人,別管自己。」這樣的教誨比較容易明白,也相對地比較適合我們。

最令我們感到驚訝的是,我們當中對靈性最不感興致的比爾,竟然堅持要前往菩提迦耶。我們不太樂意,因為從地圖上看來,菩提迦耶距離尼泊爾非常遠,而且路況很差。我和漢娜一心想著要見喇嘛傑珠,而亞倫則暫時受夠了印度。不過,車子是比爾的,更何況他是我們的朋友,單憑這兩點一切便已有了定奪。我們幾乎不怎麼聽說過有關菩提迦耶這一個地方,抵步後,我們開始對她具有了更深入的了解。

菩提迦耶位於比哈省(Bihar)迦耶市近郊。這一個坐落於印度北部的小鎮就是當年佛陀悟道成佛的地方。據說這一個遍滿了殊勝加持力的地方,是諸佛達到最圓滿證悟之處,也就是諸佛開展漸次增上悟境,達證不二境界,並圓滿了無上正等正覺的聖地。兩千五百五十年前,釋迦牟尼佛降世並說法四十五年,早在祂之前便已經有三尊佛降世。直到劫數完結之前,一共會有一千尊佛降世,救度無量有情眾生。

在菩提迦耶可見佛教各個宗派的寺院,雖然我們一心掛念著喇嘛傑珠和尼泊爾,卻仍對菩提迦耶豐富的面貌感到嘆為觀止。即使身在菩提迦耶,我們依舊被西藏人所吸引。我們非常幸運能夠親見度巴喇嘛(Dukpa
Lama)的轉世,並且領受他的加持。上一世度巴喇嘛正是喇嘛傑珠的老師。此外,我們也非常幸運能夠親見拓殿耶喜仁波切(Thubten
Yeshe Rinpoche)與拓殿左巴仁波切(Thubten Zopa
Rinpoche)。兩位仁波切之後非常積極地在西方國家推動弘法事業。聽聞拓殿耶喜仁波切的上一世乃為女尼,因此非常了解女人。當我們來到藏傳寺院時,恰好碰到令仁波切(Ling
Rinpoche
)正在傳授灌頂。令仁波切是第十四世達賴喇嘛的兩位主要上師之一。當時參加灌頂的人都分獲一盧比,由於這些金錢來自於生活也不算富裕的人們,因此讓我們感到既感動又尷尬。當時我們也分獲一塊肉,可是作為素食主義者一年了,這一塊肉讓我們有點反感。當時的我們因為無知,所以連忙將那一盧比給了乞丐,又將那一塊肉給了一條狗。一直到今天我們才知道,在灌頂時無論分獲甚麼都至少要吃一點,然後精神上接受它。儘管如此,我們還是感覺到自己領受了上師的加持。引頸長盼了三天後,我們終於驅車前往尼泊爾。

\textbf{第五章}

\textbf{黑寶冠喇嘛}

一九六九年十二月廿二日,我們抵達加德滿都。當我們開著小巴大搖大擺地進城時,只見我們的「老朋友」們站滿了新路(New
Road)兩旁。他們向我們喊著:「噶瑪巴來了!噶瑪巴來了!」原來噶瑪巴也在幾分鐘前抵達加德滿都。噶瑪巴是西藏最偉大的禪修大師,也是第一個轉世的瑜伽士。十三年前,我們第一次在加德滿都親見噶瑪巴。

西元一一一Ο年,噶瑪巴是西藏第一個自在轉世的喇嘛,他能夠在每次的投生轉世中自我認證。他也為藏傳佛教的其他傳承尋找轉世者,其中第一世達賴喇嘛便是第四世噶瑪巴弟子的學生。一九五九年,噶瑪巴帶領數百人躲開中共士兵,翻越喜馬拉雅山口逃離西藏,途中沒有任何人傷亡。自此,噶瑪巴便一直竭力在錫金(Sikkim)和不丹(Bhutan)建立道場,當時候這些國家都仍未對外開放。多年後,如今我們終於有機會在這裡親見噶瑪巴。我們是直到最近才發現,長久以來因緣際會已使事情的發展巧妙地各就其位,一層層地串聯出整體的畫面。而我們和噶瑪巴多生多世的師生法緣,恰逢時空的湊合,又再次圓滿地匯合成一了。年幼時不斷出現的深層夢境,在夢裡起伏不平的山巒間奮力擊退中國的士兵,拼死命地保護普通百姓和僧人的性命等種種情境,甚至漢娜自幼很隨興地會像西藏人般輕輕吟唱和起舞,如今這一切都說得通了。

只是當時的我們,並不曉得此種種夢兆似乎與前生的記憶有關。我們只是一心想要找到喇嘛傑珠,對於其他事並不關心。我們每天開車往返喇嘛位於馬哈拉亞坤(Maharajgunj)的住家數次,可是每次都是失望而歸。他們總是說喇嘛和噶瑪巴在一起。我們有許多事要告訴喇嘛,也想要好好地感謝他,不過由於苦無機會私下拜訪,於是便決定前往斯瓦彥布大佛塔。這裡是噶瑪巴居住的地方,喇嘛傑珠當然也會在那裡。

我們來到山腳下看見潮湧的人群就知道山上一定有甚麼特別的事正在進行著。我們從未見過那麼多西藏人同時出動,他們穿上了自家最好的傳統服飾,站在沿路的梯階上。他們不約而同朝山頂的方向仰望,臉上難掩興奮喜悅之情。他們每個人雙手合十於胸前,感恩與虔誠之心流露無遺。山坡上古老的西藏號角響起,聲聲迴旋。忽然之間所有人開始沿著階梯拾級而上,路經剛粉刷過的大佛聖像,朝著山頂上寺院的方向移動。受到號角聲所牽引,我牽起漢娜的手,越過緩緩移動的當地人,直奔山頂。當我們登上頂端後,首先映入眼簾的是大佛塔和金剛杵(表徵如金剛般堅固的證悟),此時號角聲已改成了像雙簧管般較為尖細的聲響。佛塔右側的院子裡擠滿了藏民,他們熱切地朝寺院入口處的方向望去。在半明半暗之中,寺院門口有一名身材魁梧,身著紅、黃外袍的男子坐在一個盒狀的席位上,手持著某個黑色的物體在頭上。由於陽光刺眼,一開始時我們幾乎看不見是甚麼。過了幾分鐘,他將它放下,然後裝入某個盒子內,而此時入口處的鐵柵卻迅速被拉上。所有人像遭雷擊般站著不動片刻後又再次開始移動。他們全都擠向了左側的小門,想要往寶座上的男子靠近。現場的人們像失了控般開始推撞擠壓,小孩開始尖叫,而我則乍然發現自己已經開始著手幫忙維持秩序,攔著年輕體狀者,讓老人和小孩先行通過,保護那一些在蜂擁的人群中容易遭到踩踏,較為「弱勢」的一群。後來這種維持秩序的工作似乎也成為了我的「職責」。像這樣的工作真是一種體力活,對於大多數屬「輕量級」的尼泊爾人以及營養不良的西藏人來說確實不容易。

大概是一個小時後,大多數的人也已經領受過加持,我的任務也結束了。我和漢娜加入最後的隊伍,隨著人潮的推擠,沿著昏暗的短廊緩緩前進。突然之間,就在號角聲下,我們來到了噶瑪巴的面前。當他將手放在我們的頭上給予我們摸頂加持時,我們抬頭仰望。噶瑪巴突然變得像天空般不可思議的浩瀚巨大,整個人呈現金黃色,並且閃耀著光芒。人們推擠著我們移動腳步。噶瑪巴所示現的巨大威力給予了我們極大的震撼,身體不由自主地顫抖。不知覺間,我們來到了身穿紅袍的僧人面前,他們將一條繩子圍在我們的脖子上。我們回到院子前,身體緊靠著鐵柵,思緒遠了。我們只是看見眼前莊嚴的金身大佛,正在給予隊伍中的最後一個人加持,我們知道他是任誰也無法忘記的圓滿證量。噶瑪巴的加持,已經融入我們的生命裡。

在噶瑪巴的隨行喇嘛當中,就以不丹籍醫生吉梅澤旺(Jigme
Tsewang)的英語說得最好。他總是看起來很高興的樣子,是一個在處事上面面俱到的一個人。由於他經常嚼檳榔,因此嘴唇總是紅通通的。當時他替我們翻譯,成為了噶瑪巴和我們中間的橋梁。布塔拉錫米也非常照顧我們。我們那「不斷衰退的經濟狀況」總是讓她很擔心。對於那些欲向我們兜售物品的西藏老婦人,她知道我們總是毫無招架之力,永遠無法說「不」。對於急需錢的人,我們同樣無法和他們討價還價。為了替我們儘量省錢,權宜之下她在老城區的朋友家裡替我們找到了一間小小的廉價出租房。這樣一來至少幫我們每天省下了半美元的住宿費!

我們的行李幾乎是原封不動地擱在房裡。我們從清晨起就一直待在斯瓦彥布,直到深夜大家都睡了才甘願離去。我們必須多親近噶瑪巴,噶瑪巴也表示沒問題。我們是他的第一個西方學生。過了一段時間後,不丹醫生替我們安排與噶瑪巴私下會面,這也是我們向噶瑪巴作出「最殊勝的供養」的一個機會。為了表示我們和噶瑪巴之間的法緣,我們將一個強力的丹麥馬蹄形磁鐵和一小包一千微克、最純淨的LSD(註:香港人俗稱「弗得」的致幻劑)供養給噶瑪巴。直到那個時候,這些純淨的LSD是讓我們接入本性實相和喜樂最有力的工具。噶瑪巴先是端詳我們一會兒,然後請我們吃糖果,笑了。他還讓我們用藏文跟著他說一遍表徵五佛智慧的顏色。離去之前,他將手放在我們的頭頂上,給予我們似無止盡的加持。

當天,滿月皎潔光亮,高掛天際。不丹醫生給我們送來了一個小紙包,說:「裡面裝有所有轉世噶瑪巴的頭髮。」他還說:「我從來不知道有這種東西,法王說要送給你們。」我們心裡非常激動,也非常感動。我們收下了小紙包,回程時我將它放入軍綠色襯衫左邊的口袋裡。走著走著,我忽然感覺到口袋下方的皮膚開始變暖,胸口上有刺痛的感覺。後來,刺痛的感覺漸漸加重,就像有某種東西熾熱地燃燒入體內似的。我掏出小紙包,然後放入胸前右側的口袋裡。此時,右側口袋的皮膚下也出現了同樣燃燒的刺痛感,只是沒那麼強烈。

每天晚上,我們和屋主都會上演同樣的戲碼。這棟房子建成的時候,尼泊爾仍未有現代的門鎖。我們每晚回家的時間,就尼泊爾人的標準來看算是「夜歸」,回到家時大門已經從屋內閂住。不知道為何,我總是得要幾乎把門砸了才會有人來給我們開門。更奇怪的是,那戶人家似乎對這樣的騷動不以為意。可是若他們能夠忍受的話,我們這些寄人籬下的遊客又能說甚麼呢?當天晚上,屋主被我們吵醒了兩次。另一次是我在脫掉口袋裡裝有噶瑪巴頭髮小紙包的襯衫時,因為劇烈的痛楚而大叫了一聲。

翌日早晨,噶瑪巴在博德納大佛塔傳授灌頂。博德納大佛塔可說是世界最大的佛塔。早前在尼泊爾,我們只是見過欽尼喇嘛,不然就是和朋友們派對狂歡。如今我們知道原來也有好幾位轉世得道高僧居於此地,因此決定要前往拜訪。

當天,事前並沒有人公佈噶瑪巴會在甚麼地方;他們從來都不說。只是我們很快便學會了跟隨街上的藏民隊伍。只見他們誠心專注地持誦著咒語,顯然是正在準備自己要迎接某個很重要的東西。這一群民眾所到之處,就是可以找到噶瑪巴的地方。我們在人群輕輕地推擠下緩緩前進,也許是噶瑪巴的加持,又也許是因為我們實在是太渴望親近噶瑪巴的緣故,一抬頭才發現目的地就在前方。

這天,噶瑪巴將三世諸佛證悟的能量傳遞下來。他手持著黑寶冠頂戴在頭上,通過「見即解脫」的方式,讓眾生了悟自心的真實本性。同時,噶瑪巴通過其甚深的禪定力將我們未覺的佛性和其證量銜接在一起。我們和噶瑪巴第一次見面時,他正在進行黑寶冠儀式。儘管在金剛乘佛教裡有許多強而有力的法門,此修法卻特別殊勝不共。數世紀以來,它讓見者立刻明心見性,直至了悟成佛。這也是噶瑪噶舉傳承有別於其他宗派特殊與不共的特徵。

那些年在加德滿都,噶瑪巴每天都會進行黑寶冠儀式。這一種儀式讓見聞者漸次增上地進入甚深明覺與內觀的殊勝禪定。當噶瑪巴和我們的心合而為一時,他已在我們的八識田中種下了未來得證佛果的種子。

每一次的儀軌都會產生不同的效驗和感應,並沒有「特定」的反應模式可被期待。就我個人而言,我的心性體驗是全然卻又充滿戲劇性的:跳傘、賽摩托車、「愛」,包括之前服迷幻藥等等,其他人也許會選擇較細微的體驗方式。雖然這種極端的方式能夠讓人見證心性的圓滿無缺,可是不一定比一般人隨流而來的方式來得更為優越或殊勝。就像在「愛」和禪定能量傳遞的過程中,行者不可存有任何期待。對於結果的期待只會阻礙自心本性的全然展現。一切的顯現都是殊妙的,即使沒有直接的體悟也沒關係。銘印已存於心中,結果遲早不求自得。就我而言,我需要強劇的「藥彈」來粉碎我那堅固的我執,而噶瑪巴就擁有充足的「藥彈」。這也是為何噶瑪巴的存在至今仍然形影不離,為我們帶來源源不絕的意義和喜樂。有的時候在儀軌中,我那凡俗的世界瓦解了,橙光的明淨境界豁然顯現,腦海裡只留下黑寶冠明晰的淨觀。其他時候則有一股能量之流在身體中心往上直衝,強烈得讓我失去意識,處於茫然的狀態長達數個小時。佛陀曾在兩部經典內授記,凡見過黑寶冠者,證悟的烙印將永存不滅,在中陰階段成就。處於中陰階段時,心識不再受到感官所限制,因此便能夠與寶冠的能量場全然匯合。我們將能夠了悟自心的真如本性,證入超越時空的解脫境界。令人驚嘆的是,當曾與噶瑪巴非常親近的我的父母,以及其他曾經參加過黑寶冠儀軌的人往生時,這些不可思議的能量頓時活躍了起來。他們都是在伴有不少明確瑞相的情況下往生。

在古老西藏,許多大喇嘛在儀式結束後只是象徵式地給予少數的人摸頂加持。而今,以難民的身分處於他國,這種傳統也隨著改變了,不過顯然沒有人知道。正因為如此,再加上他們當中有的人認為能夠作為最前幾個領受加持的人就是最殊勝的,因此每當黑寶冠儀軌一結束,他們都會衝向噶瑪巴。雖然眾人並不是出於嫉妒或敵意才會如此擁擠推撞,不過有時候,情況確實是挺危險的。站在前面的人會嘗試阻攔以免被撞倒,而後面的人就只是一股勁兒地向前推擠。所有人都想要領受加持,在數百人,有時甚至是上千人的場面下,真的很可能會亂了套。

博德納的情況尤其混亂。群眾已經近乎失控地胡亂推撞和踩踏。有那麼一瞬間,噶瑪巴看似會示現出忿怒的本尊形相,即諸佛為降伏萬惡時所示現的慍怒形相。由於忿怒形相看起來甚具威脅性,而且四周烈燄如劫火熾盛,外行者常會把它誤以為是魔鬼。此時的我雙手不知何時突然多出了一根竹竿,然后以突如其來的一股力量(顯然是來自於噶瑪巴),用力往前一推──我就這樣擋住了數百蜂擁推擠的人群,引導他們按秩序徐徐向噶瑪巴的寶座移動。隔天早上,我才赫然發現向來堅固的涼鞋已經變了形,全身儼然被輾過似的──每一吋的肌肉都疼痛不已。回到斯瓦彥布後,似乎有甚麼重大的事情已悄悄改變。

我們甫抵寺院便立即被帶到陽台,然後有幾位老喇嘛過來問了我們一些問題,像是我們的背景,特別是和我的大力氣有關的問題。他們問起我們的生肖,以及其他許多當時我們並不明白的事,不過顯然我們如今已經被接受作為噶瑪巴核心團體中的一員。自此,我便被要求充當噶瑪巴的貼身保鑣,為他平定四周圍潮湧的人群。 

數天後,噶瑪巴乘直升機飛往納吉貢巴(Nage
Gompa)。納吉貢巴坐落於加德滿都山谷盡頭的山腰上。祖古鄔金仁波切(Lama
Urgyen
Tulku)與妻子──她也是一名喇嘛[或尼泊爾語稱為「喇米尼」(Lamini)]──以及兩個皆為轉世活佛的兒子住在那裡。仁波切與其子孫被譽為是加德滿都山谷的守護者,他們有的在隆德寺長大和接受訓練。隆德寺是噶瑪巴於一九六一年至一九六五年間,在喜馬拉雅東部一個非常殊勝的地點上所建立。它是噶瑪巴在西藏以外的主寺,亦是當時追隨他一起逃離中共壓迫的人們的棲身之地。

漢娜和我希望能夠親近噶瑪巴,於是我們便乘著巴士前往中國鞋廠,再從那裡步行穿過竹林來到山谷盡頭的西瓦普里山(Shivpuri)山腳,然後沿著狹窄的羊腸小徑上山。漢娜很快便在途中病倒了,發著高燒(這可是獲得淨化的好兆頭啊!)。我因此負責扛著兩個背包上山,後來漢娜的身體變得非常虛弱,我幾乎得連同背包把她一塊兒扛上山。

當我們抵達寺院時,噶瑪巴正在傳授灌頂。我們所坐的位置較以往更靠近噶瑪巴,差不多就在他的正前方。當噶瑪巴將諸佛加持的力量傳授與信眾時,我們能夠真實地感覺到其甚深的禪定力量。此時的他看起來比平時更為巨大,我們倆不約而同認為眼前的噶瑪巴和平時的不同。我們問吉梅醫生,這究竟是噶瑪巴的真實身,還是他的意生身。他認為是真實身,不過以噶瑪巴而言,其實都不重要。為了更具體地說清楚,他告訴我們在前往尼泊爾時途中所發生的一件事。

當他們在孟買時,有一些政府官員堅持要噶瑪巴接受體檢。印度人視西藏人為低種姓民族,他們要求噶瑪巴進行體檢的動機不善。在印度軍隊中,最好的突擊隊和傘兵部隊都是西藏人。不過由於他們總是忙著追隨喇嘛高僧,因此這些印度人都想看看這些轉世高僧究竟能活多久。

當時噶瑪巴有其他事要處理,可是又無法對這一個收留了那麼多西藏難民的國家說「不」。後來體檢的結果頗令人吃驚。X光片顯示噶瑪巴的肺部大得驚人,可是其心臟卻如核桃般大小。他們也在噶瑪巴的尿液裡發現有糖,唾液裡有斑疹傷寒菌。由於有關當局感到無法置信,因此當噶瑪巴在數天後抵達加爾各答時,他們堅持要他接受第二次的檢查。這一次X光片顯示一個如足球般大小的心臟,以及一個非常薄的肺部。他們在他的唾液裡發現了霍亂菌而非斑疹傷寒菌,尿液裡的糖分則消失了。

就在噶瑪巴前往納吉貢巴的前幾天,他接受德國籍醫生費希爾的檢查。這名醫生多年來一直在加德滿都盡心盡力地為窮人服務,真正做到了無私的奉獻。這一次的檢驗結果顯示,噶瑪巴的心臟、尿液和唾液都很正常。

我們正聽得入神,無暇注意山谷上壯麗的景色。此時來了幾位年輕的少女並排著隊準備上樓。她們低著頭,手上拿著一條白色的絲巾。噶瑪巴顯然又要與信眾會面,我們見狀趕便緊混入隊伍中,跟在她們背後,然後溜進室內。噶瑪巴一見到我們便哈哈大笑起來。原來當時他正要為這幾位少女剃度出家,我一個大男人出現在尼姑庵內,女孩兒們可要忙了。噶瑪巴招手示意,讓我們在他身旁坐下。這一次是我們第一次真正有機會和他聊比較久。

他問我們來自哪個國家,我以身邊的朋友、弟弟和我自己作為典型的例子,告訴他有關丹麥人的種種。我大概把我們說得仍像是英勇的維京海盜,而不是如今收斂了不少(也是大部分人已變成)的那種「文明人」。噶瑪巴笑說自己也是個硬漢。他是康巴人,東藏的勇士部落。他在說話的同時也握拳輕輕打在我的肩上。當時我也忘了顧慮他的身份,很自然地也打他的肩膀。不過由於太使力,噶瑪巴差點兒從座位上掉下來,然而他卻只是放聲大笑。我這才意識到自己幹下的好事,不禁覺得汗顏。

他突然問:「你們想從我這兒得到甚麼?」

我清楚聽見自己的回答。我說了一些當時就連我們自己也不理解,更是未曾說過的話。我說:「我們希望成為利益眾生的菩薩。」

在我們所閱讀過的典籍中都有描述大乘佛教以菩薩道作為主要的修行目標。當這些菩薩的內觀覺慧漸次增長時,他們都會竭盡所能去幫助其他眾生。這一種利他的發心確實是很實在、自然。我們最懇切的心願,就是希望能夠成為利樂眾生的菩薩行者。

我的回答似乎讓噶瑪巴感到很欣慰,他送了我們一人一個錫製的佛牌。佛牌背面是一個金剛杵,象徵如如不動的證悟。噶瑪巴將佛牌掛在我們的脖子上,說:「這不是甚麼特別的東西,不過是我的一番心意。」

我們一直戴著噶瑪巴送我們的佛牌,直到幾年前佛牌破了,我們才將其中一個放在嘎烏盒裡,用以給予信眾加持。

翌日早晨,喇嘛傑珠乘著軍用直升機前來接噶瑪巴。我們很開心能夠再次和喇嘛見面。我們一直憶念著喇嘛,也為了先前的事向他道謝。喇嘛總是四處奔波,能夠見到他的機會實在不多。另一方面,我們又覺得自己愧對他,認為自己時常跟隨著噶瑪巴可能會削弱了我們和喇嘛之間的關係。當時我們仍然以西方人這種「自以為是」的眼光來看待我們和上師之間的關係,認為自己很傑出,其他上師必定會嫉妒我們所追隨的上師。當時,漢娜大概也算是一名很優秀的學生,而我則不折不扣是一個好斗又自負的人,就像一個能量不平穩的蒸汽壓路機一樣,會不顧一切將擋路的「障礙」夷為平地。

喇嘛傑珠察覺到我們的心意,不斷鼓勵「面臨兩難」的我們視噶瑪巴為上師,後來內心裡的尷尬感覺就漸漸消退了。儘管我們都經過理性思維的訓練,然而在學會真正的生命功課之前,還是得要面對很多次夢幻破滅的洗鍊。噶瑪巴能量場的加持圓熟了我們,促使我們從自己的上師處,乃至其他的人事物學習到不少的生命功課,可是我們也花了好一段時間才明白到這一點。與生命中這種深切的體驗相比,通過理論去了解佛的全知境界就相對的較為容易。

由於當天無法跟隨噶瑪巴一起前往,因此我們便和其他人一起下山。在噶瑪巴的隨行喇嘛中有四名青少年,初次見面後便對他們產生好感。他們提出了很多問題,可是似乎早就已經知道答案。我們從一塊石頭跳到另一塊石頭,緩緩走下山。憑著他們所懂的幾個英文字,加上我們所學過的幾句尼泊爾話,讓我們一路上溝通得還不錯。

雖然這幾位年輕的僧侶都很特殊,可是當山區人民見到他們時所作出的反應,卻讓我們非常訝異。只見他們連忙將頭上的帽子摘下,十分恭敬地向這幾位年輕人鞠躬行禮。而這幾位年輕人在走過他們的身旁時也會一一伸出手給予他們摸頂加持。當太陽在閃耀之際,人們便看不見月亮與星辰的光輝。我們將心思全放在了噶瑪巴的身上,幾乎不怎麼注意到噶舉實修傳承的另外四塊瑰寶。

雖然封建的西藏社會和轉世制度會對這些珍貴的身世造成某個程度上的傷害,這些年輕的轉世祖古依然是不凡的。在過去的八百年,他們其中一人或數人在歷代噶瑪巴轉世之間,對於噶瑪噶舉傳承法脈的延續,扮演著舉足輕重的角色。最受人所尊崇的是夏瑪仁波切(西藏「善規派」,又稱黃教格魯巴政府曾禁止他的轉世長達兩百年),接下來便是司徒、蔣貢康楚和嘉察仁波切。

這一段期間,有許多來自哥本哈根的朋友前來探訪,他們無不因為噶瑪巴的加持力而感到震撼。雖然他們當中此後能夠在生活上作出重大改變的沒有幾個,他們大多數都情願參加不同的灌頂,卻從未開始進行實修,可是他們仍然從中受益。在他們的內心深處,生起了一種令人動容的大信心。這一種信心能夠吸引諸佛非常明顯的護佑與加持。

在斯瓦彥布,整片山谷陷入一片鬧哄哄的景象。來自世界各地的人潮絡繹不絕,噶瑪巴日以繼夜地為信眾開示法教、加持、治療、聆聽。他總是有不得不去的地方,人們對他總是有不同的請求。可是我們未曾見他拒絕過任何人的要求,或凡事先為自己的利益著想。

在眾多儀軌中,要數「法會」(Puja)最受西方人所歡迎,尤其是祈請護法的法會。在法會中,行者以樂音為所緣進入禪定;法會上所使用的法器,其振動有時深沉得會讓人覺得自己周遭的環境都在搖動。大鼓的節奏、僧眾琅琅的念誦聲與心臟的跳動是一種同步的共鳴,是一種能夠勾攝人心,震撼意識的強烈氛圍。一般在法會上,眾僧侶分成兩排面對面而坐,手持號角、喇叭,或類似雙簧管的法器、大小不一的鼓以及金剛鈴。在伴有樂音的情況下,他們依照前方小桌上的經文,一心不亂地念誦。默念的部份往往會被響亮的鼓、鈴和號角聲所劃破;眾喇嘛與僧人的誦經聲,仿若來自另一個世界,如此超然。整個體驗往往會有一種飄然生起,不確定聲音從何傳來,卻又無處不在。

噶瑪巴和各轉世祖古就坐在盒狀的寶座上,絳紅色的僧袍上另外披了一層金黃色的外袍。出乎意料之外的是,他們並沒有因為整個法會的氛圍而變得拘謹。當法會在進行時,他們聊天、開玩笑,大笑,環顧四周,時而打打哈欠,直到某個片刻,所有人又會變得非常專注,之後又再次恢復輕鬆與隨興的姿態。

我們都會盡量帶朋友去參加法會。當我們溶入法音,心裡的鬱結便會自動鬆綁。久遠的記憶突然浮現,古老的感受又重新生起了,卻又隨著號角的聲音漸漸消退。雖然我們沒人知道加持力、灌頂、誦經等是如何運作,不過全都被它們的一體性和所能感覺到的能量所吸引。

我們懂得一些咒語,而咒語是持誦中很重要的一個部分。我們主要是重複持誦「噶瑪巴千諾」,噶瑪巴曾親自傳授我們這一個咒語。此咒語的效用廣大深遠,意即:「諸佛之事業力,請在我身上盡速彰顯」。我們覺得此咒語讓我們和噶瑪巴匯合為一,令我們有利於眾生,迅速地滿足他人的願望。我們也持誦布塔拉錫米告訴我們的那一個咒語(此咒語是洛本傑珠仁波切所傳授的咒語,當年在監獄裡時,我們有不斷地持誦),以及家喻戶曉的六字大明咒「嗡嘛呢唄彌吽」。這是西藏人為利他故而主要持誦的咒語。此六字大明咒乃為「咒語之王」,持誦以祈請諸佛慈悲之化現觀世音菩薩[藏:千瑞吉(Cherezig),梵:阿瓦洛吉蝶濕伐羅(Avalokitesvara)]。六字大明咒能將六種煩惱轉為菩提,諸煩惱如貪執等,其真如本性其實就是智慧。

嗡(Om):轉化傲慢心和我慢心

嘛(Ma):轉化嫉妒心與敵意

呢(Ni):轉化我執和自利心

唄(Pe):轉化無明與困惑

彌(Me):轉化貪欲

吽(Hung):轉化嗔恨心

單單使用咒語就能讓人對它們的良好效果感到信服。就像法會,它們能夠使內心裡喋喋不休的妄念(內在的噪音)漸漸止息,真正的平靜就會現起。

我們所遇見的種種修持上的方便都可說是常識,除了非常有用,也極具吸引力。然而,有那麼一個本尊,我們卻無法立刻適應和明白,祂就是大黑天金剛護法「瑪哈嘎拉」。隨著念誦名號的聲音越來越大與強烈時,瑪哈嘎拉的加持力也就越來越凝聚。我知道祂就是西藏畫軸和塑像裡那一個呈單身或雙運身相,周圍火燄熾盛的黑藍色力量。祂怒目圓睜,獠牙外露,雙手、四隻或六隻手上持有武器,頸項上掛有人首項鍊,身披虎皮和象皮。瑪哈嘎拉的怖畏凶猛相,讓人很難無視祂的存在。

這種威猛的力量總是讓我難受得想要握緊拳頭。我一直誤以為祂們代表絕對負面的能量,我幾乎想要與之交戰。制約於西方人的典型思維和想法,我對瑪哈嘎拉生起了這一種錯誤的知見,可是與此同時祂的能量卻又如此熟悉,深深地震撼了我。我從不情願地感受到自己與祂的緣分漸漸增長,到後來對祂欲罷不能的喜愛。我在內心裡暗中希望:這不是性格軟弱的一種表現!我知道自己懂得這種純粹、原始的力量,後來噶瑪巴除了以「戰士「(康巴)和「佛法將軍」的名字之外,也經常以「瑪哈嘎拉」來稱呼我。

藏歷新年落在每年的二月或三月初一,那天噶瑪巴會蒞臨斯瓦彥布和博德納給予加持。斯瓦彥布將會是表演喇嘛舞的地點,而博德納則會舉行盛宴,當地舞者會穿上特別的服飾和面具進行表演。

斯瓦彥布的寺院前擺放著一個瑪哈嘎拉的巨首聖像,聖像上是一個色彩繽紛,條紋瑰麗莊嚴的網狀物。在此年度公開儀式中,過去所有負面的能量會被驅入此網狀物中,然後再將此巨首聖像搬到某一個特殊的地點焚燒,以表徵瑪哈嘎拉已經摧毀了所有負面的能量。一切邪惡的負面能量將會在此盛大的慶典中,以火元素的形式重新回歸,在新的一年為人們帶來浴火鳳凰般的新生和重生。我們幾個人負責將巨首聖像搬到焚燒的地點,當我們拾階而下時,感覺從未如此辛苦過。這一個瑪哈嘎拉巨首聖像真的是沉重極了。我們將聖像翻轉過來,紅紅烈火開始將之吞沒,感覺就像戰勝了一切對我們有害的能量,我對瑪哈嘎拉所表徵的意義似乎有了更深一層的認識。我明白到祂並不負面,反之祂是一股征服萬惡的力量,祂那無窮的力量將能夠摧毀一切煩惱,以及妨礙我們成長的一切外境障礙。雖然外表怖畏凶猛,其本質卻乃為一切諸佛慈悲的化現。今天,瑪哈嘎拉與其能量場是全世界金剛乘佛教背後的大加持力。

尼泊爾的佛塔都是以相似的結構模式建造。儘管這些年已經變得相當商業化,然而在那裡所能體驗的定境和大樂,依然令人心醉神怡。我們也會盡量到噶瑪巴的房裡進行禪修,他總是會允許我們在裡頭待上數個小時,讓我們體會與他同在的感覺。每次見到我們,他都會微笑著說:「很好。」

在他的能量場中,有時來到房裡有要事和他商討的人都不會注意到我們的存在。當我們覺得應該先行迴避的時候,我們就會跑到寺院頂樓的陽台上,即使是在那裡也能感受到噶瑪巴不斷上揚的加持力,如果置身其中,很自然便能直接進入禪定中。

當我們沒有和噶瑪巴在一起時,我們總是和福特、將軍、尼爾斯‧艾貝等老朋友在一些鄉下地區裡閒逛,鬧哄哄地來到山谷裡眾多的聖地。夜晚,我們遊走於吞雲吐霧後的迷幻境界;白天則沿著佛塔周匝繞行。我們很自然地學當地的西藏人伸手去推動轉經輪。有時我們亦會有機會親近喇嘛傑珠。喇嘛本身獨具一種深沉、寧靜的力量,我們很自然地就能領受他的加持。博德納的某個銀匠替我們打造了一個管狀的容器,我們將歷代噶瑪巴的頭髮裝入,漢娜和我輪流一人戴一天。將它戴上時,皮膚仍能感覺到灼熱,觸碰過的人都有同樣的感覺。

就在春天的某個美好的一天,噶瑪巴要離開了。由於前來覲見他的人潮不斷,他已經展延了出發的日期好幾次。那天我們起得特別早,穿過濛濛濃霧上山。我們繞過陽台來到噶瑪巴的寢室,讓我們感到很驚訝的是,欽尼喇嘛已經坐在裡頭了。他看起來一臉惶惑,穿著噶瑪巴剛給他的白袍。直覺告訴我們,那是噶瑪巴要他「棄邪歸正、改過自新」的意思。隔天,欽尼喇嘛的淨罪過程便已經全幅展開。沒過多久,整個尼泊爾都在談論他們。

當噶瑪巴要出發前往機場時,數百人以各種交通工具(從拖拉機到卡車都有)一路尾隨。所有人都因為他的離去而感到非常傷心,可是大部分的人都只是默默地將這種感覺埋藏在心裡,這讓我感到有點不愉快。對此我並沒有一個合乎邏輯的理由。事情就如我們所願般發生了。噶瑪巴在我們的堅持下接受了我們的請求(當時的我們有點兒自我高估),讓我們先跟隨喇嘛傑珠學習。畢竟,每個人都是噶瑪巴的弟子,可是卻沒見過幾個喇嘛傑珠的弟子。此外,我們熱愛在尼泊爾的生活,希望可以繼續留在那裡。儘管噶瑪巴說我們很快就會相聚,可是當下的離別卻猶如截肢般,讓人感到痛徹心扉。

我們在機場見證了自在轉世者深沉的定力和臨在感。藏文「祖古」(Tulku)意即「化身」,指一些擁有一個不是他本人軀體的人。雖然數世紀以來,有不少孩童因為政治原因被選為轉世靈童,然而對於能夠保任其自心本性的明覺者,他們往往會是一個非常優秀且極具啟發性的導師。若不是深受傳統西藏認證制度的影響,許多西方的所謂理想主義者都值得這一個稱號。

在這些小祖古(Tulku)當中,其中有一位看起來也許正值學齡。他披著紅黃色僧袍,一個人坐在機場貴賓室裡的桌上,和那些穿著灰色衣服四處亂跑喧嘩吵鬧的孩童們形成很鮮明的對比。當噶瑪巴和一些政府官員在後室時,我們注意到了一個很奇怪的現象。只見有幾名西藏人和尼泊爾人走到這小男孩面前,放下供養品,然後領受他的加持。他們看起來都很開心,不過卻沒有人給其他小孩任何東西。這讓我覺得很反感。出自於內心裡強烈的「民主意識」,我突然想過去看看這小男孩兒究竟有甚麼特別之處。我這一個西方人,體型本來就較西藏人來得高大,再加上臉上留了五天的鬍子,應該怪嚇人的。我擠著鬼臉,突然跳出來大聲地喊了一聲:「Boo!」想嚇嚇他。我想要看看他的反應。結果他只是一臉淡定若無其事地看著我。他的臉上看不出絲毫的畏懼。當我們的目光交觸時,彼此間生起了純粹的心心相印的默契。這太不可思議了!這孩子很嬌小,我拉漢娜過來,希望能夠領受他的加持。我們站在他的面前,他沒有將手放在我們的頭頂上。反之,他將額頭貼近我們的額頭。我們知道這是表示「接受」和「認可」的意思。此時腦海中盡是一片光明燦爛,卻又不知其所以然,我們找了個地方坐下,試圖消化剛才的體驗。後來我們才聽說,原來他就是本樂仁波切(Ponlop
Rinpoche)的轉世。本樂仁波切是第十五世噶瑪巴七個轉世的兄弟之一,他也是大圓滿(Dzogchen)心意傳承的持有者。

\textbf{第六章}

\textbf{被遺忘的山谷}

當噶瑪巴離開加德滿都後,喇嘛傑珠又再次繼續其奔忙的旅程。由於當時在加德滿都並沒有其他特別的事,因此我們便接受了友人泰瑞‧貝克的邀約,一起去探索某個位於尼泊爾中北部,靠近赤蘇里(Trisuli)的偏僻山谷。泰瑞是和平工作團隊(Peace
Corps)的志工,最近該組織節育團隊的直升機恰好發現了這一個地區。他的某個朋友是第一個進入此地區的白人,如今我們也想去看看。我們是在一九六九年第一次來到尼泊爾時才認識泰瑞。我們是在前往尼泊爾西部的博克拉(Pokhara)的班機上相遇,當時他還教會了我們不少事情。泰瑞可說是當時唯一一個在尼泊爾境內旅行多年的西方人,他大概比誰都還要熟悉這一個國家。他走遍尼泊爾全國,目的是為了要錄下當地的傳統音樂;他已從東至西橫跨了尼泊爾國境三次。在每一次的行程中,他總是帶著許多行李隨行。由於他花錢從不吝嗇,因此挑夫們從未找他麻煩。泰瑞熱愛爬山,他也是第一個征服了數座位於尼泊爾西藏邊境山峰的人。他的攝影技術也很好,其作品可結集成幾本書出版,只是如今他身在尼泊爾而未能好好整理。當年攀山者都必須向當地政府預繳六百元的風險補償津貼,泰瑞知道這些保證金最終都會無疾而終,所以從未乖乖繳付。

尼泊爾約有十三個不同的部落,擁有各自的方言。在這些方言當中有的源於梵語系,其餘的則屬藏語系。泰瑞會說一些當地主要的梵語系方言,他也是一個擅長組織的人。我們這一次將會和他及其同伴里察(他們都是訓練有素的攀山者)一起攀登幾座「所有權不明」的山峰和路徑。尼泊爾宣稱這幾座山脈皆隸屬尼泊爾國境,中國一方又堅持它們隸屬西藏境地。由於雙方爭持不下,這些地區便成了「閒人免進」的禁地。

我們所乘搭的交通工具來自拉納斯(Ranas)。拉納斯是過去管理尼泊爾的一個非常富有的印度種姓。一千多年前,當穆斯林勢力入侵北印度時,這些拉納斯帶著寶藏逃到尼泊爾,而且很快便成為了當地的統治階級。他們所建造的巨大宮殿,迄今仍屹立在加德滿都。在兩次世界大戰間歇期間,汽車成為了身分地位的象徵。當時尼泊爾甚至尚未建有道路,亦沒有汽油供應,可是當地許多富有的拉納斯仍砸下大筆金錢購買汽車。這些汽車都是經由船運抵達印度的加爾各答,經過拆卸分解後,再由苦力們揹著它們翻山越嶺,帶到尼泊爾。今天這些汽車可在各種莊園的穀倉里找到,而且都被保存得很好,車身被擦拭得光亮如新,絲毫不見生鏽的痕跡。儘管這些汽車所走過的哩數不超過五英哩,可是車主仍不惜以賤價將它們脫手。尼泊爾的內陸交通已經由堅不可摧的福斯小巴(我們有時是二十三個大人連同行李一起擠),或是使用柴油引擎的大賓士車(多用於行駛在顛簸難行的道路上)所取代,日本進口的汽車則一般是在城鎮裡行駛。在這一種平均一夸脫的汽油抵上當地工人一天工資的國家,較老舊的車款並沒有甚麼用處。泰瑞一些較有生意頭腦的朋友買進車身巨大的古董勞斯萊斯,然後再轉手運到美國售賣,從中賺了一大筆利潤。他和朋友也剛買了一台1928年份的全新福特汽車Model
A,也就是旅途中我們所開的那一部車。

一旦我們習慣了手油門後,這一台福特汽車開起來就沒有多大問題。老實說,我挺享受開這一台車的。這一台汽車的扭矩大,開起來感覺很棒。我們沿著一路未經鋪砌的小路來到加德滿都以西、靠近赤蘇里的一個小村莊。這裡已是路的盡頭。從這一段路開始,我們必須自行揹著行囊上山。漢娜和我已經習慣較節約的生活方式,如此一來我們才擁有更充裕的金錢去學習佛法。我們大概只帶一些麵包和豆類就準備上山去,可是我們的朋友就較為講究,甚至可說是做足萬全的準備。他們買了各種各樣的東西,現在我們必須一起帶著上路。除了和我們一樣帶上衣服、靴子、煤油和睡袋,他們還帶了各式的「奢侈品」,其中包括沉重的罐裝花生醬、特製麵包和許多巧克力。這些全都是在American
PX
零售商店裡可找到的東西。我們這幾個大男人平均每個人要揹超過60磅的行李,而漢娜的背囊則差不多是我們的一半重量。

第一段路程將我們引領至一條水勢奔騰的河流,我們在途中經過一座由俄羅斯人所建造的水力發電廠。由於橋塌了,所以當地人弄了一個渡河的牽引裝置。人們在那一條看起來隨時會斷掉的麻繩上掛了一個吊籃。人坐在吊籃上,底下就是白花花的一片流水美景,溜索一拉便可渡河,到達對岸。幸好我們的背包都平安渡河,沒有掉落翻騰的流水中。

真正的攀登現在才正要開始。我們終於明白為何只有少數人曾到訪過此山谷。眼前是一座我們必須攀越的岩壁,我們感到非常振奮。後來我們才聽說,當地的婦女來到此路段後都不願意搬運任何東西,因此便由男人們分擔物品的重量,分好幾趟將東西搬到山谷裡去。無知的我們卻傻乎乎要揹著如此重的負荷就逕自登山。從那裡看往對岸山脊的景色更為壯麗,可是若不小心失足掉落,便會像自由落體般直接墜入江水中。

行囊揹在背上比頭還高,這樣除了方便雙腳的移動之外,也會讓我們看起來比途中可能會碰到的熊還大隻,具有嚇唬的作用。只是我還是犯下了許多新手的錯誤,將背包的腰帶束得太緊。由於背包太高,偶而在攀爬時會頂到頭上突出的岩石部分而使到整個過程變得非常艱鉅。然而這種艱鉅的過程卻給了我們數次機會,將寶貴人身的教義內化。我不斷推著無畏的漢娜向上攀爬,自己則最後抵達頂端。我們那兩位專業的登山者泰瑞和里察說,以後他們絕不會在不用繩索的情況下攀爬那座岩壁。

傍晚將至,我們來到一個U型山谷,只見遠處堅固扎實的小木屋林立。當地的居民讓我們睡在大大的屋簷下。由於他們都是印度教徒,因此不允許我們進入屋內留宿。我們無意中發現他們手上擁有最上等的大麻,其品質之好就連在尼泊爾時也未曾遇見過,所以我們決定幫他們和加德滿都那邊的某個朋友拉線。他們顯然並不知道自己手上的大麻品質有多好。看在他們那麼友善的份上,再加上看見他們在物資貧乏的環境裡生活,因此我們認為應該讓他們多賺點錢。隔天早晨,我們往山谷的左側前進,沿著山腰上的狹窄小路,爬了一個又一個小時。沿路的竹叢長得和我們一般高,一不小心會很容易迷路。一路上只見竹葉都呈現褐黃色,據說這些竹子每十一年(太陽黑子較多之時)就會枯萎。就連高大的槐樹亦然。乾枯的槐樹,只需要一把斧頭就能將樹幹從頂至底一劈為二。由於樹身軟得可憐,(也正因為這一個原因),在喜馬拉雅一帶,槐樹常被用作屋頂上的板瓦。當地居民為了保持現成的木板供應,於是將樹幹底部的一圈樹皮移除,使樹身變乾。只是如今處處都是腐蝕枯萎的槐樹,似乎是供多於求而未被妥善利用。

尼泊爾由於人口過剩而正在面對大量泥土侵蝕的問題,流入印度的河水也是一年比一年混濁。對於生活在山頭裡已經正在挨餓的人們來說,這確實不是一個好兆頭。

接近傍晚時分,我們路經一塊突出的大石頭,狹窄的山谷在此處微微向左蜿蜒。只見懸崖邊的狹小的梯地上建有六間草屋,若不小心掉落山谷,雖不至於像自由落體般直墜河流,不過地勢也是相當險峻。按照我們所觀察,山上的居民不可能看得見我們來時的那一條路,可是他們卻已經為我們準備好了餐點迎接我們到來。他們大部分人從未離開過這片山谷,身心早已與這一片土地合而為一,他們就是知道有人自遠方來。

山裡的男人們提出以一天六盧比(差不多是半美元)的價格替我們扛行李。這一個價錢就當時而言可說是非常昂貴。我們想要表現出慷慨大方,於是還是答應了。我們在加德滿都存了一些一盧比的新鈔,山區裡的居民認為這些新鈔比就舊鈔更有價值。這些新鈔讓我們省下了不少有失雙方身份的口舌之戰,也算是給他們的好東西。這些山區裡的居民,誠如百分之八十的尼泊爾人一樣,都是印度教徒,一戶人口不斷生育。這次我們也是睡在屋外簷下。其實這樣也好,外頭除了沒有蟲子之餘,也能更好的了解他們的生活方式。看著他們如何在如此寂靜的山谷裡頭把自己組織起來也倒是挺有趣的。

「阿嬤」無疑是這兒的老大。她的意志如鋼鐵般堅韌,言詞尖刻鋒利,把這一片山谷管理得妥妥當當。在這種嚴苛的環境下要活到這一把年紀可不容易,真是少點兒能耐也不行。她雖然年事已高,卻無損她叱喝山谷的權力。我拿起她的水煙筒吸了一口,就這麼一丁點的好奇心把我直嗆得大咳起來,差點兒沒吐。她吸的可是未經處理過的煙葉,是純毒藥也!

挑夫們的體格雖然瘦小,可是卻非常能吃苦耐勞。他們的步伐緩慢、專業,不容易疲累。他們總是赤腳而行,腳板近乎平直,像駱駝般柔軟,因此能將重量平均分布在凹凸不平的地面上。他們偶爾會發出深沉的嘆氣聲(這種嘆氣聲是喜馬拉雅一帶人們的背上負荷沉重的象徵),為整個畫面提供了一個和諧的背景音樂。一個訓練有素的苦力能承擔很大的重量,即使是一個外表看來瘦不拉幾的苦力也能夠揹起你我所無法想像的重量。這其實是和他們呼吸運氣的方式有著很密切的關係,反而與個人的肌肉發展無關。他們會根據山坡的傾斜度、高度和貨物的重量調整呼吸,再配合腳步前進,就像是在很自在地換檔一樣。不過他們需要有人先幫忙將貨物搬到背上,再由額前的帶子捆紮支撐後便能一氣呵成。他們有的人甚至能夠揹起一個很重又患了高山症的西方人,不過一天僅是能走數英哩的路程,而且最好是下山的路線。在一些較多人經過的路線,陡峭的高地下會建有一些小木屋。挑夫或苦力們會到那裡喝上幾口烈酒,在抵達目的地之前先暫時舒緩疲憊不堪的身軀上那隱約的痛楚。他們有許多只不過才三十來歲,看起來卻像六十歲般蒼老。

挑夫們負責揹那三個較重的行囊,為了更快將東西安排妥當,加上我也不習慣支使別人幹活自己卻不動手,於是便揹起了漢娜較輕的那個背包。這些挑夫教會我們的不僅是搬運的技術,他們所做的事,我們曾經也只是從一些書本上看過而已。午休時間時,他們會去收集稻草,接著便從身上穿著那種雪巴人開放式的外套拔出一些鬆散的羊毛線。他們身上就只是穿著這種開放式的外套和圍著一塊纏腰布而已。他們從口袋裡掏出一小片鋼鐵,鋼鐵的一面被磨得光亮。只見他們利用鐵片和隨地撿起的半透明石頭進行磨擦,產生火花後用來點燃石頭上的羊毛線,再用它來點燃稻草和一些乾的樹枝生火來加熱玉米稀飯。這些玉米稀飯是他們這一輩子最主要的糧食。我們幾個人站在他們的小屋旁,某隻只剩三條腿,看似缺鹽的牛走過來舔了舔我們。看著眼前正在幹活的人們,我不斷思度:,他們究竟是如何在這種環境下活過來的?

近傍晚時分,我們來到一間用樹枝搭建的廢棄小屋。從那裡可以看見整片山谷的壯麗景色,因此我們決定在此地留宿。一開始時,挑夫們都不敢靠近這個地方。他們說,以前曾經有一個法力高強的法師住在那裡,因此很危險。我們進入小屋後沒有發生任何意外事件,加上他們記得我和漢娜老是在持誦咒語,因此認定我和漢娜的法力更高,於是在請求我們給予他們保護後也進到屋裡來了。他們沒有攜帶任何墊子或毯子,只是隨意睡在地上的一些竹簡上。漫漫長夜,他們不斷翻動身子向生火的地方取暖,就這樣翻來覆去熬過了寒冷的夜晚。這樣的生活很不容易,他們卻沒有絲毫怨言。

翌日早晨,我們路經某處,那裡剛下過雪。儘管挑夫們走到哪兒都幾乎是赤腳而行,這一次他們似乎決定要自我提升。我們剛換上軍靴方便在雪地上行走,他們說想試穿我們的網球鞋。對他們來說這可是一次重大的體驗。我們把鞋借給他們試穿看看,另一方面在心裡卻暗自希望他們穿得不舒服,從此再也不想穿鞋。然而這只不過是我們一廂情願的想法。穿鞋的想法似乎已經平穩、篤定的入侵他們的心靈,對他們原來的價值觀和自信心造成了極大的威脅。

隔天中午,我們抵達一片大樹參天的美麗高原,這裡的植被分布亦已開始改變。我們在高原上發現了一群綠色的小植物。這些綠色的小植物看起來有一點像蒲公英,可是卻有著比蒲公英更厚及矮生的葉子。挑夫們高興的不得了,不願意繼續上路。他們只是一心想著要採集這些綠色的植物,然後把它們帶回家。這些稱為蔓陀羅(Datura,又稱Jimson
weed,屬茄科毒草)的植物的根部擁有強勁的致幻作用。歐洲中世紀時期,巫婆將蔓陀羅涂在人類身體感受性強的部位上以誘導各種不同的迷幻狀態。在我們那個年代也缺少一些較為強勁的興奮劑,於是這些蔓陀羅的根部有時在我們的朋友圈中會相當受歡迎。他們會趁入夜之時摸黑到植物公園去「採集」蔓陀羅,然後再踏上逍遙的迷幻之旅。雖然不需花一毛錢,可是它對身體所造成的副作用卻相對更大。正如溴化物(bromides)和東莨菪碱(scopolamine)一樣,長期下來會對脊椎造成傷害。挑夫說焚燒這些植物有軀邪惡鬼的作用,現在他們想要趕緊回去淨化他們的家。我們猜想他們也許是想要嘗一嘗這種強勁的興奮劑爽快一下,於是便給了他們每人十二盧比新鈔,要他們速去速返。如果他們的速度快,天黑之前必定能夠趕回來,否則入夜之後,即使他們再熟悉整個區域,摸黑走起來也會很危險。他們答應會在之前的地方把網球鞋留下。下山時,我們會需要換上這些鞋子。

我們又揹起所有行囊繼續前進。由於我們現在處於海拔四千米以上的山上,這些行囊揹起來似乎比之前更重。我們很快便學會挑夫們的方式走起路來。以腳跟著地,將腳步縮小,即使是陡峭的上坡路,平穩前進才是道。我們不斷地走,只是在人有三急要方便時才會停下來。這一路上一定是高度的變化引起生理上的反應,所以老是覺得尿急。天黑之前,我們登上五百米更高處,來到一個景色壯麗的高原。高原右方是藍塘雪峰群(Langtang
Himal)和戈桑坤達湖(Gosinkund);左方則是西藏群山。藏民在高原上建造了一些佛塔和一個可以留宿的小石屋。隔天,泰瑞和里察動身去勘察地形,想要繪製地圖,幸好這次他們帶上完備的器具一起上路。而我和漢娜則希望爭取時間靜坐禪修。我們爬上山脊,從那裡遠眺可見西藏,近看則是我們卸下行囊的小石屋。人來到東方國家總是很快便學會無論身在何處都要注意看管個人的隨身物品。

高原上那些用粗石建造的佛塔就如其他建造有序、光滑明亮的佛塔一樣,皆表徵五種圓覺的智慧。我選擇了一個靠近石塔的地方坐下,打開那本關於「拙火」禪修法的書籍。「拙火」禪修法非常有名,而且效果顯著。雖然此書籍和我在監獄裡時所翻閱的版本不同,可是皆同屬噶舉實修傳承的修法。我才剛開始觀想、做深呼吸,內心裡一種潛在的能量便驟然衝向身體的中心。這顯然是過去世的福德所感。當時我尚未領受相關修法的指導,或具足其他修持條件,對於體格較弱者而言很可能會產生一種類似帕金森症狀般的效果。由於內在的種種阻塞已被破除,一種無法言喻的光明、能量和喜悅將我拉開,沒有留下任何疑問和疑慮的餘地。

當年密勒日巴尊者就是在此地修持拙火定而證得了圓滿的證量。今天尊者的能量遍照著整個噶舉傳承。我們所感受的便是尊者超越時空的證悟力量。離開此地不遠之處,山谷的另一方,就在西藏境內有一座吉榮寺。吉榮寺的領導就是傑珠仁波切。仁波切曾經在附近的某個洞穴裡和杜巴喇嘛(Dukpa
Lama)一起修行。當時他每日僅靠三勺水過活,這樣的生活一直維持了數年。這是我人生有史以來第一次感覺到有一股能量從雙手湧出,於是緊緊地抓住漢娜的手,想要將能量傳遞給她。

傍晚時分,我們回到小石屋。泰瑞和里察仍不見蹤影。天空開始飄起雪花來了,翌日清晨醒來時,四周圍已經是白茫茫的一片,積雪的厚度大約有半米高。泰瑞和里察雖然無法及時趕回來,不過我相信他們一定會平安無事。他們已帶上最好的羽絨裝備隨行,而且遍滿整個區域的加持能量必定也會保佑他們無恙。當夜幕再一次降臨,他們終於拖著沉重的步履回來了。

屋內溫暖的爐火讓他們感到非常高興,甫一進門便迫不及待拿起食物要填飽肚子。那套羽絨裝備和睡墊讓他們在雪地上舒舒服服地睡了一晚,他們對此讚不絕口。不過這兩天在外頭曾碰到過幾次緊急的狀況,有一次里察更因為雪盲的關係還差點兒踩了個空從岩石上掉下去。

隔天早晨,我們準備打道回府。新雪幾乎讓我們寸步難行,後來索性仰天躺著順坡滑下,這回可是又刺激又省力,一會兒功夫便向前邁進了一大段距離。我們急著想把腳上濕漉漉的靴子脫掉,然後換上先前借給挑夫們穿的網球鞋。可是當我們來到先前約定好的地點時卻不見鞋子的蹤影。雖然心裡有點兒不太高興,我們還是前往法師住的小屋,心想他們可能會把鞋子留在那裡,結果還是撲了個空。最後,我們又再次回到阿嬤的村莊。

阿嬤很沒誠意。她說那些把鞋子穿走的挑夫到山谷的另一頭去了,不確定甚麼時候才會回來。如今這真的把我們給惹惱了。這分明就是他們不想把鞋子還回來才使的詭計嘛!結果我們鐵了心,皮笑肉不笑地告訴他們說我們會邊吃邊等,ㄧ個小時後若鞋子還未被送回來,那麼我們就會把他們家的屋頂給燒了。ㄧ個小時後,當我們掏出火柴...(我們當然只是裝腔作勢嚇唬他們而已,不是真的要把這些窮困人家的財產給燒掉)就在千鈞一髮之際,當中某個年紀最大又最英勇的挑夫挺身而出,很神奇地「找到」了我們的鞋子,然後乖乖把它們歸還。他們現在才開始懂得尊敬我們。他們想要跟隨我們一起前往赤蘇里,但先前因為向我們要求太高的工資,覺得欺騙了我們,再加上網球鞋的鬧劇,於是現在希望能將「大事化小,小事化無」,所以願意免費替我們搬運行李。這次我們只是讓他們幫忙搬ㄧ兩個行囊。由於我們已學會正確的運氣方式,所以一路自己揹著行囊也走得不亦樂乎,再加上途中已消耗了不少食物的關係,這些行囊比一開始時已輕了不少。

我們一回到當初停泊福特汽車的地點後便付錢給那一位負責看守的老翁。連同開頂的車尾箱上的三名苦力,我們一行人浩浩蕩蕩出發,沿著曲折蜿蜒的道路駛往赤蘇里。這三名苦力是第一次進城。對他們來說,開車進城是一件非常拉風的事情。他們堅持要一路鳴響車笛,好讓所有人都看得見他們。我們尷尬地投其所好,一路鳴響著車笛駛進城裡,替他們為其族人的歷史寫下了光輝的一頁。

\textbf{第七章}

\textbf{雪巴人的國度}

當我們回到加德滿都時,洛本傑珠也剛巧回來了。我們很高興能夠再次和仁波切見面,他聽了我們的歷險經歷後被逗得哈哈大笑。可是他很快又要動身遠行。這一次,他將前往不丹。仁波切若不在加德滿都,我們留在鎮裡也沒什麼意思,於是想要再次回到喜馬拉雅的山區裡去。不過這次不只是爬山而已。山區裡的雪巴國度仍然保有完整的西藏文化,我們深受其所吸引。這一次我們若想要趕在春天起程就必須要加緊腳步了,否則當雨季來臨時,我們整個月就只能受困於風雨之中。

喇嘛傑珠在離開之前已告訴我們應該前往拜訪哪個喇嘛和寺院,給予我們他的加持和護佑。泰瑞和里察在上一趟旅程中教會了我們許多事,如今我們必須要靠自己了。從加德滿都到「雪巴人的國度」雪村布(Shercumbu)最好的方法,就是搭上清晨送郵件的吉普車前往西藏邊境,然後在蘭桑戈(Lamsango)或巴拉拔希(Barabesi)下車,再從那裡往東北的方向徒步前進。

我們在中午時分抵達。河邊簡陋的鋅版屋住滿了來造路的中國工人,我們沒有多作停留便繼續登上那看似遙無止盡的山頭。一路上幾乎不見有樹,太陽閃耀著刺眼的光芒,隨身攜帶的水壺裡也只裝有白開水。當太陽消失在山後,這裡很快便入夜了。就在快天黑之前,我們找到了一個有瓦遮頭的地方。他們這裡除了米飯和茶,便沒有其他食物。

隔天一早,我們又繼續上路。我們在中午時分抵達山脊,當天的天氣清爽宜人。我們稍作休息一會兒後又開始依著蜿蜒的山路走下山。這一周下來就只是不斷重複一樣的模式:上坡來到山口、下坡抵達山谷,接著又再爬坡上山,如此周而復始。其中有一段路會經過佛塔、佛像石雕及其他的文化標誌。幾個世紀以來,西藏人一直是沿著這一條路線將商品從加德滿都運往印度。離開此路段不遠,眼前僅是一片遭侵蝕的枯黃山丘,以及那些生活貧困、棲身於簡陋小屋裡的山區居民。我們偶爾在路上會遇見前來乞討的印度苦行僧,有時則會難得遇見幾個年輕體壯的西方人。那些年,歐洲面孔(尤其是獨自在尼泊爾旅行者)並不常見,他們通常都是參加那些安排妥當,隨行還會有許多挑夫幫忙搬運行李的團體行程。每當途經農場,我們會問:「有牛奶嗎?」「雞蛋?」,農家若沒有這些食品的時候,就會問:「那有大豆嗎?」「扁豆呢?」然而,他們通常只有一些去糠的大米和大量的辣椒而已。如果不幸巧遇有一支探險隊剛路過,那麼他們的食物便所剩無幾。漸漸地就連我們自己也開始猶豫起來。這些山區的農民只有幾平方米的梯田可供耕種,若他們把食物全賣了,我們所付的金錢根本無法讓他們充飢吧?然而這些純樸的山區人民都喜歡鈔票,他們還會把錢花在那些不但會破,而且電池還很昂貴的手電筒上,甚至也會去買那些他們根本用不著的圓珠筆。

若在喜馬拉雅一帶旅行,記得要注意城鎮以外的居民都會偏愛新鈔,而且若用大鈔付錢,他們通常沒有零錢可找。一九七Ο年,十盧比(一美元)或以上都算是大鈔。因此在離開加德滿都之前,我們先將一些小鈔和新鈔留了起來傍身,以備不時之需。此外,有些山谷裡的居民對硬幣也沒什麼信心,尤其是半盧比的硬幣。如果路經某些小山谷村莊,當地人民若以許多硬幣當作零錢找還你時可要當心,因為接下來要擺脫這些硬幣也不是簡易之事。

山區裡的人民一般上都很熱心,他們會耐心地告訴我們眼前那些沒有標示的道路,哪一條才是前往目的地的正確方向。我們也很高興自己能為他們做一些事當作回報。每抵達一個村莊,村民們都是帶著各種疑難雜症,一個接著一個地來迎接我們這些白人訪客。他們通常都是患有甲狀腺腫大、流感、傷口感染或頭痛等問題症狀,一般上幾滴碘酒和創可貼就能解決所有的問題。其實假如有人教導他們給鬧腹瀉的孩童補充足夠的水分,便能毫不費力地救了多條人命。在尼泊爾活不過五歲的兒童比率,大概高達百分之六十五。有一名醫生曾經告訴我們說,因為父母親錯誤的觀念,認為孩子在腹瀉時不應該喝任何東西下肚,所以他們大部分是脫水致死。

我們經常在路上遇見一些體型巨大的流動「障礙」,底部還會露出一雙小腳在走動。這些會走動的「樹叢」其實只是一群扛著一捆又一捆樹葉的婦女,這些樹葉是男人們才剛從樹上砍下的。這種砍自草樹的葉子,將會被儲存起來以備過冬或饑荒時所用,同時也用作為餵養牛羊的糧食。儘管草樹的葉子都被剝得清光,只剩下光禿禿的樹幹,可是一年後樹葉又會再次長出來了。如今這些光禿禿,看起來張牙舞爪的枝幹,讓整個地區看似戈雅畫作裡的風景。每一片山谷都擁有各自獨特的特徵和其獨有的氛圍。有的友善親和,有的則咄咄逼人、傲慢,甚至非常商業化。這一路走來,接下來究竟會是甚麼樣的一番風景,要應付甚麼樣的人,只要來到山口便已能略知一二。有些山谷大部分居民的長相很相像;在有的山谷又會遇見幾個婦女共事一夫,閉門一家親的情況。

翌日,我們踏上一段延自西藏南部的古道。這一條古道就如先前所說的一樣,一路上佛塔、佛像、刻石等數不勝數,洋溢著濃郁的傳統文化氣息。當道路向東分岔出另一條路時,沿路可見的風景又再次變得非常沉悶。接下來的日子,我們仍是在印度教徒的家門口留宿(按習俗外人不能入屋),一路上也沒有甚麼讓人感到振奮的事。我們一直不斷向前,期待著遠處正在等待我們的另一番風景。

吉里(Giri)是座落在那一片廣袤山谷中的一個美麗的村落。有一個瑞士人在這裡建立了一座農場,雜交培育當地和歐洲的牛隻品種,然後生產出尼泊爾品質最好的乳酪,運往加德滿都販售。村裡甚至有一條可供小型飛機降落的跑道。我們下榻的地方提供瑞士式的住宿,設備文明,但價格並不便宜。經歷過先前那一段沉悶的旅途後,我們特喜歡他們那烤得香噴噴的麵包。

吉里之後,我們又開始了登山的旅程,整個地區也顯然起了明顯的變化。而今,我們身處在一個古老的佛教區域。在那裡所發生的事深深地觸動了我們的心靈。這一切似乎蘊藏著一種說不出的意義。我們感到安心、自在、歡喜。

我們才離開小村莊沒多遠,就在一條狹窄的小徑上遇上一隻欲向我們衝來的成年公牛。它似乎是想自個兒獨霸整條小徑,不讓任何人擋路。我將漢娜推向身後的那塊大石,然後趕緊從刀鞘中拔出那一把尼泊爾軍用刀。這一把長兩英尺的刀具,其刃鋒利。這隻畜生噴著鼻息,瞄準了我和漢娜後便發威猛朝我們這兒衝過來,幸虧我們都沒被牛角撞著。這隻公牛像發了狂似的,還好我不必被迫將它刺死。這一場偶發的小騷動卻意外提高了我們對此壯麗的環境的感受度。這裡四周圍是一片對稱而立的松樹林,和早些日子的一片枯黃截然不同。

近中午時分,我們仍在前往山口的坡路上,此時有一個透明、發光,擁有四隻手臂的幻影在眼前豁然顯現。此人形幻影由光與能量所組成,呈現月光石的顏色,懸浮在空中。我無法相信眼前所見,揉一揉眼睛再抬頭看,幻影似乎在我們徒步登山的同時,一直懸浮於空中長達數小時。當時人們還不知道全息圖(Hologram)。換作今天,我會把當初浮立於空中的影像形容成全息圖,只是它更為清晰,而且閃耀著光芒,有一種無以言喻的美麗。所見幻影是觀世音菩薩[梵文:「阿瓦洛吉蝶濕伐羅」(Avalokiteshvara);藏文:「千瑞吉」(Chenrezig)]所示現的主要法相。觀音菩薩乃為一切諸佛菩薩慈悲的總聚。這是我生平第一次如此有意識地親見菩薩,而且還是這麼長的一段時間。不過當時的我並沒有全然去接受祂,反而因為高興過了頭,對於如此殊妙的示現不知道該作如是想。當我和漢娜討論數次後,一致認為那必定是我們進入了佛教地區的一種徵象。

當夜幕低垂,我們來到一個偌大的雪巴人木屋。這裡所提供的住宿和膳食是平常收費的一倍,也就是兩個人差不多五角錢左右。由於天色已暗,我們又挺喜歡這戶人家,因此便決定留下來,也不討價還價。換作平常,討價還價還似乎真是一個免不了的正規作業。無論是此戶人家還是房子本身都有一種說不出的純淨,我們實在沒有心思要去壓榨他們。後來當他們發現我們也是噶瑪巴的弟子時,更是主動降價,我們也很高興地分享他們的食物。

這是一個溫馨的夜晚。這些雪巴人興致勃勃地想知道一切有關我們那身在加德滿都的喇嘛的事。後來他們還送了一串禪修用的念珠給漢娜。這一串念珠本來屬於住在該地區一名證量很高的老瑜伽女行者。漢娜將念珠纏繞在手腕上,當下便接受了其加持。漢娜頓時感到有一股熱能從手臂向上流向心臟,漸而遍遞全身,十分不可思議。稍後,我習慣性地點燃了一根大麻菸想要為此美好的夜晚加分,卻突然感覺渾身不對勁。這是我第一次感到這麼不適,心跳狂亂。我老是忘記了這裡是千佛的國度,祂們顯然並不喜歡我們以大麻菸當香薰「作供」。

翌日,我們必須翻越好幾個山口。沿途一片廣袤無垠,曾有好幾個小時的路程,路徑的右側一直有一條河流與我們作伴。當天的天氣頗為潮濕,一路狂風暴雨,偶爾還夾雜著大冰雹從天而降。然而亦因為如此惡劣的天氣,我們獲得某個熱心的雪巴人收留,到他們家去避風躲雨。這些雪巴人十分有趣,而且他們家中的供養壇亦非常有看頭。供壇上的佛像或唐卡中所描繪的諸佛聖像都擁有一雙巨型的眼睛和不合乎比例的手,與其粗糙樸素的造型相映成趣。無論是寂靜相、忿怒護法相、女相、男相、單身相或呈雙運身相,這些粗糙的佛像處處可見,護佑加持著那裡的道路和房屋。我們在這裡不需要把人當作小孩般看待,彼此間能夠作出真正的溝通與交流。即使彼此間能說的話或能做的事情不多,當下能相伴於身邊似乎就已經足夠。每當他們發現我們與噶瑪巴的關係後就不願意收下食物或茶水的錢。幸好我們也學聰明了,懂得在他們家的供養壇上留下一些盧比。這樣不但能省下口舌,對方也無法拒絕。

那一天由於我們要拜訪多個地方,因此並沒能走遠。不過,我想我們也是來對地方了。隔天早晨,他們帶了一個瘦得只剩下皮包骨的孩童前來。他快死了。我把那裝有噶瑪巴頭髮的容器放在他的頭頂上,他睜開眼睛笑了。無論那之後發生了甚麼事,那孩子的生命氣息已逐漸增強,他一定會好起來。

我們隔天在十點左右抵達班德(Bander)。班德位於海拔600米處,是一個處於兩山溪流交匯處的營地。從這裡開始一直往上走便會進入雪巴人的心臟地區。人們曾經很仔細地描述過此趟路程幾次:首先,上坡走約三公里路抵達山口,之後再沿著山脊走幾個小時;最後會穿過一片樹林走約幾百碼的下坡路程,之後再向前走差不多一個小時便能抵達雪巴人居住的地方。

山上已有降雪的痕跡。雖然時間不早了,可是我們依然想趕在天黑之前抵達目的地。人們都勸告我們不如等到翌日早晨再出發,不過我們這兩個歐洲人可沒有這等閒情逸致和耐心等待,加上我們對諸佛的加持力亦抱有極大的信心,所以選擇繼續趕路。一路上曲折蜿蜒,我們沿著那看似無盡頭的荒蕪陡坡不斷前進,走過一片無與倫比的美麗景緻。途中不時傳來啾啾鳥兒鳴聲,唯遇巨鷹在頂上蒼穹間盤旋周匝時方才嘎然停止。每走到轉彎處,又見另一片更廣闊的風景。我們靠隨行攜帶的罐裝煉乳來補充足夠的體力,馬不停蹄地趕路,沒有歇息。

在半山腰處有一座寺院,我們決定前往拜訪。寺院裡有一隻巨大如牛的藏獒在看守,一見我們兩個陌生人便狂吠不止。前來應門的喇嘛倒是很有學問,而且非常友善。他很開心地帶我們參觀寺院,而且對於我們就那些精美的佛像和畫卷所提出的問題皆能對答如流。可惜我們無法逗留太久,否則當天就很難趕在天黑之前找到一個有瓦遮頭的地方落腳。

一路上的積雪把鞋子都浸濕了。我們在近山口不遠處巧遇一支小型的馬幫商隊,他們還熱心地邀請我們加入他們的行列。他們知道在山脊的另一方有可下榻的地方,而且還說沒人能夠趕在天黑之前抵達雪巴都城尊貝喜(Jumbesi)。從這一座山口,我們必須繼續走約三個小時才能抵達目的地,那是相當累人的一段路程。我們未曾試過與馬幫同行,喬叟(Chaucer)與其他作家在著作中所描繪的種種情境,如今活現眼前。這些自由的人民無論是文化還是脾性似乎都是恆古不變,而且非常有趣。他們對動物所表現出的仁慈,尤其當我們在回教國家親眼看過那麼多動物遭到虐待的情形後,更是我們未曾預想過的驚喜。這些雪巴人風趣幽默,有一種打自內心散發出的恬淡閒適與自在。看到有人前來請求我們的加持,對他們來說似乎是再自然不過的事情,反而是我們自己覺得不甚自在。我們第一次碰到這種情況。我想這也許是和我較早前在空中看見觀世音菩薩的示現有關。這畢竟也不是一件尋常的事。我將一個印有噶瑪巴肖像的別針和那個裝有噶瑪巴頭髮的容器放在他們的頭頂上,傳承的加持力帶來了極大的滿足感與覺受。沿著山脊而行的這一條道路引領我們穿越了景色超凡的杜鵑木林。從山脊遠眺,兩側的山谷景緻幽絕;越是深入,沿途的風景就越是更勝一籌。當太陽消失在地平線上,我們的馬幫商隊終於進入了雪巴人的山谷。雖然我們每個人都已濕到膝蓋上,卻掩不住內心裡的喜悅。

尼泊爾擁有超過十三種不同的民族部落。各個部落族群之間鮮少互動,各有各的生活。大部分的印度教徒生活在茅草屋裡,其他族群的房屋就活像是博物館似的,像尼瓦族佛教徒的房子就是一個很好的例子。他們的房子總是給加德滿都山谷市中心的那些遊客們帶來很大的驚喜。雪巴人堅強、活躍,他們無論在哪兒所建造的房子都很好看。雪巴人的木屋寬敞,其建造年代之久遠可追溯至西藏人開始進入尼泊爾經商的時期。雪巴人就是在那個時候開始從事商業貿易和經營馬幫。後來當這一種職業因為中國對西藏的破壞而消失之際,他們找到了另一條出路,轉而成為許多喜馬拉雅遠征隊伍和加德滿都商賈的嚮導。他們會願意為了打拼和機會而不惜遷徙他鄉,亞洲社會向來頗受傳統所束縛,所以雪巴人這種特質算是相當難得,再加上天生的巨肺,於是成為了高山嚮導的理想人選。然而他們如今也不得已要面對年輕人外流的問題。雪巴族群就像西藏人和不丹人一樣,喜歡組織小家庭,希望家中成員都有機會受教育。不像同地區的其他族群的婦女們,肚皮從來沒有閒下來過,好像是要為真主阿拉組織一支軍隊方善罷甘休似的,不然就是希望多生兒子,好讓他們長大後能賺錢養,為自己能安享晚年作保障。這一種自我主義不但污染了原本美麗的國度,更令其變得雜亂無章。

轉眼間天色已暗,若要繼續趕路會相當不便。就在此時,我們看見前方左側有燈亮著,那是我們在旅程中所遇見的第一個農場。我們整隊人馬受到農家所款待。農場的主樓後方是一個圍起的三角形草地,草地上豎著一支長竿,長竿上懸掛著勝利幢正隨風飄揚。我們躺在剛割下的乾草上,伴著馬鳴聲與經幡隨風拍打的聲音酣然入睡,然後做了一場深沉的美夢。

翌日早晨,馬幫的那戶雪巴人邀請我們到禪修室去。他們之前看過我們在靜坐,所以現在想要給我們看一些東西。他們將鞍囊打開,然後將裡頭的東西一一陳列在桌上。昨天晚上甫抵達農場,他們拿進屋裡的第一件東西便是這一個鞍囊。我們平時難有機會看到手工如此精細的佛像和大修行者的聖像,不過聖像上的臉孔都已被搗毀,顯然是遭到中國人所破壞。它們都是非常寶貴和嚴謹的藝術作品。聖像底座完好無缺,這意味著這些聖像仍然具有加持力,這大大地提升了其作為禪修工具的功效。聖像內部空心的部分主要由經卷及大瑜伽行者與具足加持力的聖地之舍利所填實,全然接受聖像者將能感受其中殊妙的能量。雪巴家族說:「今後它們會有新的面孔,然後我們會把它們帶到一個人們能夠了解和懂得珍惜它們的地方去。」

我們很感動。他們大老遠將這些具足加持力的諸佛聖像運離西藏,好讓它們能真正發揮其作用,這是很了不起的。可是我們心裡也清楚,這些聖像將無法安然地在寺廟裡被供奉很久。有一些無恥的商賈付錢請幫派到全國各地去竊取佛像,之後再轉手把它們賣到西方國家去。這些佛像來到西方國家後,內部的裝藏物品會被海關人員(為查看裡頭是否暗藏毒品)或好奇的收藏家所掏空;不然就是淪落到一些人的家中充當一件充滿異國風情的裝飾品或是一項投資而已,這些人根本就不懂得這些佛像對心性發展的真正意義。這一個地方給了我們一個很寶貴的機會去明白業力運轉的方式,也看眾生如何種下了來世苦樂果報的種子。有一些果報未必會在來世方才顯現。業力在助緣下也許會更快成熟也說不定!

此時,整個加德滿都都在談論著欽尼喇嘛的幾個兒子。他們持械打搶被捕,如今已押庭受審。持械搶劫這等事在尼泊爾可說是前所未聞的境況。他們就在噶瑪巴給了欽尼喇嘛那一件白袍後的第二天落網。我們心裡都明白,那件白袍象徵噶瑪巴對欽尼喇嘛滌罪的支持。在審訊過程中,欽尼喇嘛的家族涉及不同領域違法犯罪行為的事隨即也昭然若揭。這些事在尼泊爾似乎老早便已是一個公開的秘密。

前往雪巴國度都城尊貝喜(Jumbesi)的路況雖好,也有明確的標示,可是並不好走。一路上不知何故總有農民不斷從田地裡拋石頭,所以我們一直小心翼翼地避開這些飛來橫「石」。我們沿途經過許多田園風的小木屋和數不勝數的舍利佛塔。這些佛塔很難讓人忽視。我們最近聽說其實舍利佛塔內安奉著許多法物和舍利,給整個環境非常殊勝的加持。人在加德滿都很快便學會沿著佛塔的中心以順時鐘的方向周匝繞行。

舍利佛塔[「Stupa」窣堵坡]是佛教文化中的一種表現形式。佛塔的造型是以內、外、密各層面對宇宙的解釋而建造,經典中亦不斷強調佛塔對人的心識會產生不斷深化的影響。它們幫助眾生將當下的娑婆穢土轉化為原本的佛淨土。它們在眾生的心裡種下了覺悟的種子,使之能明心見性,透視萬法本質,而且超越一切期望或恐懼,將眾生引領至一個無論任何境遇都是真實和圓滿的境地。因此,眾生的心識將漸漸能夠辨識其本性中的喜樂與自由。眾生的心識,將能從有緣境走向無緣境,從迷亂走向菩提。

佛塔最底部「稜角分明」的方形基座象徵宇宙的堅固性或「土」元素,以透明黃色作為表徵。塔座與諸煩惱中的「我慢心」以及「南方」相應。到證悟時,煩惱即轉化為平等性智,明瞭萬法平等,萬法本自不有,裡裡外外皆因種種因緣條件和合所成。塔座上方呈圓形、水滴狀的部分象徵水元素,宇宙的「流動性」,以透明藍色作為表徵,與「忿怒」煩惱及「東方」相應。當忿怒轉化為清淨的大圓鏡智時,它會顯現出萬物的真如本性,即不增不減。其上方長方形的結構象徵「熱能」、火元素和「西方」,以透明紅色作為表徵,象徵「自私的欲望」,一個人通過修行將之轉化為妙觀察智。這一種頓悟和樂天派的人最為相應,他們在觀察到事物之個別差異性時,也能悟出萬事萬物之整體性和統一性。在西藏與尼泊爾,一般上這部分(如同在博德納和斯瓦彥布的佛塔一樣)四面繪有巨大的「慧眼」,觀望四方。其上方有環狀物的錐形結構,一般象徵「風」的元素和「移動性」,以透明綠色,表徵「北方」和諸煩惱中的「嫉妒」與「嫉羨」二心。通過見(見地)、修(禪修)、行(行持)將此二種煩惱轉化為成所作智。塔頂乃為新月造型,新月懷裡抱有一個太陽,或被理解為「空」元素中烈火熾盛的水珠與盛滿生命甘露之碗的一種表徵。它透明、發光或像月光石。在其不淨的狀態,它表徵「無明」與「昏沉」。這種煩惱鮮少以直接的方式對治之,通常會在一個人對治前四種煩惱時間接被轉化為超越時空侷限之法界體性智。這五種智慧成就圓滿的覺悟。這五種智慧以無畏的洞察力、任運的喜樂與大慈悲心示現,並以豐富、不可思議、護佑的方式給予眾生寧靜,給予眾生加持。西藏畫軸上以金剛坐姿盤腿而席,交叉雙臂,身呈藍色的金剛總持[藏文「多傑羌」(Dorje
Chang)],在噶瑪噶舉傳承被視為是圓滿覺性的表徵,如金剛般堅固不壞,通過喇嘛,尤其是歷代噶瑪巴所示現。又稱為大手印(英文:Great
Seal;藏文:Changchen;梵文:Mahamudra),乃為一切眾生皆具有的究竟本質,無論何時何地都保持這種正知便是圓滿的證悟。

在雪巴人的國度裡,一簇一簇齊肩高的瑪尼石牆都是作為原始的「路標」而存在,為行人標明行走的路線之外,也和舍利佛塔擁有同樣的象徵意義。人們以五種顏色在石牆上刻上「嗡嘛呢唄彌吽」與其他經文,也刻有各種家喻戶曉的佛形相與吉祥圖案的浮雕。因此沿著石牆行走,也成為了能打開心性的一種修行。

儘管此地區落後純樸,佛教的影響卻尤見深遠。此地區萬事萬物都是大整體的一部分,對於心態正面的觀察者而言,這裡充滿說服力地表彰了各種趨入心意識各大層面內義的法門。漢娜與我非常享受這種氛圍,感覺自己正被所信任和喜愛的力量所包圍、啟發。我們激動也開心,不時慫恿著同行者策馬奔騰。

我們一行人路過一塊巨大的懸岩,懸岩上紋刻著巨大的咒語,似乎是才剛漆上顏料不久。只見眼前的風景變得開闊起來,高地下是一片廣袤的山谷。山谷裡約有二十來間房屋緊緊靠在一起,像一個小村寨。那裡就是雪巴國度的都城尊貝喜。當我們靠近尊貝喜時,看見在山谷左側的草坪上坐落著一棟獨立的高房。有一名義大利籍的年輕人出現在高房門口,然後朝我們跑過來。這名年輕人住在高房裡,正在學習唐卡繪畫,而且對於我們一路上碰到甚麼人感到非常好奇。他看起來非常孤獨和迷惘。他一句藏文也不懂,我們猜想他會突然跑過來搭訕,應該也只是想確定自己是否還活著而已。就在這群房屋後方的數公里處,半隱藏在山后,矗立著一座宏偉莊嚴的寺院,我們深受吸引。由於先前答應了丹麥的朋友在擬定任何計畫前會先到鎮裡與他們會面,因此我們還是先到鎮裡去了。我們在鎮裡的兩家吃店裡都不見他們的蹤影,不過卻發現他們在鎮裡無人不曉。了不起的他們似乎喝光了整個尊貝喜的「唱酒」(Chang)。唱酒是雪巴人自製的一種啤酒。我們聽說他們出遠門去了,打算要把周圍地區的唱酒都喝光。後來我們也打聽到想要前往的那座寺院正是度齊仁波切(Tuchi
Rinpoche)的寺院。我們遵守了和朋友的約定先到鎮裡會面,現在既然他們出遠門了,那麼我們便留下紙條和大部分行李,迫不及待想要登山去拜訪度齊仁波切的寺院。傑珠仁波切曾向我們大力推薦這位度齊仁波切,因此心裡非常期待能與他見面。

我們走在青山綠水之中,一路上伴隨著瞬息萬變的天氣與英式公園般的優美景緻,頗饒有詩意。寺院是典型的藏式建築,也許不久前有人慷慨地捐了些錢給寺院,只見沿途刻有的咒語圖紋,全閃耀著新漆獨有的亮麗。寺院主樓依山而建,主樓前是一座中央庭園,四周圍起了一道牆。主樓周圍建有單層的排房,樓下則是喇嘛、出家人和女尼的寮房。寺院後山上的石岩處(就在主樓右側要走好一段距離方能抵達的高地),岩壁上如蜂窩般散落的石窟是禪修室。瑜伽行者們在這些建在岩壁裡狹小的禪修室內閉關修行數月,甚至是數年的時間。這些禪修房一般上小得只能供人坐著。在外行人眼裡看來,這些小禪修室似乎是可怕又折磨人的地方。可是對於修行者而言卻是一個能生起大樂之處。像這樣的一個處境能夠讓人避開外在環境的惑亂,心往內觀,瞭解真如本性。唯心才能體悟,才能有所成就;而且通過適當的教誨,大樂便能任運而生。這是所有人永恆的本質精髓。

有一隻很吵的巨型大狗負責看守這座寺院。不過就在我們瞄準向它丟小石頭幾次後,它總算很識相地知難而退。可是我們就是沒法子讓那一群叫聲粗劣嘶啞的烏鴉閉嘴,因此敲門敲了大半天才有人前來打開一條細細的門縫,瞧瞧外邊來者何人。

前來應門的僧人聽說我們是噶瑪巴的學生,而且是喇嘛傑珠派來的人之後,他才把門打開。他看起來非常高興,一路領著我們到廚房去招待我們喝茶。在西藏人的家裡或寺院裡,廚房是主要的活動場所。倘若你受得了火爐裡的冒煙,西藏人的廚房倒是個好地方,只是對我們西方人來說,這煙有點兒太刺眼了。那位僧人是個天才。他非常擅於模仿人,就連所有旅客的噩夢──在加德滿都那個討人厭的簽證官員,他也能模仿得唯妙唯俏,把我們逗得捧腹大笑。我們完全沒有預想過在如此嚴肅莊嚴的地方會有這麼一個逗趣且擅於模仿的人。他亦和我們分享了他們的故事。他跟隨度齊仁波切與其他僧侶,從珠峰北部地區逃到這裡來,逃離中國的壓迫。他向我們展示舊寺院裡的一幅壁畫(如今寺院已毀)和某張全新的相片:照片裡噶瑪巴正施賜福加持手勢,腳板和掌心彼此直線相對。

幸好我們留了一些阿斯匹林、創可貼和碘酒在身邊,這時候就可以用來當作小禮物送人。那位僧侶把我們帶到一間裝飾精美的房間裡,說我們可以在那兒留宿。他說:「仁波切剛進入了為期六個月的閉關。他就在樓上禪修,沒人能見他。歡迎你們在這兒留宿。當是在家好了,別拘束!」

就在我們踏上雪巴國之旅前,我們收到了人生中第一本有關金剛乘修法的英文書籍。送書人是喜馬拉雅東部小鎮卡林邦的一名華裔瑜伽行者。他曾要求我們從丹麥給他找來一些書籍,那些書籍的類型可真是有點兒讓我們不知所措。在那一個純真的年代,這些描繪「女性美」的書籍使到我們的國家名聲大噪。只是我們不明白一個女人的裸體究竟對他會有何作用。難不成西方人和東方人的內部脈輪存有微妙的差異?好吧!姑且先不管究竟是甚麼原因,他確實給了我們一份很棒的回禮。如今,這一本綠救度佛母[英文:Liberatrice;又藏文:多瑪(Dolma);梵名:多羅(Tara),乃諸佛菩薩慈悲的化現,以女相度化眾生]修法的英文翻譯本已經安然地躺在我的背包裡。那個年代,許多禪修方面的書籍都是藏文版本,因此這英文譯本就顯得特別珍貴。

當時我們並無意識到自己缺少了成為一個正信佛教徒所需的「印證」。噶瑪巴在尼泊爾的兩個月,我們參加了一次又一次的黑寶冠儀軌和灌頂。噶瑪巴的威力讓我們為之暈眩,不過周圍也沒人主動向我們提起有關皈依的事。即使尚未「正式」成為佛教徒,(當然我們依然是佛教徒),在觀想度母莊嚴圓滿的形相,念誦其心咒「嗡 塔列 度塔列 度列 梭哈」時,感覺仍舊是相當美好的。我們只是是很難去辨識散亂的自心其本來面目其實是與度母不變的覺悟本質無二無別。無論如何,度母就如慈母般愛護著我們,心門一開,便能感受到祂如潮湧般的大愛與加持。

由於翌日早晨醒來後證悟未至,我們心想也許應該要藉由隨行攜帶的那些高純毒品來提升美好的經驗。漢娜一次攝取了平日雙倍的份量,而我則依平日的份量攝取了數次後便出門去了。我們此行的目的地是光禿禿山坡上的一塊小平地,就在寺院往上走幾百米處。那個地方看起來非常適合靜坐禪修。LSD(俗稱「弗得」的一種致幻劑)的藥效很快,我們在上坡途中便已開始發揮迷幻的作用。那一條陡峭的小徑沙塵滾滾,不時還有小石頭滾落。小徑仿若剛從沉睡中甦醒過來,整個大地也像會呼吸般律動、搖晃起來。我們來到山坡上的小平地。在和煦的陽光下,放眼就能遠眺整個雪巴國度的美景,我們能在這一個距離寺院只有數百米的山坡上靜心打坐,心裡除了喜悅還是喜悅。,我們的氣息與群山大地一次又一次地融為一體。紫銅和黃銅的元素能量通過灰色的岩石溶入我們體內。它的意義不斷見次彰顯,身體的的各個部分,每一個細胞和原子都處於大樂之中。忽然,我注意到周圍有馬蠅在飛舞,也許它們從剛才開始就一直在那裡。我讓第一隻馬蠅吸飽了血,心想這隻馬蠅若因為吸了我的血而莫名其妙地踏上迷幻之旅的話不就很好笑麼?後來當它飛走時,我有注意到它很踉蹌地轉了幾個圈。當另一隻馬蠅停落在我身上想要大快朵頤之際,我沒多想便將它一掌拍落──扁了。就在我的手臂上還躺著一隻被拍扁卻仍然微動著身子掙扎的馬蠅之際,原先那種美好的感覺乍地消失不見了。我雖然擁有讓世界生起又幻滅的能力,但卻無法對一隻小動物生起慈悲心。這讓我感到相當困擾。

傍晚時分,我們沿著下坡路回到寺院,藥效仍持續著。把馬蠅打死後,我的內心一直非常不安,似乎有甚麼正努力想要掙脫出來。我們在下坡途中巧遇幾個正在建房子的雪巴人;在我眼裡看來,他們又矮又醜。我深知一個人外在的際遇是其內心的一種投射,這實在不是一個好的跡象。現在問題是我該如何從當下的迷幻異境中抽離,重新恢復正常的意識狀態,回到那個友善、正面又積極也饒有衝勁的自己。就在我們快要抵達寺院之際,我們注意到一棵小樹,樹上繫了多卷法本。這些法本已十分破舊,無法在法會儀軌中使用了。人們將這些法本掛在樹上,想必也是希望將它們的正面能量傳播給眾生。這一個舉動背後的慈悲心才是關鍵,其餘的都已不重要。當我了悟了這一點後,緊緊揪著的內心終於能釋懷。我的雙眼泛著感恩的淚水,內心裡一遍又一遍重複承諾著要致力於推廣這些寶貴的教法。漢娜站在我的身旁,我們誓言要將大智、大悲、大力統合的教法傳播出去。

進入寺院後,我們在諸佛面前再次堅定地許下誓願,使之穩固、加深,並且一直維持著此願力。我們認為這必定是過去的一種因緣,未來生生世世也必定會延續下去,其實這不外是常理:如果我們老是為了自己,問題就會生起。如果我們一心為他人著想,則自然能如意滿願。

我們的丹麥同胞終於回到尊貝喜。他們已將周邊地區所能找到的啤酒全都喝光,現在回到尊貝喜,心裡盼望著這裡的人已釀好新酒等他們歸來。能再次見到他們真好!我們都有很多故事想要和彼此分享。他們很快便融入了雪巴國度,成為了這裡的一分子。他們那維京人式的行事作風廣為當地人所接受,和他們完全沒有隔閡,很是親近。最近,他們才剛和某個歷史學家談論過雪巴人的歷史,簡要說一下大概是這樣的:

從十四世紀開始,雪巴人便分成三波南下,每一波的遷徙相隔了數百年的時間,分散至目前他們所居住的各個地方。他們來自藏東地區(如今的康巴省),顯然是遭到好戰且擁有部分歐洲人血統的外來部落所驅逐。新一代的康巴是強悍的藏民。直到今天,他們仍然堅毅地抵抗著無情的中國軍隊。有人說,康巴族一半是強盜,一半是聖人。在這一片受到保護、美麗的喜馬拉雅山谷中,雪巴人仍然保留著一些在西藏已近乎失傳的古樸習俗。在這些習俗當中,最廣為人知的就是人們不分晝夜,飲酒歌舞長達一周的宗教慶典。

這一種盛宴不僅是單純的狂歡活動。它們有助於消除人與人之間的侵略性與隔閡。在場的喇嘛引導其證悟的能量,使到這些雪巴人能感受彼此間的同一無二,達至一體的境界,因此這些活躍的人們的心境才會如此平和,整個民族如此團結的緣故。這些喇嘛有的來自「舊派」的法教傳承,也有的來自噶瑪巴的傳承。他們當中許多人亦擁有明妃。作為具有高證量的上師,他們能夠同時受用且又超越一切世間的欲樂。

尊貝喜的每一個人都知道度齊仁波切已進入長達半年的閉關,通過禪修轉化此險惡世界中的種種苦厄。這裡的人們在生活上總是不離皈依。人們確信圓滿的證悟乃為人生中最究竟的目標;引導他們走向此目標的方法就是佛陀的教法,也就是佛法;在道上,他們所能信賴的朋友,就是僧伽。若要以最有效的方便取得最快速的成果亦需皈依喇嘛,即上師。他(她)將會是信心、精神上的力量與佛法事業的根源。

這一種對於究竟皈依的確信不但啟發了雪巴人,同時也啟發了貧困的西藏難民。如今越來越多西方的修行者也能彰顯他們心靈圓滿的特質,當心靈不再執取於希望與恐懼時,一種溫暖圓滿的狀況便能任運無礙地顯現。這一種任運的能量能夠超越二元對立與主客二體的分別,而且是任運自然地現起的。

在這裡房子都是以灰色的粗石和全木板所建造,堅固、扎實,也不難進入。每一間屋子都會有一間供奉佛的房間,有的會用以修行或置空。後者通常會有一個供宗教舞蹈的院子。這些房子以手工粗糙的木柵欄圍起,和加德滿都山谷裡頭那些雕工精細的房子有著很大的差別。供壇上有許多已具上百年歷史的佛像,就如沿途所見的許多畫卷一樣,它們都擁有非常明顯的標記:一雙巨大無比的手和眼睛,和一個極長的腹部。這些獨特的標記特徵表示它們都是由農村(而非城鎮)裡的工匠所製造。

在尊貝喜放眼一望總是不離佛塔的風景,古魯仁波切(Guru
Rinpoche)的加持力更是遍蓋了整個區域。古魯仁波切乃為一個證量圓滿的瑜伽行者,也是他將佛教傳播至藏地。古魯仁波切「珍貴的上師」,也稱為「貝瑪桑巴哇」(Padmasambhava)或「貝瑪炯聶」(Pema
Jungne),意即「蓮花生」,乃為密乘佛教最高法教之持有者。大約在西元741年,蓮花生大士(簡稱「蓮師」)將一個究竟、超越自我的證悟的視野,和眾多能讓行者達至證悟的金剛乘修持方便,從如今已隸屬阿富汗的某個地區帶到了西藏。他曾將眾多法教傳給了其中一位明妃,後來外道入侵,佛法受到慘重的破壞,蓮師的教法以伏藏的方式被保留了下來,而且只能在因緣具足的時空下才會通過瑜伽行者取藏。他的教法受到舊譯寧瑪派所認可,但卻不太受到統治西藏的黃派或「善規」格魯派所認同。

數天後,我們一行約七或八人,展開了一趟旅程。我們在平均海拔3000米的地方遊走,計畫環繞雪巴國度的心臟地帶一圈,拜訪一些重點景點和寺院。此趟行程要避開的地方就只有著名的天波齊(Tangboche)和龐波齊(Pangboche),這兩個地方是挑戰珠峰者的基地營。澳洲和日本的兩大遠征隊伍將一路上的食物都掃光了,致使沿途的食物價格高漲得離譜!

我們從地勢較低的某條路徑出發,取道某條路經全鎮最大佛塔的道路,再走過濕滑的木板塊,左轉渡河,然後沿著某座山腰順勢而上。蜿蜒的小路帶領我們穿越猶如無人到過的一片樹林,沿途的景色,美麗得讓人窒息。每到轉角處就是一片全然不同的風景,無論遠近皆美不勝收。

途中倘若路經雪巴人的房子,他們大部分都會邀請我們到屋內做客,招待我們食物。他們會向我們探消息,對於我們隨身攜帶的物品都會觸摸和討論一番。幾乎每到一處,我們隨行攜帶的泡沫膠墊都會被用來交易五次以上,其他的物品則至少一次。他們也會迫不及待將噶瑪巴贈與我們的舍利放在頭頂上,領受他的加持。大部分的人會問噶瑪巴目前住在甚麼地方,然後也會很自豪地說自己的哪個叔叔或阿姨曾在加德滿都見過黑寶冠,領受過噶瑪巴的加持。

雪巴人的食物選擇極為有限。他們主要吃馬鈴薯,都是直接將馬鈴薯扔進壁爐裡烤,然後用手扒開來吃。他們也有烘過的青稞粉,叫做「糌粑」。他們將青稞粉和茶摻拌在一塊兒,和勻的麵糰也叫做糌粑。如今在大部分情形下,他們不得不將就著用小麥、玉米,甚至是大豆。這種傳統的主食是此民族的文化中最著名的東西之一,有說只要一個地方有人對喇嘛抱著極大的信心,且以糌粑為食,那裡就是西藏。雪巴人的餐盤中有時亦會有酸奶,如果周圍恰好住著會屠宰動物的穆斯林,又或剛巧碰到死了的動物,那麼他們便有肉可吃。蔬菜會把雪巴人給嚇跑!許多西藏人是在流亡他鄉後才知道「蔬菜」的存在。即使到了今天,他們當中仍有許多人會以為這些綠色的東西只是給牛吃的而已。

製作糌粑的方法有幾種。一般上,人們會先將穀物倒進一大鍋的熱砂子裡,然後開始拌炒鍋裡的砂穀,直至穀物爆裂開來。之後利用篩子過砂,剩下的穀物將會被放入一個袋子裡搖晃,以去掉外殼。最後,剩下的穀粒就會被細磨成粉。雪巴人大量喝的「茶」,西方人實在不敢恭維。原因是西方人對「茶」一字的定義與他們大不相同之故。對西藏人而言,「茶」是一種以牛奶、鹽巴和牛油,再加上一點兒低檔茶調味混合而成的「湯」。對於居住在高山上的人而言,這種茶的脂肪含量高,喝了有助於保暖禦寒,抵擋刺骨的大風和寒意。如果要將它理解為我們所熟悉的「茶」,確實是挺可怕的;不過作為「湯」倒是不賴。此外,旅遊書籍裡常寫說要用變味的牛油來煮茶也不是事實。他們當然也肯定會喜歡以新鮮的牛油來煮茶,只是對於生活貧瘠的他們來說,也就只能物極其用。數天後,我們也嚐了一口茶,箇中的滋味,恐怕畢生難忘。

雪巴人的家都設有煙囪,這種基本設施也許讓他們的生活變得輕鬆一點,然而無論我們進入哪戶雪巴人的家,不到幾分鐘便會開始眼淚直流,接著便會火速踉蹌逃離屋內那一個開放式的壁爐。壁爐裡的排煙理應是會通過屋頂上的洞口(只要把木瓦板蓋推開便可)被排出屋外。木瓦板蓋的位置取決於風的強度和方向,不過這一個系統顯然無法有效地發揮其功能。再說,他們所焚燒的木材內似乎含有過量的酸性物質。因此即便所排出的煙不多也仍會感覺刺眼,不一會兒工夫眼淚便開始嘩啦啦地流。

我們這幾個西方人因為刺眼的排煙不斷在揉眼睛,雪巴人們似乎沒有受到任何影響。大體上,他們都不是太嬌氣的人。他們能空手將燒得火紅的碳塊從火堆裡拉出,也能在沒有手柄的情況下,將一個裝有沸水剛燒開的大銅鍋扛起。對我們來說,無論是生活方式、身材或五官,他們每個人看起來都差不多並不特別吸引,不過他們在的日常生活上所經歷的事,可能遠比我們這一些生活在一個消費主義世界裡的人還要來得更多姿多采。雪巴人有一種說不出的古肅,像從那些灰色的岩石裡蹦出來似的,所以那幾個較「摩登」的(就像模仿簽證官員的那個出家人)就會顯得特別出眾。

就像在尼泊爾時一樣,無論我去到哪兒,人們都想要將我繫在腰帶上的那把廓爾喀彎刀買下來,我只好每次都禮貌地藉機轉移話題。大部分在尼泊爾所購得的刀,質量都很好,它們都是利用裝在貨車裡的那種彈簧所鑄造,像我這一把廓爾喀彎刀就是一把軍用的刀具,刀身的平衡性很好,甚至能砍斷一棵小樹。無數次的冶煉鍛造使它成為了一把完美的刀具。按照傳統,「彎刀出鞘必見血」,因此廓爾喀士兵們都會在指頭上留一個開放性的小傷口。如今我已將這把廓爾喀彎刀交由弟弟去保管,就在他位於瑞典南部樹林裡的家中。

旅程在一開始時的步調很緩慢。我們的朋友約翰是某個印度教斯瓦米的信徒,他堅持要穿著他那礙手礙腳的長袍上路。他甚至希望比當地人更道地,竟然打著赤腳走在那些即使是當地人也會很樂意穿著鞋子走的路面上。我們為了遷就他而把腳步放慢,沿途不知看了多少左右兩側的花草樹木。後來我們索性不管悠哉的他,直直地向前走。我們穿著舒適的網球鞋,帶著輕便的行李上路,那種感覺還真不賴!

我們沿途一直思憶著西藏最著名的瑜伽行者密勒日巴尊者,大地似乎也甦醒了過來。九百多年前,密勒日巴尊者曾經在此地區靜坐禪修。自從我在那被遺忘的山谷中修持拙火定開始,便一直能夠感覺到尊者的加持。當地人向我們指出數個尊者曾經在那裡禪修的洞穴。尊者的圓滿證量及所傳唱的道歌一直利樂無數眾生。關於尊者的書籍,像《西藏偉大的瑜伽行者密勒日巴》(Tibet's
Great Yogi Milarepa)、《飲入山泉》(Drinking from the Mountain
Stream),以及《密勒日巴尊者的十萬首證道歌》(The Hundred Thousand Songs
of
Milarepa)都是藏傳佛教中非常顯著的代表。他在人生中各個階段的發展,迄今仍然啟發了許多人。尊者以生命實踐佛法,他的故事比任何學術性的論文更能讓我們了解金剛乘佛教所追求的目標和修行方式。

這整個地區似乎仍和九百多年前一樣並沒有多大改變,就連當地人民的生活亦然。除了多了一些塑料製品和些許技術上的改進,他們的生活方式依然保有了舊時的傳統。前往曲旺(Tjuang)的道路有兩條。曲旺是山谷上方一個非常老舊、宏偉的寺院綜合建築。我們選擇路程較短卻相對較陡峭的登山者路線。不一會兒工夫,我們便被四周林立的寺院所包圍。

寺院主樓的周圍依然建有許多小型的禪修房。一些受過特別訓練的瑜伽行者會在秋收前到高原上作法,將可怕的冰雹趕走。在防雹儀式中,行者吹響長號角和海螺施法,將雹雲引走。山上的冰雹可不是鬧著玩的,所降下的冰粒可達至雞蛋般的大小,輕易就能取人和牲畜的性命,將農民辛苦勞作的莊稼破壞殆盡。這裡的寺院已顯得非常破敗和荒涼。如今只剩下幾名僧侶和女尼住在寺院裡,就連寺院大堂內極其精美的藝術創作也已失修多年。這裡不像印度的拉達克(Ladakh)或西藏的中西部,在高海拔地區氣候乾燥,只要沒有外來入侵或受到像文革運動這種事的干擾,無論是建築物或藝術品都能保存長達世紀之久。這裡的氣候潮濕,屋頂一旦破洞,霉菌很快便會出現,那些沒受到良好保護的物品便會很容易因此壞掉。總之,寺院周圍的地區給人一種陰鬱頹喪的感覺。當地甚具影響力的喇嘛們都已經離開,已經沒有人將人們凝聚在一起,就連許多農民也漸漸遷徙至其他地區去生活。

美國人理察在曲旺上課已有一段時間。他是最早開始修持金剛乘法門且深感受益的西方人之一。理察希望能夠過修行的生活,卻因為遇到我們這幾個飢渴的丹麥人而頻頻受阻。他也很容易受人慫恿而動搖!當我們剛抵達曲旺時,「將軍」伊恩便已經前去拜訪他。「將軍」可是一個難纏的客人,說一不二。他有一個了不起的本事,就是總是非常清楚地知道哪戶人家正在釀酒,而且他也擁有堅毅不拔的精神和耐力,能扛起兩大桶裝滿了全發酵的青稞「湯」沿著漫長又蜿蜒的道路上山。有人曾經開玩笑調侃他說,如果他能夠將這些良好的品質都運用在修行上的話,這個世界便會多一個偉大的上師。

 

我們只在曲旺待了一晚。那天早晨,漢娜將某個已往生的女尼在生前所使用的小金剛杵和鈴買了下來。我們付了七美元,就當時的行情來說是太貴了。當時的我們天真懵懂,不曉得許多穿著僧袍的人就像大部分的亞洲人一樣,都是天生的生意人。我們總是不殺價就按對方所提出的價格乖乖付錢,甚至認為這也是一種加持。即使是今天,我們對他們的了解多了,若要和他們交易也仍然會覺得很奇怪。然而有許多特殊的事情已聚合了在一起,如今這些殊妙的禪定加持物賜予了世界各地的金剛乘禪修中心一種不共的加持能量。

放眼遠眺脚下的山谷,只見蜿蜒的小徑一路引領至塔信都(Taxindu)寺院。從珠穆朗瑪峰吹襲而來的雪霧與風籠罩了一切,當霧散去時,一座巨大的山峰冰壁就在視野裡的左方。就在長長的山坡上那一條小徑旁,矗立著一間小木屋。這裡住著一名年老的女瑜伽行者,身上總是穿著一襲襤褸的紅色長袍。大家說她的家總是不缺牛奶和酸奶。我無法將視線從她的身上挪開。她擁有一種優雅卻又帶點不安的女性能量。這一種女性能量自幼便深深地吸引著我,總是能激發起我的保護欲。

就在毗鄰珠穆朗瑪峰的這一個地方住著像這樣的一名女性,如果她是生活在哥本哈根這種現代城市,必定會成為那種婦女組織領導人或在各種領域都能創造出偉大藝術的傑出女性。我看著她,內心裡泛起了難得的思鄉之情,也讓我想起了母親,想起了業力的運轉,想起了導致我們投生各處的種種業因行持。塔信都位於南池市集(Namzhe
Bazaar)和基地營之前的山口上。雪峰出現在雲霧瀰漫之間,彷彿觸手可及,周遭的景觀亦瞬息萬變。打從我們踏入那片土地開始,心裡便一直希望有朝一日能再次回到那個地方,逗留較長一點的時間,然後在山脈與雲霧之間靜心地閉關修行。

塔信都寺院的住持喇嘛是一名年輕快樂的轉世活佛,他要麼已證入世間與出世間法的不二境界,不然就是對教授佛法毫無興趣。不管怎樣,他老是在逃避我們的問題,而且手持法器念誦了祈請文後,便開始向我們兜售各種法器供物。我們一度很武斷,近乎以自以為是的道德意識去判斷他。當時我們並沒有意會到也許他的寺院需要經費,而且過去他也想必曾試圖向一些對佛法興致缺缺的西方人講解過佛法。對於他所兜售的這些珍貴的物品及所提出的合理價格,我們非但沒有心存感激,反而立即就將他歸類為「商業喇嘛」,絲毫沒有察覺到他那細微、開放的能量。後來,我們向他購買了一些畫工非常精細,表徵佛的淨化能量的金剛界曼陀羅畫像和一些莊嚴的法器供物。

我們的最後一個行程是位於札爾沙(Jalsa)的西藏難民營。我們在途中經過一座小寺院,那是到加德滿都去為人民祈雨的那個喇嘛的寺院。布塔拉錫米就是向這位喇嘛請得我們生平第一個護身符。喇嘛並不在,寺院裡只有他的家人。我們留下了供養物,感謝喇嘛護佑我們遠離一切麻煩。直到今天,在他的加持下,我們不但躲過超速監視,那一輛又一輛雖殘舊,速度卻很快的德國轎車和摩托車也依然耐用。儘管難民營的外觀看來相當貧瘠,營內的感覺卻截然不同。直到西藏人涉及政治之前,在他們身上都能普遍感受到這種很殊妙的能量與特質。營內家家戶戶都設有安奉佛像和喇嘛聖像的供壇,樹木與木桿之間也總是掛滿了彩色的經幡,隨風飄揚。我們途經一些房子,由於大門敞開著所以能看見一群婦女正在邊聊天邊編織著色彩絢麗的藏式毯子。這些婦女們的心情似乎不錯。不久前,人口控制局的人員曾到訪,分派了一些避孕套給她們。現在她們可以安心行房卻又不必擔心懷孕的問題。我們似乎每走到一處都會遇到一些在噶瑪巴蒞訪加德滿都時認識的老朋友。

和這些老朋友一同坐在難民營裡的食堂,我有那麼一瞬間浮現出「似曾相識」的幻覺記憶。有人認為那是早期生活中的一種記憶重現,有的人則認為那只是內心裡對某些感官印象所產生的雙重反應。直到獲得客觀的證實之前,我想還是不要冒然對它們強加太多固定的觀念較為明智。當目光停留在門外的風景時,三名康巴戰士的剪影在布滿紅霞的天空下顯得分外出眾。我知道自己曾經見過完全相同的情景。而且那是只有在雪域西藏才會看見的景觀。

我們從札爾沙抄捷徑回到尊貝喜去拿其他的行李。雨季快要來臨,我們必須立即動身回到加德滿都。我們連續趕了四天路,漢娜仍然保持著輕盈的步伐,而我在沉重的背囊下只能拖著緩慢的腳步前進。當我們趕到蘭桑戈(Lamsango)時,巴士正準備駛離。當天傍晚,我們抵達加德滿都。那一趟登山行之前種種未知的可能性,似乎正漸漸開始具體地彰顯。

\textbf{第八章}

\textbf{最後的迷幻之旅}

壞消息正在加德滿都等待著我們。我們早在噶瑪巴離開加德滿都的一個月前就已開始申請錫金的簽證。德里內務部竟然在兩個月後的今天斷然拒絕了我們的申請。這是一個障礙。只要想到我們因為這些無能的印度官僚被迫與噶瑪巴分開,心裡就非常難受。除此之外,我們也不得不考慮一些很現實的問題。儘管布塔拉錫米一如往常的熱心,幫我們在喇嘛傑珠的住所附近找了一個很棒的出租房,只是現在我們不只是身上的錢已所剩無幾,就連尼泊爾的簽證也快要到期。我們尚有很多想要學習的東西,不想就這樣打包行李回歐洲去。

我在監獄裡第一次留了長髮。在多年的走私生涯中,我總是把頭髮剪短,這樣會顯得我比較平庸均可和低調。其實長髮搔撓著肩膀的感覺挺好,而且讓我看起來較為溫和,所以我就這樣留了一段時間的長髮。蓄長髮唯一的壞處就是頭髮很容易纏結。一天,當我正在洗髮時,因為突然覺得要花太多時間打理頭髮,於是便決定要將頭髮全部剃掉。我頂著光禿禿的頭從火警瞭望塔旁的理髮店走出來,頭皮上仍然可見早些年留下來的一道傷疤。泰瑞恰好騎著腳踏車路過,停在我的面前。他一如既往的冷靜,帶著淡淡地口吻對我說:「聽著,你是大學生吧?有意到美國領事館教英文嗎?」我的誠實終歸得報。這是我們繼續留在尼泊爾的大好機會。這一份工作不但會為我們提供簽證,所獲工資也可給我們解燃眉之急。我當然毫不猶豫便一口答應了。

漢娜和我就這樣成為了尼泊爾六個月的居民。我們是那少數非常幸運的人,竟然能夠在像天堂般的這一個地方生活和工作。如今那越發嚴格的簽證審批條例,或是銀行的錢幣匯率再也不會對我們構成任何影響。我任教的「學校」是位於新路(New
Road)的美國圖書館頂樓的語言課室。我每個星期只需教幾天課,而且每一天只是幾個小時的課程,遇上宗教或國定假日還可以休息。這份工作挺有趣的。我的學生們來學英語的目的是為了到第三世界的大學裡學習農業。他們充滿好奇心、聰明,也擅於作弊。

我們每個月的固定收入是八百盧比,當時折合美金差不多是四十元。這筆收入已經足夠用以應付生活上的開銷、幫助別人、購買一些藏傳佛教的法器,甚至還會有多餘的錢剩下。唯一讓我們擔心的事,就是漢娜的肚子不知道得了什麼怪病。當地的一家俄羅斯醫院一直無法揪出病因,不過我想即使漢娜身懷九個月身孕,他們大概也無法診斷出來!除了這一個問題之外,我們在尼泊爾的生活安好,甚至沒有受到雨季的影響。每當雨季來襲都會對許多人造成不少影響。俄羅斯領事館在不久後也聘請我去教導他們的某個大人物英語和德語,而且當時整個情況就很妙。俄羅斯方面付的工資較高,每次都會派出其中一輛ZIL禮科特夫豪華轎車到美國領事館的大門口接我。他們雖然長得牛高馬大,可是看起來總是一臉的不快樂。我在課堂上有一半的時間是用來鼓勵他們要對自己所曾受過的高等教育具有信心,幫助他們克服「害羞的心理」,大膽開口說話。

我們通常會在西藏邊境渡週末,除了為這一個已遭摧毀的國家祈福,我們也會去泡溫泉。儘管我們所專注的事,甚至是思維模式都仍未改變,可是我們卻出乎意料地成為了加德滿都裡外來自不同世界與世代的人們所依賴的對象。我們一直試圖避免受到這些資產階級所依賴,因此如今感覺非常微妙。然而更奇怪的是,我們卻開始不斷在他們身上發掘到很多「嬉皮世代」的我們所嚴缺的品質。

路比是我們的一個好同伴。它也是房東所飼養的一隻迷你黑色藏犬。有一天傍晚,它喝下拌有一些大麻小碎片的煉乳後就「茫」了。它斜著身子坐著,一直盯著我們看了幾個小時,似乎知道只有我們才會明白發生在它身上的事。當天我們徹夜未眠,試圖幫助路比渡過此次的迷幻經歷。它看起來似乎沒受到多大影響,臉上絲毫不見困惑或恐懼。雖然路比無法說話,可是它所經歷的過程和我們平時幫助渡過迷幻旅程的人們非常相似。那次以後,路比便對我們生起了極大的信心,決定要跟隨我們,讓我肩負起保護它不受到其他嫉妒的流浪狗所傷害的大任。這一段時間讓我更了解狗狗,也讓我開始了解人類心靈裡那一段灰色的地帶:我剛才究竟是因為慈悲,抑或是惱怒才會踹了這些流浪狗一腳呢?

我們每天都會向牆上噶瑪巴的聖像祈請他的加持,以金剛鈴專注所緣,一心不亂,然後誠心許願。我們無法接受在德里的那群草包竟然拒絕了我們的錫金簽證,因此私底下也同時在想辦法該如何在不需審批的情況下出入境。喇嘛傑珠勸我們繼續嘗試以合法的途經入境。後來我們還是乖乖聽取勸告,申請了為期幾天的遊客簽證,只是就連這種簽證也得要等上一個月的時間就讓我們覺得渾身不對勁。雖然我們從外表看起來不錯,可是其實內心裡已充斥著諸多不滿。我們每天傍晚都會去領受喇嘛傑珠的加持,而且總是在吸了大麻一頭茫然、雙眼通紅的情況下去見喇嘛。除了加持之外,喇嘛並未傳授我們任何教法。

一天傍晚,喇嘛把我們召到寺院裡去。我們興奮不已,心想:「哇嗚,終於等到這一天了!喇嘛終於要傳授我們灌頂,就像密勒日巴尊者生平故事裡寫的一樣。」結果,喇嘛只是請我們幫忙某個不丹喇嘛賣一些木製的杯子籌錢。我們走在回家那短短幾步的路程中,心裡除了感到莫名的空洞,亦深感自己受騙上當。

我們收到一封遲來的信,通知我們就在我三月份的生日那天,我們位於哥本哈根的寓所被某個德國嬉皮士給燒毀了。這一個地方曾經是我多次迷幻經歷的基地。對於此事,喇嘛傑珠只是微笑著說:「好啊!火具有淨化的作用。」不久後,我們在尼泊爾租房裡的供壇也在某天我們去參加宴會時莫名地燒了起來。除了噶瑪巴的照片和一些LSD,供壇上所有的東西都被燒成了灰燼。那些LSD是別人送我們的禮物,對方似乎對它的強勁藥效一無所知。

就在某個內心空洞茫然的夜晚,我們服用了這些LSD。此時此刻,我們至少可以確定會發生甚麼事。供壇雖被燒毀,這些致幻劑卻安然無恙。我們認為這是鼓勵我們去服用它的徵兆,而且非常期待前額內的那一股力道能夠轉化成明性。此LSD的藥性對我們來說也算強勁,它很快便開始發揮了藥效。我們感覺自己離開了軀殼在附近遊走,返回軀殼後又再次出離,然後繼續四處飄移。我們也有觀想蓮花生大士的聖相。當天亮在際我與漢娜交歡時,四周出現許多呈金屬色的菱形狀能量。我有一種感覺,就是這奇怪的能量欲使漢娜懷孕。我嘗試去驅逐它,迷幻的體驗不斷延續。即使幾個小時過去了,我仍然在語言課室裡看見那些菱形狀的能量。我的聲音變得非常沙啞,沙啞得甚至連話也說不清楚。當我在講解有關最近挪威沿海一帶所發現的石油會對經濟帶來的重要影響時,腦袋裡突然冒出一個想法:我想到加德滿都某個出口中國貨的商店去買一瓶人蔘口服液來喝。

我知道人蔘對身體非常有益。雖然指示寫說服用劑量只是幾滴,可是我卻一口氣把整瓶口服液喝光。當時我只感覺到有一種非常奇怪的能量進入喉間,而且那些菱形狀不見了。我突然覺得自己像回春似的年輕、充滿活力,彷彿回到了早些年前正在練拳擊時幾乎每周都會與人幹架的那一段時光,回到了那一個又急性子又魯莽的自己。可是這一個我卻魯莽地將某件事給鬧大了。換作是早幾天前,我應該會作出不同的反應。

是這樣的。我們乘著一輛全新的塔塔巴士在離開西藏邊境的路上。當時我們就坐在靠近前窗的位置。這輛巴士的引擎是在印度合法安裝的簡化版賓士引擎。在步伐緩慢的東方國家待著的時間長了,如今能夠感受到如此漂亮的一副引擎在腳底下嗡嗡作響,的確會讓我感到心醉神迷。只是那個長得牛高馬大,看起來像是頗具威嚴的退役廓爾喀士兵的司機似乎正在盡其所能在破壞它,這讓我非常反感。他就像當地其他的司機一樣,為了賺錢只要是站在路邊的東西都載,因此車廂裡很快就塞滿了許多人和動物,有時候甚至連車頂也擠滿了人。我有好幾次試圖告訴他,若超載的話,他必須輪流以不同檔駕駛,否則巴士的引擎就會被毀掉。他聽懂了我的話,不過卻完全沒有想要搭理我的意思。

果然不出我所料,這時問題出現了。由於引擎內的活塞被卡住了,司機只好讓巴士靠慣性滑行下坡到某個村口。當他下車檢查巴士的損傷時,我跑到他的面前,說:「是你害我們受困於此動彈不得的!現在我們需要搭另一輛巴士離開,你把車資退還給我們。」司機很惱火,用那雙油漬漬的手揪著我的襯衫,我一使力便直接把他扔到溝里去。他爬了起來試圖反擊,我又再次把他往溝裡扔去。當他第三次嘗試反擊時,手上還拿了一根鐵棍,不過巴士裡的其他乘客制止了他。我已把原本可用來擋拳頭的背囊丟下準備和他拼了,幸好事情就暫且在此打住。他要是拿著武器衝向我,我應該會把他傷得更重。

這一個狀況真的非常發人深思!其他人並不知道我和司機之前的對話。在他們眼裡,我這一個頂著光頭,頸項裡掛著一條念珠,襯衫上還別著一個印有噶瑪巴聖像小別針的白人,痛毆了一個倒大楣的司機。所有人的目光都停落在我的身上,我趕緊將念珠和小別針收進口袋裡。我尷尬得恨不得鑽進地洞裡,於是踉蹌地溜進當地的一家茶館,想要躲避眾人的目光。只是該負面的氣場卻一直如影隨形。平時的我對人可和藹了。可是現在每個人看到我都唯恐避之不及,就連狗狗也不太敢吃我給它們的食物。當天傍晚回到加德滿都後,我們便直奔喇嘛的住所,將所發生的事全都告訴了他。當我向他描繪那個金屬色的菱形狀能量時,他說:「啊!我知道那是甚麼。它總是到處惹事生非。別擔心,我會好好處理。」當天一直到深夜,喇嘛的搖鼓與金剛鈴的聲響和他祈請護法的陣陣念誦聲一直在耳邊迴盪。翌日早晨醒來時,我的雙眼洋溢著喜悅的淚水。那「東西」不見了。我感覺自己如釋重負。

一九七Ο年九月初一的前幾天,好福報和好運氣終於降臨。忽然之間,所有事情似乎開始各就其位,將我們引向錫金。簽證局裡有一名官員認得我和漢娜。那些年,除了簽證局裡的官員,要在尼泊爾遇見一個討厭鬼還真是不容易。我們和這名官員第一次見面時,因為他對漢娜無禮,所以我讓他吃了些苦頭。現在他升職了。在尼泊爾有很多高級官員會經常因為財富突增,與其工作收入不成正比而丟了工作,這也就是說他們會私賣簽證來賺錢。如今這位先生的職位正好可讓他一報當年之仇。他竟然勒令我們在一周之內離開尼泊爾。現在該怎麼辦呢?這時應該依靠美國人還是俄羅斯人好呢?還是要動用喇嘛的名義?我們知道他在尼泊爾的影響力甚至比領事館還大,因為當地皇族(儘管他們都是印度教徒)在往生時都必須依靠喇嘛的力量來轉移神識。還是我們乾脆自己動手,就利用經典的熟雞蛋來偽造簽證好了?當我們仍在權衡各種可能性時,德里方面傳來了消息說我們的簽證終於順利獲批。

幾天後,我們帶著許多藏族朋友交託要供養噶瑪巴與喇嘛們的禮物,踏上了前往錫金的旅程。按照西藏人的傳統,出門遠行者同時也要扮演郵遞員的角色。若先撇開遇到盜賊,或行李的負擔會將背脊壓斷不說,其好處就是無論你走到哪兒都會受到歡迎。丹麥的朋友們答應幫忙將我們這段期間在尼泊爾收藏的畫卷、佛像以及其他禪修用的法器用他們的福斯小巴一起載返歐洲。他們出現得正是時候!我們根本不可能帶著這些行李一起上路。出發之前,我們和幾個密友狂歡了一夜,而且還抽了不少大麻,害我隔天走起路來都顯得晃晃悠悠。在這種茫然的狀態下,我們壓根沒注意到裝著書本的那個背包在火車上被扒走了。當火車抵達某個印度北部的城市時,我為了避免踩到那些躺在門口的人而從窗戶爬出車廂外。我亦因為這樣而將某人所交託的熱水壺給打破了。熱水壺破了固然可惜,可是書本不見了可事大!我們把這視為是「僅限於知識層面的學習是時候要告一段落」的跡象,接下來應該是作出實修的時候了。

漢娜很肯定地認為抽大麻的習慣對於靈性方面的發展毫無益處可言,這也許是喇嘛傑珠遲遲未授予我們任何灌頂的原故。後來我們才發現原來是噶瑪巴先讓喇嘛「拖延」我們,聲稱我們是他(噶瑪巴)個人的弟子。當然漢娜也說得不錯。儘管喇嘛傑珠要將重要的程序留給噶瑪巴去進行,可是他之前也許曾嘗試讓我們明白吸大麻與靈性的發展(修行)之間所存在的衝突:大麻會導致我們的心變得散亂,修行卻能使之專注不二。就在前往覲見噶瑪巴的途中,我們毅然決定要放棄身上所有的毒品,一丁點也不留下。

我的九年毒海生涯(漢娜大概是我的一半)終於告一段落。這是我始料未及的事。這一種覺醒與能即時了斷的自由力量讓我感到非常震撼。儘管某些長期的後遺症仍會伴隨我,像我在甩頭時就會發出「卡嗒卡嗒」的聲響(這是開玩笑的!),至少我踏出了第一步。更令我們為之詫異的是,我們再也不想碰任何毒品了。

\textbf{第九章}

\textbf{皈依我佛}

就像印度北部其他的鐵路旅程一樣,錫金之行仿若將我們帶到另一個世界。在此鐵道系統上穿梭幾個星期後,若是一個敘事風格多變的作者,必定已經擁有足夠的題材去寫一本精彩絕倫的書了。北印度鐵路不僅是一個交通系統,其實它更像是一個人間大熔爐,各式各樣的印度人生就在此坦蕩地上演。

若要前往喜馬拉雅東部地區,首先必須從尼泊爾乘搭當地的普通列車一路顛簸南下,之後再轉搭臭名昭著的「阿薩姆號」(Assam
Mail)。沿途經過的小村莊通常也只是零落的幾戶人家,住在由茅草和竹子所搭建的簡陋小屋裡。只要親眼看見山麓下那一座瘧疾肆虐的森林就不難明白當初三支欲遠征尼泊爾的隊伍為何會在途中消聲匿跡。這一座森林以東是一片人口過密的低地平原。西方人為這裡的居民送來高效的青霉素時,忘了留下避孕的藥具。這一片低地從阿富汗邊境的開伯爾山口一直延伸至緬甸,如果當地人口只剩目前的十分之一,那裡會是一片天堂。在這一片廣袤的平川上,菩提迦耶和靈鷲山(佛陀曾說法的地方)那幾座挺拔而立的小山峰,宛如寬廣的胸膛上突起的乳頭,劃破了四周的單調與靜謐。

鐵路上沿途的風景不斷在改變,火車上的乘客亦然。火車廂裡先是擠滿了皮膚黝黑,對瘧疾擁有頑強的抵抗力,體格較瘦小的「雅利安入侵前期」的土著印度人。當時,從來不會有人在這一段路程擔心車票的問題,因為沒有列車員敢上車來查票。當火車漸漸駛入佔地廣闊的北部城鎮時,車廂裡的土著印度人逐漸由皮膚白皙、體格高大,擁有「歐洲血統」的印度人取而代之。大約三千年前,這些人種從今天的烏克蘭遷徙南下,將黑皮膚的原住民推向了邊緣。他們與當地土著的偶然交合,繁衍出比哈爾人(Biharis)等族群。

這些「烏克蘭人」利用吠陀(Vedas)和奧義書(Upanishads)保護基因的純正性,建立了基本上算是獨特的印度教文化,也就是當今世上所普遍體驗的典型「印度」文化。種性制度為社會階層間設下了嚴密苛細的藩籬,各種行為舉止上的規則在特定的情況下對於各階層的控制亦相當管用。在這一種制度下,佛陀本身是屬於戰士種性。儘管今天大部分人會想像佛陀擁有非常「亞洲人」的五官特徵,然而經典裡所描繪的佛陀身材高大挺拔,而且擁有一雙藍色的眼睛。

火車以蜗牛般的速度緩慢行駛著,這一趟長達好幾千公里的路程很快就會讓人感到沉悶。你大概會慶幸自己手上有一本可讀的書或能持誦咒語,這樣才不至於無所事事。窗邊掠過的風景連續兩天都是一片片的平地、茅草屋、棕櫚樹、一棵一棵垂懸著氣根的大樹,以及腐蝕破落的寺院建築。沿途也看見一些瘦骨如柴,牛角寬大的白牛拉著輕便(偶爾木製)的犁或那些不會卡在泥沼裡的大輪推車。印度人把牛視為神物,因此非常善待它們。只見四處擠滿人潮。他們大部分身著白色衣物,當中有許多人正在乞討。這些林林總總的人與風景完整了整個畫面。

我總是感覺有一半的印度人民其實就在這趟火車上或在這四周圍。旅者若不表現得落落寡交,反而是神態自然地四處觀望,那麼一天之內難免會被問上百次:「你是哪個國家的人?」這裡的人民似乎真的都很想和人溝通、說說話。我們當然也希望給與回應與支持,只是他們的英文水準就只侷限在同一個問題:「你是哪個國家的人?」他們大概也只從朋友那裡學過這麼一句話而已。我們可以隨便回答:從下斯洛波維亞(Lower
Slobovia),月球或隨便說個甚麼地方都好,他們都會以一副「我知道」的模樣點點頭,年紀稍大的會附和說:「噢,英國啊!」年輕一輩的就會說:「噢,美國啊!」有一個毛髮凌亂的印度人突然跳進我們的車廂裡,他搞不好是一個偽裝自己的「存在主義」大師也說不定。他盯著我們,大喊:「你在哪裡?」,接著又消失在人群裡。

我們在西里古里(Siliguri),距離加德滿都以東兩天路程的小鎮下火車。在這一個小鎮上,我們又再次被體格嬌小、圓面孔,總是笑容滿面,精力充沛的山區居民所包圍。這裡的居民操著一口我們都聽得懂的尼泊爾話,四周又再次恢復平靜,沒有刺耳的嘶喊或尖叫聲。我們知道自己很快又可以和噶瑪巴見面,喜悅又再次在心裡滋長。

在火車站的二樓,我們跨過不少在報紙或薄紙皮上就地而睡的人們,才領到了所謂的「入境」許可證。旅者必須持有這一張許可證才能進入喜馬拉雅東部山麓地區的大吉嶺(Darjeeling)。他們通常不會在上山時檢查,不過若想要延長逗留的時間,或進入像是錫金等限制區時就必須向當局人員展示此證件。由於負責發許可證的警察先生睡著了,我們必須先把他搖醒,所以整個過程多花了一點兒時間。幸好他們幫我們延緩最後一輛的吉普車。這一趟車程倒是出乎意料的舒適。

吉普車裡只坐了六或七個人。在尼泊爾,一台福斯小巴擠廿三個大人、幾個小孩,再加上幾個麻袋的大米是很平常的事。我們也沒料到車身會那麼新。我們曾經有過一個很奇怪的想法,以為人們是故意讓這些吉普車看起來「老舊」的。不過,我相信沒多久這些吉普車就會像古董陸虎(Land
Rover)一樣,每行駛十公里就要停下來作各種不同的修理。目前,司機先生至少不需要用力轉動方向盤兩次才能驅動車子。

上山的道路是單一與雙車道交替,偶而路段上還會有上世紀末期在蘇格蘭建造的迷你鐵軌穿過。造路的工程師與建鐵道的人關係應該不怎麼好。我們的吉普車必須停下來幾次,等待玩具般大小的火車駛過,而且路上也無任何鐵路道口的標記,相當危險難行。然而,每一次的顛簸就意味著距離噶瑪巴更近一步,所以我們也沒有任何怨言。

從西里古里開始,我們駛經一條很長的直路,穿越沿途茶園林立的一片低地,再蜿蜒上坡越過一片亞熱帶雨林。這一個區域與阿薩姆邦(Assam)接壤,阿薩姆邦是全世界最潮濕的地方,每年在短短的幾個月內降雨平均13至17米。經歷了兩個小時沿途只有幾隻木馬,彎路又多的車程後,我們的司機將車子停在可宋(Kurseong)火車站對街。那裡位於海拔約一千米處,是西里古里之後的第一個城鎮。此中途休息站還不錯,我們嚐了些茶點,也欣賞了優美的平原風景。從那里前往索納達(Sonada)也是差不多兩個小時的車程。後來,索納達也成為一個對我們來說非常重要的地方。沿著曲折蜿蜒的道路行駛了約八公里的路程後,車子漸漸駛入山口。沿著上坡路行駛約2.5公里處有一座稱為古木(Ghoom)的小鎮。由於這一座小鎮常常籠罩著厚厚的雲霧,因此許多當地人就索性仿英語單字把它稱為「葛隆木」(Gloom),即「陰鬱」的意思。

道路將會在山口的不遠處分出一條岔路。司機說我們很快會駛入一條非常陡峭的下坡路,然後會經過茶園來到堤斯塔(Tista)的大橋。道路將會在那裡分岔,一條前往卡林邦(Kalimpong),另一條道路則前往錫金。在通往錫金的路上,吉普車靠慣性滑行了約6或7公里後,我們終於抵達目的地。大吉嶺是此區域的主要城鎮與行政中心。其實大吉嶺原本被稱為「多傑嶺」(Dorje-Ling),意即「堅不可摧的覺悟之地」,多傑喇嘛的主要道場也在此地。原來的錫金-\/-西藏殖民地[如今已稱為「布提雅布斯提」(Bhutia
Basty)]就在山腳下。而主城鎮則延續了英國殖民文化的烙印。每年雨季,人們會來到此地以避開肆虐印度東部的瘧疾。

大吉嶺以一個非常戲劇性的方式迎接我們到來。我們看見有一名婦女撲倒在昏暗的街頭,並且瘋狂地嘶喊,頭上的傷口還流著血。一開始時我們以為她被人打搶,於是想幫忙扶她起來,殊不知我們一靠近她,她便力竭聲嘶地大叫。後來我們發現原來她只是喝得爛醉,而且她不是因為恐懼或痛苦才力竭聲嘶地大叫,反而更像是憤怒。我們猜想等她酒醒後應該就會沒事,所以便由她去。不過,她倒是很像一名天生的現代芭蕾舞者或歌劇演員,讓人留下了深刻的印象。

大吉嶺的空氣清新,即使是到了夜晚,整個城鎮看起來比加德滿都更顯明亮與莊嚴。只是這裡缺少了美麗的佛塔與令人印象深刻的文化。我們下榻的旅館雖欠缺了尼泊爾旅館的那種魅力,不過亦少了灰塵和汙垢。在這裡幾桶溫水是非常奢侈的享受。我們就在大吉嶺冰涼的空氣裡,伴著深沉的美夢酣然入睡。

翌日早晨,巍然屹立的干城章嘉山脈與其世界第三高峰在陽光下閃耀著光芒。大吉嶺充滿了誘人的魅力。我們從「下市集」(鎮上較為貧困的一區,也是下榻旅館的所在地)發現了通往郵政總局上方較高尚區域的一條路徑。看見這裡的大別墅、商店和餐廳完美地保存了英式風情,頗讓我感覺不可思議。一般上,當殖民地脫離了殖民母國後,殖民母國的文化、語言在當地的發展都會面對停滯不前的局面。然而,這個地方卻比今天的英國更充滿傳統的「英式」氛圍。只是這裡缺少了願意消費的遊客,對當地的經濟確實造成危機。像我們這種凡事先看價格的節約型旅客對此地的經濟發展更是毫無幫助。

最近,隨著富有的印度人一波接著一波地來到此地遊玩,這裡的服務業又開始蓬勃發展起來。這裡的居民擁有像天使般的耐心,然而要滿足外來的殖民者似乎比自己的同胞更容易。那些來自孟買和德里的有錢人可不是一般的傲慢與難伺候。

在大吉嶺呼吸著清新、冰涼的空氣,不得不讚同這裡確實是一個讓人能好生休養之地。我們在郵局裡找到成堆來自家人和朋友的信件而高興不已。在尼泊爾,如果信件堆疊太多無人認領,他們會乾脆把信件都燒掉。大吉嶺一直是印度的一片綠洲。儘管昔日的風光不在,卻依然不失其獨有格調。

法院大樓要十點才開門。錫金的簽證都是在這裡簽發。我們很早就抵達,所以就在法院大樓外邊打坐邊等待,同時祈請噶瑪巴的加持,讓我們順利取得簽證。由於在亞洲待了很長的時間,我們知道在喜馬拉雅中部收到消息說簽證獲批了待領,來到東部地區後並不保證簽證真的批下來了。幸好我們還是如願取得了簽證,內心裡感到雀躍不已。

接下來我們將會需要一台吉普車。若要搭乘前往錫金的吉普車,我們必須回到那天下車的地點。候車站就在下市集的屠宰場旁,那裡也是苦力、計程車司機和樵夫居住的地方。我們的第一站是古木的山口。由於這一次是在白天出發,所以沿途看見舍利佛塔就很興奮。雖然這裡的佛塔在造型結構上略顯稚嫩,無法與尼泊爾的繁複與精細匹比,不過我們仍很開心能在山口和山丘上看見它們的蹤跡。隨著雲霧被風吹散,我們還看見一座巍然聳立的藏傳佛教寺院。

我們在古木左轉,不一會兒便行駛在陡峭的下坡路上。我們驟然從海拔2500米的地方落到海拔數百米處。這裡有一座大橋,橫渡從西藏奔騰而至的堤斯塔大河。大吉嶺的高原氣候瞬即由熱帶氣候所取代。空曠的山坡上不再是一片片的茶園,取而代之是較短粗、果肉厚實的香蕉樹,以及總是空空的橘子樹。這裡的香蕉要比歐洲那些低劣的東西好吃多了。尤其在山上吃太多油膩的食物,現在正好能吃這些便宜又鮮甜的水果解膩。

我們必須在大橋處填寫個人資料,通常這一個過程將會花半個小時以上的時間。因此我們的司機就擁有了充足的時間為他那已很擁擠的吉普車再多找幾名乘客。就當我們仍在疑惑他們到底是如何保護這座大橋而不受中國軍隊或當地共產黨的炸彈所威脅時,吉普車已經卡嗒卡嗒地行駛在陡峭的道路上,取右道前往卡林邦。車道左側是一條河流,沿途經過一排又一排的軍營。這些軍營都藏在森林裡,不過一如既往地沒甚麼了不起。畢竟當年英國只需五千名英軍就征服了印度!我們的證件在橋頭和橋尾兩端一共被複印和檢查了四次。眼見我們為這麼多人提供了這麼多那麼「具有意義」的工作,真是讓人感到「欣慰」啊!

我們的吉普車先經過孟加拉邦(West
Bengal),這裡「酒不多見」,然而來到錫金後卻處處可見推銷當地威士忌酒的廣告牌,標榜著這些威士忌酒是錫金最偉大的貢獻(我想是掀起化學戰的最大貢獻吧?!)。西方人把這種威士忌酒稱為「Monkey
Whiskey」,據說若不小心將酒滴落桌面,它會腐蝕桌上的清漆和色漆面。當地甚至有一家賣酒的店家叫作「噶瑪巴」,不過我們說服店主將標誌撤了。雖然西藏人看不懂,不過在西方人的眼裡看來卻非常滑稽與諷刺。

當我們看見第一個以藏文標示的路牌時,心情非常雀躍。我們知道很快便能與噶瑪巴見面。吉普車裡的其他乘客,他們平時的話題不外是久下不停的雨或農作物的收成。如今我們這兩個總是轉動著念珠持誦咒語,對那些他們覺得再也平常不過的事物展現出如此強烈熱忱的白人巨人,應該給了他們不少茶餘飯後的話題。

吉普車再行駛約八公里路便抵達岡托克(Gangtok),然而司機先生卻突然在左車道分叉出的一條垮了的小路上把車子停下。他指著前方蜿蜒向上的道路,說:「再往上走就是隆德(Rumtek)了,大喇嘛噶瑪巴就住在那裡。」由於我們帶著很多行李(主要是他人交託的禮物),天空又下著倾盆大雨,我們建議付他雙倍的價錢,把我們直接載至隆德。不過對方卻說自己沒有該路段的通行證,因此不敢鋌而走險。他建議我們先到岡托克住一晚,等隔天早晨天氣好轉後再回來。我們對他的話置若罔聞。噶瑪巴就近在呎尺,沒人能夠拉遠我們與他的距離。我們將所有的行李和禮物揹在身上,活像一棵會走路的聖誕樹。就在此時,有一輛沒載客的吉普車從逆向的車道駛過來。要麼這名司機的膽子比較大,不然就是因為他沒有一個經營旅館的叔叔,所以他願意僅以三十盧比的價格冒險開上這一條壞損的公路,把我們送到寺院去。雨水就像瀑布般直瀉而下。這場豪雨著實非比尋常,吉普車的遮篷或塑料布都無法防止雨水入侵。窗外的可見度非常低,司機更像是憑記憶沿著蜿蜒的道路緩緩向上行駛了11公里。我們的一些行囊,即使擁有層層保護也全都濕透。

忽然之間,吉普車的車燈照射在一道灰白色的牆壁上。牆上是許多鎖孔形狀的深色窗戶。車燈接著照在了敞開著的大門上。寺院的台階上未見任何人的蹤影。就連我們在某個類似休息室的地方卸下行李時也不見有人。我們陸續敲了好幾道房門,終於有人前來應門。每個人見到我們都詫異不已,其中包括一些我們在尼泊爾認識的喇嘛。他們聚集在一起,七言八語地討論起來,有的大概忙著去找會說英文的人過來幫忙翻譯。後來,Jigmela(-la:尊稱)來了,他是噶瑪巴的侄子,也是今天噶瑪巴駐歐洲的官方代表。Jigmela很委婉地說服我們說時間太晚,不方便覲見「滿願如意寶」。西藏人都稱呼噶瑪巴為「滿願如意寶」。後來,我們也贊同說明早再來覲見噶瑪巴會比較殊勝,幸好喇嘛們在寺院對街的旅館裡幫我們找到了可下榻的地方。

經過半年深切的盼望後,我們終於在翌日早晨見到了噶瑪巴。我們興奮地衝向噶瑪巴,他慈悲地給予我們大力的加持,然後說:「你們來了就好。你們可以把我視為導師。現在告訴我,你們想要甚麼?」我們想要的東西很難用三言兩語就說清楚。其實我們也未曾真正思考過這一個問題。我們告訴噶瑪巴,我們希望能追隨他為他服務,噶瑪巴知道怎麼做才是最好。「好吧!」噶瑪巴說:「明天十五,是一個特別的日子,也是結夏安居日的最後一天。明天我會授與你們在家居士戒(Genyen)。」我們不知道在家居士戒究竟是甚麼,不過只要是噶瑪巴給的就一定是好東西。我們把頭髮剃掉,只在靠近頭頂漩渦處留幾根頭髮。我們不知道只有和尚或女尼才會把頭髮剃光。女尼帕嫫幫忙我們準備在明天的儀式中所需的傳統禮品。這名老派的老女尼原籍英國,給人的印象非常深刻。她之所以能長年留在隆德,是因為擁有印度國籍的原故。早前她曾跟隨達賴喇嘛,爾後因為想要專注修行而非辯論,所以轉而跟隨噶瑪巴學法。她為了西藏的難民與文化,比任何人都付出了更多努力,只可惜不是每一件事都取得成功。一九六七年,她將自己在喜馬拉雅西部所培育的四名西藏轉世活佛送往蘇格蘭,其中三位爾後因為鬧出醜聞而令整個藏傳佛教蒙羞。無論是過去或甚至到今天,他們雖以西方人透明化的作風為前提,卻嚴重地阻礙了金剛乘佛教在西方國家(尤其是英國和北美)自力的發展。

漢娜的氣質,無法言喻。皮膚白皙的她,頂著光禿禿的頭,散發著睿智的光芒。我們帶著雀躍的心情等待明天降臨的同時,也不忘把握時間探索四周。我們也儘量爭取時間靜坐,對噶瑪巴的禮物保持著開明的態度。

夜幕很快降臨,寺院裡的燈火也一盞接著一盞地熄滅。遠處傳來一陣陣的長號角聲,聲嘯長空,迴盪在山谷之中,似乎向居民們保證,護法常在,居民常獲加被。在萬籟俱寂的夜裡,有一扇窗戶,總是亮著燈。那是噶瑪巴的房間,就座落於寺院的最頂樓。這一盞燈照亮了整片山谷。我在半夜醒過來,匆匆辦點事,只見那一盞燈依然亮著。後來我因為心情太興奮實在無法睡著,一大早就爬了起來,窗戶裡的那盞燈依然亮著。清晨的寺院一片寂靜。後來我們才知道,噶瑪巴雖然在晚間休息,可是卻沒有真的入睡。他那當下現前、無所不在的明覺性恆常是了了分明,無所障礙的。

樓上的禪修室布置得華麗而莊嚴,噶瑪巴身處其中,依然散發著一種攝人的光芒與氣韻。噶瑪巴席地而坐,背向著窗戶,面前是一張小桌子。他微笑著接過我們準備的禮品。進入禪修室時,幾位喇嘛向我們展示傳統的頂禮方式。首先雙掌合十,依次放在頭、喉嚨和心間上,接著跪地磕頭。這一個步驟依次重複三次。那一刻之前,深受西方國家開放式教育所影響的我們未曾親身體驗過這一種象徵「敞開與放棄對自我的執著」的動作。我們認為那只是一些熱帶國家的原住民才會做的事兒,不過卻未曾想像自己有一天會向任何人事物彎腰頂禮。無論如何,如今的我們真的非常喜歡身邊的這些出家人,那一天是我們第一次在噶瑪巴面前彎腰頂禮。一開始時感覺有點兒彆扭,不過這種感覺很快便因為我們對噶瑪巴的信心一掃而空。其實一直到九十年代中期,西方國家也有類似對喇嘛或祭壇行禮的文化,只是後來他們儘量避免進行任何看起來與回教相似的儀式。

噶瑪巴向前傾身,將我們頭上僅剩的頭髮剪掉。我們被納入「僧伽」(Sangha)作為世尊的追隨者,修行佛法的一員。接著,噶瑪巴還授與我們法名。我們滿心期待他會授與我們一些特殊法門的灌頂,騰空而飛的神通力,我們亦期待會有其他特殊的事發生,可是接下來的事卻和想像中的截然不同。在這一個似無止盡的儀軌中,我們蹲俯在地板上,跟隨噶瑪巴念誦一大段經文。不一會兒,我們便開始覺得骨頭隱隱作疼,然而在非凡莊嚴的噶瑪巴面前,我們對此冗長的儀式並無任何質疑。後來終於聽到噶瑪巴說:「當我打一個響指時,皈依境自然會現前。注意集中精神。」其實當下皈依的境界便已歷歷在目,非常清楚明白。就在僧眾們繼續念誦經文的當兒,噶瑪巴朝我們這兒撒了一些米粒,顯然重要的一刻已發生了。最後,噶瑪巴說:「你們必須像我一樣,對佛陀具足信心。」我們心裡卻想:「我們對你比較有信心。」

就在寺院屋頂的陽台上,吉梅醫生替我們翻譯剛領受的法名。漢娜的法名是「巨力荷花」,而我則是「智慧之洋」,我們當然沒有理由抱怨了。每一個人的法名的第一個字都是「噶瑪」(Karma),意即金剛乘佛教噶瑪噶舉傳承的子弟。同時它具有為了廣度眾生而開展證悟事業的願力。其實,皈依的意義是無量無盡的,直至證悟佛果為止,其證量是不可思議的。以利他的發心接受皈依將使一切變得具有意義起來,那也是當下最自然的修行。之後,法的新滋味便會不斷現前。這種皈依是內外悲智合一的境界,不只是一般所謂的「佛教修行的簡介」而已。

在過程中,人們展呈對「三個珍貴稀有之寶」(又稱三寶)的信心。三寶乃為佛、其教法(佛法)與修行上的善友,即諸菩薩(僧)。「佛」超越諸天神與人的境界。此名相表徵圓滿證悟的境地。人類歷史上的第一尊佛,乃為兩千五百五十年前的悉達多太子。佛境地乃每個人堅不可摧、永恆的本質,只是煩惱阻礙了人們對佛性的覺知。世尊的教法展示了何謂世俗與勝義,並且結合了其仍然活躍的加持力與清楚的指示,教導人們該如何達至證悟。此外,在修行道上的夥伴與助手亦很重要,否則大部分人都難以有所進展。目標、方便與靈性上的善友因此對於佛教裡的各個宗派都很重要。他們為成千上萬的人創造許多機緣,讓具有意義與獨立的靈性生活不再遙不可及。

若只是要單純的「走路」的話,這三個就足夠了,可是若想「騰空飛翔」的話就必須導師的指引。上師的覺性就是弟子與佛聯繫的橋梁。他傳授有利的教法,他的事業行持將會是弟子利他的楷模。證道上最直接、快捷的金剛乘佛教中,第四個不可或缺的皈依境,藏文稱為「喇嘛」(Lama,意即上師),表徵三寶的總集體。「喇嘛」在此意指具有傳承證量的上師們,代表諸佛的加持力不斷延續至今。許多人對此的體驗是溫暖和喜悅的,是人們可依賴的正面能量。「本尊」(Yidam)是指心性與證悟的結合。祂帶來世間與出世間的特殊功德,代表諸佛證悟境界的各種形貌狀相,譬如說是寂靜的、忿怒的、男的、女的、單身或雙運身相。這一種超越凡夫一般的概念思維,在灌頂或禪修時被引介入我們的潛意識中,使之漸次增長,與諸佛的智慧合而為一。

其實這些在藝術書籍和博物館裡常見的本尊形相與人的心識一樣真實。當心能清明開放如虛空時,純淨的本尊相和咒音將會自然顯現。祂們源自於諸佛菩薩悲智合一的本性,為喚醒眾生本有的覺性而出現。雖然在藏傳佛教中擁有各種不同的本尊形相,然而若因為本尊的數量與種類眾多而放棄修行是錯誤的。各種本尊相純粹只是心識的表現形式,祂們其實都擁有相同的本質。這好比一家好的餐廳會備有一個豐富多樣的餐牌供食客選擇一般。就修行而言,一本尊即等於一切,只要證悟一個本尊就等於證悟了一切。

護法讓一切佛法事業變得可能。不同的護法擁有不同程度的證量,噶瑪噶舉傳承所祈請的護法都是已達至圓滿證悟者。其佛力能為利益眾生而當下立刻顯現。祂們的忿怒相巧妙地消除了眾生內在的阻塞和外在的障礙。噶瑪噶舉傳承的主要護法乃大黑袍金剛護法,二臂瑪哈嘎拉,祂代表忿怒的威猛力量。這種忿怒威猛的力量和方便二者密不可分。將內心的狂亂及外在有害的障礙當下轉化,金剛乘的法門是全天候無休的修法。它喚醒個人最深層的潛意識,讓多生多世的記憶重新活化起來。因此,這些一般只在生老病死或中陰階段的關鍵時刻才會出現的記憶便獲得釋放。如今在最短的時間以最快的速度將之激活起來,這一種淨化業力的過程是痛苦的。雖然這之後喜悅和自由將會增長,但是人們總是脆弱而善忘的,稍有障礙便會氣餒,生起退心,所以這種大威猛的智慧力量不可或缺,以便打破人們慣有的模式,並且吸收其黑暗的力量,進而讓體驗任運生起。

因此,無論此第四皈依境被稱為「上師」或「上師---本尊---護法」,意思都是一樣的。後者使皈依的意義更能多層次的展現出來。

即如先前所說,皈依的圓滿意義只有在證悟的當下方能完全展現。其正面的力量卻能在當下便能感受得到。它能勾攝眾生,使之與佛結緣。受之啟發的禪修能夠清除眾生阿賴耶識中的負面業習,把惡業化解。皈依使心更加全面完整,因此更能幫助其他眾生。

九月十五對隆德眾人而言是一個很特別的日子。這一天是為期六周的「耶涅」(Yerne)或「結夏安居日」的最後一天。在這段閉關期間,眾出家人將會集結在一起修行、打坐、誦經。這一天也會舉行祭典舞蹈。從大清早開始,長號角與雙黃管的奏鳴聲便一直在耳邊迴盪。我們連走帶跑地從下榻的旅館趕往寺院湊熱鬧。旅館距離寺院也只不過是幾步路的路程。只見僧眾正在院子裡繞著寺院主樓周匝走三圈。他們的頭上頂戴著與密勒日巴尊者相關的傳統金黃與紅色的寬大法帽。雖然身上穿著最好的僧袍,然而他們的自我意識卻沒那麼強烈,不像西方人的遊行般拘謹。反之,他們笑著,不時開開小玩笑,和從各地遠道而來看他們的親人與朋友點頭示意。

在上師面前時的寧靜感覺是前所未有的強烈,眾人更是深受其所感動。燃燒檜樹時所產生的濃煙裊裊升空,與降雨雲交織在一起,因此無法看清是否就快要下雨。雨季仍未結束,如果在舉行祭典舞蹈時又開始降雨的話必定會非常掃興。寺院外的一座小山丘的平地上有一偌大的白色帳篷,篷頂上貼著八吉祥的圖案。帳篷的另一邊設有一個較小的帳篷,帳篷的一面敞開。從帳篷上可看見整個場地的概況,寺院的屋頂,山谷的遠處,西藏的岩峰山巒。當僧眾完成了三圈的繞行後,整個隊伍沿著台階拾級而上來到帳篷處。大喇嘛們在小帳篷裡坐下,其餘的人則圍繞著祭典舞蹈的場地坐下。我們在一群大人物中找到一個靠近噶瑪巴的位子坐下。從那裡除了能夠看見噶瑪巴的身影,也經常能夠聽見他那爽朗豪邁的笑聲。噶瑪巴厚實的笑聲發自丹田深處,總是輕易地便能感染身邊的人。在眾多喇嘛之中,噶瑪巴在展現其證悟的大樂方面有著其獨特的一套方式。這種證悟的大樂與他是無二無別的;這種喜樂的爆發力更是無與倫比,像大自然的力量般震攝人心。

儘管天空中烏雲密布,看起來快降下滂沱大雨,可是不知何故卻一直沒有動靜。此時,祭典舞蹈開始了,以生動的方式詮釋了佛陀的教誨。第一場演出由年輕的夏瑪仁波切所呈現。他是眾年輕一輩的轉世活佛中年紀較長的一位,乃為無量光佛的化身。無量光佛,藏文「Opame」,梵文「Amitabha」,日本人稱之為「阿彌達佛」,中國人則稱之為「阿彌陀佛」。「非個人化」的大智慧能讓人掌控自己的生理基因,像夏瑪仁波切便是最好的例子。他的面貌表情就和佛教文化中世代傳頌的阿彌陀佛一模一樣。

此舞蹈表演和一個未經過訓練的心識在中陰期間的種種經歷有關。中陰(Bardo)是指死亡(神識離開肉體)與再生之間的一個中間狀態。直到獲得真正的解脫之前,眾生的神識都是由投射所制約,因此它會認為所體驗的過程都是真實的,有關這一點在著名的《西藏生死書》中便有詳細的解釋。以各種標誌性的動作和甚具張力的面具,夏瑪仁波切具體地展現了中陰身的種種心識活動與狀態。就如先前所說,當新的感知能力在死亡之際遭到切割時,未受調訓的心不會比平常更具智慧。對於不斷改變的種種投射,它仍會把它們視為是真實的,因此產生嗔心或慢心,貪心或妒心,執著或愚痴等無可避免的煩惱。在不超過七周的這段期間,所強烈偏向的煩惱將會決定神識在六道中的投生處所。因此無論是地獄或天堂,作為鬼魂或天人,投生人道或畜牲道,這種種的體驗不外是眾生所造作和儲存的善惡業行,或稱「業力」(Karma)。眾生所經歷的生與死是一組不斷變幻的夢境,生生世世的延續。簡單來說,這一段舞蹈所要展呈的重點是,在心識了悟其開明、澄淨與無限的本質之前,眾生便會不斷在輪迴中流轉。唯達至證悟後才能無止盡的無畏無苦、喜樂與慈悲,這種力量更是不可思議地強大。

在場的觀眾都能感受到昆吉夏瑪巴的表演所欲傳達的信息。就實際層面而言,他將勝義諦與世俗諦結合在一起,把道和果統合起來。物質主義者強調無常跟變幻的真理,而對於虛無主義者來講,一切只不過是因果的呈現而已。死後的善業以明淨的白色狀相顯現,惡業則以黑色的恐懼或驚慌的狀相顯現。這一個表演對於固執和膚淺的人是不錯的教誨。

接下來的節目是一個非常古老的舞蹈。那是一個有關某個願意將財產、妻子,甚至是雙眼拱手讓人的國王的故事。那個國王的確非常慷慨,只是冗長浮誇的演說讓整齣戲變得十分沉悶。不久,我們便開始覺得那可憐的王后能從他身邊逃脫實在是一件可喜可賀的事。其實在場無論正在進行任何表演,我們的焦點永遠落在噶瑪巴及其周圍所發生的事上。對我們而言,他就是比甚麼都重要。

將近中午時分,我們巧遇一群參加豪華遊的義大利遊客,於是便開始向他們講解有關藏傳佛教的種種。儘管南歐民族在西藏和尼泊爾人身邊看起來是如此高大魁梧,可是同時亦顯得我們是如此的不成熟和分化。這是我第一次正式向西方人講解有關西藏人的宗教和文化,這一個經驗給與了我一個很大的衝擊。爾後,這亦成為了生命中很重要的一部分。我們的講解引起了他們的興趣,他們之後還一起進入帳篷內領受噶瑪巴的加持。現場沒有人說話,可是噶瑪巴和喇嘛們的臉上卻洋溢著無比的喜悅。我們知道他們一直希望能夠將教法傳播至西方國家去。這一個覺知給了我們一個很大的衝擊,我們永遠無法忘記:往後這亦成為了我們生命中非常重要的一個任務。

當所有表演結束後,陰鬱的天空仍未下雨,在場的觀眾很即興地上前展現他們的技巧,娛樂大眾:有的表演摔角、唱歌等等;也有人叫我秀出幾招,於是我便拿出看家本領,向他們展現各種高難度的彈跳。西藏人鐵定都沒有看過這些表演。從尼泊爾到隆德,眾人因為我看起來像藏東的戰士,因此都稱我為「康巴」。那天之後,我的大力氣很難不成為眾人矚目的焦點,一些天真無邪小喇嘛更不時會把我的身體當成樹木般攀爬嬉戲。我那一身結實的肌肉大概是最受人們所贊揚的特質了。

在表演的間歇期間,某人拿了一台收音機前來。那一台收音機是不丹國王送給噶瑪巴的禮物。一九七Ο年,這一台收音機在喜馬拉雅區域看起來可是件奇怪的東西。那台收音機似乎是壞了,他們希望能把它修好。大部分的西方人認為西藏人都擁有不可思議的神通力(當然這不是真的!);反之,西藏人估計我們都擁有甚麼都能修理好的「技術神通」。看來我的瑞士刀又再次派上用場了。當我將刀子攤開來時,他們都驚嘆不已。雖然我比較偏愛摩托車等會動的東西,這台收音機乃屬芝麻綠豆的小問題,未幾便將它修好了。收音機還即時播放了有關當地暴風雨的報導。在場的人都感到十分詫異。漢娜與我亦然。那天整片山谷幾乎被洪水淹沒,可是我們這兒卻一滴雨也沒落下。一直到表演結束了,眾人進入了有瓦遮頭的地方後,雨水才開始嘩啦啦地落下,一直持續了好幾個小時。接下來兩天的祭典亦然:祭典開始前停雨,傍晚活動結束後才又開始下起雨來。就在第三天活動的那個傍晚,我藉故隨口問一名會說英語的僧人為何會出現像這樣的情況。對方簡單回答說:「滿願如意珍寶(噶瑪巴)不希望在表演期間下雨,你看那兒的僧侶們都正在忙著燒香念咒。」每一次和噶瑪巴在一起,我們都知道自己必須要學習的事物很多,在學校學過的事物太少,心裡總是恨不得向學校討回以前繳過的學費。我們所接受的正規教育著實應具有噶瑪巴這種能帶來解脫的實際教授。

\textbf{第十章}

\textbf{不丹之路}

每天總是有川流不息的人潮湧至寺院。他們都想要見噶瑪巴,領受加持,向他請益,通過黑寶冠儀軌感受其證悟之心流。噶瑪巴的出現帶來了加持力和淨化力,親見他的次數越多,我們所感受到的改變便越多。一開始時,它們只是一種純粹的具體能量,像受到雷擊般。我全身劇烈地顫抖,後來我們倆雖感到筋疲力竭,不過卻處於一種大樂與圓滿的狀態。漸漸地,這一種經驗變得更流暢、無限與持久。每一次的儀軌都將我們引領至超越了時空的境地,因此只要一有機會,我們便會好好把握,與眾一同分享噶瑪巴的智慧心。

既然再見噶瑪巴,我們當然不願意離開了。我突然冒出想當噶瑪巴的司機的主意。這些年來,我在高速公路上累積了不少經驗,不但開車的速度很快,也能不依照慣例駕駛,十分敏捷。其實我們心裡是希望這些印度人沒那個膽子驅逐噶瑪巴的司機而已。按規矩,外國人只能在錫金逗留幾天。錫金距離中國駐西藏的軍隊非常非常靠近。

當我們想盡辦法留在錫金的同時,那種無時無刻都想要見到噶瑪巴的渴求也逐漸消失。噶瑪巴的能量無所不在,我們不再堅持一定要一直貼近他身旁寸步不離了。太多人希望爭取他的關注,他所要兼顧的事務真的太多。慈悲的噶瑪巴有求必應,他總是耐心地回答每一個人的問題,替他們排憂解難,也滿足他們的願望。對於其他人會認為是無可救藥或敗局已定的對象,他也從不逃避。他似乎對任何狀況,無論好壞皆一視同仁。他不受過去和未來所限,平等關照一切眾生。他能和任何眾生結下深厚的法緣,待時機成熟之際,所種下的菩提種子就會開花結果。

僅通過日常的黑寶冠儀式,便能感受到噶瑪巴源源不絕的加持力,而且這一段期間,只是待在隆德寺便能啟發心性的潛能。不久後,漢娜開始認為我們是時候作出實際的修持,而非只是一昧沉醉於驚嘆之中不能自已。我們向噶瑪巴請法,他很高興地大聲喊出三個不同的咒語,接下來我們應依能力儘量持誦。他所傳授的第一個咒語乃為第二世噶瑪巴,偉大的瑜伽行者噶瑪巴希(Karma
Pakshi)的心咒。馬可波羅(Marco
Polo)聲稱曾在蒙古親見噶瑪巴希,並且描述了噶瑪巴希如何通過其法力教化了忽必烈(Kublai
Khan)。此心咒將能激活噶瑪巴希殊妙的能量;由於噶瑪巴希也是蓮花生大士的化現,因此心咒也帶來了舊譯派(又稱寧瑪或大圓滿)的加持。第二個咒語乃與金剛亥母(多傑帕嫫)相應的鑰匙。金剛亥母表徵一切諸佛的內在智慧,過去密勒日巴尊者也曾以金剛亥母作為觀修的本尊形相。金剛亥母乃現女相之本尊,身裸,呈紅色,舞立姿,全身發光透明。第三個咒語乃為「金剛黑袍」大黑天護法之心咒。大黑天乃噶瑪噶舉傳承之事業護法。我們很訝異噶嗎巴並未以耳語傳授的方式將這些密法傳給我們。不過噶瑪巴說這些殊勝的加持力與能量將會在我們的自心裡外不斷增長。

隆德寺裡眾僧侶每天的生活大致上是這樣的:喇嘛和出家人們會在清晨三點鐘左右起床,然後各自開始打坐和誦經。六點鐘左右,他們快樂地聚在院子裡供洗刷的地方,這裡的水龍頭也是唯一供水的水源。接著,他們會各自回到寮房吃早餐。早餐一般是藏式酥油茶拌壓碎的玉米或大米。當時隆德從慷慨的國王處獲得不丹紅糙米,因此他們會較其他吃當地白米的僧侶更為健康。儘管如此,他們當中仍有百分之八十的人患有肺結核病,而且全部人都營養不均,缺少維生素與蛋白質。當他們從家人或功德主處獲得金錢時,他們總是會去購買烘過的新鮮糌粑。糌粑可說是他們最喜愛的食物。

早上七時正,眾僧侶進入寺院大殿誦經共修。眾人集合共修的力量為一切眾生,為特定的家庭或個人祈福除障。他們通過專注持誦諸佛的經文和心咒與甚深的禪定來激活證悟的能量,而且所祈求之效果往往會以意想不到的速度彰顯。此過程稱為「法會」(梵:Puja,藏:Cheupa),意即「通過供養承事祈請」。供養可分為外、內、密三個層次和虛空供養。只有捨下我執才能領受佛力的加持。法會中所響起的每一個音符都是一種意念的表達,與身體中的覺性產生共鳴,讓我們的心從分別概念的思維中解脫出來。心的界限一旦瓦解,大樂便即刻現前,深層的淨化因此而產生。當一個人在法會上強烈的音樂中找到自己的生命力時,內在強烈的證量便會恆常現前。

到了午休時間,寺院裡傳來一陣陣的琅琅聲響。小童學習拼字,較年長者則在背誦各種經文。一般的現代教育著重於個人的獨立性、理解能力以及對資訊的迅速掌握。然而,西藏人卻知道更好的學習方法和更好的教導方式。他們在前三年用背誦的方式學習,先打好根基,讓未來的學習更有次第的增長,在生活上對他們有所幫助。儘管當地疾病猖獗,他們當中有的人又經常受到一些惡和尚的殘酷毆打,外在的生活環境條件悲慘,可是這些兒童卻比我們所見來自其他文化背景的孩童更顯得和睦。

中午時分,眾僧相續進行其他法會,或分擔寺院內外的雜務。他們以木刻版一頁一頁地印書;他們製做草藥、教學。當夕陽西下,長號角聲再次響起,眾人又分別繼續修持個人的晚課,回到寮房裡靜坐禪修,他們當中根基較好者甚至能將禪定延長至睡眠中。

當噶瑪巴蒞訪時,這些日常的規律便會完全亂了套。除了訪客川流不息,寺院裡從早到晚都有各種儀式和灌頂在進行著。不過,西藏人的應變能力實在是令人佩服。即便是針對一個很特別的儀式,一切已準備就緒,他們對於突如其來的發展和變化都能夠從容且毫不猶豫地作出改變和調整。他們總是能讓所眾人覺得是特別為他們所作出的改變。多年的禪修經驗使他們消除了慣性的思維,變得不受時空所限制。

漢娜與我希望能擁有更多自由,尤其是時間上的自由。我們希望和噶瑪巴在一起的時間能長一些(最好是能在他身邊待一輩子),只是外在的條件並不允許。印度官員達斯先生一天內打了好幾次電話過來。他要我們立即離開錫金。在此之前從未有外國人在錫金待上超過七天的時間。一般人在三天之後便要申請展延簽證。從寺院沿著蜿蜒的道路北上二十公里是乃度拉山口(Nathu-La)。從那裡可通往西藏的準比山谷(Chumbi
Valley)。那裡有許多中國軍人駐守,不久前才爆發了一些小衝突。邊界的印度軍人遭到中國軍人相當無禮的對待,導致他們非常急躁。個性開朗,操得一口流利英語的吉梅醫生嘗試利用自己的影響力和相關要員交涉過幾次。他告訴對方說噶瑪巴即將啟程前往不丹,屆時我們也會隨行。可是無論他說甚麼,始終無法改變結果。

漢娜和我才不屑去理會那些政府官員。如果我們繼續逗留下去,他們也只能將我們逐離錫金,但前提是他們必須先找到我們。不過,西藏難民在印度沒有法律權利,在印度人面前又處於相當不利的位置,作為這些難民的客人,我們也不得不違背意願遵從他們的意思。可是一想到必須與噶瑪巴分開,即使只是短短數天,我們仍感到萬分不捨。當時的我們非但沒有把握機會學習忍辱,觀察自心,反之卻不斷埋怨這些印度官員,認為他們的無能已經到達了一個全新的境界。當我們在寺院頂樓噶瑪巴的寢室內與他道別時,他所說的一番話激發了我們的慈悲心。他說:「因為無明,所以他們才會拆散上師與其弟子。這樣的舉動會在日後帶來麻煩。他們對此是毫不知情的。」噶瑪巴臉上的笑容更是直接觸動了我們的心弦。他繼續說道:「就以我為例好了。我每天都在為毛澤東祈福,他很需要幫助。」後來我們約定在卡林邦見面,然後我們將負責開他的其中一輛汽車或貨車一起前往不丹。我們會先在卡林邦申請不丹護照,因為我們一旦持有不丹護照,以後若前往印度或其他喜馬拉雅地區時便不再需要簽證,所有阻礙我們留在噶瑪巴身邊的障礙將會一一結束。

在位於錫金和印度之間的邊界小城朗博(Rangpo),我們逾期逗留讓當地的官員感到非常生氣。在那一個動盪不安的年代,戰爭可能隨時爆發,多疑的他們總以為處處都有間諜勾串之虞。在連綿不斷的雨中,我們並沒有渡過堤斯塔吊橋,離開此嚴受管制的禁區。反之就在抵達橋口之際,我們截停了一輛計程車,取左道開上了蜿蜒的山路前往卡林邦。聽人們談論卡林邦,口吻裡盡是唏噓。直到一九五九年西藏遭入侵之前,卡林邦一直是西藏、尼泊爾、錫金、不丹和印度在經濟與政治上交手的相匯之處。儘管如今中國已關閉通往西藏的乃度拉山口,卡林邦依然是喜馬拉雅東部地區最多事物看、聽、購買和體驗的一個地方。此地也住了一名在廿四來不曾離開其住所超過七步的華裔瑜伽行者。當他初抵卡林邦時,其居住的地方是卡林邦郊外一個環境清幽的住宅區。不過隨著這一個小鎮不斷開發,如今其居所周遭已多了成群吵鬧喧囂的兒童。這名瑜伽行者不時與世界各地的人保持聯繫,那一本我們在尼泊爾時所閱讀的度母法本正是他所贈送。由於噶瑪巴與隨行喇嘛至少還需要一天方會抵達卡林邦,因此我們仍有時間去拜訪這名華裔瑜伽行者。

走在鎮上,我們一路避開了辛苦傳教的基督教與印度教傳教士。試想想一個沒有傳教士的宗教!這名瑜伽行者在他的小黃屋前接引我們,甚至還向我們多次鞠躬。以前走私金條到印尼時曾經感受過類似的氛圍。這裡和新加坡的那種古早味中國風情如出一轍,到處都有小相框和手工藝品的擺設。這名華裔瑜伽士個子矮小、圓潤,充滿著活力,可是卻看不出他的實際年齡。後來,我們才聽說原來他已經七十歲。他告訴我們幾則非常奇怪的故事,關於他如何在一個道教風行的社會追求佛教。說故事的同時,他健步如飛地在工作室內來回走動,四處從箱子裡、從架子上掏出一些小本子。他顯然寫了不少作品,如今就疊擺在我們面前。當時,他所說所寫的東西,對於我們那「前嬉皮」,最近卻剛「矯正」過來的心而言甚是一種滋擾。這些小冊子將多層次的佛法在同一個時候全體展現,這其中包括道德主義者所誤會的部分,以及需要更多前行準備及額外口訣指示的部分。此外,直到我們對金剛乘佛教的開展負起全然的責任之前,我們會受他對幾位喇嘛的批判所影響。由於我們看不清這些僧袍與笑臉的背後所隱藏著的古老的政治問題,我們一廂情願地認為所有法教如此圓滿,上師也必然同樣完美。無可置否,西藏是一個中世紀社會。西藏社會裡沒有所謂民主、自由投票、透明制度、人權或新聞媒體。眼看西藏周邊國家如斯殘暴不仁,西藏境內又存在如此顯著的社會不平等現象,在任何情況下,我們其實不難想像在這個地方可會出現多殘暴的處罰,背後的政治紛爭又會有多齷齪。虔誠心的力量是如此強大,但是當我們在喜馬拉雅的時候,幾乎沒有想到可用業力的義理為眾生帶來改變的契機。幸好,我們從未聽說過任何關於噶瑪巴的閒話。有一次,這瑜伽行者陳氏被一條狗咬住了頸項,「噶瑪巴千諾」[註:,「噶瑪巴千諾」乃噶瑪巴心咒]讓他死裡逃生。

這些事情有很多是在後來的拜訪中才得知的。我們第一次見面時,瑜伽行者陳氏對我們而言只是牆上那一些精緻的小唐卡的主人而已。所有的唐卡都非常精緻,描繪著我們未曾見過的許多佛聖像,諸佛聖像的加持力更是將我們重重包圍。他也是一名說故事的好手,我們對他的了解也逐漸增長:當他出生時,母親的乳房上多長出了兩顆乳頭為他哺乳。這裡更深層的意義其實是指他領受宇宙四方的加持。

年輕時的他曾在中國當老師。當時,他經常害怕自己會遇上非時死亡而十分惶恐不安。他曾經花多年時間的修練道教的養生延壽之術,不過後來卻開始追隨佛教,稍後更前往西藏修行。他在康區(Kham)遇見了著名的法國婦女亞歷山大‧戴維德‧尼爾(Alexandra
David-Neel)。她可說是第一批入藏,而且又寫了不少關於修行上的神通等甚具啟發性的宗教書籍的西方人之一。陳氏沒提及當時在西藏梅毒肆虐一事,可是他卻不斷強調這名法國婦女不願意與她到山區裡拜訪的那些瑜伽行者們進行雙修。他覺得很可惜,否則所有人都能速證菩提也說不定。他也告訴我們有關自己在西藏某個山洞內閉關修行的故事以及前往拜訪他的婦女們,感慨現世「空行母」(覺悟的女性能量)真的是可遇不可求。

當我們仍在聊天的當兒(他的英語既古怪又逗趣),他喚了某個在窗外看熱鬧的小男孩進到屋裡來。外頭有一群人不斷盯著屋裡這奇怪的男人與他那奇怪的訪客狂看。陳氏給了小男孩一些錢,讓他去買些「莫莫」(momo)回來。

我們都知道「莫莫」。麵糰擀皮以肉碎包餡,再上鍋蒸熟了,就是「莫莫」(註:中國人的蒸餃)。西藏人非常喜歡吃。從進行療癒的那天開始迄今,我們已經茹素三年,此時並不想破戒吃葷,於是便告訴陳氏我們不吃肉餃子的原因。我把小男孩喚回,跟他說不必買了,不過又聽見陳氏跟他說了甚麼「莫莫」的,接著小男孩又跑了出去。我們以為他讓小男孩去買一些起司包餡的無肉餃子,不過小男孩拎著回來的很明顯是肉餡包餃。我們再次向陳氏強調說我們不吃肉,他卻回答說以前佛陀是施主供養甚麼,他就吃甚麼。絲毫沒有扭轉的餘地。我們心想這次一定會失去所有的加持力了,麻煩也必定接踵而來。當我們想到這些動物們在被屠宰前,因為極大的恐懼(它們怎麼看也不像會安祥地死去)所以大量分泌的荷爾蒙會殘留在肉裡,我十分擔心吃了這些肉後自己較具侵略性的一面會因此被喚醒。最重要的是,我們不禁猜想手上的銀製手鐲會否因此而失去了療癒的能量。

在哥本哈根第一次的療癒發生前的數小時,我們因為受到「先知」哈拉爾所啟發,因此將兩片火腿留在餐盤上不吃。當時我們嘗試要戒肉一個月。就在第一次的「療癒奇蹟」發生後,我們確定這一定是和戒肉有關,自此便決定轉為素食。此外,我們亦對白度母(我們殊妙的女護法)感到愧疚。我們正將祂所護佑的眾生吃進肚子裡呢!出乎意料的是在吃肉之後,或許除了晚間的睡眠變得更為深沉,自覺性較低以外,似乎對我們的身心沒有其他不利的影響。

接下來幾個星期,我們避免吃肉,甚至要求噶瑪巴授戒讓我們遠離肉類,奈何他怎麼也不肯。他顯然是希望我們能靈活變通。如今我們就如世界各地的佛教徒一樣:坦然接受所施予之物,私下儘量少買肉類食品。我們絕不允許任何動物因為我們的口腹之慾而遭宰割,絕不讓自己成為殘酷的殺戮背後的主因。我們可以通過對盤上食品祈願念咒「嗡阿彌疊瓦舍\textsubscript{利}」,幫助這些動物與屠夫,令之在未來世能投生善趣。如果念咒時距離動物往生該日未逾七周的話,這一個動作將對他們大有助益。

在卡林邦,我們下榻的貢普斯酒店不但地理位置優越,環境也想當優美,從那裡可以俯瞰所有進出鎮裡的公路。酒店的主人塔希與其妻子和母親和各區的人都有些交情,因此若有甚麼事情發生,他們必定會擁有第一手消息。塔希的上師是一名蒙古籍,年逾百歲的喇嘛。他對上師非常虔誠。這名蒙古籍的喇嘛就住在一座小山坡上,據說他擁有特殊的療癒能力。我們想要前往拜訪,順道去看看蓮花生大士閉關的洞穴,不過由於噶瑪巴的車子隨時會抵達,所以未能成行。從早晨開始,我們便帶著所有行囊,躲在酒店入口處附近,一來是要避開警察,二來只要車子一到達我們就可以馬上跳上車疾塵而去。一個下午晃晃悠悠又過去了,沒人出現。這次似乎不像是「西藏人慣性耽擱」這麼簡單。儘管要順利聯繫錫金的機會渺茫,但是我們還是嘗試撥了一通電話到錫金,而且出乎意料之外的是電話馬上就撥通了。電話的另一端傳來Jigmela嘹亮的聲音,他也確定了我們今早就已聽說可是卻不願意相信的一個謠言:連接錫金與卡林邦的公路又再次被大雨沖走了。噶瑪巴前一晚已從另一條地勢較高的軍用公路出發。按時間計算,他現在也許已經抵達不丹。

這意味著若我們想見噶瑪巴,那麼我們就必須非法入境不丹。不過似乎也沒什麼差別,不丹本來就不對外國人開放。若按當初的計畫充當噶瑪巴的司機,那麼過境時鐵定不會有人阻攔。只是現在計畫有變,這下我們也只能自己想辦法矇混過去。整個情況因此開始變得微妙地有趣刺激。

我們原本光禿禿的頭上又再次長出了一些頭髮,加上曬黑的膚色,我們和當地的康巴看起來沒兩樣。此外,我們也穿上了藏式傳統外袍「楚巴」(Chuba),巧妙地掩飾了我們的真正血統。如今,我只是要小心我的藍眼珠不被發現,否則麻煩就大了。當我們登上那一台即將從卡林邦開往彭措林(Phunchoeling)的巴士時,司機先生露出了奇怪的神情打量著我們。彭措林是我們進入不丹都城聽布(Thimphu)之前將會路過的第一個邊陲小鎮。幸好司機先生是西藏人,和他溝通好後,他答應不會向邊境守衛洩露我們的身份。

那一條地勢較高的軍用公路穿越一座座風景如畫、綿延起伏的山巒,路經破落的軍營後抵達山口。之後,道路蜿蜒下山路經一片一片濃郁蔥蘢的草木植被,儘管此區域非常潮濕,這些植物卻看起來像是典型的地中海型灌木叢。巴士沿路進入低地,一直順利地向東前進。我們穿越一片片無盡的茶園,茶園裡的平頂合歡樹綠蔭如傘,視野的左側是連綿起伏的山麓小丘。當巴士抵達第一個檢查站時,我們躲在椅背後。我們一上巴士就挑了較暗且又不易被發現的後座。結果,檢查站裡的印度官員只是在門邊探了探頭就讓巴士開走了。第二個檢查站的情況較緊張。幾名海關人員上了巴士,他們其中一人沿著狹窄的通道走向巴士末端。我們儘量卷縮著身子保持低調,佯裝暈車導致身體不適。當他走到我們的座位旁時,我彷彿能感覺到他的氣息就打在頸項上。不過對方似乎對架子上的行李較感興趣。在昏暗的燈光下,他並未注意到我們的膚色較淺。當他轉身離開後,巴士準備行駛,我們帶著成功矇混過關後的一陣狂喜,隨著巴士浩浩蕩蕩駛入了不丹境內。

抵達彭措林時天色已晚,開往都城聽布的巴士都已停駛,據說抵達聽布之前一路上還會經過七個檢查站。我們認為應先在彭措林待一晚,隔天再去辦理所需的入境文件。在這一個國家,人人都對噶瑪巴非常虔誠,所以我很願意奉公守法。我甚至還作了一個夢,夢見自己開著一台官用吉普車上山,看看多快能抵達都城聽布。

決定留宿不見得是個壞主意。打從我們越過邊境開始,某些較重的業報顯然已經成熟。漢娜突然病得很嚴重,我們找到留宿的地方後便卸下行李、掛上蚊帳酣然入睡。漢娜在半夜醒過來,迷糊之中看到有一個男人正在翻我們的背囊。漢娜顯然已將「勿以小人之心,度君子之腹」的教導深植在心裡。她沒有把我叫醒,迷糊間甚至責怪自己胡亂猜忌別人會居心不良想偷我們的東西,她以為那只是自己發燒腦袋不清楚亂做夢,結果不予以理會又再次沉沉睡去。

隔天早上醒來,只見行囊裡的東西有一半攤了在地板上,另一半則在窗戶外面。我們把東西收拾好後,發現隨行所攜帶的器材並未遺失。甚至那台舊佳能相機和彩色膠卷也沒被拿走。這些東西在此地可算是相當貴重的東西。只是有兩件物品不見了:噶瑪巴頂戴黑寶冠的照片,以及對我們護佑有加的白度母唐卡。噶瑪巴的照片是我們從卡林邦11英哩的「OM」照相館購得。我們經常購買這一類型的法照寄給朋友,或當成禮物送人。遺失了噶瑪巴的相片已經很糟,不見了唐卡更雪上加霜。這一幅唐卡活靈活現,栩栩如生。白度母(藏:Dolkar,
卓噶)乃為女性覺者,是觀世音菩薩慈悲眼淚的化現,祂的能量場在我們的時代非常地活躍和強大。

白度母以七眼凝視無垠虛空。七眼象徵佛母能夠觀一切眾生的苦難,並以無上智慧救度他們的能力。除了臉龐上的一雙眼睛,其餘五眼分別長在額頭、手心,以蓮花坐姿朝上的腳板上。度母的右手在膝前結施願印,左手當胸捻著蓮花。蓮花綻放於耳際,表徵純淨的本性。

遺失了這一幅唐卡尤其讓我們覺得心疼。雖然那只是度母聖像的簡單速寫,在我們西方人的眼裡看來,畫中度母的臉部比例是如此地完美。畫這一幅唐卡的人是隆德寺裡的一個老畫匠。老畫匠天賦異稟,只可惜已漸漸面臨失明。如今,老畫匠已經架著兩副眼鏡在頭上;不久後,也許他只能單憑「心眼」來作畫,或由其他弟子來秉承他的技術了。這一個奇怪的賊/密探(?),管他是誰,他的舉動可真讓人覺得匪夷所思。我們甚至覺得這些照片和畫像現在搞不好就在山上,在噶瑪巴那裡。依據金剛乘佛教的說法,「塞翁失馬,焉知非福」,若遺失此類型的聖物一般會被看作是消災擋劫,在佛力的加持下躲過了像是意外事故或疾病等磨難。

我們向人打聽了當地SDO(搞不好他是DSO也說不定)(註:SDO與DSO乃當地政府官員的職銜)的住所。我們總是搞不清楚這些前英殖民國所使用的職銜縮寫,聽起來是如此華而不實。不過,我們其實也能夠明白這些頭銜對當事者而言又有多重要。

第二天清晨,天色仍早,我們已來到這名官員的家門口。我們天真地以為早前成功矇騙邊界的印度官員混入境一事必定能惹得對方大笑,接著他會熱心地替我們辦理入境所需的文件,最後再請我們替他向噶瑪巴問好,然後送我們離開。我們甚至幻想自己會開著吉普車上山到聽布去。當這名官員睡臉惺忪出現在自家洋房的大門口時,他的第一個反應卻和我們所想像的相差十萬八千里。他臉上的表情彷彿寫著「我應該只是在作夢」般驚愕不已,回過神來後,他大聲囔道:「你們怎麼會在這兒啊?」這名SDO看起來就像典型的孟加拉印度人(這倒是出乎我們意料之外),不過如今我們確實是身處在不丹境內沒錯。我們向他解釋說其實我們是噶瑪巴的學生,因為在卡林邦錯過了,所以他邀請我們前來聽布會合,因此希望能夠獲取在檢查站所需的通行文件。這名官員漸漸從錯愕中恢復過來,熱心的他亦有意幫忙。只是上頭有令禁止白人上山。他說,若他們在更高處發現我們的話,大概已將我們關進牢裡了。他答應幫我們和聽布當局聯繫,只是眼下此事不易辦妥,因為政府部門裡的官員大概都已休假參加噶瑪巴的法會去了。他說,直至將所有事情弄清楚之前,我們將會以國賓的身份待在鎮邊上緣的賓館裡。他們為我們準備了一間大房,房外不僅是一片美麗的草坪,還能將延綿山坡的美景盡收眼底。他們也提供了一切物質上的享受。這裡有幾名尼泊爾籍的傭人將房裡上上下下擦拭得一塵不染,更費心為我們張羅一天三餐,這一切比起這些年來我們為自己準備的有過之而無不及。儘管我們過著像有錢人般的生活,可是心思卻始終無法落定。

我們的全副心思都在噶瑪巴的弘法事業上,心裡也清楚知道山上正在進行著很重要的事。正當我們每天受困此地動彈不得,只能焦灼地等待相關當局發下邀請函的同時,噶瑪巴在山上一定傳授了許多很有意義的灌頂。由於我們已經開始認識這些儀軌的意義與其所帶來的影響,因此錯過了更覺惋惜。以前我們純粹只是參加一個又一個的法會儀軌享受瞬間的快感,卻從未預想過它們會帶來長遠的影響。

所謂「灌頂」(藏:Wang),意指力量與力量的傳送。在使用或未使用法器的情況下,上師進入禪定,啟動無限佛性的某種功德和能量,然後傳授給受灌者。這種傳承加持力是代代相傳、不曾間斷的;一個清淨無間的傳承加持力是最大的禮物。受灌者與所處的世界將會有所提升,直至能達證「任運生起」,不假造作的證悟境界。屆時,我們將與佛合而為一,一切將會超越時空而圓滿無礙地自然現前。

通過灌頂,開悟的種子將會種入眾生的潛意識中,加上金剛乘見地上的修持以及有關禪修的訓練,就能夠完全開展心性的潛能。解脫之樂即是最大的安樂。這一世雖然無法具體地開展自心,可是所領受過的灌頂卻無論如何也不會白費。眾生在過去世通過灌頂所種下的菩提的種子將會在死亡之際或未來世一一從潛意識中再次浮現。這一種強大的力量能夠轉化業力,讓眾生在下一世與金剛乘結下法緣。這種能量場無論在任何狀況下都能運作,生生世世都能漸次增長,直到完全成佛為止。即使不明白其意義,只要有智慧,多參加這些佛教儀軌也是好的。然而,若無法(或無意)遵守某些灌頂的戒律,就不應該領受相關灌頂以免破戒。此外,把許多傳承的灌頂混亂的話也會帶來迷惑。所以在不看低其它傳承為前提,你必須一心專修與自己最具法緣的傳承。

當我們待在賓館裡享受著優質服務的同時,快樂的不丹人每天上山去見噶瑪巴時都會路過這裡。我們托人給噶瑪巴與吉梅醫生帶信,同時也嘗試打電話聯絡聽布的相關部委。當時整個不丹政府部門只有三部電話,各部門之間也沒設有分線。因此當我們終於撥通電話時,接電話的是西藏難民部。我們還未來得及說出我們的意願,電話就被掛斷了。

日復一日,我們越來越不耐煩。漢娜很不開心,而我也是極度焦躁,內心裡負氣到了極點。我們那狂妄的自尊心怎麼也不願意面對現實。我們貴為噶瑪巴「萬中選一」的弟子千里迢迢來到不丹就只能乾坐在一旁,白白浪費許多寶貴的時間。對此噶瑪巴似乎一點也不在意。我們心裡實在是咽不下這一個事實。雖然心裡從未停止這麼想過,可是我們也無法真的就這樣步行上山直抵聽布。我們曾向該名官員保證會乖乖在賓館裡等消息,我們也把護照交到了他手上。一個星期後,他告訴我們說,他無法留我們在彭措林。聽布方面遲遲未有消息,我們不得不返回印度繼續等待。

如今我們能做甚麼呢?我們身上只剩下約二十美元現金,又沒有錫金簽證,可是我們真的不想折返加德滿都。當我們拖著沉重的腳步,帶著極度鬱悶的心情離開彭措林時,錯愕的印度邊界的士兵不斷探頭過來。在官方程序上,他們壓根不知道我們的存在,即使看到他們的反應,我們一點也不覺得好笑。將這種過去曾幫助過我度過難關的力量付諸於行,我說:「我們直接回丹麥吧!這真的夠了!」其實我們誰都不願意回去。我們千里迢迢來到這兒,目標尚未達成又如何能離開呢?我們也尚未正式展開修行。更糟糕的是,我們並沒有一個清楚的方向,沒有任何東西可與一直等待著我們的朋友分享。儘管我們不時因為各種加持而感受到瞬間的快感,實際上我們心裡都清楚,我們在實質上根本就沒有任何改變。欲馴服自心的這個目標,我們尚未達成。

如今還剩下甚麼可能性呢?除了打道回府,再次回到哥本哈根賺錢,與人分享這一路上的種種經歷,我們還能做些甚麼呢?我們站在邊界旁灰塵滾滾的馬路上,此時有一輛豪華巴士開了過來。這裡實在不像是會出現這種豪華巴士的地方。旅巴裡坐滿了富有的法國遊客,他們每個人都穿戴著最時尚的服飾與珠寶。原來這一群遊客從加爾各答出發,前往盛產茶葉的阿薩姆旅行,如今他們也想見識見識「時髦」的不丹。印度邊界那裡的印度人先前因為我和漢娜的「突然」出現已經引起了騷動,現在又來了一巴士彷彿是來自另一個世界的入侵者的外國旅客,更是把他們給嚇壞了。這些駐守邊界的士兵除了大聲叫囂之外,同時也將槍頭瞄準了巴士,試圖驅逐他們離開,而我和漢娜則趁亂請求他們載我們一程。善良的他們讓我們上了巴士送我們一程。這是一段奇妙的旅程,我們就這樣莫名其妙地坐在了一輛舒適且引擎運作良好的旅巴內,和一群陌生人談論著文明卻毫無意義的話題。我們坐在他們之間,省思著這一輩子所奮鬥的種種事物。幾個小時後,我們在西里古里的火車站下車,如今我們有兩個選擇:從西里古里出發前往德里,再搭飛機回家,或前往大吉嶺。我們可以很肯定的一件事是:我們不回去歐洲。我們無法就這樣離開這群西藏人,無法就這樣離開噶瑪巴。

我們在大吉嶺等信,也打算給父母捎一封家書。由於我們不再靠走私金子或大麻賺錢,父母親知道我們留在東方國家純粹是為了學習,所以也很樂意給我們寄錢應急。如今我們要求他們每月給我們寄大約五十美元,這一個數目對他們來說不多,不過卻足以應付我們在印度的開銷,甚至有剩餘的可與他人分享。在等錢寄到之前,如果運氣好的話我們將能靠身上僅剩的這一點錢過活。其實吸引我們前往大吉嶺的最大原因是先前在斯瓦彥布某個露天派對中聽說的某個傳言,漢娜恰好在此時想了起來。在靠近大吉嶺有一個叫索納達(Sonada)的地方,那裡住著一名年老的高僧。這名高僧是在噶瑪巴的允准下,第一個向西方人教授傳統金剛乘法教的西藏上師。雖然只是在吞雲吐霧的一片迷茫中見過他的照片,卻也足以讓我們著迷不已。如今他的名字又再次在腦海中浮現。他就是卡盧仁波切(Kalu
Rinpoche)。

\textbf{第十一章}

\textbf{向卡盧仁波切學法}

抵達大吉嶺後,我們換上最體面(平常只會在特殊場合穿的)一套衣服,前往申請學生簽證。當時外交部的警員相當厭惡嬉皮士,我的短髮白襯衫,加上漢娜清麗的面孔讓他留下挺好的印象,因此沒有阻攔我們提出申請。德里腐敗的官僚制度十年如一日,這意味著在他們發現我們的存在之前,我們至少擁有六個月的自由時間。之後,他們會需要更多時間才會發現領著學生簽証的我們其實只是留在限制區與一群西藏人一起生活,而不是花錢報讀印度國內任何的大學。最後,當決定性的時刻來臨時,我們也知道如何拖延而不被趕走的方法。這樣屈指一算,無論想做的事是甚麼,我們也已為自己攢了不少時間。

我們下榻的小旅館就在郵局的附近,一天收費住宿包早餐才不到一美元。沒過多久,我們便發現其實這一棟低矮的木樓鬧鬼。旅館內處處充斥著十分詭異的能量。此書較早期的版本,我並沒有透露旅館的名字。一來不希望影響這位和藹的老婦人的生意,更不想將這家旅館變成旅遊勝地,把旅館內的幽靈嚇跑。如今轉眼廿八年過去了,時過境遷,說出名字也無妨。「Shamrock
Lodge」(新樂旅舍),是旅館的名字。我也不是說旅館內的鬼魂或靈異能量特別負面,「它們」只是非常迷惑,不過由於它們的能量很強,以至於有時會相當惱人。一些相熟的朋友曾說,它們有時會將桌上的書本推倒在地,又或者會將一些經文法本隨便堆放在地上。隆德的吉梅醫生曾經下榻此旅館一個星期,他每晚都得先把自己灌醉才睡。他總覺得那些鬼魂在猛擊他的背部。住宿期間,我和漢娜只是間接受到一點影響,它們似乎知道我這人會還以顏色,所以惹不得。只是有它們在身邊,其混濁的能量令人在每天早晨醒來時有一種空茫的感覺,加持的力量也消失不見。此外,當它們在周圍時,我們與噶瑪巴在一起時所感受到的大樂覺受便會削弱,我們非常抗拒這種感覺。除此之外一切安好,而且看女房東與她飼養的那隻黑貓咪說話也頗為詭譎有趣。

我們在等待著父母把錢寄過來,同時也忙於收集各種有關噶瑪巴的消息。此時,拜訪卡盧仁波切的契機終於出現了。我們的一名美國朋友曾經拜訪過那個地方,這次我們將與他同行。在晴朗清新的天氣下,我們打算走到目的地。在前往古木的前四英哩路,綿延起伏的干城章嘉山脈就在我們身後右側巍巍而立。接下來走向西里古里的五英哩路,印度一片片遼闊的低地平原就在眼前。就在離索納達不遠處,我們找到了卡盧仁波切的寺院。再往上走兩公里,山坡上的加油泵對面偏左側,矗立著一棟棟陳舊、漆成綠色的木屋。我們在路上遇見的第一個西藏人是一個非常討喜的年輕和尚嘉千,他主動提出要帶我們去見卡盧仁波切,順便充當我們的翻譯。他帶領我們走進一間長形的房間內,室內的牆板高及腰間。房間最內側,小木桌後方的床鋪上,坐著一個男人。那一張苦行者般的莊嚴相貌,讓人無法忘記。他就是大喇嘛卡盧仁波切。

雖然卡盧仁波切看起來就像是一個非常虛弱的老人,可是卻擁有很強的心力。實際上,仁波切並沒有外表看起來般虛弱。後來有好幾次我們曾開車載著仁波切到各地遠行,他比起其他年輕的喇嘛更能吃苦。一九八三年雨季,仁波切以八十歲高齡在五個月內傳授了兩千個灌頂,直到一九八九年圓寂之前仍然活躍於弘法利生的事業。我們向仁波切頂禮。雖然頂禮時仍會覺得很彆扭,可是我們不希望失禮隆德,同時希望藉機表現出我們有教養的一面。卡盧仁波切微笑著接見我們,同時也給了我們一個強而有力的加持。仁波切接著拿出一張世界地圖,我就像在亞洲其他國家時一樣,指出格林蘭和丹麥的位置,然後補充說人們都住在地圖上標示綠色的那一小塊,其餘白茫茫的一大片主要是雪地。我也向仁波切吹噓著丹麥人究竟有多強壯和堅毅。他提出許多和我們在尼泊爾與錫金的喇嘛有關的問題,對於噶瑪巴的消息尤其感興趣。當他問及我們希望逗留多久時,我們思考了一會兒,回答說:「直到噶瑪巴從不丹回來。」仁波切說:「好,你們可以留在這裡和其他人一起學習。這是我的姪兒嘉千。」他指著替我們翻譯的年輕人,繼續說道:「他會協助你們找住的地方。」

我們出去後,嘉千說有兩個美國人:蘇和里察,就住在山坡上數百米處一棟偌大的老房子內。他說屋子很大,一定會有多出的房間。一開始時,我們並沒有很想要住在索納達。這一個地方毫無魅力。寺院本身也甚無吸引力,整個地區就是缺乏了一種獨特的文化。索納達就只有一條破爛不堪的道路,以及兩旁遭雨水摧毀的木屋和小攤子而已。這一個地方也非常吵鬧。每當有小型的火車路過,尖銳的鳴笛聲便不絕於耳。而且在這個地方似乎所有東西都鋪著一層薄薄的煤塵。索納達不像大吉嶺能將干城章嘉綿延山巒的美景盡收眼底。最糟糕的是,這裡距離辛格先生和我們在東部唯一相信的郵局太遠。這裡的警察也不喜歡西方人與當地西藏人有太多的接觸。他們希望遊客們能安分守己,萬一有甚麼事不合乎心意或與其意見相左,他們甚至會以「取消延長逗留」為藉口偷偷恐嚇遊客。然而,我們也不忘卡盧仁波切的建議,爾後當我們覺得大吉嶺不再具有我們可以學習或對成長有所幫助的事物時,我們便往南搬到索納達,而且依然是向當局申報新樂旅舍的地址。

我們隨同幾名美國、加拿大和法國人(他們當中有六人旅居索納達,其餘者只是過路人)來到卡盧仁波切的房內,參與了第一次的開示。神奇的是,在場除了替我們翻譯的嘉千,便不見寺院裡或鄰近西藏難民營裡的藏民、當地的雪巴人或塔芒族人(Tamang)。這些族群沒有語言方面的障礙,一定都能聽懂仁波切的開示,可是卻半個人影也不見。他們對佛法甚具信心,可是卻似乎對深入學習佛法不甚感興趣,當時的我們對於這種矛盾的現象甚為不習慣。我們心裡亦清楚知道,像他們這種學習的傾向必定不能在西方國家裡發展開來。

仁波切的開示和地獄有關,各種不同的地獄。我們完全沒有料到仁波切會作出相關的講解。除了以前在上學時的基督教課(也許當時的我們曾乖乖上課,間中也許試過直接曠課),漢娜與我一直都在一個人文主義而非宗教的背景中成長。我們對生活感到相當滿意,不需要一些滑稽可笑的神靈,又或「人死後會墮地獄遭火燒」之類的恐懼為生活添味。這種所謂的教言對我們來說只是一種用來操控弱者,然後從中詐取金錢的卑鄙手段而已。

我們原以為卡盧仁波切會作出心靈方面更深層的開示,看他展示神通或如雷電般迅速的證悟的跡象,現在可尷尬了。仁波切坐在眾人面前,指著一個畫像幼稚的畫軸。畫軸上所畫都是眾生被火燒,遭切割,受群山所擠壓,或遭冰封於巨大冰塊中的種種情景。他說了幾則陰森森的故事,都和「八個熱地獄」、「八個冷地獄」、「鄰近地獄」和「間歇地獄」有關。這些地獄各各不同:在有的地獄中,眾生以不同的方式互相殘殺,一陣冷風吹過又使之復活,眾生又再次自相殘殺,同樣的過程不斷重複。在別的地獄,眾生被放入一個裝有金屬熔液的大熔爐裡,或遭蟲子所蠶食。這些地獄一個比一個痛苦,一個比一個噁心。卡盧仁波切還說,眾生一不小心就很容易墮入這些地獄中。這是很沉重的課題。我受不了了!西方國家的靈性書籍一致認為眾生若已為人,便不再會墮入動物、鬼或地獄惡道中,一切惡行只會令其「凍結」(受困)在同一個境界(人的境界)而無法有所提升而已。我向仁波切提出此觀點,並且告訴他在西方國家,無論是唯心論者、通神論者、人智唯靈論者等一概人都同意此觀點,不過仁波切卻回答說:「這是佛陀的教言。」

當時我們仍住在大吉嶺,當我們乘著老舊的吉普車翻越過古木山口時,我們不斷在討論這一位老喇嘛。我們認同他所說的故事充滿了異國的情調,然而這些傳統故事會更適合作為民間故事來傳誦,或者用來恐嚇或操控一些人,與廣妙精深的佛法根本搭不上邊。雖然我們知道他能看透我們的心思,想必也知道我們這裡的情況,我們依然忍不住會想:「也許他是真的老了。」

隔天,負責翻譯的是喜樂塔欽。他是美國某個非常富有的銀行家族後代,操一口流利的藏語。他和嘉千顯然是輪流上陣,替仁波切做翻譯。昨天的開示結束後,我們連忙去翻閱伊文斯‧溫茲(Evans-Wentz)的書籍,他的書是我們主要的信息來源。我們相當確定自己可以不必去理會這些地獄的故事。如果這天仁波切講一些有趣的課題,我們一定非常樂意原諒他老人家。怎知卡盧仁波切又再繼續解釋有關地獄的種種,這次的主題是地獄中的酷刑,以及受苦的期限。不一會兒,我們便開始覺得仁波切的開示枯燥煩悶極了,耳邊就只剩一連串天文數字在嗡嗡作響。

和我們一起參加開示的那幾個美國人、加拿大人和法國人並沒有像我和漢娜般對地獄的課題感到厭煩。就在我和漢娜都快氣得冒煙時,他們總是乖乖坐著聆聽,並且將仁波切的開示寫成筆記。雖然我們喜歡這名老戰士(仁波切是東藏的康巴),喜歡他那無懈可擊的臉龐和笑容,可是也忍不住捫心自問,自己是否真的需要這方面的開示。不過,我們希望再給他一次機會。隔天,我們依舊前往索納達參加開示,看看這次他會講解些甚麼。難以置信的是,關於地獄的課題還是沒完沒了。這天仁波切描述種種會令我們墮入地獄受苦的憤怒、惡行、起心動念與言詞。第四天,當仁波切又開始講解地獄時,我簡直是氣瘋了!

我們並不是為了好玩才這樣每天舟車勞頓來到索納達。從大吉嶺到索納達大概一個小時車程,我們就這樣每天往返兩地。我們搭乘的吉普車屬於半密封型,裡頭大概擠了約十五人。有的人半蹲,有的人身子一半是在車外,人人咳個不停。這些老舊的「路虎」(Land
Rover)或「威利」(Willy)吉普車皆由廉價材料所建造。吉普車的馬達已經磨損,加上使用的是第三流的機油,所排出的廢氣把人嗆得十分難受。每一段車程,吉普車都必須停下來好幾次進行不同的修理。這一路的顛簸,加上為了參加開示所花費的開銷,我們認為自己該聽一些較具吸引力,或至少有意義的開示。我打斷他的話,說:「我們已經聽過了。」他看著我,嘴角勾起了一抹諷刺的笑容,回答說:「沒錯,但你聽懂了嗎?」忽地,我才驚覺原來我們一直沒有領會他的意思。他說這些地獄故事,不是因為他對受苦的眾生無動於衷。他亦沒有妄下評價或欲伸張道德正義的意味,又或者想要藉所說的話獲得任何報償。他當然也沒有要操控任何人的意思。他講述這些有關痛苦的故事,是因為他真的相信,他更希望聽的人能引以為鑑,遠離麻煩。然而我們卻一直漠視仁波切的用心良苦,不但沒生起慈悲心,反而感覺厭惡,只想聆聽美好中聽的事。若一直這樣下去,我們將永遠無法進步,這似乎也提醒了我們過去到處漂流冒險,或不斷參加各種灌頂儀式的日子也該告一段落。如今我們應該通過實際的修行去完全吸收妙法的精髓。我們慣性地四處遊走,讓我們很容易墮入只吸取自己想聆聽的事物的陷阱,我們只是假藉靈性之名與自創的一套想法自我欺騙,避免去揪出我執的根源而已。

我們對卡盧仁波切作為上師生起了強烈的信心。仁波切絕不會縱容我們。他一定會教導他認為對我們最為有效益的教法。我們就在那個時候決定遷往索納達,全心全意追隨仁波切學習佛法。消除了對仁波切的抗拒心理以後,我們也較能理解他所講解的地獄了。地獄裡的酷刑不是壞心眼的天神給與眾生的懲罰,所謂地獄純粹只是一種負面的心理狀態。地獄是過去種種惡行的結果。過去我們協助不少人度過可怕的迷幻經歷時,不就曾見識過心識是如何將奇怪的事都當真嗎?既然人會受苦是因為種種自我造作的妄想所造成,那麼便沒有所謂「正常」狀況下的受苦,不是嗎?基本上,地獄不就是一種心理疾病嗎?雖然我們難以親見受苦中的地獄眾生,可是眼下不就有許多人生活在恐懼和疑慮的地獄中嗎?絕對的證悟之心,也會有著相對受限制的體驗,這種體驗是毫無堅固自性的。儘管這一切不外乎是一種心理投射,可是由於感覺是如此真實,所以才會招致痛苦。

現在這些觀點上的分歧似乎說得通了:西方人相信「人類的持續性進化」使到兩大領域產生了混亂。其一是身體上的遺傳進化。人類的身體確實會不斷產生基因上的進化,直到廿世紀初期青黴素(又稱盤尼西林)和機關槍的出現背離了「物競天擇」的原則而使之出現了大逆轉。其二是有關心性能量的成長。其實這一種對心性的檢驗應該更早──就從「我」的轉世投生開始(每一個人都認為自己有一個的獨立的「我」存在)。這應該包括在受胎之際,心和潛意識相應的環境與身體結合的過程。更複雜的是,我們必須明白在未來世中未覺悟的心會持續地認為自己的業力跟感官所呈現的世界是真實的。固此,佛陀才會告誡我們業力甚深不可思議,所以不要對業力產生不必要的妄想和推測。

然而,這並不意味著本性「明空」的心具有形體。就像那些相信輪迴的物質主義者所相信的那般,當意識離開身體之後,它必須和另一個身體相結合。就如《西藏生死書》中所講的一樣,此過程就像俄羅斯輪盤般不斷輪轉。當人在死後其粗淺的意識心消失時,細微的潛意識便會現前。其中最強烈的業習和動機將會徐徐將我們引入六道中相應的一道,然後就投生了。這一個過程就如虛空一般是無始的,但卻有終──成佛之後便會終止。這是一個必然而充滿痛苦的過程。

因此,我慢心重者,其異熟果報是投生天道;妒忌心重者則其異熟果是投生阿修羅道;執著心重者若具足福德則會投生人道;愚痴的異熟果則是墮畜牲道;貪心重者則其異熟果報是墮鬼道;而嗔恨心重者則其異熟果報就是投生剛剛仁波切所說的地獄之中。對覺悟者而言,這種種的過程不外乎是一場夢境,空無自性,如遊戲般幻現,充滿無限的可能性。從純淨的層次而言,所有現象的顯現都是任運無礙而圓滿的──不過對於追逐這些境界著而言,一直以來都是充滿痛苦的。就算是在受限的因緣中現起的大樂,其實也是自心本性中大樂的其中一個微弱的影子而已。就算如此,大多眾生就連有限的快樂也時常是了不可得。

山坡上的「克里斯大宅」仍有空房。這一棟以都锗風格建造的巨大木樓是由一名義大利籍的百萬富翁所建造。據說,這位百萬富翁想要在地勢較高處生活以治好自己的肺結核病。殊不知大宅尚未建成,他就已經病逝。如今住在那裡的是一名英國牧師的尼泊爾籍年輕寡婦,她也和來自兩至三個不同世界的人們住在一起。大宅的底層和周圍的小屋住著年輕寡婦的親戚,大家都維持著尼泊爾傳統的生活型態。在如此偌大的家庭中,他們彼此皆以兄弟姊妹相稱,不易辨是哪家的孩子。他們的父親,通常是休假時來到此地的廓爾喀士兵,不過若要追查下去,恐怕也沒人能確切知道孩子的父親究竟是誰。無論如何,底層總是熱鬧轟轟的。這名年輕的寡婦替牧師生了一個兒子。這兒子是一名搖滾音樂人,無論是穿著或思想都在追尋歐洲的時尚潮流,只是他的「潮流」總是免不了像印度人般遲了半拍而已。最後,大宅的最高層,這裡大部分的空間都被我們這群口中總是唸唸有詞,持誦著咒語的西方人所占據。

此時,父母親的首一百五十美元已經寄至。附近的銀行只願意以一美元兌六盧比,大吉嶺的自由市場兌九盧比,不過由於在加爾各答能獲得一美元兌十二盧比的高價格,因此我們便展開了往返索納達---加爾各答的快速行程,為了節省時間,我們每次都會選搭夜班火車。像這樣的一日行也成為了往後幾個月的常事,而且這些行程其實也成為了我們在學習上相當重要的補助。我們從平靜的卡盧仁波切的寺院,來到了一個荒涼喧囂的城鎮,娑婆世界裡的痛苦在此地尤其明顯可見。回到仁波切的開示,我們只能贊同說:「真的是如此的。這些痛苦不斷延續著。我們剛剛都親眼看見了。」

我們每次都會儘量小心拿捏行程,等到仁波切講解至各種地獄的結尾時才出發,期間會缺席幾個法會儀式,然後趕在仁波切開始講解有關「餓鬼道」和「畜牲道」之前回來。餓鬼道眾生因為執著心與貪心重,所以才會墮入此道受苦。餓鬼眾生缺他同心,心路過程由貪慳所主導:它們總是努力想要得到或抓住甚麼,不過最終由於無法真正「擁有」或「保留」甚麼,因此總是感到十分沮喪。它們非常執著於吃和喝的東西,這亦對感知造成扭曲。許多餓鬼眾生因為業力所致,只要接近食物,眼前的食物便瞬間化成了穢物和火焰。有的餓鬼眾生會則產生假象,其喉嚨和嘴巴如針孔般細小,再怎麼吃偌大的肚子也無法飽足。有關餓鬼道眾生的種種描述與博斯(Bosch)和勃盧蓋爾(Breugel)所畫的畫像十分吻合。仁波切在這方面的開示非常湊效。直到今天,當我們想要購買任何東西時仍會很自然地先問自己:「真的需要嗎?」

眾生由於無明,自覺性地扭曲事物,沒有善用內心的潛能,因此才會投生畜牲道。家畜最主要的痛苦來自於人類的奴役和屠宰。至於生活在大自然界中的動物則免不了互相殘殺和吞食。大部分畜牲道的眾生散居於土地裡或水中。它們沒有能力控制自心,也苦無追求解脫的機會。動物、餓鬼和畜牲道統稱「三惡道」。它們皆為心識中負面印記的異熟果報,因此應避免種下投生此三惡道受苦的種子。這三種境界實在沒有任何值得我們去追求的事物,所以應敬而遠之;因為一旦下墮三惡道就難以出離,亦將苦無機會自利或利他。

卡盧仁波切接著告訴我們說:「即便是三善道的眾生也受苦。」「六道中沒有絕對的庇護之處。」當所主導的欲望和過去世所累積的善行令眾生得以投生人道,八苦亦自然會隨之而至。其中最明顯的乃為生苦、老苦、病苦與死苦四大苦,以及另外較「間接」的四苦。他們分別是愛別離苦、怨憎會苦、求不得苦及五蘊盛苦。仁波切很仔細地解釋了每一種苦,告訴我們這些種種被我們視為是很「自然」的事物其實是無法令人滿意的,它們其實都是偽裝了的痛苦。他說:「天人和阿修羅也不值得羨慕。」阿修羅因為過去生生世世的善行所以具有莊嚴妙好之身軀和神通力,不過他們強烈的嫉妒心使到他們的世界中充滿戰爭和打鬥。他們永遠不滿足,而且往往在極憤怒和嗔恨的狀態下往生,因此下一世的投生一般會變得異常痛苦。

即使是娑婆世界中最為喜樂的「天道」,快樂也是有限,因為強烈的「我」見依然堅固。不管是色界、無色界的天道眾生,儘管只要心有所想就能心想事成,享受各種天樂而舒心適意,可是也不會是永恆不變的。天人壽命雖長,但亦有盡頭,一旦福報消盡後就會再次進入輪迴,墮入惡趣。

輪迴中的各道皆為如此。佛智慧能消除由無明引起的二元對立心或分別心,所以在達至證悟之前,眾生忘卻自性本空,心中存有強烈的「我」見,因此才會生起主體和客體(亦即你我)之分別。由於對體驗與所體驗之事物有所分別,進而導致如是情況:對於喜好之物生起貪著,反之便生起嗔恨心。由於所有人都想留住美好的事物,因此欲望很自然便會導致貪執心的生起;而另一方面,憎恨則導致嫉妒(因為任誰都見不得敵人好嘛!)愚痴引起我慢心,讓人覺得自己較他人優秀,結果導致孤獨和頑固。我們以為自己等同於那些無常變幻的意念、身體和財富,因此這種認同感矇蔽了自心的光明本性,如果是這樣的話,即便是天人也會墮落。這一種不安的情緒會引起惡口和惡行,最後免不了招致痛苦和不滿。

這一種絕大部分眾生無法控制或無視的混亂,我們稱之為「生命」。六道中無論是哪一道,即使是天道也會有同樣的情況發生。卡盧仁波切以實際的例子告訴我們,六道中到處都遍滿著二元對立的思維,就算是天道也是一樣的。仁波切語氣堅決地又說了一遍:「只有一個境界是圓滿及永恆的:佛的境界」他教導我們,只有一個目標是真的值得我們去追尋!只有到達佛的境界,心的障礙便會完全清除,度眾生的願力和能力才會自然地完全圓熟起來。

卡盧仁波切在講解了一個星期有關有為法的種種顛倒後,開始改變其開示的方向。如今他告訴我們今生能投生人道是何其幸運的事。在遍布法界的云云眾生之中,他們是如何才能擁有如此珍貴的機緣學習佛法,得以求解脫。他說:「能擁有一個學佛修行的人身,是因為避開了八種無暇的障礙,具足了八有暇,十種自他圓滿的善緣的原故。」仁波切一一列舉,繼續說道:「如今就是這種種的善緣為你提供了一個難得的學佛修行的機會。」

在仁波切的指導下,要撕開心中的無明和煩惱的面紗不再是一件遙不可及的事,那是許多人會認為是自己能力所及的事。他所描繪的圓滿境地亦非夢想,而是每個人內心中不變的真正本性。仁波切生動的描述甚至啟發了我們這群人中最懶散的成員。之後,仁波切開始講解「無常」,更讓我們開始認識到修行刻不容緩。萬物無常、稍縱即逝;生命的結局必定是死亡,我們的壽命剎那變滅,他所講的字字句句使到我們對周遭的一切無時無刻都充斥著變幻與衰減的跡象有所警覺。他說:「你們有機會在此生尋找此不死亦不滅的東西──即你們本具之佛性。」「這是一個非常殊生難得的機會,然而生命短促。如果現在不好好把握,以後你必須經歷多少世的痛苦和迷惑,無法自利也無法利他,才能再次遇上這些殊勝的機緣求得解脫呢?」

儘管我和漢娜已開始能接受仁波切的開示,可是我們依舊無法具體地將「痛苦」和現實生活作出聯想。我的心知道痛苦和失望是一種瑕疵,因此會將它們剷除,可是在索納達的這些人卻沒能徹底了悟到仁波切所開示的觀點其實是從佛的層面所表達的一種真理。按照仁波切傳統的表現方式,其實聽起來和我們的生活是毫不相干的。我們認為自己是最快樂的人,認為一切事情都是有意義的,令人感到振奮的,因此能夠引導我們去修行的是因為前方更美好的未來,而非背後的恐懼不斷在鞭策我們的原故。因為即使是目前我們知道的最快樂的境界也只是證悟之心的無限喜樂的其中一個微弱陰暗的影子而已。證悟的無限喜樂才是我們最大的動力。

這一種認知是我後來開始說法的時候才出現。卡盧仁波切開示說:「有為法皆不離三苦:(1.)苦苦---種種苦緣逼而生惱,極度煎熬難捱的狀態。(2.)壞苦---一切樂境或所現變壞時之苦。(3.)行苦---由於內心無明而看不清楚萬物的遷流變動之苦。唯有覺悟者才會完全超越這些漏失。」

\textbf{第十二章}

\textbf{噶瑪巴確定了我們的方向}

我們在克里斯大宅的美籍鄰居蘇和里察,每天都會花好幾個小時進行某種奇怪的體操。只見他們雙手合十,依次先高舉於頭頂,然後置於喉嚨和心間,就如我們在皈依儀式中所學的一樣。不過與其是雙膝跪地叩頭,他們是一叩掌心貼地,身體向前傾滑,展五體投地之姿,瞬即又再站直身子,不斷重複著此一步驟無有歇息。加拿大人阿健曾在寺院大殿裡進行法會時很「突兀」地重複了做這些動作,其他人則在屋裡重複做著這些動作。也許我們對噶瑪巴死心塌地,再加上先前曾向卡盧仁波切提出刁鑽的問題而不經意得罪了一些虔誠的弟子,又或者是我們提問的方式不對;不管怎樣,大家似乎都在忙著做自己的修行功課,我們很難問出個所以然。即使內心裡非常好奇,可是不久後我們也忘了這麼一回事兒。直到卡盧仁波切快要結束有關「痛苦」的開示時,他開始對這種五體投地式的運動作出了講解。

原來這是四加行中的第一加行:「大禮拜」。藏傳佛教的眾具德上師曾說過,像是「止」(梵:Shamatha,藏:Shiney)和「觀」(梵:Vipashyana,藏:Lhagtong)等簡單的修持,以及其他較高階的「密續」修法是否能圓滿修持,很多時候是賴於加行的功夫。行者需要通過這種準備工作集聚無量功德與智慧後,在修法上才會有真正的成果。大禮拜除了能有效地消除我慢,對於身語意也具有極大的影響。它能清理身體中心的能量脈輪,使心轉向菩提。卡盧仁波切時常說:「十萬次大禮拜是菩提道上關鍵性的第一步。」儘管人們稱此第一加行為「十萬次大禮拜」(藏:Bum),不過行者所必須完成的實際數量其實是111,111次。

漢娜已準備好要馬上展開大禮拜的修持,然而我的驕慢心卻察覺到事情不妙,於是趕緊找了個藉口推搪,我說:「如果噶瑪巴要我們做大禮拜,他應該早就跟我們說了。而且如果真要開始的話,應該一氣呵成把它完成。所以我想直到和噶瑪巴再次見面之前,我們最好還是先專心打坐。」我自以為這是避開大禮拜的絕妙之計。即使我們一天做一千次,我們也需要花超過三個月的時間才能完成十萬次大禮拜。我們肯定不想在無聊的索納達待這麼久。儘管我們很開心能擁有像卡盧仁波切般的導師,感恩之心也每日不斷增上,可是我們的心始終只向著噶瑪巴──我們心目中那偉大的笑佛。

我總是看見他人在做大禮拜,自己卻不曾親身嘗試,心裡難免不是味兒。於是某天出自於好奇心,我清理了房內地板上的一些木板,然後鋪上睡袋作為護膝之用,再在腹部會接觸地板的位置鋪一個棉墊,雙手穿上一雙穿了洞的襪子以便利於滑行,準備親身體驗一下大禮拜。我就只是想嘗試看看,其他人沒必要知道。大禮拜帶來了很特別的感覺,其實那種感覺相當振奮人心。十七歲那一年,我參加了AFS學生交換計畫前往美國,每當在練習美式足球時,凌空搶球都會讓我感到極大的快樂。如今做大禮拜也讓我嚐到了相似的亢奮與快感。在這過程中專注的進行一連串動作,似乎會讓人上癮,我一口氣做了大概五至六百次大禮拜。我只是想要知道大禮拜究竟是怎麼一回事,除了腹部肌肉出乎意料的痠痛,我後來也只是將它歸類為一種極具「異國風情」的運動而已。之後當我們去參加卡魯仁波切的開示時,仁波切微笑著說道:「你已經開始做大禮拜,很好很好!它們對消除身、語、意方面的障礙非常有幫助。」他對我做大禮拜一事極力地讚揚,在場的一些西藏人也不斷附議,熱切地說著十萬次大禮拜的種種好處。我不能當眾說出自己其實只是試驗性地做大禮拜,而且並無意願想要繼續。這樣不但會對他人的修持造成負面的影響,也會破壞仁波切的弘法工作。如今我陷於一個騎虎難下的處境,實在是非常尷尬。這一次我們著著實實地感覺自己被擺了一道。我和漢娜就這樣開始了大禮拜,一天三遍,每遍兩百次。即便如此,我們每天仍具有充足的時間去做其他事,像去探索那些位於錫金邊境的山麓地區等等。

我們聽說在大吉嶺並不只是只有一個像卡盧仁波切般的大瑜伽士。就在鎮外山坡上的那一片樹林裡,住著一名具德上師堪珠仁波切(Kanjur
Rinpoche)。堪珠仁波切是噶瑪巴在西藏的弟子,亦是寧瑪傳承地位最高的其中一名上師。儘管我和漢娜一直認為最好只是緊隨一個傳承,不將各傳承的法教參雜一起(換言之,不明白一件事,也總比搞混幾件事來得好),可是我們卻對堪珠仁波切感到非常好奇,同時也很渴望能領受他的加持。於是某一天,我們動身穿越那一片美麗的樹林,踏上了拜訪仁波切之路。

寧瑪與噶舉兩大派在一些名相上存有許多差異;然而多世紀以來,寧瑪和噶瑪噶舉傳承的法教和加持力曾相結合過數次。如今寧瑪傳承的法教也是噶舉傳承重要法教的一部分。這兩大傳承經常同時由一個喇嘛所持有,第三世噶瑪巴朗炯多傑(Rangjung
Dorje)就曾將寧瑪傳承的大圓滿教法(藏:Dzogchen)與噶舉傳承的大手印(藏:Chagchen)融合一體。一般而言,大圓滿法教對憤怒心重者非常有效用,而貪執心重者則適合修持大手印。對於「自心本具佛性」具信者,這兩大教法將會是最快成就的方便法門。堪珠仁波切與其明妃和身為轉世活佛的兒子就住在檜樹參天而立的山腰處。他們美麗的石灰房屋正在建造中,身邊不時圍繞著一群富有的法國籍弟子。若要進入仁波切的居所必須先通過大柵門,那裡由兩隻德國狼犬守著。這兩隻狼犬兇惡極了,聽說前幾天才咬死了人。坐在那裡的幾個西方人絲毫沒有要前來幫忙的意思。就像和卡盧仁波切較親近的那幾個西方弟子一樣,他們都只是專注於個人的發展,視無旁人。我們對自己承諾,往後若在西方國家建立藏傳佛教的禪修中心,任何挑起事端導致不和睦者一定會被攆走。後來當我真的如願開始創立禪修中心時,我並沒有忘記當年許下的承諾,而且相當樂於執行它。甚麼官僚制度,傲慢乖戾的態度,自我中心─這些通通皆與金剛乘的法教不符,它們完全背離了噶瑪巴的精神。

當我們成功打開大柵門時,這兩隻德國狼犬並沒有給我們帶來任何困擾,堪珠仁波切亦及時出現。仁波切踩著涼鞋,穿著長睡衣直直走進屋裡。他的妻子讓人感覺慈愛和藹,我們立刻就對他們產生好感。仁波切坐在其法座上,問了我們幾個問題,然後便進入禪定中,我們亦跟隨他一起靜坐。這比平時更具意義。此時我們就安住於自心本性中,超越一切概念思維,了了分明。當我們回歸日常生活中的分別和造作時,幾個小時已經過去了。我們搭晚班巴士回到索納達後,依然能感覺到仁波切的加持力。後來當我們再次前往拜訪仁波切,亦出現了類似的情況。仁波切在我們面前的椅子上坐下,開始持誦「嗡啊吽邊札咕嚕貝瑪悉地吽」。忽然之間,體內一股洶湧澎湃的能量往上直衝,我整個人都在顫抖,感覺到無限的可能性突然展現。我的內心充滿感恩與歡喜,以額頭碰觸仁波切的膝蓋俯身頂禮。一九七七年,當我們與隆德寺的喇嘛們在日內瓦時,傳來了仁波切圓寂的消息,當時噶瑪巴承諾說仁波切很快便會乘願歸來。一九八一年十一月,噶瑪巴在芝加哥圓寂之前,他認證了堪珠仁波切以及其餘廿名很重要的喇嘛的轉世。堪珠仁波切曾是噶瑪巴在西藏的僧侶之一。他擁有很高的禪修境界,智慧空行母出現與之雙修,達證了不二的境界。

恰札仁波切(Chagdral
Rinpoche)也是寧瑪傳承的大喇嘛。仁波切位於古木附近的寺院建有許多莊嚴的舊卡當巴式的舍利佛塔。這一座寺院也因為這些美麗的佛塔而成為了當地的地標。恰札仁波切並非一名轉世活佛,可是他已在此生證得了圓滿的證量。仁波切是卡盧仁波切的好友,他因其非常準確的占卜和嚴格實際的教法而著名。有一個太妹型的西方女生,仁波切給她傳法之前曾令她清理馬房長達半年之久,後來她還患上了肺結核病才下山。當仁波切獲得金錢時,他總是會下山前往西里古里,將能買的魚隻都買下,對它們念咒後再帶到河裡放生。仁波切是素食主義者,就像古魯仁波切所有的弟子一樣,除了供香之外,其餘無論任何形式的煙對他們來說都是有害的。他們說:「煙會使諸佛不願靠近。」至於酒精,他們的態度則較為開放。適量地飲酒甚至有延年益壽的作用。仁波切在西藏時曾經拜訪過許多古魯仁波切和密勒日巴尊者閉關禪修的洞穴。他有兩隻驢子,一隻負責搬書,另一隻則負責搬運他的食物,他則按照個人的心意隨時隨地靜坐禪修。不久後,慷慨的不丹皇室協助他在尼泊爾創立了一所閉關中心。一九八一年的秋天,那是我們最後一次見到仁波切,他看起來仍非常健康。

當時寧瑪傳承的持有者乃為敦珠仁波切(Dudjom
Rinpoche)。仁波切於一九八七年在法國圓寂。就像噶瑪巴,仁波切身邊總是圍繞著許多追隨他的人和眾多前來請求幫助的信眾與弟子。仁波切往返卡林邦與尼泊爾兩地,在七十、八十年代期間,他曾前往西方國家數次弘法,更在倫敦、巴黎和紐約成立了許多禪修團體。我們在卡林邦時曾嘗試拜訪仁波切,不過當時仁波切的居所已被丟空多時無人居住。如今在大吉嶺,我們恰好趕在仁波切離開的前一天見到他。敦珠仁波切患有嚴重的哮喘病。他的弟子認為那是他承擔了許多眾生的惡業的緣故,而瑜伽士陳氏卻按照道教的一套說法,認為仁波切的哮喘病是因為性事過於頻密所致!不管人們怎麼想,仁波切虛弱的身體倒是讓很多人積下侍奉上師的功德。仁波切讓我們免於頂禮,我們問候他過得如何,他的回答簡潔有力。他說:「我很痛苦。」後來仁波切還給了我們倆一包包的喇嘛法藥(瑜伽士和喇嘛在修法時加持過的不規則顆粒草藥),然後輕輕地給予我們摸頂加持。一九八一年,我們第三次前往朝聖,當時仁波切仍能給替我們整百個朋友加持。

位於大吉嶺與古木之間的山丘上(從汽油泵再往上的地方)住著竹千仁波切(Drukchen
Rinpoche)與德澤仁波切(Thugtse
Rinpoche)。如今竹千仁波切也已轉世。這兩位仁波切似乎不怎麼喜歡卡盧仁波切,原因是卡盧仁波切在索納達的地位較他們更為德高望重。這兩位仁波切對此地區具有重大的貢獻,他們經常辦大型的課程和誦經法會。他們的努力使得當地的佛教徒擺脫過往「消費型」的被動角色,進而投入較積極的實修行列。他們的寺院建築相當現代化,法會上的樂音律動亦相當震撼人心,只是看到一個受認證的活佛失心瘋讓我們覺得非常不安。當時的我們並無法明白這種事。

後來我們經常拜訪座落於大吉嶺城外的布提亞布斯地提(Bhutia
Basty)。這裡非常有趣,它是一個在家人的團體,然而在運作方面卻仍有很大的進展和學習的空間!與我們西方人的團體比較起來的話,它顯然不夠活躍和透明。

雖然我們會去拜訪鄰近地區的其他喇嘛,可是我們並沒有忘記卡盧仁波切才是我們學法的導師。噶瑪巴經常會直接把學生送到卡盧仁波切這裡來學習佛法的基礎,而我和漢娜則是在較間接的情況下接觸了仁波切。一直到今天,噶瑪巴在西藏古老的寺院所流傳下來的傳統法教與僧團制度也成為了他在法國、加拿大、挪威和美國所建立的寺院的基礎,所有的修法都是以藏文進行。直到ㄧ九九八年,我們和朋友所創立的超過兩百間在家與瑜伽禪修中心都是直接為噶瑪巴而立,尤其在昆吉夏瑪巴的拜訪之後,更為我們增添了不少創新和重要的主張與理念。我們都是法友,大家都有權力決定團體內的事務。這一種開放的風格,大手印的教法,以及噶瑪巴上師禪修法,使不少人更加成熟起來。後來創立的團體屬於在家人團體,行事上亦更為透明化,人們能將生活和佛法結合起來,讓這些聰明和獨立的心智擁有更開明和平等的平台充分地展現自己的才能。

我們幾乎每天都跟隨著卡盧仁波切學法,亦在他的指導下修持大禮拜,可是我們卻從來不從仁波切那裡領受任何灌頂。由於灌頂將會以不同層次觸動受灌者的心靈,因此除了噶瑪巴,我們不希望讓其他任何人觸碰到這一塊。有ㄧ次,我們在仁波切開始傳授灌頂之前離開了他的房間,所有人都覺得匪夷所思。卡盧仁波切注意到我們的壓力,也知道我們很猶豫;有一天他告訴我們噶瑪巴即將從不丹東部前往西部,途中必須繞道印度而行,因此勸誡我們應該藉此機會向噶瑪巴請益有關灌頂的事。這是我們聽他說過最動聽的話了!我們的心情頓時雀躍萬分,激動得無法言語。當我們知道自己即將再見噶瑪巴後,所有事物似乎在散發著光芒。在接下來的幾天,我們不斷收集有關他的行程資料,以及該地區警員的分布狀況。儘管我們從來都不知道消息從哪兒傳來,可是那些無所不在的康巴人似乎總是對噶瑪巴的一舉一動瞭如指掌。他們傳來了消息說噶瑪巴數日後入境印度時將會經過的某個路段。由於連結不丹東西兩端的道路尚未完成,因此必須繞道印度方能抵達目的地。由於路只有ㄧ條,所以我們會在上次登上豪華旅巴的那個地方與噶瑪巴會合。這ㄧ次我們輕而易舉便領到了通往卡林邦的所需文件,也很慶幸自己非常熟悉那ㄧ條軍用公路。當那條主要的道路被關閉時,所有車輛就只能取道此軍用公路,而且在這條路上一直到抵達不丹邊境之前,都不會有任何檢查站。

這ㄧ次我們在巴士抵達關口前便下車,然後步行至邊境檢查站問看我們是否可以在那裡等待噶瑪巴。雖然值班的官員並不是上次那ㄧ班「天才」,不過他們同樣是緊張兮兮的,而且還威脅我們若不趕緊離開就將我們逮捕。他們大概是擔心會丟了工作才這樣吧?在印度打破飯碗可不是鬧著玩的事。我們告訴他們的長官說我們是噶瑪巴的弟子,不過他怎麼也不相信。他看來是曾接受過古典印度教的教育,所以才會不斷重複說:「那是需要幾世修來的緣份。」無可否認,兩個金髮的歐洲人說自己與德高望重的噶瑪巴關係如家人般親密(噶瑪巴可是西藏眾多偉大的瑜伽士和日益漸增的印度信徒的上師呢!)這聽起來確實是挺奇怪的。可是對方卻遺漏了ㄧ點:許多來到東方的大塊頭白人,過去世也許曾經在此居住過也說不定。就像那些深受基督教文化所吸引的亞洲人一樣,他們不只是為了金錢或表示自己高人一等,他們只是曾與基督教有過一段淵源,重拾過去世所遺留下來的印記而已。

其實ㄧ個偉大的上師若要通過弟子的個性而認出他們並非難事,只是還是避免談及前世會比較好。因為人們會不斷推測,而且亦無法真的可以肯定,加上每個人要應付今生的自我意識已經足夠疲累了,多一事還不如少一事,因此為人師者還是專心教導弟子,繼續提升他們的心性潛力就好。

這名邊境檢查站的官員對我們試圖改進他的ㄧ套說法似乎很不以為然。不過由於這只是在浪費時間,而且一不小心還會惹出大麻煩(他們已經近乎迫切),所以我們乖乖地揹起背包打算轉身離開。在ㄧ排松樹直立的沙地上走了ㄧ小段路後,我們登上了ㄧ輛來自不丹的吉普車。吉普車把我們送到毗鄰的村莊,那裡仍屬限制區的範圍。由於這ㄧ條公路沒有其他分岔路,噶瑪巴的車輛必定會經過這裡,所以只要守在這兒就一定會萬無一失。我們趕緊離開大街避開警察的耳目(管他是便衣還是穿著制服的警察,我們一定不可被發現),同時亦開始四處打聽當晚可下榻的地方。除了可以避開那些惱人的官員,這ㄧ個小鎮大概是我們到訪過喜馬拉雅南部最美妙的ㄧ個地方了。小鎮上相思樹參天而立,綠蔭如蓋,鎮上的居民性情溫和自然。與他們相處時我不必說教,真是讓人鬆了一口氣。當我們問他們哪裡有旅館時,他們直接把我們帶到寺廟裡去,鋪好蓆子邀請我們留下。我彷彿又感覺到了在嬉皮式湧入這些東方國家前的那種熱情。這ㄧ個天堂般的地方居然還有ㄧ個仍然管用的電話,當我們聽到從彭措林傳來的消息說噶瑪巴要到隔天才會抵達時,內心裡不禁讚嘆他們的殷勤招待來得真是合時。

蓆子就鋪在主殿旁,每隔一段時間就會有當地的居民進入主殿,敲響銅鑼把諸神喚醒,供養食物或硬幣,再許個願後方才離開。他們有的人也會為我們帶來食物。過了一會兒,有幾個塔芒族人前來邀請我們為他們的寺廟加持。塔芒族主要是佛教徒。此刻情景,一個小動作便遠勝千言萬語。我們究竟是應該帶上背囊,抑或是相信他們好呢?在印度這麼久,這是我們第一次冒險將背囊留下,然後跟隨他們走開。總之,我們能感覺到他們很開心。

他們帶領我們走在一條剛種了樹的坡路上,沿路樹影成蔭,我們心想:「印度在人口爆炸之前(當年白人把盤尼西林帶到印度卻忘記留下避孕方法嘛),應該就是如此的美麗。」這些塔芒族的寺廟離村莊不遠。只見寺廟建在木柱上,這種高腳建築的結構是為了防止蛇鼠的侵害,具有很明顯的塔芒族風格。就像在其他山麓地區處處所見的唐卡一樣,唐卡上是擁有一雙超大的手和眼睛的四臂觀音聖像。唐卡旁還有一些金剛杵和金剛鈴,看得出其手工生疏粗糙,就連表面上的雕刻和輪廓線條也顯得模糊不俐落。

他們顯然並不常使用該寺廟,而且他們大部分的習俗都涉及了大量的酒精。不過只要想到諸佛菩薩,基本上還是能感覺到正面的能量。我們邊撒米邊持誦著所修持的咒語為他們加持寺廟。

翌日早晨,我們選擇了ㄧ個能看清楚路況又不會被發現的「風水位置」等待噶瑪巴的到來。這裡是村莊的入口,而且路上設有低坎,所有路過的車輛都必須減速,因此我們一定不會錯過噶瑪巴。就在我們等了幾個小時,吃了許多香蕉下肚後,噶瑪巴的路虎終於出現,行駛在其護航車隊的最前頭。我們跑向噶瑪巴的吉普車前,他的司機嚇得連忙剎車。噶瑪巴見到我們後開懷地大笑,並與我們寒暄了一會兒。

稍後,他吩咐人把我們藏在卡車上的行李堆內,之後我們便跟著大隊浩浩蕩蕩地進入了不丹境內。

在彭措林,車子就停在靠近以前曾下榻的那ㄧ家賓館的某個旅館前。我們躲開了那些想要和我們聊天的隆德寺的朋友,直奔噶瑪巴身邊。我們並沒有注意到那些正在排著隊等待著要覲見噶瑪巴的貴賓,直接就跑進了噶瑪巴的休息室內。我們將護照交給噶瑪巴,告訴他說我們希望能領取不丹護照,那麼以後便不會再與他分離。當時的我們完全沒有想要回到歐洲去的想法,一心只想親近他。結果噶瑪巴握著我們的護照,語重心長地對我們說道:「你要為自己擁有一個國籍和明確的身份而不是身為難民而感到幸運。你們應該要明白這ㄧ點。不要放棄你們的護照,日後會有用的。」噶瑪巴答應會儘他所能幫助我們,同時邀請我們和他一起坐在竹椅上。

四周正在醞釀著一種很奇怪的氛圍,我們也說不出個所以然。此時,有一個健壯魁梧的男人進入室內,來到噶瑪巴面前席地而坐,我們認出他的身份,靜悄悄地溜了出去。人們領口上的別針與四處張貼的照片上都是他的肖像:不丹的ㄧ國之君。

夜已深,我們恰好有機會再次覲見噶瑪巴。一踏入房內,便被眼前的景象震撼得說不出話來。噶瑪巴整個人散發著金色的光芒。我們回過神後,便趕緊向噶瑪巴請益有關灌頂的問題。噶瑪巴微笑著說道:「真是傻瓜!難道你們看不出來卡盧仁波切代表著我嗎?你們當然應該從他那兒領受灌頂。回去索納達,好好地修持加行,有機會便來錫金看我。」

後來我們有幸覲見了不丹的王室成員。一九七九與一九八七年,他們邀請我們前往不丹作客。不丹國王如今迎娶了傑珠仁波切的四位姪女為妃。此外,我們亦見到噶瑪巴的侄子Jigmela,他不時強調閉關中的行者很快便能完成大禮拜的修持。完成了最後一場的黑寶冠儀式後,我們在凌晨三點離開。這次我們和老友吉梅醫生一起坐在某輛卡車舒適的座位上。由於噶瑪巴必須作多站停留,給不同的人們和地方加持,因此我們的車子較早抵達堤斯塔大橋。我們捧著許多要要寄回家鄉送給朋友們的照片和聖物法器,等待噶瑪巴到來時請他加持。怎知一回過神來,噶瑪巴的車子已經悄悄駛過,到了橋的另一端。

\textbf{第十三章}

\textbf{家在索納達}

我們從堤斯塔(Tista)坐上一輛開往大吉嶺的吉普車,然後在古木(Ghoom)下車,再轉搭一輛貨車前往索納達。這是我們第一次在索納達找到歸屬感。既然噶瑪巴要我們留下來學習,那麼我們就遵從他的意思。我們儘量把握時間在索納達學習和修行,先從每天兩千次的大禮拜開始,然後再漸漸增加至三千次。除了大禮拜,我們其餘的時間就是在吃飯、參加開示、和歐洲的家人朋友聯繫。當時,寫信是最主要的溝通方式,我們會趁空閒的時間提筆給家人和朋友寫信。由於睡眠時間變短了,於是每天多出了一些時間。在噶瑪巴的加持力之下,再加上大禮拜疏通和矯正了體內的脈輪,我們每天只睡四至五個小時就會自動醒過來。由於室內很冷,所以從大清早開始做大禮拜感覺還不賴,身體會因此而變得暖和起來。修持大禮拜讓我們的身體變成一個可以運用的工具,一切苦痛的精進變成了樂意的精進,這是我們從未料想過的事。

漢娜在修持大禮拜時所展現的智慧讓我留下很深刻的印象。我們總是肩並著肩一起大禮拜,由漢娜負責設定節奏。她很能吃苦,面對如此吃力兼消耗體能的運動,她總是能夠連續進行而不中途喊停。我們會以她所做的大禮拜為計數標準,而我偶爾會加快速度以促使更強烈的體驗。

結合了身(動作本身),口(所持誦的咒語)與意(禪定與保持正知)三門,大禮拜是一個全面並具有轉化能力的密乘修持。大禮拜能為行者帶來持久的改變;我這麼多年來一直飲酒、吸毒不斷,積累了不少混濁的能量,因此沒有其他方法能像大禮拜般更有效地消除這些負面的能量和我那自以為是的一套人生觀或怠惰的障礙了。只是我們應該相信古老的智慧,遵從他人所教導的修持方法。我曾經擅自修改修持的方法,在胸口會碰觸地面的位置擺放一個硬物想藉此釋放心輪的束縛,可是卻無其裨益。儘管心間處產生了一些奇妙的感受,我的自作聰明帶來最顯著的效果,便是斷了一根肋骨。我的胸口真的非常疼痛,在進行最後三萬次的大禮拜時,我只靠左手臂支撐著整個身體,若說我是在持誦咒語,其實更常是在痛苦呻吟。通過修持大禮拜,內心裡有許多業習將會圓熟、出現。在這一整個星期,大黑袍金剛護法擋住了一切,然而脆弱無知的我們卻不知那是一種加持。祂就懸浮在前方的虛空中,如如不動。

修持大禮拜消除了許多毒品帶給我的影響和多年來的謬見,我在過程中往往會不斷顫抖,其中的痛苦都寫在了臉上。幸好我知道這只是一種淨化的過程,它正在將負面的能量排出體外。我們在大吉嶺的布提亞布斯提寺(Bhutia
Basty
Monastery)完成最後一萬次的大禮拜,平均一天做四千次,就在那充滿加持力的古老蓮師聖像旁。我們非常感恩喇嘛拓殿(Thubten)與洛迪(Lody)的安排,讓我們擁有如此殊勝的機緣在布提亞布斯提寺內完成最後的大禮拜,我們亦非常感恩吉梅醫生及其家人為我們送飯。一九七一年新年當天早晨,我們完成了四加行中的第一加行,亦即111,111次大禮拜。當天我們還泡了很長的溫水澡作為慶祝。

回到索納達後,卡盧仁波切傳授了一個灌頂。那是我們從仁波切處所領受的第一個灌頂。當時仁波切所傳授的是諸佛菩薩慈悲的總集,身呈白色的四臂觀音菩薩灌頂。一五ΟΟ年代,印度人稱觀音菩薩為「阿瓦洛吉蝶濕伐羅」(Avalokiteshvara),西藏人將之翻譯成「千瑞吉」(Chenrezig)。我們擠到仁波切面前,他的臉龐散發著智慧的光芒。仁波切琅琅的誦經聲與手鼓聲無處不在。忽然之間,我能清楚看見一個透明,身呈白色,擁有四隻手臂的菩薩行者就在眼前。那是一種不可思議的加持力,我的雙腳像釘在原地般無法移動半步。我永遠無法忘記那一次的灌頂。

我試圖「改良」大禮拜的修持方式一事惹得這群西藏人捧腹大笑。儘管胸口上一直綁著繃帶,也塗上了傳統的藥酒,可是我的肋骨卻一直不見好轉。不管怎樣,我竟然會面對如此愚蠢的障礙,讓我覺得自己遜斃了!由於我們有意前往卡林邦的星期日市集去看一些西藏文物,所以亦打算藉此機會去嘗試當地某個醫療喇嘛的醫術。

貢普斯酒店的塔希很清楚我們上一次來到卡林邦時所面對的種種苦樂。他對我們曾下榻的地方亦是瞭如指掌,這意味著我們也是他們西藏人閒話家常的對象之一。如今聽說我們有意要完成四加行的修持,他感到非常開心。塔希帶我們上山,為我們引見一名蒙古籍的喇嘛。我們還在途中購買了一些新鮮的糌粑。以前我們老是對上坡路感到非常厭惡,如今每一步走起來卻是如此地輕鬆、歡喜。通過修持大禮拜,我們的身體找到了自己的節奏,開始能順暢自在地運作自如。大禮拜是我們所踏出的很重要的一步。如今我們的身體已是一個聽話的僕役,不再是執拗難纏的主人了。它們不再限制心的自由或讓它變得懈怠。

我們抵達山脊的頂端後便沿著一道石牆走,最後來到了某座寺院的大門;通過大門入內,我們來到一排小屋前。塔希從角落處的那扇門進小屋內,我們尾隨其後。外頭陽光刺眼,與屋內的一片漆黑形成很大的對比。由於我們已經習慣漆黑的環境,因此眼睛很快便適應過來,只見屋內放滿了箱子和袋子,有一位和善的老人示意讓我們進來。我們向他問候,他給與我們摸頂加持。他就像喇嘛傑珠給人加持時一樣,會先將雙手擺放在我們的頭部兩側,然後緩緩向上移動至頭頂的位置。我們以剛買的新鮮糌粑和一些金錢作為供養,然後開始聊起彼此的故事。

就像大部分的蒙古人一樣,這位老喇嘛屬於「善規」格魯派。儘管現代人也許很難相信,這一個名字其實是由他們自己所選。格魯派重視對佛法的如理聞思和周密的組織和制度。他們管理西藏的政權,並且創立了許多容有千名僧侶的大寺院,就像色拉(Sera)、哲蚌(Drepung)與甘丹(Ganden)三大寺院。許多僧侶就在這些大寺院裡學習並通過因明學和辯論藝術的考試。

佛陀留給世間最美好的禮物,是其教誨不僅是能對治問題本身,亦能對治問題的成因。佛法尤其針對造就一切煩惱的根本,亦即「無明」。不管你是選擇修持三大舊傳承(也有人稱之為紅帽傳承)的禪修法門,或是選擇追隨黃帽傳承研習教理,這一千年來它們都保存了佛教最精華的法教。無論你是對佛陀和上師具有強烈的虔誠心,還是處於較慢的聞思修階段,無論是任何根器者都適合修持佛法。

這名蒙古老喇嘛和一般的喇嘛看起來不太一樣。塔希說他差不多有一百歲,可是他看起來完全不像,而且他看起來像是一名瑜伽士多過班智達(學者)。老喇嘛主要修持「斷行法」(Choed),這是一個從根本上斷除我執的法門。行者在大手鼓、金剛鈴和脛骨號的伴奏下,持誦著醉人的樂音,然後通過禪定力將個人的身體佈施給餓鬼及其他有需要的眾生,同時將意念傳送給上師。藏傳佛教的舊傳承有在修持斷行法,其中的經驗更是讓人留下深刻的印象。首先,所祈請的能量將會在禪修者的想像中出現,之後便會具體地顯現出來。行者在掌握此修法後,對於他人所恐懼或會感到生氣的事必能一笑置之。

老喇嘛建議傳授我們此修法,這讓我們覺得有點兒受寵若驚。只是我們必須留下來六個月,和他一起住在當地的墓地裡。然而,我們想要在卡盧仁波切的指導下,先完成四加行的修持。我們在骨子裡也能感受到第一加行所帶來的效果,因此內心裡非常期待繼續修持接下來的三組加行。而且我們心裡清楚知道在向上師請益之前,不應該隨便更換目前所修持的法門,否則日後肯定會出問題。我們感謝他對我們的信心,同時也坦白地告訴他目前我們正在卡盧仁波切的指導下修持著四加行。我們答應說會儘快展開斷行法的修持。離開之前,我突然想起斷了的肋骨,老喇嘛聽我說了傷勢的由來後更是忍不住大笑。其實在胸口觸地的位置擺放硬物的這一個想法來自山下那位華裔瑜伽行者的小冊子,不過老喇嘛似乎未曾聽說過他。不過,塔希卻對這名華裔瑜伽士一點兒也不陌生。原來上次那些難吃的肉餡餃子就是從他的酒店買來的!

老喇嘛用長長的勺子從兩個不同的瓶子舀取了一些灰黃色的粉末,然後用米紙摺成三小包。老喇嘛對著三包粉末念了幾句祈願文後,對我說:「這兩包粉末是讓你分別在明天與後天一大早配以熱水服用,而這包粉末則是在明晚配上溫水服用。」當我服用了老喇嘛的粉末後,胸口便不再感覺疼痛。我們也不曉得把我治好的究竟是他的那些粉末,還是他的加持力。不過搞不好是兩者相輔相成也說不定!

不久之後,我們又再動身前往加爾各答(Calcutta)。大吉嶺的警員已無法再替我們延展簽證,所以現在我們必須拜訪他們的頂頭上司申辦新的簽證。辦事處位於作家大樓(英:Writers
Building,印度西孟加拉邦的政府大樓)。那個看起來受過良好教育的年輕人無疑是擁有很好的福報,不然就是有一個有權有勢的父親在背後撐腰才讓他坐上了這個職位。他很同情我們,我們也很同情他。他答應會為我們辦理大吉嶺區長達半年的有效簽證。我們如常以醫療理由提出居留申請,說漢娜在尼泊爾得了痢疾,如今需要山區清新的空氣來調養身體。這些公務員大概也懷疑我們是傳教士,不過對此似乎並不反感。我知道若我們向他們坦承留下來的真正目的,他們一定會立即取消我們的簽證。他們經常用這一個手段來對付我們其他經驗尚淺的朋友,後來他們也曾試圖要用同樣的手段來對付我們,不過我們很少會讓他們得逞。對他們而言,西藏人是低種姓,齷齪的難民,而且還很可能是中共政府派來的情報員。他們無法接受西方人(他們的秘密偶像)如今會來到他們的國家向這些貧窮至極的難民學習。後來我們的朋友很快便學會了在前往任何官方機構之前會先將身上念珠和金剛繩藏起來,不讓他們有機會刁難。

由於簽證需要十天才能辦好,我們也不想一直對他們施壓以免壞了好事,因此決定在乖乖等待簽證的同時到山區走走。那個位於孟加拉灣加爾各答南部的小鎮普里(Puri)聽說不錯,它是冬天的好去處。無論是遊客或嬉皮士對此地皆讚嘆有加,可是真正吸引我們的原因是因為約瑟夫住在普里。約瑟夫曾經是艾文斯‧溫茲(W.Y.
Evans-Wentz)的僕人,一九二Ο年代在錫金專司承侍。這意味著說當年艾文斯‧溫茲教授獲得某些很重要且只能秘密傳授的法本的第一手(大概也是最好的)翻譯時,約瑟夫也在場。第一次世界大戰結束後不久,喇嘛卡孜‧達瓦桑珠(Lama
Kazi Dawa
Samdub)的四本翻譯《西藏生死書》、《西藏大解脫書》、《西藏大瑜伽師密勒日巴》與《西藏瑜伽與秘密的教義》由牛津大學出版社出版。《西藏瑜伽與秘密的教義》一書更是我們在牢獄裡時的良伴。我們希望這兩個先進的文化能夠做到最好的結合,如今我們終於有機會見到當時在場的見證人。遍布於世界的北歐文明和藏傳佛教在各層面上可說是相輔相成。一直到今天,儘管書中有許多法門必須在修好加行法以及在上師的指示之下方能修持,我們依舊認為這四本書是第一流的翻譯,不過書中的注釋卻相當難以理解。艾文斯‧溫茲教授的博士論文專攻是凱爾特宗教中的德魯伊(Druids),他顯然並不理解佛教。如果將不同宗教的名詞或見地混淆在一起的話,就會令其失去原有的鮮活和能量,這樣對任何人都將會是無其裨益。這兩種宗教無論是在修持方式或目標上都存在著極大的差異。但更重要的是,現今社會必須檢驗哪一個宗教能利益及幫助眾生解脫,否則都應加以遏止。

我們與約瑟夫的會面並不像想像般達到交流的效果。他的孫女戰戰兢兢地打開門讓我們進到屋裡,年老的約瑟夫臥在床上,已經不太能說話。我們有看見艾文斯‧溫茲的大禮帽與拐杖,不過約瑟夫的家人只是一味在談論著他們的鄰居。只因為他們是基督徒,這些信奉印度教的鄰居老是向他們扔石頭,他們的孫女也因此找不到結婚的對象。否則若按照當地大鬍子的標準來看,其孫女的長相也算是相當標青。

我們在普里靠近海邊的一家旅舍渡過了一段美好的時光,我們享受著海水(雖然沒想過那裡的海水會如此渾濁),也吃了許多許多的食物,當作是犒賞自己完成了一萬遍大禮拜的修持。我們也趁此期間將會在第二加行中持誦的百字大明咒默記起來,和當地的嬉皮士打交道收集消息;我們也協助一群人渡過了惡劣的迷幻之旅。他們當中有人開始對佛教感興趣,有意進一步去發掘和了解,後來更來到索納達求法。如今每當我提及噶瑪巴便很自然會真情流露。其實我們能在印度這一個愚人的天堂,而非在家人或朋友們充滿批判的氛圍之下展開如此艱鉅的修行是不錯的。否則,我們開口閉口談論的都是和解脫或靈性有關的課題,可能會給他們造成很大的困擾。佛陀教法的力量已將心維持在一個單一的明覺境界中。此外,重複持誦咒語與佛力的加持亦非常有效。這種種的修行方便能以更快的速度將我們的業習和煩惱淨化,直至只剩下赤裸裸的明覺。而且我可以肯定的是,從現在開始,心靈的開展將會是最自然不過的事,但我亦知道這只是漫長旅程的一個開端而已。

\textbf{第十四章}

\textbf{修持}

我們領了長期簽證後便回到索納達,那裡已有好消息正在等待著我們。兩個月前,我們申請錫金的簽證,就在我們快把這件事遺忘的當兒,我們通過了申請,而且簽證來得正是時候。噶瑪巴即將在隆德傳授瑜伽度母(Dolma
Naljorma)大灌頂。此重要的大灌頂,歷代各個噶瑪巴也只是傳授數次。這一次的灌頂將會延續幾天進行,期間噶瑪巴也會傳授其他灌頂。索納達的一行人大部分已經上山去了。我們尚未來得及卸下行李,又急忙趕往隆德,這一次與我們同行的還有一對在普里認識的澳洲籍夫婦。他們對噶瑪巴一無所知,對於佛教也僅是略為懂得一些。他們在喜馬拉雅西部待過一天後就毅然決定申請簽證留下。一切和噶瑪巴有關的事情再次圓滿展現,各就其位。如今這一對夫妻,男的在協助土著居民的相關組織裡任居高職,女的則支承悉尼附近的一間南傳上座部佛教的禪修中心。

當天傍晚,我們渡過最後一座橋進入錫金境內。我們在抵達岡托克(Gangtok)前的八公里處下車,打算步行上山。這裡的天氣乾燥漸涼,我們沿路抄捷徑前行。此時天色已暗,沿途不時遇見成群躺在路上的牛羊。大部分牛羊的頸項仍栓著繩索。寺院主樓前的院子中亮著幾盞燈,為四處亂轉的人群照明。許多西藏人與大德高僧千里迢迢來到此地參加這一次的大灌頂法會。鄰近的旅館也只剩下睡地板的空房。耶喜說這一次因為亦有許多高僧大德蒞訪的關係,所以吸引了大批人潮。眼看旅館已經客滿,耶喜想必會讓許多前往求宿的人失望了。

翌日早晨,噶瑪巴主法灌頂前行法會。身為西方人的我們很幸運。由於我們是客人,加上我們來自於遙遠的國度,因此他們安排我們坐在最前排的位置,能將噶瑪巴的一舉一動看得清清楚楚。人們可以很具體地感受到噶瑪巴的舉動。伴隨著雙手鼓、脛骨號和金剛鈴,噶瑪巴不斷持咒,進入禪定,他在佛殿周圍建立起一個能量場。昨晚在路上碰到的那些牛羊似乎也來湊熱鬧,緩緩走在會場的通道上,顯然是被正在進行中的事吸引而至。每當有大法會時,山谷裡經常發生動物們用盡辦法掙脫繫繩,然後來到會場共襄盛舉的奇事。這些動物們的主人也都已經放棄去牽它們,因為它們之後又會再次逃脫,反正當法會結束後大部分都會找到回家的路,所以就索性由得它們。

當天晚上,所有人都非常疲憊,然而卻能感受到滿滿的加持力。噶瑪巴給所有出席法會的人分派兩根吉祥草。晚間睡覺時,我們必須將較長的一根豎放在睡墊下,較短的另一根則橫放在枕頭下,形成一「T」字型,而且要注意夢境。當晚所顯現的任何夢境也許具有重大而深遠的意義。

我做了兩個非常深沉且清晰的夢。在第一個夢境中,我和某個丹麥的朋友正在看著一大叠非常漂亮的唐卡。當我看到大黑袍金剛護法的唐卡時,我將它拉到面前,說:「這幅唐卡我要了。」說畢,唐卡瞬即強行溶入心中,消失不見了。在第二個夢境中,某個相貌莊嚴的藏族老婦女不小心從山坡滑落,我及時抓住了她,後來還扶她上山。我記得當時內心裡生起了強烈的慈悲心,山上的風景不可思議的美麗。漢娜也做了一些甚有意思的夢境,隔天早晨醒來,我們都感到非常滿意。

大灌頂法會的正行部分較前行部分更讓人為之印象深刻。瑜伽度母具有一面八臂,身色綠色,呈半忿怒威猛女相。我們無法相信眼前所見,噶瑪巴開始顯化出本尊瑜伽度母的形象特徵和活力。雖然我沒有看見噶瑪巴的身色轉變成綠色,不過無疑能感覺到有一股女性的能量向我們湧來。當噶瑪巴成為一切能量的中心時,幾位轉世活佛亦開始在人群中走動。他們以具有噶瑪巴的禪定加持力量的法器給予每一個人加持,將他的能量與加持力傳授至我們的頭部及雙手。如此一來,證悟的種子便已被植入我們的心識中,我們圓滿地領受了灌頂。日後我們也將能證得與佛的身、口、意、功德與事業同等的證量。金剛乘的灌頂的力量能在眾生往生時將其帶往解脫的境界。當眾生不再受到感官的銘印所影響時,對自性潛能具信者將能消除對我相的執著。接著,各種現象都將具有解脫的法味,了知道一切皆為心的遊戲幻現。如此一來,我們就能免除墮落惡道的因緣。當內觀漸次增上時,能觀、觀和所觀就在剎那間融合為一體,我們便現證佛身。

人在死亡時,感官的銘印無法影響我們。受灌的智慧本尊身將會從我們潛藏的意識中以光和能量的方式湧現。噶瑪巴的黑寶冠就能達到這樣的效果。如果我們與這種超越時空的智慧顯現繼續保持聯繫的話,那麼我們便能進入智慧本尊超越時空的能量場。

在大乘佛教,人會在臨終時祈請無量光佛(阿彌陀佛)的淨土。人間一切的主要動力來自於貪念,若能當下轉化便能帶來立即的效果。擁有死亡意識的人都知道應該要多持誦無量光佛的心咒「嗡阿彌疊瓦舍\textsubscript{利}」。我們亦應從具德上師處領受蓮花部之佛的灌頂。

灌頂法會在隔天傍晚圓滿結束,人們陸陸續續回到居所品嚐法味。此時沒有人說話,大家似乎擁有太多感悟,什麼話語也說不上來。在返回旅館途中,我們巧遇一名年紀看起來和我差不多,相貌清秀的出家人。由於他懂得說英語,又希望能多接觸西方人,因此在接下來的幾天,我們對他的了解又多了一些。他是阿陽祖古(Ayang
Tulku),來自噶舉傳承的其中一個小支派。

儘管三大舊傳承[又有人稱之為「紅帽」傳承]具有臨終時遷移神識的修法,此傳承通常會向大眾傳授,同時檢視眾人是否具有開頂的徵兆。這一種禪修能使意識從肉體離開。意識就在頭頂(髮際八指處)出來,與無量光佛融合為一體。此修法外在的徵兆就是一個小傷口、小膿包或一滴血,在心的能量穿透皮膚之處就會有這些徵兆。臨終時的體驗比修法時更為強烈,但是過程同樣涉及我們的意識從中脈的光管中向上攝出。意識從肉體中飛出,進入不可思議的大樂和覺性的境界中。這種大樂及圓滿的體悟會漸次增長,直到所有二元對立的概念消溶為止。這裡就是無量光佛的能量場,祂的「淨土」。從一九八七至一九九七年間,我曾向世界各地兩萬六千名弟子傳授過這一項修法,那是一個極美妙的過程。如今,我每年大概會傳授破瓦修法十二次左右。

阿陽祖古的夢境亦很殊勝。他在夢境裡遇見無量光佛的化身昆吉夏瑪巴(Kunzig
Shamarpa),並且向他祈請加持和供養水果。他亦告訴我們西藏人詮釋夢境的方式。阿陽祖古所居住的地方位於邦加羅爾(Bangalore)和芒加羅(Mangalore)之間,他和我們談及該地區西藏難民營的情形。我們現在才知道原來有上千來自高原地區的西藏難民如今住在氣候炎熱的印度南部。他們大部分是西藏西部的遊牧民族,在其他藏民逃離西藏七年後才放棄他們的家園、牲畜、帳篷和大部分財物,於一九六六年文革時期來到印度。中共政府聲稱這些溫和的難民是罪犯,施壓印度將他們送遣回國。正因為如此,印度將他們從北部邊界送往南部一座無人居住的森林裡。他們當中有三分之一的人因此而丟了性命,這其中的衝擊實在太大了!生存下來的難民展現了他們的求生技能。他們以驚人的速度從一個寒帶地區的遊牧民族化身為森林裡的農民,他們亦向印度政府證明了一點:只要努力就能培養出優良的莊稼。阿陽祖古大力邀請我們前往拜訪,希望我們能助他一臂之力。我們答應他會保持聯絡,並且會將他的請求銘記於心。

在接下來幾天,噶瑪巴為弟子、信眾傳授了一系列的灌頂。我們也看見某個對學生很嚴格的加拿大籍喇嘛。這是我們第一次遇見西方人的喇嘛,我和漢娜都一致覺得他的行為甚為誇張。然而,噶瑪巴並沒有揭穿他。噶瑪巴將諸佛的喜樂、智慧、慈悲、事業和守護的能量種植在受灌者心中,這一系列能量將會帶來佛果的成就。當我們進入加行修法最後的階段時,這些灌頂其實是一種加持而非修法的開端,它們對後續的禪修會有很大的幫助。顯然這次是噶瑪巴動用了一點政府關係才讓所有人都能在錫金逗留十天參與大法會。我們心裡慶幸不必為了簽證的問題與印度當局爭持不下,省下了不少麻煩。

回到索納達後,我們也是時候展開111,111遍金剛心菩薩[又稱「金剛薩埵」(Vajrasatvva)或「多傑森巴」(Dorje
Sempa)]百字明咒的持誦及禪修。由於在持誦此百字真言時必須字字清晰,所以要完成此加行所需的時間會較大禮拜來得更長。金剛心菩薩乃為一切諸佛淨化能量的總聚,其修法具有巨大的力量。當有害的銘印離開我們的阿賴耶識時,大樂就能增長,最後證悟自心圓滿光明的本性。

世間有器的境界將無法和心的這一種圓滿狀態相比。尚未證悟就是表示有缺陷。怎麼說呢?佛陀解釋說這一種痛苦、遍在的現象是這樣生起的:二元對立的幻象(主體、客體和所作有所分別)從自心裡無始的無明中生起,就如烏雲蒙蔽晴朗的天空。當我們的心執取於空性(其「虛空」,虛偽的「我」),再去體驗其光明本性時,在此虛空中所發生的一切就會被視為是「你」或「他者」。從這一個分化的感受中,基本的煩惱便會生起:對於所喜歡的事物會產生貪執,對於不喜歡的事物會生起嗔怒。人總是很自然便會想要保留一切美好的事物,因此執著便會帶來貪染;嗔怒會帶來嫉妒,因為人人都不希望自己的敵人過得好。最後,無明會產生愚笨和貢高我慢。這都是從我們認為「這些短暫的現象都是真實不虛」的幻象中而生,認為自己比他人優秀,殊不知他者亦是虛妄的假象而已。

佛陀後來宣說貪嗔癡這三種根本煩惱各組合又將延伸出八萬四千種煩惱,導致種種惡行、惡語與惡念。它們會帶來有害的習慣,蒙蔽我們以至於無法體驗到證悟的自心本性。儘管這些障礙感覺起來是如此真實,可是它們基本上就像是鏡子上的塵埃而已,並無自性,所以這些障礙和痛苦都能被清除。為了促成這一點,一切諸佛將祂們淨化的能量化為金剛心菩薩的形式來顯現。金剛心菩薩(又稱金剛薩埵)呈晶瑩剔透之相、慈悲,其如如不動的覺性將能消溶人們潛意識裡的心結。即使沒有投入上述的修持中,單單在臨睡前念誦幾次百字明咒也會是一個睿智的習慣,給予新的一天一個美好的開始。

當我們了解因果、業力的運轉方式以及如何避免惡業所能帶來的苦果後,淨化的過程就會特別有效。以下是我們從喇嘛處所領受的眾多相關開示中的精華,若想更深入了解的話可以參考《The
Way Things Are》一書,同樣是由Blue Dolphin
Publishing(藍海豚出版社)所出版。因果關係的確立可分為「生起」、「結果」和「消溶」三大組四大條件。以下是簡要之概括:

「業力」若要植種於眾生的意識,帶來或樂或苦的結果,必須先符合以下四大條件中至少一項。

(一)我們必須要知道整個處境。

(二)我們必須擁有讓某件事情發生的念頭。

(三)自己作或教他人作。

(四)之後對所作感到滿足。

當其他因緣聚合時,也會增強業力的效果。同樣的,業果也將以四種方式顯現:

(一)死亡後,潛意識中的種子被喚醒。這是在缺乏新的感官體驗之下發生,其中的投射會被體驗為是一種真實的狀況。它們帶來了或樂或恐、或貪或惑、或妒或執的覺受。在不超過七周(四十九天)之內,最強烈的種子(銘印)將會引導我們投生六道中的其中一道。

也許是過了很長的一段時間後,當心識再次和人的肉體相結合時,接著就會產生以下三種情況。

(二)投生到心所吸引的軀體,意即適合它的基因,這也就決定一個人的健康狀況,是否長壽或反之。

(三)所誕生的環境──百分之十在富強的國家或百分之九十在貧困的國家,在市郊或貧民區裡,這裡是說明所誕生的地方狀況或財富差距。若對他人造成性方面的傷害會投生於沙漠地帶;若擁有好福報卻抱有邪見的話,就會投生於西北歐等地。

(四)最後,我們的習慣模式或習氣將會造作新業。比如說,有些人天生很友善,很容易親近;有的控制慾很強,不太友善。

對於捆綁著未證悟的眾生的因果業力具有大概的了解後,我們該如何脫離這一個混亂、無始的桎梏呢?

這裡也同樣涉及四大步驟:

(一)我們必須確信生命中有某些部分是不稱心如意的,譬如我們尚未成佛,我們不斷體驗痛苦,我們傷害了他人,為他人付出犧牲的不夠等等。

(二)我們利用最快速的金剛乗的法門來清除罪業,例如金剛薩埵的禪修。

(三)我們發願不再重犯惡行或惡語,若不小心重犯則再次發願。

(四)儘量多做善行。我們將能感受到喜樂,並從過去所造作的痛苦中獲得解放。

卡盧仁波切經常說:「經過這四大步驟後,不管多麼嚴重的業力都將會被淨化。」他又繼續說道:「只有當你運用它們時才會帶來效果。這就好比一塊香皂,若你只是把它收藏在口袋裡,即使過了一百年後,它也無法將你洗滌乾淨。唯有實修才能實證。」

我們從仁波切處領受了這方面的口訣後,便開始111,111遍百字明咒的修持。我們儘量觀想本尊金剛薩埵就在頭頂上,有如水晶般晶瑩剔透,祈願本尊能將潛藏在意識中的障礙統統清除乾淨。我們重複持誦了幾遍咒語後便已能感受到本尊金剛薩埵的能量開始凝聚,向我們湧來。雖然在過程中生起各種令人不安的感覺,特別是煩躁感,不過我們知道這些都是自心獲得淨化的徵兆。它們只是我們所面對的障礙中的冰山一角而已。如果現在不通過修持將之瓦解,日後恐怕會有更大的問題顯現。我們每天在克里斯大宅的臥床上禪修,窗戶外總是會有流雲路過,靜坐的時間往往一晃便已過了六個小時。雖然知道這一個過程能夠燒毀一切惡果的種子,可是經歷過了十萬遍大禮拜後,我們也花了相當久的時間才讓膝蓋習慣長時間的靜坐。

此具有清淨作用的咒語有效地清除了過去所造作的惡業。甘露的流注使到過去因為無數暴力和毒癮所組塞的脈輪獲得清淨和消溶,所有機械化的反應也有所改變,進而產生ㄧ種像行雲流水般順暢,任運自在的行為模式。在修法進入尾聲時,我們作了很深沉、強烈的夢境。這些夢境就和文本中所提及的如出一轍──與黑色的巨獸搏鬥,喝下白乳,吐出黑色的穢物和液體──這些都是修法成就的徵兆。我們猶如漫步雲端般生起了一種輕安的感覺。

\textbf{第十五章}

\textbf{菩薩境地}

這時我們聽說早前申請的錫金簽證已發了下來。如今我們都已學乖,每次在離開錫金時就會重新申請簽證,那麼下一次若有任何重大的活動便能趕得及參加。當時在隆德,我們是唯一的外國人,那一段回憶因此而尤其難忘。

噶瑪巴問及較早前我們按照一些書所進行過的那些高級禪修,我們坐在他的面前,如實一一作答。我們向他描述了當下所感受到的巨大加持力與能量,而且當時的我們對皈依和灌頂是一無所知的。噶瑪巴解釋說即便是如此強烈的體驗,若缺少了他的加持,這些經驗並不會帶來持久的效果,所以此時我們所修持的加行法其實會更為有效。當時也恰好發生了一件事,只是我們在一開始時並無法明白。

且讓我稍微說一下。當時噶瑪巴要我在不看手錶的情況下告訴他時間。我的答案偏離了實際時間十五分鐘,噶瑪巴說若一個人的內在脈輪完全清淨的話,總是能作出完全準確的預測。就在他說話時,我們有留意到他突然翻白眼數次,之後他便一聲不響地站了起來回到房裡,隨即便傳來他的誦經聲,伴隨著金剛鈴和手鼓的聲響。噶瑪巴正在將某個往生者的神識遷轉至淨土。我們不禁在猜測:「究竟是誰往生了呢?」正在四處奔忙的僧侶們和我們一樣震驚。兩個小時後,噶瑪巴仍在房內,而我們從收音機廣播中得知了答案。此時有關不丹國王因心臟病發在奈洛比(Nairobi)去世的消息迅速地流傳開來。不丹國王是隆德寺的大功德主,也是噶瑪巴的弟子。噶瑪巴顯然已經將其神識遷移。

當時的不丹國王曾在英國留學。他採取了那個時代較世俗和物質的見地,忘了自己的根。當然這和他在國外所受的教育有關。當他回國接管不丹政權時,恰好噶瑪巴就在不丹。他以非常傳統的方式提醒國王不要忘記心識的潛能。當地的寺院尋找三名轉世靈童多年,當噶瑪巴列出了這三名轉世靈童身體上的瑞相、家庭狀況、誕生時辰以及其他詳細的資料後,如今這些重大的職責將得以填補。這些轉世靈童住在偏遠的不丹山谷裡。不丹國王根據噶瑪巴提供的線索立刻派人前往尋找,儘管噶瑪巴未曾到過該地區,可是他所提供的資料卻非常準確。此外,這幾名兒童亦經過傳統的甄別試驗,從眾多相似的物品中辨認出前世活佛的遺物。他們也能辨認出幾位舊侍者。不丹國王深受這種不可思議的力量所震撼,後來更歸順成為噶瑪巴的弟子。這是一個明智的選擇。這位不丹國王在生時的生活富裕也充滿意義,如今往生後也獲得解脫。其實有一個很重要的事業挽救了他,也就是成為一個大力護持佛法的大功德主。這位不丹國王利用了自己的影響力與財力支持藏傳佛教三大「舊」傳承的發展,這些教派一般上很少獲得官方方面的贊助。即使是在心靈豐裕的東方國家裡,這種財物上(或世俗)的資助是很重要的。就像在現今的西方國家一樣,若少了教會或社會團體的支持,任何事情都將難以推行。人們必須以世俗的方式獲得靈性上的解脫,然而缺乏布施的慷慨心卻是我們行佛事業最大的瓶頸。有些人因為貪圖免費的心態所以才選擇佛教,他們所得到的加持其實才是最廉價的!

隔天,噶瑪巴帶著五十名喇嘛前往不丹。他必須出席不丹國王的葬禮以及新任國王的登基典禮。噶瑪巴將頭髮完全剃光,當他正在凝聚諸佛殊妙的加持力量時,他的身體一如既往地會看起來較平常更為魁梧龐大,聲音亦非常響亮。當他穿過院子走向吉普車時,我們近乎能看見圍繞著他的那一層保護能量。當他經過身旁時,我不經意觸碰到他的僧袍,此時身體只感覺到一陣酥麻,像觸了電般。

噶瑪巴在離開之前囑咐我們與四名年輕的活佛,亦即昆吉夏瑪巴、大司徒仁波切、嘉察仁波切與蔣貢康楚仁波切一起留在隆德寺。他要我們從眾年輕活佛的大師兄夏瑪仁波切處領受菩薩戒。

這一次恰好除了能讓我們有機會更進一步了解他們之餘,我們也正好能解答他們對西方國家的種種疑問。那些印度人似乎暫時忘了我們的存在。就這麼一次,我們身在隆德卻沒有接到任何電話提及有關簽證的問題。夏瑪仁波切向我們解說菩薩戒,其餘三名仁波切則一起坐在一旁聆聽。憑著他們日漸進步的英語,再加上我們零零落落的藏文和尼泊爾文單字,我們設法將要點都搞清楚。在偌大的佛殿裡,我們能夠和他們的心融合在一起,伴隨著上百名僧侶琅琅的誦經聲,這其中殊妙的體驗,竟久久無法散去。在回程途中,坐在吉普車裡,我們仍能感覺到他們安住在心中,繼續為我們開示廣妙的佛法。這一趟車程真的太美妙了!

「外在」的戒律能幫助眾生防非止惡,釋放內在的能量,以便能在靈性上取得發展。而密戒則是在灌頂或像修持大手印等殊勝法門時所領受,將世俗世界轉化為清淨之佛土。菩薩戒結合了這兩種層面,以「自律」及「利他」為主。凡事若以廣利眾生的角度出發,各個善行之功德將無量增廣,對於痛苦的見地亦會有所改善。對於一切不好的事不再將之視為是一個痛苦的淨化的過程,反之會視之為學習的其中一部份。只有了解痛苦,然後有意識地去體驗這些痛苦才能令智慧增長,日後方會有力去幫助其他陷於類似困境中的眾生。

菩薩戒分為兩大部分。第一個部分和「發心」有關:為了利樂一切眾生而發願成佛;不生起憤怒或嗔恨心;對所有眾生不分親疏遠近,一心為他們著想。這相當於是為人寬容大度的意思。毫無怨言地忍受那些有害的行為只是意味著我們縱容某些特定的人或文化更快趨向於滅亡而已。我們必須探討事物的「成因」,而不是在其「結果」上費心。今天,這意即要幫助窮苦的國家或社會降低生育率,同時普遍提高當地兒童的教育程度。只有具有智慧的慈悲以及影響力廣泛的行動才能帶來真正的成績。這能有效地消除大部分促成爭鬥的原因,沒有鬥爭便不會有痛苦。以善巧的方式消除任何憤怒的成因非常重要,因為沒有其他事物像憤怒的煩惱般更能徹底的摧毀辛苦累積的福報了。若從慈悲和空性的至高動機中墜落,又若將所發生的事以假當真並且作出負面的回應,這並不意味著受戒者完全破功。受戒者仍能清淨內心裡的忿怒,只是應儘快著手以便消溶掉其破壞力。當妒忌心和嗔恨心慢慢消失的時候,你就因此會覺得鬆了一口氣。

第二部分與「實際的行持」有關。受戒者決定能利樂眾生以及能夠讓他們達至證悟的行持、言語和意念,也就是修持六度:

\begin{enumerate}
\item
  \textbf{布施}:這包括身、口、意、財物受用、教育與護持。布施將眾生連結在一起,而且能豐富心靈。這意即不把美好的事物與經歷占為己有,而是大方地分享出去,並視施者、受者和施物皆悉本空。
\item
  \textbf{正確的行持}:這裡我們不用「道德觀」這種說法,因為這是長久以來教會和政府用以控制人民,使一切合理化的一種說法。活著時若沒有被政府逮獲,死後就由教會定罪送入地獄。在此,正確的行持意即善巧地利用一個人的身、口、意,作出會為眾生帶來永久快樂的事,同時避免造成任何傷害和痛苦。
\item
  \textbf{忍辱}:這裡主要的意思是避免生氣。如此一來便能夠避免摧毀過去通過布施和正確的行持所累積的功德。
\item
  \textbf{精進},或樂精進:這有助於將一個人現有的可能性擴大,永遠不會為目前所獲得的小小成績而沾沾自喜,反而會不斷向前邁進。戀人、律師和藝術家都能了解這四種行持的力量。對於不了解證悟的人而言,它們一般上也會被認為是不錯的品質。
\end{enumerate}

\begin{quote}
在這一個系統中,(五)\textbf{禪定}並不意味著是證得自心本性的最高方法。禪定的目標反而是要製造一種距離感,以讓我們能夠抽身出離去選擇生命的喜劇而非悲劇。理論上的內觀會漸漸成為一種全然的體驗,讓人能夠實際利用禪修的體驗於生活中而受益。這種「止」(藏:Shiney,梵:Shamatha)的禪修其實是更多更高的禪修技巧的基礎。
\end{quote}

(六)\textbf{般若智慧}:這不是一種任運的內觀,而是仍然具有概念的思維。如果之前五個是基礎,此種內觀便像是明亮的眼睛,指引我們前進的方向。其本質乃為知道行善是最自然不過的本份,因為主體、客體及所作都是相互緣起的,皆為「一因一果」,「自作自受」。

這六種波羅密多只是一個開端,圓滿了此六種波羅密多後尚有其餘四種不常提及的波羅密多。依據藏文的名相,分別是「Thab」(方便善巧)、「Monlam」(願)、「Thob」(力)與「Yeshe」(智)。此瞬即自生的不二境界讓人能圓滿地進行利生之事業。

昆吉夏瑪巴在一個誦經聲低沉而悠婉的儀式中傳授了菩薩戒。在整個過程中,我和漢娜多次拭淚。ㄧ開始時,守戒一點兒也不容易。我們受戒不久後便碰上了第一個考驗。那些印度人又再次請我們離開錫金。有很長的一段時間,我們偶爾必須甚麼都不想。後來我們才漸漸地明白到菩薩戒是一個無止盡的加持。它使我逐漸退去好鬥、極具侵略性的一面。像不殺、盜或妄語等「外在」律儀,只要受戒者不破戒或希望復戒,那麼此戒只需領受一次即可。然而像「內在」律儀或菩薩戒,若親近擁有這種戒體的上師亦能強化自己的戒體。此外,受戒者在日常生活中亦須對戒律常念不忘。「利樂一切眾生」是殊勝的大願力,沒有甚麼比這更美好的事了!

\textbf{第十六章}

\textbf{第一次回國}

回到索納達後,我們開始了第三不共加行之獻曼達。在此修持中,行者將七撮彩色米粒放在一金屬盤上,與之前第一、二加行一樣重複修持111,111
遍。觀想這些彩色米粒乃為宇宙中的一切珍寶,行者在前方置放一簡化的皈依境的表徵,然後無有執著誠心作供養。接著,行者再將米撮拭入方巾中,再建立起另一個完美的世界供養出去。在這一個修持中,行者的身口意三門合一,一起運作。其成果並非是概念的遊戲,反而是自我完整性的一種豐化。重複上百次此簡短的供養後,行者將亞洲社會認為最為珍貴的三十七種物品作觀想供養出去。這其中包括大象寶、舞女寶和將軍寶等等。沉醉於這種傳統供養的歡喜樂受中,行者也可將任何現代社會中會讓他人感到快樂的物品,如寶馬摩托車、跑車、ROLO的時尚服飾、藍綠藻、摩登女郎或夜生活等供養出去。

獻曼達是四加行中需時最短的修持,大概二十天便可完成。我們在布提亞布斯提寺的蓮師聖像前完成最後一萬遍的修持。有時候,當我們在巴士上被一些美國人傳教士纏著不放時,便會和他們說起這一座寺院的歷史。我們不希望對方浪費時間,於是會很有禮貌地告訴對方,宗教應被視為是一種良藥,它們唯一的功能就是幫助眾生。因此,對宗教感興趣者應該自行檢視最適合自己的宗教信仰,也應給予他人作出選擇的自由。基本上只要該宗教不會促使其信徒壓制女性或發起所謂的「聖戰」,信仰宗教應是很個人的事情。然而,經驗尚淺的傳教士會相當堅持與急切,怎麼說也不肯退一步,非得要迫使我們使出非常手段來反擊,把他們惹惱才善罷甘休。因此每當我們對他們說出以下這一個有關迫使他人接受某種宗教的故事時,耳根就會頓時清靜下來。

布提亞布斯提寺是為了佛教的護法所興建的一座寺院,最初的建設地點就在目前的地點隔鄰的一座山丘上,也就是在朝拉薩廣場(Chowrastra
Square)和溫德米爾酒店後方,前往該寺時沿途還會看見許多乞丐。大黑天金剛護法瑪哈嘎拉曾在此山頭示現數次,因此人們才將寺院建在那裡。不過後來英國人把寺院拆了,改建在目前的這一個地點上。原因是他們想要在這個能俯瞰尼泊爾、錫金和不丹的絕佳地點上建一座教堂。然而,教堂在竣工不久後就因不明原因塌了。他們再重建一次,接著又塌了。該教堂就這樣屢建屢塌三次。如今這一個地方又再次成為瑪哈嘎拉之地,不過其左邊的那一塊地區已由印度教和一些當地宗教所佔據。後來英國人又在大吉嶺的大街上興建教堂,此昔日的教堂如今亦已變成供人消遣的戲院,由始至終根本就沒有人真正從這座教堂中受益。

噶瑪巴這次將會在不丹逗留很長的時間。前任國王去世後,他要辦的事情不少,而且他剛找到新的轉世活佛,也要到不丹全國各地去傳授灌頂。就在我們剛完成獻曼達的修持時,卡盧仁波切和其侄子亦剛好抵步,我們非常感恩。他們在寺院裡主持了一些儀軌,正好也使用我們在修持曼達時所使用過的米粒。

卡盧仁波切離開之前,就像其他上師般捉弄了他的某個弟子。有些雪巴酒鬼經常來到寺院裡向仁波切投訴。這次他們要投訴天氣:天空遲遲不下雨,莊稼都旱得蔫頭耷腦了。卡盧仁波切非常了解這些人的脾性。這些雪巴酒鬼每到周末就會搖身一變成為共產黨,醉醺醺地在鎮內吵鬧地遊行示威。當他們在街上遇見仁波切時,他們會脫隊跑向仁波切請求他的加持,之後又再回到隊伍中繼續高喊示威口號。波卡祖古(Bokhar
Tulku)天性靈敏,十分善解人意,他原是藏北草原的遊牧民族。波卡祖古自幼便跟隨著卡盧仁波切,由仁波切一手撫育長大。他有大部分的時間都是在閉關禪修中渡過。每當那些雪巴人邀請他去主持法會時,他總是會驚愕失色地一臉慘白。而卡魯仁波切每次卻甚麼都不說,只是掛著一臉牽強的笑容一次又一次派他前往。

大吉嶺地區和雪巴人的家園只不過是隔了幾個山谷的距離。這些雪巴人來到大吉嶺後都變成了一些廉價的政治口號的犧牲者。我們在尼泊爾所遇見的雪巴人個個性情平穩和諧,然而這裡的雪巴人卻個個情緒沮喪,也盡顯壞脾氣。在缺乏了佛教傳統為基礎的情況下,宗教和酒精之間的關係便受到打擾,而他們就傾向於較容易取捨的一方。這一次他們來到寺院向仁波切求雨,仁波切說:「嗯,你們將會得償所願。」通常當仁波切想要給人教訓時,臉上就會出現異乎尋常的平靜表情,所以我們便等著看好戲。果然不出所料!就在幾個小時內,山上忽然烏雲滿布,隨即便下起了滂沱大雨,山上到處開始淹水,就連梯田邊緣多處亦被水沖塌。這些雪巴人冒著傾盆大雨正努力地修補損壞的梯田,實在不難想像在宿醉下與洪水搏鬥該會是多麼折騰的一件事。

噶瑪巴總是有辦法鼓勵和強化人,使人不斷成長。他會把美好的體驗帶給大家,直到愚昧的習氣和業力使他們從喜樂的頂端墮下。之後,他又會另有辦法使他們再次振作起來,直到弟子們了悟到這一切的起伏不外是心性的顯現而已。當我們能跳脫出對自我的執取,專注於為他人帶來永恆的覺悟時,我們自然而然就成為了噶瑪巴的法侶,而非學生了。另一方面,卡盧仁波切則以較個人化的方式趨近學生。他會以較強烈的方式打擊學生的自我。他甚至會催化一些狀況,讓學生們能迅速了解到當下其修行的進展。就如他的學生當中若有人試圖以表面的佛教道德觀來審判或約束其他人時(漢娜和我是極度厭惡這種事,可是有些學生很輕易便會墮入這種陷阱),卡盧仁波切對這件事是了了分明的。當他們尚未真正體驗某些事卻擅自發展出自己較頑固的意見或主張時,卡盧仁波切心裡也是一清二楚的。這時仁波切會以一貫的語調,從一般的傳統教法轉為告訴我們有關「狂慧」或瘋行者竹巴袞列(Drukpa
Kunle)等人的故事。

竹巴袞列(DruKpa
Kunle)是不丹的一名瑜伽行者,也是噶瑪巴的弟子。他曾與不少女弟子發展出不凡的親密關係。有關他的故事已經出現英文譯本,儘管他這種強而有力的說法風格和人格特質難有人可媲比,他可說是證悟事業最具體的一種展現。他的每一個行動都將眾生引向解脫,以一種強悍的方式將人們從有限的妄想和謬見中搖醒。不管人們的修行之道是小乘的逃避、大乘的發心或者是金剛乘的純淨見地,他們都能從他向拉薩大昭寺等身佛的喊話中學習:「你有今天的證悟,因為你心裡只想別人;而我會在這兒,是因為我只想到自己。」

卡盧仁波切亦知道該如何趨近我和漢娜。我們(尤其是我)總是自我感覺超好。對於一些不好的事,我們也能從中找到趣味。在困難的情境中,我們總會有轉身的餘地,並且將一切當作笑話來看待。漢娜的個性和諧,她通常只會因看見他人受苦而感到痛心,而我總是覺得所有的經驗都充滿刺激。對我們倆而言只有一方面是不堪一擊的弱點:也就是愛的這一個部分,我們對彼此的執取。這是我們的家族遺傳,全家人都是這個樣兒!無論是我的父母親抑或是祖父母,他們的關係都非常親近。就像是具有兩個頭的眾生,配偶之間的運作方式就是男人成為女人的力量和安樂的泉源;而女人則是男人的智慧與開放性的搭配。我們能夠感覺彼此一體同心,每一次的分離都會讓我們覺得十分不快。卡盧仁波切當然知道這一點,然而當我們在四月份的結婚三週年紀念日當天開開心心請求他給與加持時,他微笑著說道:「真不錯。接下來的三年,你們可以分別出家當和尚和女尼。這樣對你們都好。」如果他的目的是要激怒我們,那麼他可說是成功了。接下來的幾個小時,我們不得不努力去克制心裡許多不好的想法,否則之後又要花不少工夫去淨化內心。

最後一個不共加行(或稱第四加行)之上師相應法需要上師進一步的指示。這一個修持具有很長的前行準備階段,引介行者逐漸向傳承開放。之後行者將會領受有關智慧方面較深的教導,發成佛之願心,最後祈求諸佛菩薩賜於加持和護念。由於卡盧仁波切不在索納達,我們不希望從其他人處領受這方面的口訣與指示,因此唯有等待仁波切歸來。一切事物就像以往般都是最好的安排。時間並沒有被白白浪費。在等待的這段期間,我們的朋友來到索納達。金‧文斯、尼克和伊華、歐蕾與涵、克勞斯、力奇與荷拉的因緣成熟了,為領受噶瑪巴的加持而來到這一個地方。之前卡盧仁波切要我們去宣揚他的教法,如今我們能明白他的意思了。這時也恰好能和他們一起去拜訪之前一直未有時間前往的一些當地的佛教聖地。當仁波切和嘉千抵達索納達時,所有人還特地跑到小型火車的鐵道那裡去迎接他。同時我和漢娜也發現,我們真的非常敬愛這名老戰士。在領受了仁波切的加持後,我能感覺到他的雙手在頭上良久。

我們這幾個西方人在克里斯大宅像尼泊爾人般生活了太久,狹小的空間內總是擠滿許多人,亦作出太多無意義的談話。當身邊的朋友們即將開始實修時,我和漢娜鬆了一口氣,因為這意味著我們又將能繼續各自的修行。卡盧仁波切在傳授我們觀音菩薩的灌頂後,第一次讓我負責講解觀音菩薩的修持法門,朋友們亦正式踏上修行之路。這對我來說是莫大的榮耀。若一個人要教授金剛乘修法,他必須非常確定其中的細節,同時對於他人而言是一個良好的楷模方可。若他人因為教授者的行為怪異而對修法失去信心,那麼即使是再殊勝的法門也沒有用。

當我和漢娜領受了最後一個加行的口訣和指示後,我們終於可以展開相關的修持。在這一段間歇期間,我們的發心其實變得更為鞏固和強烈。與這群老朋友相伴的日子,我們的心所開展出來的警覺性已被一些慣性的思想、業力和言語所消溶和擊退。我們顯然需要額外的距離,以便不被這些妄習所左右。我們必須這麼做,這是毋容質疑的事。我們深深地了知這些習氣與煩惱的根本就是卡盧仁波切所講的執、嗔、貪、慢種種心毒。這些洞見讓我們覺得「修心」其實就是刻不容緩的一件樂事。

就在我們即將完成四加行的修持之際,索納達周圍的環境也出現了變化。巴基斯坦東部爆發了一場戰爭,導致民不聊生,許多難民因此紛紛湧入山裡。爆生戰爭的地區距離我們的住所只不過是十公里路,儘管雨季時節索納達的天空總是雷聲隆隆難以分辨,不過我可以肯定自己有時確實是聽到了戰場上的砲火聲響。在前往大吉嶺的路上,沿途擠滿了身穿粉紅色長袍的巴基斯坦難民。身處異鄉的他們在寒冷的天氣裡凍得慌,ㄧ雙眼眸也凍得通紅。在前往加爾各答途中,這些難民的身影亦處處可見。鐵道旁已經廢棄多年的水泥造儲水設備,如今已然成為這些難民一家大小的「家」。

我們當中有人主動提議要向這些難民伸出援手,可是似乎不得要領。此時,在我們內心裡也有一種非常強烈的感覺,逗留在索納達的日子即將結束,這一種傳統的學習是時候告一段落。我們亦漸漸意識到一個能讓自己專注於修行而不受打擾的外在環境條件可遇不可求,因此更積極把握在索納達所剩無多的日子。

外事警察已經通知說要找我們談話。在德里的官員已發現我們假借學生之名申請簽證留在限制區裡生活,並未入讀任何印度大學。如今,我們必須選擇離開山區,或離開這一個國家。我們寫了一封友好卻甚無意義的信件,心想按照當地官員的辦事速度,我們仍可享受幾個月「自由的空氣」,不過這一招再也不管用了。大吉嶺那裡有幾名官員也是佛教徒,曾試圖幫忙我們拖延,然而由於在德里的長官已經緊盯著我們,他們也無可奈何。不久後,我們又再次接到要我們選擇離開山區或直接離開印度的通知。這一次我們給他們寄了某醫生開的醫藥證明書,不過我們也心知肚明這拖不了多久。就在我們加緊步伐,希望能趕在離開索納達之前完成加行的修持時發生了一件事,讓我們覺得像是被雷電擊般備受衝擊。某個曾從越戰逃離、神經質的美國人匆匆忙忙跑到我們房裡來,然後大喊說:「卡盧仁波切要去美國。他們已經在收拾行李了!」那是一九七一年的秋天,我們一直以為自己很冷靜,隨時準備好迎接一切,然而這突如其來的消息對我來說真的太震驚了,我直接就跑到洗手間去。這是這的嗎?我們開始感覺自己全身發燙。

消息是真的。現在正是將佛法事業弘揚至西方國家的時機。這位老喇嘛將會前往西方國家傳授密勒日巴尊者和噶瑪巴的傳統法教。他與嘉千以及阿尼措加剛從噶瑪巴那兒聽說他們的不丹護照已經辦好。

在這最後一段紛擾的日子裡,事實證明佛法也能攝化動物眾生。我們在前往大吉嶺途中,看見有一隻大黃狗被一輛吉普車輾過,躺在馬路的中央。我們連忙跑到它身邊,不斷重複為它念誦「嗡阿彌疊瓦舍\textsubscript{利}」,然後將念珠放在它的頭上。大黃狗沒有驚慌,鮮血從它的嘴裡湧出,然後它就斷氣了。我們知道它已解脫,將會投生更好的地方。看見了念咒對小動物的功效,我們的內心除了是喜悅,也感恩。

卡盧仁波切在出發前給弟子們傳授了長壽佛灌頂作為離別的禮物。這一次有許多西藏人前來參與,仁波切以手中的法器給予每一個人加持,為眾人消除種種逆緣與障礙,增福延壽。在整個過程中,我一直感覺右手很燙、很痛。當我攤開手一看,乍然發現手掌上多出了一條非常深刻的手紋。由於不知道這一道手紋意味著甚麼,我們因此將它視為是諸佛大德鍛鍊我們的象徵,為日後將教法弘揚至西方國家作準備。

看著這些藏民依依不捨地與仁波切道別,實在是令人為之動容。他們對上師非常執著,每一次的離別就像是截肢般讓他們感到痛心。法教中不時強調行者應視上師為一面鏡子,反映眾生真實圓滿的佛心。然而,許多人仍然非常執著於上師外在的形相。雖然上師外在顯現的形相會加深我們對加持的覺受,可是同時亦會令我們變得脆弱。當我們的藏族朋友們為了仁波切的離開而感到痛心的同時,我們這幾個西方人卻感到非常振奮。卡盧仁波切即將前往西方國家,我們的朋友將會有機會見到仁波切。這真是太棒了!

這一次卡盧仁波切的幾名西方學生將會與他同行,一起從索納達出發。喜樂塔欽負責支付所有人的費用。兩名加拿大人阿建與英格麗德亦一同隨行。他們剛完成了加行修法。另一對法國夫婦丹尼斯和露斯瑪莉則飛往巴黎為仁波切在當地的活動打點一切。

卡盧仁波切在飛往美國途中拜訪了以色列,也在羅馬與教皇會面,最後抵達巴黎。他在巴黎逗留約三個星期,我們有幾位朋友亦前往拜訪。我的弟弟伯恩在瑞士的農地裡收到我們的來信後,與一群朋友開著他的沃爾沃轎車南下。正當他要開始尋找法會的地點時才愕然自己已經來到大門口。當時仁波切正在傳授觀音菩薩灌頂,他們正恰好趕上。之後,仁波切從巴黎直飛北美,而且還在美國滯留了一年之久。由於仁波切平白無故給了喜樂塔欽性生活方面的建議,後者老羞成怒,拒絕支付仁波切回國的費用。

我們在索納達的日子結束了。德里方面傳來的消息越來越具威脅性,而且我們覺得老是讓當地的警察替我們掩護對他們也很不公平。然而由於各個部門都不知道其他部門的情況,因此我們學會周旋於這些不同部門之間,然後從中取巧。我們下山前往西里古里,多攢了額外兩個星期的時間,恰好可以完成最後的加行。當這兩周的期限快結束時,另一個部門給我們發下了錫金的簽證,我們又開開心心地上山去見噶瑪巴。

這次我們能夠再次回到隆德與朋友們相聚,感覺非常美好。這一個地方已然成為我們的「淨土」。噶瑪巴再次為眾人傳授一系列的灌頂,其中包括勝樂金剛(梵:Chakrasamvara,藏:Pal
Khorlo
Demchok),以及傳承中其他廣被修持的法門。整個過程中,我能感覺有一股強烈的能量湧入,那些西藏人對我的面部表情甚為讚嘆。在他們眼裡看來,這是強烈虔誠心的一種展現。但是往深一層想,這其實意味著我的身體上部的脈輪仍有淨化的空間。不過,我的身體已慢慢停止抖動。在格隆姆帕嫫(Gelongma
Palmo,又稱帕嫫比丘尼)灌頂期間,該名嚴肅的英籍女尼為我們傳授了下一個修法的口訣和指示。那是以第八世噶瑪巴及其殊勝加持力為對象的修法。由於她的藏文不好,再加上她具有很強烈的基督教、印度教和格魯巴學派的背景,後來我們也不得不放棄其中的一些指示。

那些印度官員又再次被我們嚇壞了,尤其是岡托克的達斯先生。他認為我們這一次真的太過分。我們要申請延長簽證,負責的官員必須先經由加爾各答辦事處去處理。我們給對方撥電,不過不用說也知道,對方當然是不肯幫我們延長任何簽證。他甚至出言威脅,我們要是再敢賴在印度不走,他可要派士兵來將我們押走。我在整個通話中的態度非常友善,我這是想要讓對方知道他已經無法激怒我們,不過這一招並無法為我們爭取更多的時間,所以我們只好藉噶瑪巴的幫助,迅速決定去向。我們心裡盤算著這次應該會回到尼泊爾,回到和藹友善的喇嘛傑珠身邊,不過噶瑪巴似乎另有安排。他說:「你們回家去。」我們錯愕不已,問道:「回家?哪兒的家?」噶瑪巴說:「當然是歐洲的家了。」

對我們而言,這是一個巨大的打擊。我們在心中盤算了各種可能性,心想若不是尼泊爾,那麼也許會前往錫蘭(Ceylon),又或者在噶瑪巴的加持下,我們也許還能到不丹去。可是歐洲,我們卻連想也未曾想過。噶瑪巴的字字句句就如冰水從頭頂澆下。當我們的心情平復下來後,回到歐洲的想法其實也頗讓人振奮。我們已準備就緒,如今已獲准在接下來的日子必須自力更生,這將會是我們的一大挑戰。作為第一批學習這些廣妙精深的法教的西方人,如今我們要將這些法教傳遞至一個擁有截然不同的文化的西方國家去,繼續將其發揚光大,這其中必須付出堅持不懈的努力。

噶瑪巴在離開之前送了我們一幅漂亮的唐卡。唐卡上所描繪的是象徵佛心的大悲、大智與大力的聖像。他答應會時常護佑我們,並且喚了一台貨車將我們送往岡托克路,在那裡已有一輛吉普車等候多時。突然之間,我們已踏上歸途,重返歐洲,噶瑪巴的話語仍在耳邊縈繞。他說:「我會一直與你們同在。」我們特地繞道大吉嶺,讓那些警察們能「正式」驅逐我們出境以便洗清「縱容」我們延長逗留的嫌疑,再從那裡乘火車前往德里。我們在一名友善的婦女的家(她只收留斯堪的納維亞人)留宿了一、兩天後,終於收到父母親寄來的電報和一百美元。我們手上仍有五十美元,加上這些錢就足夠我們回程所需的旅費,也就是差不多七十五美元左右。我們趕緊去找雅克,一個魁梧強壯的法國人。他打算開那一台老舊的貝德福巴士回歐洲,只是旅費比我們所預算的超出了許多,一個人需花一百美元左右。然而,雅克卻不介意收下我們僅有的錢作為車資,還為我們提供一日三餐。他其實可以不必虧錢就能找到其他旅客將巴士坐滿,可見他真是一個善良的人。

我們就這樣離開了印度。我和漢娜坐在巴士前窗的位置打坐,也不時和雅克聊天使他保持清醒。雅克的駕駛技術可不是蓋的,他一路鳴響著車笛,載著約五十名嬉皮士在印度和巴基斯坦的大街小巷上,穿越過人群、動物、拉車。阿富汗的夜晚凍得讓人發慌。我們那廉價的睡袋使用了兩年後已經變薄,現在我們終於明白當地人為何會如此喜歡陽光了。某個夜晚特別寒冷,漢娜搬出其餘的衣服蓋在身上,而我差不多每半個小時就會醒過來,凍得直跺腳。正當這寒冷的天氣越來越難熬之際,突然有一群看起來非常乾淨的狗兒從沙漠的邊緣出現。這些狗狗的頭像我們的頭般大,短毛,奶油色,它們來到我們身邊安靜地躺下,我們那虛弱的身體漸漸地暖和起來。我們也為它們念誦咒語和給予它們加持作為報答。雅克仿若不知疲倦地一路開車,每晚就只是睡幾個小時,不過他的食量倒是很驚人。這輛巴士的駕駛盤已經非常老舊,我們看著雅克操控著它,每到轉彎處就似乎更接近歐洲一步。

我們在伊斯坦堡(Istanbul)停留約一、兩天,換不同的乘客,也拜訪了一些曾經熟悉的地方。如今在沒有迷幻藥的作用下,這些地方不再擁有醉人的魅力了。其中最糟糕的不外是古漢酒店(Gulhane
Hotel)。這一個地方曾經是歐亞之間吸毒者的聚合地點。如今這一個地方可不是鬧著玩兒的糟糕。人們簡直是可以直接把它當作是一家醫院,立即開始治療。這裡頭的人幾乎沒有一個不是生病的,最糟糕的肝炎患者,其身色已成暗金綠色。此外,這裡的氛圍也已有所改變。早兩年前的那一種開放與理想的同志之誼已不復存在。如今在土耳其若被發現持有大麻會被抓去坐牢,判刑三十年。西方人幾乎無法在那里的監獄裡生存。因此,所有人都慌張了。最近就有一個美國人被抓。他似乎知道美國政府不會幫他,於是搶了某個警員的配槍,然後瘋狂掃射整個警局裡的人後才中槍斃命。

土耳其和其他回教相關的事物都屬於亞洲的一部分。可是我們卻很難定義保加利亞(Bulgaria)與南斯拉夫(Yugoslavia)。格拉茨(Graz)是一個美麗、充滿文化氣息、自由的地方,歐洲的各種風貌和氛圍都在此地聚合,完美有力地展現。看見這種種景象,我們便知道自己已經抵達。早在一九七一年,我已經有想法要在這一個城鎮裡建立佛教中心。這也是我第一次對一個地方抱有這樣的想法。這一定是一個美好的願望!今天在格拉茨,一些與我關係非常親近的學生正在管理和領導兩間很棒的禪修中心。鎮內的那一間較小型,鎮外的則較大型。我們很開心能夠再次感受歐洲房子的寬敞與潔淨。儘管我們仍乘著那一台老舊的巴士走在前往荷蘭的路上,可是看見高速公路上的汽車都以每小時超過100英哩的速度行駛著,那種感覺實在是太美妙了!內心裡有一種說不出的自由與暢快!在這裡,計程車司機即便是在倒車也比印度大部分正在向前行駛的汽車來得快。感受到一個地方無限的可能性以及其充沛的活力是很美妙的一件事。在一個步伐緩慢的國家待了這麼久的一段時間後,美麗的歐洲又再次瑰麗登場。

我們和特地開車南下的父母親在阿姆斯特丹重逢。我們彼此激動地擁抱,雙眼無法止住喜悅的淚水。我們一家人開開心心地一起開車回到丹麥。漢娜的父母亦看起來氣色不錯;我們大部分的老朋友在幾天內一個接一個出現。消息很快傳了開來。他們認為這金剛乘一定是甚麼非常有效用的東西,才能讓他們這個在過去總是惹麻煩的老朋友歐雷脫胎換骨。儘管我們自己並沒有察覺,如今我們已擁有全然不同的視野,顯然和兩年前離開哥本哈根時已經大不相同。

我們回到歐洲後的首要任務,就是要讓人們認識到過去這兩年來我們在喜馬拉雅山區裡所學習到的寶貴知識。金剛乗之道需以一種超越文化區隔的方式向西方人引介。許多善良的人們由於被一些奇怪的書籍所誤導而嚴重感到混亂,因此一本針對藏傳佛教作出清楚解說的書籍必然能讓他們從中受益。他們必須知道,藏傳佛教不僅僅是局限於一些冗長沉悶的儀軌或短暫的神通力而已;反之,它是結合了實際的心理學和哲理,並且是擁有兩千五百年經驗的一套方法。這一套方法教導我們如何活著、面對死亡、作出更好的投生、創造一個具有意義的生活,甚至是如何利益他人。這些教法必然有它的用處。若要讓人重新認識藏傳佛教,我們手上就擁有很好的資源幫助我們達成目標,那就是卡盧仁波切在索納達的指示和教法,我們一共寫滿了厚厚五本的筆記本。筆記本裡出現很多不斷重複的內容。每當有新加入的學生,卡盧仁波切就會從頭開始教起。不過即便如此,那也是一個很好的例子。雖然我們應該發願追求更高深的教理,但是先從基本功做起,穩固現有的基礎,這會為日後的修行作出良好的準備。我們手頭上的資料足以用來編輯一小本書。在接下來的日子,我們寸步不離樹林中的小屋,專心投入書本的編寫工作。

我們平均每天會花十至十二個小時實修以第八世噶瑪巴為對象的上師相應法,其餘的時間不是在撰寫小本子的內容就是在睡覺,小木屋內也逐漸形成一種能量場。這一個能量場防止一切干擾,我們能清楚地感覺到它的存在。我們家的廁所很「鄉下」,外建在距離小木屋約二十呎的不遠處,每次在前往廁所途中,心裡總會冒出許多奇怪的想法。可是每當我們回到小木屋或能量場所籠罩的範圍時,內心又會再次變得明亮澄淨,繼而又能專心地投入工作。即使是朋友們騎著摩托車從哥本哈根遠道而來,他們也無意留下來找我們閒話家常。他們每次只是將食物放在窗戶上後便會自行離開。我們在小木屋裡寫書寫了將近一個月,從初一一直寫到大黑天金剛護法日(下一個初一的前一天),終於完成了《Teachings
on the Nature of
Mind》(心之本質)一書。這是我所寫過關於佛法的六本書中的第一本。今天這些書籍已被結集整合為《The
Way Things Are》一書,並且被翻譯成多種歐洲語言。

接下來,我們最緊迫的問題就是資金。我的父親曾經撰寫過約五十本德文課本,因此「尼達爾」一名在丹麥學術界頗吃得開。我們很快便找到工作,在哥本哈根以西約六十公里的一所學校任教。我們每天必須冒著風雪開車去上班,而且運氣似乎還不賴。儘管我們所開的那台福斯小巴的輪胎花紋已被磨光,有時甚至得在結冰的路上推車發動引擎,可是總是能及時趕到學校。除了全職教課,晚間我們也在哥本哈根的另一所學校當保潔人員。朋友們有任何疑問時就會來到學校找我們,也順便幫忙。在早晚身兼兩職的情況下,我們賺了不少錢。如今再次投入那麼多體力活中,其實蠻多樂趣的。

我們其實也知道待在歐洲的這一段日子並不會持續太久,噶瑪巴一定會給我們指示該在何時回到印度。有一天,當漢娜仍在課堂裡教課,我來到學校裡的某一個小房靜坐。我們在這裡工作了幾個月,也存了一些錢,心裡能感覺到某些事情即將會發生。我在禪修中祈請噶瑪巴為我指示方向。我在靜坐中看見房門被打開,然後有三個小孩扛著偌大的白板走入房內,白板上是一幅亞洲的描略圖。地圖上除了印度南部,便沒有其他邊界、城鎮或地點的標示。地圖上印度南部的位置畫有一個圓形,圓形內有一偌大的笨拙字跡寫著「ANGALORE」一字。這幾個字如雷電般擊中了我,它可以是「Bangalore」(邦加羅爾),也可以是「Mangalore」(芒加羅),而在這兩個地方的中間恰好是南部西藏難民營的所在地。這一個地方就是我們的下一個目的地。翌日早晨,我們收到阿陽祖古的來信,更加確定了這一點。阿陽祖古就是在隆德的八臂瑜伽度母灌頂法會中告訴我們有關南營的那一個喇嘛。他在信中再次邀請我們前往拜訪。指示已經很明確。我們和家人朋友一起渡過溫馨的聖誕節和新年後,便收拾好行李,再次動身往東出發。

\textbf{第十七章}

\textbf{印度南部的西藏難民營}

我們選擇廉價的阿拉伯航空,經由敘利亞(Syria)飛往孟買(Bombay);當時人們對自己所支持的文化並不會想太多。由於在過去幾裡天忙著慶祝各種節日而不怎麼睡覺,漢娜和我在飛機上基本上就是不斷在補眠,間中醒過來幾次吃了一點東西果腹,看了看高空下那一片廣袤的沙漠幾眼,轉眼間飛機便已在印度降落。孟買是一個很難讓人振奮起來的地方。如果說德里給我的印像是「憤怒」,加爾各答給我的印象是「混亂」,那麼孟買便是以「傲慢」稱王。我們在抵達孟買後的當天傍晚,搭火車前往邁索爾(Mysore),再從邁索爾轉搭巴士前往西藏難民營。我們在擁擠的車廂內顛簸了一夜,隔天又換了幾次火車,所以根本沒辦法好好休息。再加上這一趟行程,我們帶了不少伴手禮隨行。我們千辛萬苦才順利通過海關的檢查,因此實在不希望因為一時的疏忽而讓人有機會順手牽羊。當第二天早晨又要換火車時,我們看了看地圖才發現自己壓根還未離開孟買的範圍。這時我們才知道原來在印度西部,搭巴士和汽船會比坐火車更有效率。沿途上不斷改變的地勢與風景讓我們大開眼界,相較於印度北部或東部,這裡能看見更多不同的族群。火車每到一站就會有不同相貌特徵,給人感覺迴然不同的人群登上火車,喧鬧之間「盧比」和「派薩」(印度的貨幣單位,約百分之一盧比)二字不斷響於耳邊,姑且無論各族群的人們說的是甚麼語言,字字句句之間,彷彿這兩個字才是不變的真言。

我們在火車上顛簸了一天一夜後,終於在隔天中午抵達邁索爾,一個位於芒加羅與邦加羅爾之間的小城鎮。我們坐上馬車,負責拉車的馬兒非常健壯,這種情景除了在阿富汗和佛教國家之外,在其它亞洲地區並不多見。正如所料,伊朗的情況最為糟糕,那些負責拉車的馬兒都瘦得只剩皮包骨,在拉車前往聖城瑪什哈德(Mashad)途中不斷被馬夫猛烈地鞭打。我們帶上一些橘子果腹,及時趕到巴士總站,擠上了一輛已經非常擁擠的巴士。邁索爾的感覺很不好。沿途有好幾個地方都在播放一些反嬉皮士的影片,而這些印度人總是一廂情願地對眼前所見深信不疑。這些影片一定是把嬉皮士描繪得非常不堪。他們顯然是對我們很反感,有時我甚至必須奮力去守住我們的座位。巴士穿越過無數的小巷與胡同,一排排的樹開滿了花,幾個小時後,那些新舊不一的西藏五色經幡終於第一次出現在眼前的右方。我深受眼前的畫面所打動,無論如何也止不住奪眶而出的淚水。我試圖以咳嗽、打噴嚏,甚至是別過頭去佯裝欣賞窗外的風景來掩飾激動的自己其實是在哭泣。看見佛法的智慧與修持出現在像這樣的一個地方,我真是感到太開心了!我頓時生起了一個非常強烈的願心,誓要力保佛法在此地繼續發展下去。

印度人不要西方人出現在這些西藏難民營裡和他們有所掛勾,所以我們沒在難民營的大門口下車,反而多坐了幾公里路到另一個地方去,再從那兒按照幾個西藏人的指示,抄另一條羊腸小徑,穿越田野和樹叢,直抵我們的目的地──第四營。這一條小路也能讓我們避開駐守難民營大門口印度官員的盤問。白人對印度人來說是金錢與地位的象徵,儘管他們無法對我們做些甚麼,不過卻非常嫉妒這些西藏難民擁有如此完善、具體的組織。所以任何能讓他們用來抑制和控制這些西藏難民的客人的手段都將無盡歡迎。

如今正值二月初旬,藏暦新年的各種慶典尚未落幕。西藏人居住的庫格(Koorg)一區座落於海拔一千米處,這裡的天氣宜人,白天不會太熱,夜晚也不至於過於寒冷。沿途一片片的玉米田種滿了玉米,西藏人的井然有序與道地的印度人形成強烈的對比。這些精進的西藏人只要擁有一個平和的生活,即使離鄉背井來到印度也一樣能夠取得成功。

第四營並不難找。由樹枝搭建而成,泥土地面的二號小屋亦然。阿陽仁波切與他的三名兄弟一起住在小屋的左方。西藏人一如既往地消息靈通,未幾便來了一群彬彬有禮的小童帶我們到八號小屋去。小屋擁有堅實的牆壁,屋裡也做好了接待我們的準備。他們為我們準備了許多對歐洲人來說還算是可消化的小吃,我們邊吃邊聊,與他們有不錯的接觸。我們想向喇嘛請法,學習波瓦法門,不過喇嘛已出外遠行幾個星期,不過我們已經給他傳了電報,再過幾天喇嘛就會回來。直到喇嘛回來之前,我們將會和他的三位兄弟在一起。他們再過幾個小時就會抵達。這段期間,我們和招待我們的那戶人家相處和睦,也正好有機會讓我們練習已生疏不少的藏文。我真的很欣賞他們家的供養壇。供養壇佔了整面牆。雖然和我們的傳承傳統不同,(招待我們的主人家來自藏傳佛教的薩迦教派),不過我們仍很開心能再次被佛像和唐卡所包圍。

我們花了一整天的時間組裝帶來的禮物,教導招待我們的那戶人家如何使用膠帶和打字機。這時喇嘛回來了。喇嘛主修波瓦遷識法門。我們即將成為第一個學習此修法的西方人。那一片開闊的田野上只有一棵樹。隔天,我們就在樹下靜坐禪修,怎知一群好奇的人不斷湧上前來盯著我們看,讓我差點兒沒暈倒。後來我們到鄰近的村莊去,這一個村莊和佛陀選擇進入涅槃的地方擁有同樣的名字:「幸福的小鎮」拘尸那羅(Kushinagar)。我們在一個潮濕,尚未建成的洋灰屋裡找到了在印度難能可貴的靜謐,好讓我們專心靜修。

在孟各(Mundgod)的難民營距離嬉皮士的聖地果阿(Goa)只不過是幾個小時的車程。或許這正是印度政府嚴禁西方人出入的原因。我們搭夜車在抵達胡布利(Hubli)之前的一個小鎮下車,然後在清晨乘著一輛擠滿了工人的貨車直抵難民營。我們必須趕在那些印度官員醒來之前到達。我們躡手躡腳在幾個熟睡的官員身旁走過,沒被發現截查遏止。這個地方不小,從那個靠近合作商店的高地上遠眺可清楚看見整個地區的十二座村莊。這些村莊周圍是一片一片的田園,分布在一片廣袤的丘地上。這一片土地原是一座森林,直到最近才被剷平。

不知何故,這一個地方總讓我們覺得不凡。這不僅是因為那一座新蓋的寺院或那些新舊不一、隨風飄揚的五色經幡而已。我們當然也很開心能看見這些寺院與經幡,只是這一個地區就是有一種無法言喻的獨特。後來,我們才漸漸發現到這一個營地獨有的特質:原來我們來到了一個「完整」的世界。在孟各期間,我們仿若是在西藏生活,這其中有某種無法言喻的完整性與平衡性,我們只曾在不丹偏遠的山谷居民身上感受過這一種特質。當然,這裡無論是氣候或周遭的環境都和真正的西藏截然不同,然而古老的西藏那種不曾間斷的能量場卻被帶到此地,繼續發揮著其功用。我們將會在這一個難民營中生活六個星期,與當地居民融為一體,以他們的方式過生活。這些西藏難民融入文明的物質世界只不過是五十年的時間。他們那質樸、簡單的生活條件,人與人之間所展現的那種本能的合作精神,期間各種的體驗在我們心中留下了很深刻的印象,久久無法忘懷。我們終於開始明白在這幾千年來究竟是甚麼讓脆弱的人類在種種艱苦與磨難中生存下來。

在這一段期間,某個灌頂法也向我們展示了加持力的意義與力量。有一天,喇嘛正在傳授無量光佛的灌頂,此時有人將一個患了肺結核病的病人抬進來。我敢說他的體重一定不超過六十磅。我們讓出前排鋪有地毯的座位,讓病人能夠躺在柔軟的地毯上,自己則退到後方與眾人一起。他能活著真是奇蹟;他整個人已經瘦得只剩皮包骨,還不時朝杯子裡咳出血和痰。我們幫忙用紙張蓋著杯口,向眾人解釋那些漫天飛舞的蒼蠅會到處散播病菌,因此不得不慎。

當喇嘛手持著法器給予他加持時,我們從他的眼中看見了變化。他的雙眸變得較為明亮,臉上痛苦的神色亦漸漸舒緩開來。他顯然已經進入大樂漸次增長的境界中,感受萬法萬物最純粹的本質。

就在灌頂法會的兩個小時後,他已能以意生身自由進入任何想去的地方。他離開了這已被摧毀的臭皮囊,追尋大自在的境界。如今,他已經持有通往極樂世界的「門票」。其實醫生早就診斷出他所剩時日不多,他卻奇蹟式地撐了一年。

孟各人的政治性沒有像大部分中、西部的西藏人般強烈。他們倒是擁有不少文化上的「遊戲」。不過,他們很快便發現和我們玩那些遊戲是毫無意義的。有一些體格矮小、身材豐滿的婦女發現漢娜其實是個女的,那麼我很自然便不是出家人,因此在背後竊竊私語說了一些閒話。我根本就不在意,不過有時卻也會令那些玩弄我的人非常難堪。他們當中有許很人的態度自在、彬彬有禮,不難看出他們內心的圓熟。這應該是和他們在日常上的修持有關:每當日出日落,難民營裡處處充斥著誦經聲以及清脆悅耳的鈴聲和手鼓聲響。這些藏民的屋子大多是由洋灰或樹枝所搭建而成,小小的屋子總是過於擁擠。撇開屋內悶熱不通風的情況、肺結核病與貧困的生活不說,他們當中有許多人可說是典型的在家修行者與瑜伽行者。他們將生活上的體驗和金剛乘佛教的見地與方法相結合,他們的行為模式日後也給西方國家帶來了深刻的啟發。

這裡的難民大多來自西藏的西部和北部地區,靠近拉達克(Ladak)和蒙古(Mongolia)。他們是遠方草原上和平的遊牧民族,與驃悍的康巴戰士不同。一九五九年,康巴戰士帶領約八萬五千名同鄉和大部分喇嘛逃離西藏,而這群西、北方的遊牧民族則是直到入侵事件爆發八年後才成為紅軍的目標。文化大革命奪取了他們的自由,因此這些人民被迫摧毀靜修之地,約有八千人在身無分文的情況下經由拉達克逃到印度。在中國政府的壓力下,印度人將大部分的難民安置在運畜的小車上,然後將他們載到南方的這座森林中,途中約有百分之三十的難民因為無法適應驟變的天氣和疾病的侵害而丟了性命。葬火一直持續燃燒了幾個月。那些倖免存活者在這片土地上發展了農業方面的技術,如今更指導身為東地主的印度人。多虧這些來自北方的人,如今將近所有當地的原居民都擁有不錯的工作。

他們剛開始來到此地定居時,曾有一群野象闖入他們居住的地方踩死了好幾個人。這些野象顯然比較喜歡這一座森林原來的模樣。每經過一些地方,就會看見有人指說:「塔希就是在這裡被踩死!」或「多瑪就是在那裡被追!」有人請求噶瑪巴在此地區設下護佑的能量,那麼這些動物便不會再來侵犯。我們恰好聽說在拜拉庫(Bylakuppe)也發生過類似的事。那裡的營地也是不斷受到野象的侵害,直到達賴喇嘛蒞訪該地區後才恢復太平。

難民營就像是他們自己的一個小小世界,不過這一個純樸的世界似乎不會保持永恆不變。這裡已出現了兩個很危險的跡象:傳統文化已逐漸開始衰退。大部分的年輕人已經離開難民營,來到城鎮裡謀生。此外,電源供應亦開始入駐當地。只見桅杆已經建起,而且也接上了第一條電線;這等同是為無意義的電台噪聲開路。我們曾以輕描淡寫的方式提醒他們要注意提防這種改變,感覺像是對他們擺出屈遵俯就的樣子似的。密續行者會憑自己的能力去轉化與接受事物,不會去逃避。沒有其他方法在能像它般使心靈更為成熟與堅毅。然而,這一個方法雖能快速帶來效果,可是同時亦很危險。至於在孟各招待我們的那戶人家是否會受到這些新進的發展和改變所影響,未來會告訴我們答案。

我知道有一天我們必須離開這裡,就像早幾個月前在丹麥接獲指示要來到南部的難民營時一樣。我們再次感受到噶瑪巴的召喚,不過我亦知道去見噶瑪巴之前,我們應該先到尼泊爾拜訪喇嘛傑珠。我們向孟各的這一群朋友道別,再次揹上行囊踏上旅途。我們在南部的難民營裡生活了六個星期,當我們從大門口離去時,那一群印度警察簡直是嚇得目瞪口呆,彷彿我們是從天上掉下來的一樣。

\textbf{第十八章}

\textbf{人生如夢}

加德滿都並沒有多大的改變。冬天的冷霧已散,山谷裡的田野與樹木又是一片萬綠蔥鬱。如今博德納佛塔已是最「流行」的地方,我們大部分的朋友都已搬到那裡附近居住。他們也很快幫我們在那裡找到了一間出租房,房裡還有一個面向大佛塔的小陽台。我們學習波瓦法門後,心裡很想祈求喇嘛傑珠傳授夢瑜伽的法門。在耐心地等待喇嘛傑珠回到加德滿都的同時,我們在充滿活力的加德滿都度渡過了一段美好的時光。

這段期間,我們見識到波瓦法門的效用。佛塔周圍有很多饑餓的野狗,它們會撲向任何能當作是食物的東西,甚至是小孩童的排泄物;我們丟給它們的干酪外皮和不新鮮的麵包尤其受歡迎。我們通常會先念咒和祝願後才將食物布施給這些狗狗,這些動物對此會作如是想我們是不清楚,可是它們似乎有注意到施主的一番心意。有一天,我們很晚才從鎮裡回來,當我們走回家時,路經佛塔外的那一片空地,有一群野狗突然從四面八方出現。我們大概被十二隻野狗所包圍,它們全都朝向我們狂吠,當下我只覺得大事不妙。要是被這些生病的狗狗咬傷,情況可不堪設想。我把漢娜推向塔壁,拔出刀子準備和它們拼了。就在這個時候,我們才發現原來事情不是像我們想的那樣。這些狗狗根本沒有要攻擊我們的意思,反而是在向我們「道謝」,頓時有一股暖流湧上心頭。它們那粗而低沉的嗥吠聲(它們也只能發出這種聲響),聽起來像是在對我們說「謝謝你的食物」。我們心裡非常感動,希望能為這些可憐的動物付出更多,而這一個機會很快便來臨了。

有人(也許是遊客)投訴博德納周圍這些患有狂犬病的野狗。有一天,博德納來了一群警察。這些警察從麻袋裡掏出一大塊一大塊下了毒的肉塊扔給路邊的這些野狗。這些毒肉將會使它們慢慢中毒身亡,整個過程非常痛苦。有些西藏人試圖將狗狗藏在家裡,可是他們也只能保護那幾隻和他們較為親近和相信他們的狗狗。中午時分,警察已經離去。大佛塔的入口處躺著十來隻奄奄一息的狗。當漢娜去取水讓它們解渴的當兒,我用噶瑪巴的舍利給與那些垂死的狗狗加持,好讓它們大部分能安寧、輕鬆地死去。前來收屍的貨車等到我加持完了後才開始辦事,周圍的藏民都感到非常欣慰。

當時有人開玩笑說,以後我們必定會遇到許多聲音粗而低沉,鼻子長長的學生;我們在如此殊勝的地方幫助過這些狗狗,以後他們將會投生轉世為我們的學生。

不久後,喇嘛傑珠回來了。我們很開心能夠再次與喇嘛見面,他對我們這一段期間的點點滴滴感到十分好奇,要我們仔細告訴他所做過的一切。當我向喇嘛請法時,他一直低著頭,似乎是在等待指示似的。後來他看了我們一眼,說道:「好,我會傳授你夢瑜伽(藏:Milam)的修法,這是行者在睡夢中繼續禪修,發展明性的ㄧ種修持法門。」

甚至早在喇嘛答應傳法之前,我們在感知上便已經有所改變。一切事物彷彿變成了一場夢境,接下來和喇嘛在一起的六個星期,一直持續地處於這一種狀態之中。夢瑜伽(禪修)創造一種境界:在任何「知道」的情況下,行者無論是清醒或熟睡,都存有「觀者」。它是一個巨大、透明和受到保護的空間。行者的心將會在這種狀況下學會了知所經歷的一切過程,甚至去掌控它們。這種修持將會增強行者的覺知和觀照力,知道即使在清醒的狀態下,那也只是集體夢境的一部分,所有意識上的體驗其實只是與他人所共享的一組投射而已。當背景和見地越是相似便越能合作無間,共創一個集體的體驗。這一個在觀念上的基本差別最能在種族和文化的摻和之間顯現。當這種光明的覺性扎根時,從我們強烈的心念中而生,誤以為「萬物皆為真實」的煩惱便會自動瓦解。

六個星期過去了。我們就住在喇嘛傑珠樓下一個布滿灰塵的儲藏室裡,不管是清醒或睡著總是處於夢中的狀態,並且一直感受到喇嘛的能量。有一隻鼻子特別長的河鼠,一天裡總會來到我們的房門前好幾次,自在地四處觀望,然後悄悄走到我們留給它的食物旁,叼起,然後離開。除了這隻河鼠之外,我們便沒有其他外來的訪客。這是一個殊勝圓滿,帶來了甚深覺受的閉關。自此,沒有任何事物是「狹隘」或「沉悶」的了。

完成修法後,喇嘛傑珠又再次恢復他原來的樣子。他說,現在是時候去見噶瑪巴。我們為了能更快抵達,再加上機票只是九美元,所以這次搭飛機前往尼泊爾最東部的比拉特納加爾(Biratnagar)。傑珠仁波切親自載我們到機場,兩天後我們已經身在索納答,那裡的人已確切知道噶瑪巴會回到錫金的日期。仁波切再次很準確。噶瑪巴隔天就會抵達錫金。

索納達在一片愁雲慘霧中。我很訝異看見他們因為喇嘛出外遠行而痛苦。卡盧仁波切已滯留加拿大超過一年,儘管他們請他的弟弟到寺院來坐鎮,寺院裡卻始終沒有任何動靜。看他們對上師如此依賴,我雖覺得感動,但也未免會認為這樣的行為很幼稚,所以我們決定要以不同的方式來做事。佛教的目標就是要讓人變得獨立,發掘出個人的力量,像索納達這樣的情況著實不該。當我們在西方推廣佛法事業時,啟發人們的潛能成為了我們的首要之事。

我們的時間拿捏得正好,在提斯塔橋等了一會兒便看見噶瑪巴的車輛。噶瑪巴對我們微笑,然後將手指放在我的軍裝夾克上。這時我才想起來在夾克內的暗袋裡裝有一封我們要交給噶瑪巴的信件。噶瑪巴給我們加持,然後我們開開心心地跳上他的其中一輛貨車。在一片藍天之下,我們在車內打坐禪修,跟隨噶瑪巴一起進入錫金。

接下來在隆德的日子非常殊勝,我們時時刻刻都保持警覺。我們被視為是噶瑪巴的「初級法侶」,和噶瑪巴的關係亦邁進了一大步。他曾多次嘗試告訴或讓我們明白一些事,可是我們一直到自己開始對他人的發展負起責任時,方能深刻地體悟和了解他所說的話。每當他在給予他人有關管理禪修和閉關中心方面的建議,又或是當他正在講解必須完成的修法,以及其方式時,他都會安排我們在場旁聽。他不時會詢問我們的想法,在西方國家會如何處理一些事等等,他總會耐心地領聽,可是卻從不說「好」或「不好」。我們可以確定在接下來靜坐禪修時或夢境中,必會向我們展示一個明確的解答。

一天,一切似乎準備就緒。就在印度方面開始施壓之前,噶瑪巴把我們叫到面前。他送了我們與友人金‧文斯一份非常殊勝美好的禮物,然後對我們說,作為第一批西方弟子,如今我們擁有他的加持去開創禪修中心。我在丹麥展開第一次的弘法事業後,便開始在歐洲以及其他地方積極弘法。噶瑪巴允諾會給與我們指引,讓他的加持力與教誨在我們身上儘速彰顯。在震驚之餘,一股炙熱的能量湧上心頭,這一股能量即使是到了今天仍不斷在持續燃燒著。噶瑪巴將我們送回歐洲,為此生弘法利生的事業正式掀開序幕。

就這麼決定了!我們已迫不及待地想要展開這一項任務,沒有多餘的空間躊躇不決。如此重大的責任固然會讓人的內心激動不已,然而我亦清楚它絕容不得我們半途而廢。我們要將這一種能讓眾生解脫的教法推廣至西方國家,一定不能令上師乃至整個噶瑪噶舉傳承蒙羞,因此必須打起十二分精神。在這三個星期裡,一直到我們在孟買擠上了一票難求的班機之際,我們不斷注意到有許多殊勝的事發生,我們驚嘆之際,心裡亦深深感恩。這些殊勝的事都是預示我們未來定會取得成功的好兆頭。當我們在菩提迦耶匆匆離開佛塔內巨大的金身佛像之前,我們看見佛像對我們微笑。在那輛載著我們的行李的小摩托車絕塵而去之前,我們衝出佛塔奮力追逐,心裡只有一個想法,覺得「佛陀也頗幽默的嘛!」不然我們還能說些甚麼,又能作如是想呢?

看見佛陀微笑的那一眼激起有力的淨化作用,在前往拜拉库(Bylakuppe)途中,甚至在那裡逗留的期間,我竟然生病了,而且痛苦不堪。我的整個頭顱和喉嚨就像不斷裂開一樣。我們乘坐加爾各答---馬德拉斯---邁索爾列車抵達拜拉库(出乎意料的所有乘客都有位子坐!)我們恰好趕上一座新噶舉巴寺院第一棟樓為地板抹上水泥的儀式。這一座寺院就建在山丘上,可以俯瞰整個難民營。我雖然因為發燒而整個人迷迷糊糊,我們認為在西方國家展開任務之前必須先在東方國家做一點事,而且我知道身體上所經歷的疼痛不會是阻擾我們的障礙,這一點是很重要的。前往孟買的火車非常緩慢,幸好途中我們在各地發了電報,才保住了之前預訂的機位。

就在十月份某一個寒冷、浮雲儘斂,皓月當空的傍晚,我們抵達美麗的哥本哈根。

\textbf{第十九章}

\textbf{正式開展弘法利生之事業}

我們的父母親早在機場等候多時,他們很高興我們這次沒有離開太久。在他們身邊站著的是我們的一群老朋友。他們當中好幾個人和我們在同一個時期脫離毒品,我們的第一本書《Teachings
on the Nature of
Mind》(暫譯:心之本質)更為他們帶來了深遠的影響。基於過去豐富的人生經驗和種種善因福報,如今他們已具有能力,並且很樂意去利益他人。他們今天會出現在機場,這其中的象徵意義已經很清楚,我們亦感到非常高興:我們在西方弘法利生的工作,將會從這一群形形色色、富有經驗以及關係親近的理想主義者開始。

隔天傍晚,噶瑪巴的口傳及加持力便已經在一群朋友中覺醒。雖然覺得不可思議,可是我們在當下便明白他最後那句話的意思:「無論身在何處,我將永遠與你們同在。」在丹麥的某個海岸旁,美麗的黃昏下,我第一次在歐洲大陸傳授噶瑪巴的教法。我能聽見自己在說法,也能看見自己用手握著噶瑪巴的舍利給大眾加持。這是一種比自我認同、苦苦掙扎的「我」的意識更為強烈和清楚的覺受,這本身就是一種大樂。

若以更廣的角度來看,這一個位置意味著漢娜和我作為一個被動、無憂無慮的佛教徒的日子將漸漸結束。若只是針對個人的人生,我也許不必對自己太嚴苛,可是倘若必須對他人的生命負責便不能如此。

這一本書的續集《Riding the Tiger》(註:暫無中譯)(亦是由藍海豚Blue
Dolphin
Publishing所出版),道出佛教來到西方國家後的發展健康與不健康的一面。不過,此書續集最主要仍是描繪噶瑪噶舉傳承在西方國家的獨特發展。最近,昆吉夏瑪巴在德里(Delhi)、多頓(Dordogne)、伊利斯塔(Elista
)和弗吉尼亞(Virginia)建立了噶瑪巴國際佛學院K.I.B.I.。這是在法國開展的一個寺院組織。目前為止,分布在全世界各個城市的兩百所在家與瑜伽禪修中心皆由我所負責管理。

從一九七二年迄今,多年來的弘法工作一直都很振奮人心。我很開心能夠看見西方國家能夠吸收傳統佛教的精髓,並且在過程中賦予以全新的生命。

一千兩百五十年後,蓮師的預言終於兌現:

「當火牛以輪奔馳(火車),鐵鳥騰空(飛機)之時,當藏民如蟻分散世界各地,吾之法教將會降陸白人之地。」

願有更多聰明獨立的朋友有機會接觸殊勝之法教,解脫輪迴之疾苦,利樂一切諸有情!

                                      ─ 全文完 ─
