\chapter{向卡盧仁波切學法}

抵達大吉嶺後,我們換上最體面(平常只會在特殊場合穿的)一套衣服,前往申請學生簽證。當時外交部的警員相當厭惡嬉皮士,我的短髮白襯衫,加上漢娜清麗的面孔讓他留下挺好的印象,因此沒有阻攔我們提出申請。德里腐敗的官僚制度十年如一日,這意味著在他們發現我們的存在之前,我們至少擁有六個月的自由時間。之後,他們會需要更多時間才會發現領著學生簽証的我們其實只是留在限制區與一群西藏人一起生活,而不是花錢報讀印度國內任何的大學。最後,當決定性的時刻來臨時,我們也知道如何拖延而不被趕走的方法。這樣屈指一算,無論想做的事是甚麼,我們也已為自己攢了不少時間。

我們下榻的小旅館就在郵局的附近,一天收費住宿包早餐才不到一美元。沒過多久,我們便發現其實這一棟低矮的木樓鬧鬼。旅館內處處充斥著十分詭異的能量。此書較早期的版本,我並沒有透露旅館的名字。一來不希望影響這位和藹的老婦人的生意,更不想將這家旅館變成旅遊勝地,把旅館內的幽靈嚇跑。如今轉眼廿八年過去了,時過境遷,說出名字也無妨。「Shamrock
Lodge」(新樂旅舍),是旅館的名字。我也不是說旅館內的鬼魂或靈異能量特別負面,「它們」只是非常迷惑,不過由於它們的能量很強,以至於有時會相當惱人。一些相熟的朋友曾說,它們有時會將桌上的書本推倒在地,又或者會將一些經文法本隨便堆放在地上。隆德的吉梅醫生曾經下榻此旅館一個星期,他每晚都得先把自己灌醉才睡。他總覺得那些鬼魂在猛擊他的背部。住宿期間,我和漢娜只是間接受到一點影響,它們似乎知道我這人會還以顏色,所以惹不得。只是有它們在身邊,其混濁的能量令人在每天早晨醒來時有一種空茫的感覺,加持的力量也消失不見。此外,當它們在周圍時,我們與噶瑪巴在一起時所感受到的大樂覺受便會削弱,我們非常抗拒這種感覺。除此之外一切安好,而且看女房東與她飼養的那隻黑貓咪說話也頗為詭譎有趣。

我們在等待著父母把錢寄過來,同時也忙於收集各種有關噶瑪巴的消息。此時,拜訪卡盧仁波切的契機終於出現了。我們的一名美國朋友曾經拜訪過那個地方,這次我們將與他同行。在晴朗清新的天氣下,我們打算走到目的地。在前往古木的前四英哩路,綿延起伏的干城章嘉山脈就在我們身後右側巍巍而立。接下來走向西里古里的五英哩路,印度一片片遼闊的低地平原就在眼前。就在離索納達不遠處,我們找到了卡盧仁波切的寺院。再往上走兩公里,山坡上的加油泵對面偏左側,矗立著一棟棟陳舊、漆成綠色的木屋。我們在路上遇見的第一個西藏人是一個非常討喜的年輕和尚嘉千,他主動提出要帶我們去見卡盧仁波切,順便充當我們的翻譯。他帶領我們走進一間長形的房間內,室內的牆板高及腰間。房間最內側,小木桌後方的床鋪上,坐著一個男人。那一張苦行者般的莊嚴相貌,讓人無法忘記。他就是大喇嘛卡盧仁波切。

\fullPageFigure{figures/kalu-rinpoche.jpg}{卡盧仁波切是尊聖噶瑪巴派遣至西方國家的第一個喇嘛。\\仁波切特別活躍於法國和加拿大弘法}

雖然卡盧仁波切看起來就像是一個非常虛弱的老人,可是卻擁有很強的心力。實際上,仁波切並沒有外表看起來般虛弱。後來有好幾次我們曾開車載著仁波切到各地遠行,他比起其他年輕的喇嘛更能吃苦。一九八三年雨季,仁波切以八十歲高齡在五個月內傳授了兩千個灌頂,直到一九八九年圓寂之前仍然活躍於弘法利生的事業。我們向仁波切頂禮。雖然頂禮時仍會覺得很彆扭,可是我們不希望失禮隆德,同時希望藉機表現出我們有教養的一面。卡盧仁波切微笑著接見我們,同時也給了我們一個強而有力的加持。仁波切接著拿出一張世界地圖,我就像在亞洲其他國家時一樣,指出格林蘭和丹麥的位置,然後補充說人們都住在地圖上標示綠色的那一小塊,其餘白茫茫的一大片主要是雪地。我也向仁波切吹噓著丹麥人究竟有多強壯和堅毅。他提出許多和我們在尼泊爾與錫金的喇嘛有關的問題,對於噶瑪巴的消息尤其感興趣。當他問及我們希望逗留多久時,我們思考了一會兒,回答說:「直到噶瑪巴從不丹回來。」仁波切說:「好,你們可以留在這裡和其他人一起學習。這是我的姪兒嘉千。」他指著替我們翻譯的年輕人,繼續說道:「他會協助你們找住的地方。」

我們出去後,嘉千說有兩個美國人:蘇和里察,就住在山坡上數百米處一棟偌大的老房子內。他說屋子很大,一定會有多出的房間。一開始時,我們並沒有很想要住在索納達。這一個地方毫無魅力。寺院本身也甚無吸引力,整個地區就是缺乏了一種獨特的文化。索納達就只有一條破爛不堪的道路,以及兩旁遭雨水摧毀的木屋和小攤子而已。這一個地方也非常吵鬧。每當有小型的火車路過,尖銳的鳴笛聲便不絕於耳。而且在這個地方似乎所有東西都鋪著一層薄薄的煤塵。索納達不像大吉嶺能將干城章嘉綿延山巒的美景盡收眼底。最糟糕的是,這裡距離辛格先生和我們在東部唯一相信的郵局太遠。這裡的警察也不喜歡西方人與當地西藏人有太多的接觸。他們希望遊客們能安分守己,萬一有甚麼事不合乎心意或與其意見相左,他們甚至會以「取消延長逗留」為藉口偷偷恐嚇遊客。然而,我們也不忘卡盧仁波切的建議,爾後當我們覺得大吉嶺不再具有我們可以學習或對成長有所幫助的事物時,我們便往南搬到索納達,而且依然是向當局申報新樂旅舍的地址。

我們隨同幾名美國、加拿大和法國人(他們當中有六人旅居索納達,其餘者只是過路人)來到卡盧仁波切的房內,參與了第一次的開示。神奇的是,在場除了替我們翻譯的嘉千,便不見寺院裡或鄰近西藏難民營裡的藏民、當地的雪巴人或塔芒族人(Tamang)。這些族群沒有語言方面的障礙,一定都能聽懂仁波切的開示,可是卻半個人影也不見。他們對佛法甚具信心,可是卻似乎對深入學習佛法不甚感興趣,當時的我們對於這種矛盾的現象甚為不習慣。我們心裡亦清楚知道,像他們這種學習的傾向必定不能在西方國家裡發展開來。

仁波切的開示和地獄有關,各種不同的地獄。我們完全沒有料到仁波切會作出相關的講解。除了以前在上學時的基督教課(也許當時的我們曾乖乖上課,間中也許試過直接曠課),漢娜與我一直都在一個人文主義而非宗教的背景中成長。我們對生活感到相當滿意,不需要一些滑稽可笑的神靈,又或「人死後會墮地獄遭火燒」之類的恐懼為生活添味。這種所謂的教言對我們來說只是一種用來操控弱者,然後從中詐取金錢的卑鄙手段而已。

我們原以為卡盧仁波切會作出心靈方面更深層的開示,看他展示神通或如雷電般迅速的證悟的跡象,現在可尷尬了。仁波切坐在眾人面前,指著一個畫像幼稚的畫軸。畫軸上所畫都是眾生被火燒,遭切割,受群山所擠壓,或遭冰封於巨大冰塊中的種種情景。他說了幾則陰森森的故事,都和「八個熱地獄」、「八個冷地獄」、「鄰近地獄」和「間歇地獄」有關。這些地獄各各不同:在有的地獄中,眾生以不同的方式互相殘殺,一陣冷風吹過又使之復活,眾生又再次自相殘殺,同樣的過程不斷重複。在別的地獄,眾生被放入一個裝有金屬熔液的大熔爐裡,或遭蟲子所蠶食。這些地獄一個比一個痛苦,一個比一個噁心。卡盧仁波切還說,眾生一不小心就很容易墮入這些地獄中。這是很沉重的課題。我受不了了!西方國家的靈性書籍一致認為眾生若已為人,便不再會墮入動物、鬼或地獄惡道中,一切惡行只會令其「凍結」(受困)在同一個境界(人的境界)而無法有所提升而已。我向仁波切提出此觀點,並且告訴他在西方國家,無論是唯心論者、通神論者、人智唯靈論者等一概人都同意此觀點,不過仁波切卻回答說:「這是佛陀的教言。」

當時我們仍住在大吉嶺,當我們乘著老舊的吉普車翻越過古木山口時,我們不斷在討論這一位老喇嘛。我們認同他所說的故事充滿了異國的情調,然而這些傳統故事會更適合作為民間故事來傳誦,或者用來恐嚇或操控一些人,與廣妙精深的佛法根本搭不上邊。雖然我們知道他能看透我們的心思,想必也知道我們這裡的情況,我們依然忍不住會想:「也許他是真的老了。」

隔天,負責翻譯的是喜樂塔欽。他是美國某個非常富有的銀行家族後代,操一口流利的藏語。他和嘉千顯然是輪流上陣,替仁波切做翻譯。昨天的開示結束後,我們連忙去翻閱伊文斯‧溫茲(Evans-Wentz)的書籍,他的書是我們主要的信息來源。我們相當確定自己可以不必去理會這些地獄的故事。如果這天仁波切講一些有趣的課題,我們一定非常樂意原諒他老人家。怎知卡盧仁波切又再繼續解釋有關地獄的種種,這次的主題是地獄中的酷刑,以及受苦的期限。不一會兒,我們便開始覺得仁波切的開示枯燥煩悶極了,耳邊就只剩一連串天文數字在嗡嗡作響。

和我們一起參加開示的那幾個美國人、加拿大人和法國人並沒有像我和漢娜般對地獄的課題感到厭煩。就在我和漢娜都快氣得冒煙時,他們總是乖乖坐著聆聽,並且將仁波切的開示寫成筆記。雖然我們喜歡這名老戰士(仁波切是東藏的康巴),喜歡他那無懈可擊的臉龐和笑容,可是也忍不住捫心自問,自己是否真的需要這方面的開示。不過,我們希望再給他一次機會。隔天,我們依舊前往索納達參加開示,看看這次他會講解些甚麼。難以置信的是,關於地獄的課題還是沒完沒了。這天仁波切描述種種會令我們墮入地獄受苦的憤怒、惡行、起心動念與言詞。第四天,當仁波切又開始講解地獄時,我簡直是氣瘋了!

我們並不是為了好玩才這樣每天舟車勞頓來到索納達。從大吉嶺到索納達大概一個小時車程,我們就這樣每天往返兩地。我們搭乘的吉普車屬於半密封型,裡頭大概擠了約十五人。有的人半蹲,有的人身子一半是在車外,人人咳個不停。這些老舊的「路虎」(Land
Rover)或「威利」(Willy)吉普車皆由廉價材料所建造。吉普車的馬達已經磨損,加上使用的是第三流的機油,所排出的廢氣把人嗆得十分難受。每一段車程,吉普車都必須停下來好幾次進行不同的修理。這一路的顛簸,加上為了參加開示所花費的開銷,我們認為自己該聽一些較具吸引力,或至少有意義的開示。我打斷他的話,說:「我們已經聽過了。」他看著我,嘴角勾起了一抹諷刺的笑容,回答說:「沒錯,但你聽懂了嗎?」忽地,我才驚覺原來我們一直沒有領會他的意思。他說這些地獄故事,不是因為他對受苦的眾生無動於衷。他亦沒有妄下評價或欲伸張道德正義的意味,又或者想要藉所說的話獲得任何報償。他當然也沒有要操控任何人的意思。他講述這些有關痛苦的故事,是因為他真的相信,他更希望聽的人能引以為鑑,遠離麻煩。然而我們卻一直漠視仁波切的用心良苦,不但沒生起慈悲心,反而感覺厭惡,只想聆聽美好中聽的事。若一直這樣下去,我們將永遠無法進步,這似乎也提醒了我們過去到處漂流冒險,或不斷參加各種灌頂儀式的日子也該告一段落。如今我們應該通過實際的修行去完全吸收妙法的精髓。我們慣性地四處遊走,讓我們很容易墮入只吸取自己想聆聽的事物的陷阱,我們只是假藉靈性之名與自創的一套想法自我欺騙,避免去揪出我執的根源而已。

我們對卡盧仁波切作為上師生起了強烈的信心。仁波切絕不會縱容我們。他一定會教導他認為對我們最為有效益的教法。我們就在那個時候決定遷往索納達,全心全意追隨仁波切學習佛法。消除了對仁波切的抗拒心理以後,我們也較能理解他所講解的地獄了。地獄裡的酷刑不是壞心眼的天神給與眾生的懲罰,所謂地獄純粹只是一種負面的心理狀態。地獄是過去種種惡行的結果。過去我們協助不少人度過可怕的迷幻經歷時,不就曾見識過心識是如何將奇怪的事都當真嗎?既然人會受苦是因為種種自我造作的妄想所造成,那麼便沒有所謂「正常」狀況下的受苦,不是嗎?基本上,地獄不就是一種心理疾病嗎?雖然我們難以親見受苦中的地獄眾生,可是眼下不就有許多人生活在恐懼和疑慮的地獄中嗎?絕對的證悟之心,也會有著相對受限制的體驗,這種體驗是毫無堅固自性的。儘管這一切不外乎是一種心理投射,可是由於感覺是如此真實,所以才會招致痛苦。

\fullPageFigure{figures/kalu-rinpoche-sonada.jpg}{這是卡盧仁波切在索納達開示時最喜歡的座位}

現在這些觀點上的分歧似乎說得通了:西方人相信「人類的持續性進化」使到兩大領域產生了混亂。其一是身體上的遺傳進化。人類的身體確實會不斷產生基因上的進化,直到廿世紀初期青黴素(又稱盤尼西林)和機關槍的出現背離了「物競天擇」的原則而使之出現了大逆轉。其二是有關心性能量的成長。其實這一種對心性的檢驗應該更早——就從「我」的轉世投生開始(每一個人都認為自己有一個的獨立的「我」存在)。這應該包括在受胎之際,心和潛意識相應的環境與身體結合的過程。更複雜的是,我們必須明白在未來世中未覺悟的心會持續地認為自己的業力跟感官所呈現的世界是真實的。固此,佛陀才會告誡我們業力甚深不可思議,所以不要對業力產生不必要的妄想和推測。

然而,這並不意味著本性「明空」的心具有形體。就像那些相信輪迴的物質主義者所相信的那般,當意識離開身體之後,它必須和另一個身體相結合。就如《西藏生死書》中所講的一樣,此過程就像俄羅斯輪盤般不斷輪轉。當人在死後其粗淺的意識心消失時,細微的潛意識便會現前。其中最強烈的業習和動機將會徐徐將我們引入六道中相應的一道,然後就投生了。這一個過程就如虛空一般是無始的,但卻有終——成佛之後便會終止。這是一個必然而充滿痛苦的過程。

因此,我慢心重者,其異熟果報是投生天道;妒忌心重者則其異熟果是投生阿修羅道;執著心重者若具足福德則會投生人道;愚痴的異熟果則是墮畜牲道;貪心重者則其異熟果報是墮鬼道;而嗔恨心重者則其異熟果報就是投生剛剛仁波切所說的地獄之中。對覺悟者而言,這種種的過程不外乎是一場夢境,空無自性,如遊戲般幻現,充滿無限的可能性。從純淨的層次而言,所有現象的顯現都是任運無礙而圓滿的——不過對於追逐這些境界著而言,一直以來都是充滿痛苦的。就算是在受限的因緣中現起的大樂,其實也是自心本性中大樂的其中一個微弱的影子而已。就算如此,大多眾生就連有限的快樂也時常是了不可得。

山坡上的「克里斯大宅」仍有空房。這一棟以都锗風格建造的巨大木樓是由一名義大利籍的百萬富翁所建造。據說,這位百萬富翁想要在地勢較高處生活以治好自己的肺結核病。殊不知大宅尚未建成,他就已經病逝。如今住在那裡的是一名英國牧師的尼泊爾籍年輕寡婦,她也和來自兩至三個不同世界的人們住在一起。大宅的底層和周圍的小屋住著年輕寡婦的親戚,大家都維持著尼泊爾傳統的生活型態。在如此偌大的家庭中,他們彼此皆以兄弟姊妹相稱,不易辨是哪家的孩子。他們的父親,通常是休假時來到此地的廓爾喀士兵,不過若要追查下去,恐怕也沒人能確切知道孩子的父親究竟是誰。無論如何,底層總是熱鬧轟轟的。這名年輕的寡婦替牧師生了一個兒子。這兒子是一名搖滾音樂人,無論是穿著或思想都在追尋歐洲的時尚潮流,只是他的「潮流」總是免不了像印度人般遲了半拍而已。最後,大宅的最高層,這裡大部分的空間都被我們這群口中總是唸唸有詞,持誦著咒語的西方人所占據。

此時,父母親的首一百五十美元已經寄至。附近的銀行只願意以一美元兌六盧比,大吉嶺的自由市場兌九盧比,不過由於在加爾各答能獲得一美元兌十二盧比的高價格,因此我們便展開了往返索納達---加爾各答的快速行程,為了節省時間,我們每次都會選搭夜班火車。像這樣的一日行也成為了往後幾個月的常事,而且這些行程其實也成為了我們在學習上相當重要的補助。我們從平靜的卡盧仁波切的寺院,來到了一個荒涼喧囂的城鎮,娑婆世界裡的痛苦在此地尤其明顯可見。回到仁波切的開示,我們只能贊同說:「真的是如此的。這些痛苦不斷延續著。我們剛剛都親眼看見了。」

我們每次都會儘量小心拿捏行程,等到仁波切講解至各種地獄的結尾時才出發,期間會缺席幾個法會儀式,然後趕在仁波切開始講解有關「餓鬼道」和「畜牲道」之前回來。餓鬼道眾生因為執著心與貪心重,所以才會墮入此道受苦。餓鬼眾生缺他同心,心路過程由貪慳所主導:它們總是努力想要得到或抓住甚麼,不過最終由於無法真正「擁有」或「保留」甚麼,因此總是感到十分沮喪。它們非常執著於吃和喝的東西,這亦對感知造成扭曲。許多餓鬼眾生因為業力所致,只要接近食物,眼前的食物便瞬間化成了穢物和火焰。有的餓鬼眾生會則產生假象,其喉嚨和嘴巴如針孔般細小,再怎麼吃偌大的肚子也無法飽足。有關餓鬼道眾生的種種描述與博斯(Bosch)和勃盧蓋爾(Breugel)所畫的畫像十分吻合。仁波切在這方面的開示非常湊效。直到今天,當我們想要購買任何東西時仍會很自然地先問自己:「真的需要嗎?」

眾生由於無明,自覺性地扭曲事物,沒有善用內心的潛能,因此才會投生畜牲道。家畜最主要的痛苦來自於人類的奴役和屠宰。至於生活在大自然界中的動物則免不了互相殘殺和吞食。大部分畜牲道的眾生散居於土地裡或水中。它們沒有能力控制自心,也苦無追求解脫的機會。動物、餓鬼和畜牲道統稱「三惡道」。它們皆為心識中負面印記的異熟果報,因此應避免種下投生此三惡道受苦的種子。這三種境界實在沒有任何值得我們去追求的事物,所以應敬而遠之;因為一旦下墮三惡道就難以出離,亦將苦無機會自利或利他。

卡盧仁波切接著告訴我們說:「即便是三善道的眾生也受苦。」「六道中沒有絕對的庇護之處。」當所主導的欲望和過去世所累積的善行令眾生得以投生人道,八苦亦自然會隨之而至。其中最明顯的乃為生苦、老苦、病苦與死苦四大苦,以及另外較「間接」的四苦。他們分別是愛別離苦、怨憎會苦、求不得苦及五蘊盛苦。仁波切很仔細地解釋了每一種苦,告訴我們這些種種被我們視為是很「自然」的事物其實是無法令人滿意的,它們其實都是偽裝了的痛苦。他說:「天人和阿修羅也不值得羨慕。」阿修羅因為過去生生世世的善行所以具有莊嚴妙好之身軀和神通力,不過他們強烈的嫉妒心使到他們的世界中充滿戰爭和打鬥。他們永遠不滿足,而且往往在極憤怒和嗔恨的狀態下往生,因此下一世的投生一般會變得異常痛苦。

% NO IMAGE TODO: 西藏的六道輪迴圖

即使是娑婆世界中最為喜樂的「天道」,快樂也是有限,因為強烈的「我」見依然堅固。不管是色界、無色界的天道眾生,儘管只要心有所想就能心想事成,享受各種天樂而舒心適意,可是也不會是永恆不變的。天人壽命雖長,但亦有盡頭,一旦福報消盡後就會再次進入輪迴,墮入惡趣。

輪迴中的各道皆為如此。佛智慧能消除由無明引起的二元對立心或分別心,所以在達至證悟之前,眾生忘卻自性本空,心中存有強烈的「我」見,因此才會生起主體和客體(亦即你我)之分別。由於對體驗與所體驗之事物有所分別,進而導致如是情況:對於喜好之物生起貪著,反之便生起嗔恨心。由於所有人都想留住美好的事物,因此欲望很自然便會導致貪執心的生起;而另一方面,憎恨則導致嫉妒(因為任誰都見不得敵人好嘛!)愚痴引起我慢心,讓人覺得自己較他人優秀,結果導致孤獨和頑固。我們以為自己等同於那些無常變幻的意念、身體和財富,因此這種認同感矇蔽了自心的光明本性,如果是這樣的話,即便是天人也會墮落。這一種不安的情緒會引起惡口和惡行,最後免不了招致痛苦和不滿。

這一種絕大部分眾生無法控制或無視的混亂,我們稱之為「生命」。六道中無論是哪一道,即使是天道也會有同樣的情況發生。卡盧仁波切以實際的例子告訴我們,六道中到處都遍滿著二元對立的思維,就算是天道也是一樣的。仁波切語氣堅決地又說了一遍:「只有一個境界是圓滿及永恆的:佛的境界」他教導我們,只有一個目標是真的值得我們去追尋!只有到達佛的境界,心的障礙便會完全清除,度眾生的願力和能力才會自然地完全圓熟起來。

卡盧仁波切在講解了一個星期有關有為法的種種顛倒後,開始改變其開示的方向。如今他告訴我們今生能投生人道是何其幸運的事。在遍布法界的云云眾生之中,他們是如何才能擁有如此珍貴的機緣學習佛法,得以求解脫。他說:「能擁有一個學佛修行的人身,是因為避開了八種無暇的障礙,具足了八有暇,十種自他圓滿的善緣的原故。」仁波切一一列舉,繼續說道:「如今就是這種種的善緣為你提供了一個難得的學佛修行的機會。」

在仁波切的指導下,要撕開心中的無明和煩惱的面紗不再是一件遙不可及的事,那是許多人會認為是自己能力所及的事。他所描繪的圓滿境地亦非夢想,而是每個人內心中不變的真正本性。仁波切生動的描述甚至啟發了我們這群人中最懶散的成員。之後,仁波切開始講解「無常」,更讓我們開始認識到修行刻不容緩。萬物無常、稍縱即逝;生命的結局必定是死亡,我們的壽命剎那變滅,他所講的字字句句使到我們對周遭的一切無時無刻都充斥著變幻與衰減的跡象有所警覺。他說:「你們有機會在此生尋找此不死亦不滅的東西——即你們本具之佛性。」「這是一個非常殊生難得的機會,然而生命短促。如果現在不好好把握,以後你必須經歷多少世的痛苦和迷惑,無法自利也無法利他,才能再次遇上這些殊勝的機緣求得解脫呢?」

儘管我和漢娜已開始能接受仁波切的開示,可是我們依舊無法具體地將「痛苦」和現實生活作出聯想。我的心知道痛苦和失望是一種瑕疵,因此會將它們剷除,可是在索納達的這些人卻沒能徹底了悟到仁波切所開示的觀點其實是從佛的層面所表達的一種真理。按照仁波切傳統的表現方式,其實聽起來和我們的生活是毫不相干的。我們認為自己是最快樂的人,認為一切事情都是有意義的,令人感到振奮的,因此能夠引導我們去修行的是因為前方更美好的未來,而非背後的恐懼不斷在鞭策我們的原故。因為即使是目前我們知道的最快樂的境界也只是證悟之心的無限喜樂的其中一個微弱陰暗的影子而已。證悟的無限喜樂才是我們最大的動力。

這一種認知是我後來開始說法的時候才出現。卡盧仁波切開示說:「有為法皆不離三苦:(1.)苦苦---種種苦緣逼而生惱,極度煎熬難捱的狀態。(2.)壞苦---一切樂境或所現變壞時之苦。(3.)行苦---由於內心無明而看不清楚萬物的遷流變動之苦。唯有覺悟者才會完全超越這些漏失。」
