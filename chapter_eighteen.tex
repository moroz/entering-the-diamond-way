\chapter{人生如夢}

加德滿都並沒有多大的改變。冬天的冷霧已散,山谷裡的田野與樹木又是一片萬綠蔥鬱。如今博德納佛塔已是最「流行」的地方,我們大部分的朋友都已搬到那裡附近居住。他們也很快幫我們在那裡找到了一間出租房,房裡還有一個面向大佛塔的小陽台。我們學習波瓦法門後,心裡很想祈求喇嘛傑珠傳授夢瑜伽的法門。在耐心地等待喇嘛傑珠回到加德滿都的同時,我們在充滿活力的加德滿都度渡過了一段美好的時光。

這段期間,我們見識到波瓦法門的效用。佛塔周圍有很多饑餓的野狗,它們會撲向任何能當作是食物的東西,甚至是小孩童的排泄物;我們丟給它們的干酪外皮和不新鮮的麵包尤其受歡迎。我們通常會先念咒和祝願後才將食物布施給這些狗狗,這些動物對此會作如是想我們是不清楚,可是它們似乎有注意到施主的一番心意。有一天,我們很晚才從鎮裡回來,當我們走回家時,路經佛塔外的那一片空地,有一群野狗突然從四面八方出現。我們大概被十二隻野狗所包圍,它們全都朝向我們狂吠,當下我只覺得大事不妙。要是被這些生病的狗狗咬傷,情況可不堪設想。我把漢娜推向塔壁,拔出刀子準備和它們拼了。就在這個時候,我們才發現原來事情不是像我們想的那樣。這些狗狗根本沒有要攻擊我們的意思,反而是在向我們「道謝」,頓時有一股暖流湧上心頭。它們那粗而低沉的嗥吠聲(它們也只能發出這種聲響),聽起來像是在對我們說「謝謝你的食物」。我們心裡非常感動,希望能為這些可憐的動物付出更多,而這一個機會很快便來臨了。

有人(也許是遊客)投訴博德納周圍這些患有狂犬病的野狗。有一天,博德納來了一群警察。這些警察從麻袋裡掏出一大塊一大塊下了毒的肉塊扔給路邊的這些野狗。這些毒肉將會使它們慢慢中毒身亡,整個過程非常痛苦。有些西藏人試圖將狗狗藏在家裡,可是他們也只能保護那幾隻和他們較為親近和相信他們的狗狗。中午時分,警察已經離去。大佛塔的入口處躺著十來隻奄奄一息的狗。當漢娜去取水讓它們解渴的當兒,我用噶瑪巴的舍利給與那些垂死的狗狗加持,好讓它們大部分能安寧、輕鬆地死去。前來收屍的貨車等到我加持完了後才開始辦事,周圍的藏民都感到非常欣慰。

當時有人開玩笑說,以後我們必定會遇到許多聲音粗而低沉,鼻子長長的學生;我們在如此殊勝的地方幫助過這些狗狗,以後他們將會投生轉世為我們的學生。

不久後,喇嘛傑珠回來了。我們很開心能夠再次與喇嘛見面,他對我們這一段期間的點點滴滴感到十分好奇,要我們仔細告訴他所做過的一切。當我向喇嘛請法時,他一直低著頭,似乎是在等待指示似的。後來他看了我們一眼,說道:「好,我會傳授你夢瑜伽(藏:Milam)的修法,這是行者在睡夢中繼續禪修,發展明性的ㄧ種修持法門。」

甚至早在喇嘛答應傳法之前,我們在感知上便已經有所改變。一切事物彷彿變成了一場夢境,接下來和喇嘛在一起的六個星期,一直持續地處於這一種狀態之中。夢瑜伽(禪修)創造一種境界:在任何「知道」的情況下,行者無論是清醒或熟睡,都存有「觀者」。它是一個巨大、透明和受到保護的空間。行者的心將會在這種狀況下學會了知所經歷的一切過程,甚至去掌控它們。這種修持將會增強行者的覺知和觀照力,知道即使在清醒的狀態下,那也只是集體夢境的一部分,所有意識上的體驗其實只是與他人所共享的一組投射而已。當背景和見地越是相似便越能合作無間,共創一個集體的體驗。這一個在觀念上的基本差別最能在種族和文化的摻和之間顯現。當這種光明的覺性扎根時,從我們強烈的心念中而生,誤以為「萬物皆為真實」的煩惱便會自動瓦解。

六個星期過去了。我們就住在喇嘛傑珠樓下一個布滿灰塵的儲藏室裡,不管是清醒或睡著總是處於夢中的狀態,並且一直感受到喇嘛的能量。有一隻鼻子特別長的河鼠,一天裡總會來到我們的房門前好幾次,自在地四處觀望,然後悄悄走到我們留給它的食物旁,叼起,然後離開。除了這隻河鼠之外,我們便沒有其他外來的訪客。這是一個殊勝圓滿,帶來了甚深覺受的閉關。自此,沒有任何事物是「狹隘」或「沉悶」的了。

完成修法後,喇嘛傑珠又再次恢復他原來的樣子。他說,現在是時候去見噶瑪巴。我們為了能更快抵達,再加上機票只是九美元,所以這次搭飛機前往尼泊爾最東部的比拉特納加爾(Biratnagar)。傑珠仁波切親自載我們到機場,兩天後我們已經身在索納答,那裡的人已確切知道噶瑪巴會回到錫金的日期。仁波切再次很準確。噶瑪巴隔天就會抵達錫金。

索納達在一片愁雲慘霧中。我很訝異看見他們因為喇嘛出外遠行而痛苦。卡盧仁波切已滯留加拿大超過一年,儘管他們請他的弟弟到寺院來坐鎮,寺院裡卻始終沒有任何動靜。看他們對上師如此依賴,我雖覺得感動,但也未免會認為這樣的行為很幼稚,所以我們決定要以不同的方式來做事。佛教的目標就是要讓人變得獨立,發掘出個人的力量,像索納達這樣的情況著實不該。當我們在西方推廣佛法事業時,啟發人們的潛能成為了我們的首要之事。

我們的時間拿捏得正好,在提斯塔橋等了一會兒便看見噶瑪巴的車輛。噶瑪巴對我們微笑,然後將手指放在我的軍裝夾克上。這時我才想起來在夾克內的暗袋裡裝有一封我們要交給噶瑪巴的信件。噶瑪巴給我們加持,然後我們開開心心地跳上他的其中一輛貨車。在一片藍天之下,我們在車內打坐禪修,跟隨噶瑪巴一起進入錫金。

接下來在隆德的日子非常殊勝,我們時時刻刻都保持警覺。我們被視為是噶瑪巴的「初級法侶」,和噶瑪巴的關係亦邁進了一大步。他曾多次嘗試告訴或讓我們明白一些事,可是我們一直到自己開始對他人的發展負起責任時,方能深刻地體悟和了解他所說的話。每當他在給予他人有關管理禪修和閉關中心方面的建議,又或是當他正在講解必須完成的修法,以及其方式時,他都會安排我們在場旁聽。他不時會詢問我們的想法,在西方國家會如何處理一些事等等,他總會耐心地領聽,可是卻從不說「好」或「不好」。我們可以確定在接下來靜坐禪修時或夢境中,必會向我們展示一個明確的解答。

一天,一切似乎準備就緒。就在印度方面開始施壓之前,噶瑪巴把我們叫到面前。他送了我們與友人金‧文斯一份非常殊勝美好的禮物,然後對我們說,作為第一批西方弟子,如今我們擁有他的加持去開創禪修中心。我在丹麥展開第一次的弘法事業後,便開始在歐洲以及其他地方積極弘法。噶瑪巴允諾會給與我們指引,讓他的加持力與教誨在我們身上儘速彰顯。在震驚之餘,一股炙熱的能量湧上心頭,這一股能量即使是到了今天仍不斷在持續燃燒著。噶瑪巴將我們送回歐洲,為此生弘法利生的事業正式掀開序幕。

就這麼決定了!我們已迫不及待地想要展開這一項任務,沒有多餘的空間躊躇不決。如此重大的責任固然會讓人的內心激動不已,然而我亦清楚它絕容不得我們半途而廢。我們要將這一種能讓眾生解脫的教法推廣至西方國家,一定不能令上師乃至整個噶瑪噶舉傳承蒙羞,因此必須打起十二分精神。在這三個星期裡,一直到我們在孟買擠上了一票難求的班機之際,我們不斷注意到有許多殊勝的事發生,我們驚嘆之際,心裡亦深深感恩。這些殊勝的事都是預示我們未來定會取得成功的好兆頭。當我們在菩提迦耶匆匆離開佛塔內巨大的金身佛像之前,我們看見佛像對我們微笑。在那輛載著我們的行李的小摩托車絕塵而去之前,我們衝出佛塔奮力追逐,心裡只有一個想法,覺得「佛陀也頗幽默的嘛!」不然我們還能說些甚麼,又能作如是想呢?

看見佛陀微笑的那一眼激起有力的淨化作用,在前往拜拉库(Bylakuppe)途中,甚至在那裡逗留的期間,我竟然生病了,而且痛苦不堪。我的整個頭顱和喉嚨就像不斷裂開一樣。我們乘坐加爾各答---馬德拉斯---邁索爾列車抵達拜拉库(出乎意料的所有乘客都有位子坐!)我們恰好趕上一座新噶舉巴寺院第一棟樓為地板抹上水泥的儀式。這一座寺院就建在山丘上,可以俯瞰整個難民營。我雖然因為發燒而整個人迷迷糊糊,我們認為在西方國家展開任務之前必須先在東方國家做一點事,而且我知道身體上所經歷的疼痛不會是阻擾我們的障礙,這一點是很重要的。前往孟買的火車非常緩慢,幸好途中我們在各地發了電報,才保住了之前預訂的機位。

就在十月份某一個寒冷、浮雲儘斂,皓月當空的傍晚,我們抵達美麗的哥本哈根。
