\chapter{被遺忘的山谷}

當噶瑪巴離開加德滿都後,喇嘛傑珠又再次繼續其奔忙的旅程。由於當時在加德滿都並沒有其他特別的事,因此我們便接受了友人泰瑞‧貝克的邀約,一起去探索某個位於尼泊爾中北部,靠近赤蘇里(Trisuli)的偏僻山谷。泰瑞是和平工作團隊(Peace
Corps)的志工,最近該組織節育團隊的直升機恰好發現了這一個地區。他的某個朋友是第一個進入此地區的白人,如今我們也想去看看。我們是在一九六九年第一次來到尼泊爾時才認識泰瑞。我們是在前往尼泊爾西部的博克拉(Pokhara)的班機上相遇,當時他還教會了我們不少事情。泰瑞可說是當時唯一一個在尼泊爾境內旅行多年的西方人,他大概比誰都還要熟悉這一個國家。他走遍尼泊爾全國,目的是為了要錄下當地的傳統音樂;他已從東至西橫跨了尼泊爾國境三次。在每一次的行程中,他總是帶著許多行李隨行。由於他花錢從不吝嗇,因此挑夫們從未找他麻煩。泰瑞熱愛爬山,他也是第一個征服了數座位於尼泊爾西藏邊境山峰的人。他的攝影技術也很好,其作品可結集成幾本書出版,只是如今他身在尼泊爾而未能好好整理。當年攀山者都必須向當地政府預繳六百元的風險補償津貼,泰瑞知道這些保證金最終都會無疾而終,所以從未乖乖繳付。

尼泊爾約有十三個不同的部落,擁有各自的方言。在這些方言當中有的源於梵語系,其餘的則屬藏語系。泰瑞會說一些當地主要的梵語系方言,他也是一個擅長組織的人。我們這一次將會和他及其同伴里察(他們都是訓練有素的攀山者)一起攀登幾座「所有權不明」的山峰和路徑。尼泊爾宣稱這幾座山脈皆隸屬尼泊爾國境,中國一方又堅持它們隸屬西藏境地。由於雙方爭持不下,這些地區便成了「閒人免進」的禁地。

我們所乘搭的交通工具來自拉納斯(Ranas)。拉納斯是過去管理尼泊爾的一個非常富有的印度種姓。一千多年前,當穆斯林勢力入侵北印度時,這些拉納斯帶著寶藏逃到尼泊爾,而且很快便成為了當地的統治階級。他們所建造的巨大宮殿,迄今仍屹立在加德滿都。在兩次世界大戰間歇期間,汽車成為了身分地位的象徵。當時尼泊爾甚至尚未建有道路,亦沒有汽油供應,可是當地許多富有的拉納斯仍砸下大筆金錢購買汽車。這些汽車都是經由船運抵達印度的加爾各答,經過拆卸分解後,再由苦力們揹著它們翻山越嶺,帶到尼泊爾。今天這些汽車可在各種莊園的穀倉里找到,而且都被保存得很好,車身被擦拭得光亮如新,絲毫不見生鏽的痕跡。儘管這些汽車所走過的哩數不超過五英哩,可是車主仍不惜以賤價將它們脫手。尼泊爾的內陸交通已經由堅不可摧的福斯小巴(我們有時是二十三個大人連同行李一起擠),或是使用柴油引擎的大賓士車(多用於行駛在顛簸難行的道路上)所取代,日本進口的汽車則一般是在城鎮裡行駛。在這一種平均一夸脫的汽油抵上當地工人一天工資的國家,較老舊的車款並沒有甚麼用處。泰瑞一些較有生意頭腦的朋友買進車身巨大的古董勞斯萊斯,然後再轉手運到美國售賣,從中賺了一大筆利潤。他和朋友也剛買了一台1928年份的全新福特汽車Model
A,也就是旅途中我們所開的那一部車。

一旦我們習慣了手油門後,這一台福特汽車開起來就沒有多大問題。老實說,我挺享受開這一台車的。這一台汽車的扭矩大,開起來感覺很棒。我們沿著一路未經鋪砌的小路來到加德滿都以西、靠近赤蘇里的一個小村莊。這裡已是路的盡頭。從這一段路開始,我們必須自行揹著行囊上山。漢娜和我已經習慣較節約的生活方式,如此一來我們才擁有更充裕的金錢去學習佛法。我們大概只帶一些麵包和豆類就準備上山去,可是我們的朋友就較為講究,甚至可說是做足萬全的準備。他們買了各種各樣的東西,現在我們必須一起帶著上路。除了和我們一樣帶上衣服、靴子、煤油和睡袋,他們還帶了各式的「奢侈品」,其中包括沉重的罐裝花生醬、特製麵包和許多巧克力。這些全都是在American
PX
零售商店裡可找到的東西。我們這幾個大男人平均每個人要揹超過60磅的行李,而漢娜的背囊則差不多是我們的一半重量。

第一段路程將我們引領至一條水勢奔騰的河流,我們在途中經過一座由俄羅斯人所建造的水力發電廠。由於橋塌了,所以當地人弄了一個渡河的牽引裝置。人們在那一條看起來隨時會斷掉的麻繩上掛了一個吊籃。人坐在吊籃上,底下就是白花花的一片流水美景,溜索一拉便可渡河,到達對岸。幸好我們的背包都平安渡河,沒有掉落翻騰的流水中。

真正的攀登現在才正要開始。我們終於明白為何只有少數人曾到訪過此山谷。眼前是一座我們必須攀越的岩壁,我們感到非常振奮。後來我們才聽說,當地的婦女來到此路段後都不願意搬運任何東西,因此便由男人們分擔物品的重量,分好幾趟將東西搬到山谷裡去。無知的我們卻傻乎乎要揹著如此重的負荷就逕自登山。從那裡看往對岸山脊的景色更為壯麗,可是若不小心失足掉落,便會像自由落體般直接墜入江水中。

行囊揹在背上比頭還高,這樣除了方便雙腳的移動之外,也會讓我們看起來比途中可能會碰到的熊還大隻,具有嚇唬的作用。只是我還是犯下了許多新手的錯誤,將背包的腰帶束得太緊。由於背包太高,偶而在攀爬時會頂到頭上突出的岩石部分而使到整個過程變得非常艱鉅。然而這種艱鉅的過程卻給了我們數次機會,將寶貴人身的教義內化。我不斷推著無畏的漢娜向上攀爬,自己則最後抵達頂端。我們那兩位專業的登山者泰瑞和里察說,以後他們絕不會在不用繩索的情況下攀爬那座岩壁。

傍晚將至,我們來到一個U型山谷,只見遠處堅固扎實的小木屋林立。當地的居民讓我們睡在大大的屋簷下。由於他們都是印度教徒,因此不允許我們進入屋內留宿。我們無意中發現他們手上擁有最上等的大麻,其品質之好就連在尼泊爾時也未曾遇見過,所以我們決定幫他們和加德滿都那邊的某個朋友拉線。他們顯然並不知道自己手上的大麻品質有多好。看在他們那麼友善的份上,再加上看見他們在物資貧乏的環境裡生活,因此我們認為應該讓他們多賺點錢。隔天早晨,我們往山谷的左側前進,沿著山腰上的狹窄小路,爬了一個又一個小時。沿路的竹叢長得和我們一般高,一不小心會很容易迷路。一路上只見竹葉都呈現褐黃色,據說這些竹子每十一年(太陽黑子較多之時)就會枯萎。就連高大的槐樹亦然。乾枯的槐樹,只需要一把斧頭就能將樹幹從頂至底一劈為二。由於樹身軟得可憐,(也正因為這一個原因),在喜馬拉雅一帶,槐樹常被用作屋頂上的板瓦。當地居民為了保持現成的木板供應,於是將樹幹底部的一圈樹皮移除,使樹身變乾。只是如今處處都是腐蝕枯萎的槐樹,似乎是供多於求而未被妥善利用。

尼泊爾由於人口過剩而正在面對大量泥土侵蝕的問題,流入印度的河水也是一年比一年混濁。對於生活在山頭裡已經正在挨餓的人們來說,這確實不是一個好兆頭。

接近傍晚時分,我們路經一塊突出的大石頭,狹窄的山谷在此處微微向左蜿蜒。只見懸崖邊的狹小的梯地上建有六間草屋,若不小心掉落山谷,雖不至於像自由落體般直墜河流,不過地勢也是相當險峻。按照我們所觀察,山上的居民不可能看得見我們來時的那一條路,可是他們卻已經為我們準備好了餐點迎接我們到來。他們大部分人從未離開過這片山谷,身心早已與這一片土地合而為一,他們就是知道有人自遠方來。

山裡的男人們提出以一天六盧比(差不多是半美元)的價格替我們扛行李。這一個價錢就當時而言可說是非常昂貴。我們想要表現出慷慨大方,於是還是答應了。我們在加德滿都存了一些一盧比的新鈔,山區裡的居民認為這些新鈔比就舊鈔更有價值。這些新鈔讓我們省下了不少有失雙方身份的口舌之戰,也算是給他們的好東西。這些山區裡的居民,誠如百分之八十的尼泊爾人一樣,都是印度教徒,一戶人口不斷生育。這次我們也是睡在屋外簷下。其實這樣也好,外頭除了沒有蟲子之餘,也能更好的了解他們的生活方式。看著他們如何在如此寂靜的山谷裡頭把自己組織起來也倒是挺有趣的。

「阿嬤」無疑是這兒的老大。她的意志如鋼鐵般堅韌,言詞尖刻鋒利,把這一片山谷管理得妥妥當當。在這種嚴苛的環境下要活到這一把年紀可不容易,真是少點兒能耐也不行。她雖然年事已高,卻無損她叱喝山谷的權力。我拿起她的水煙筒吸了一口,就這麼一丁點的好奇心把我直嗆得大咳起來,差點兒沒吐。她吸的可是未經處理過的煙葉,是純毒藥也!

挑夫們的體格雖然瘦小,可是卻非常能吃苦耐勞。他們的步伐緩慢、專業,不容易疲累。他們總是赤腳而行,腳板近乎平直,像駱駝般柔軟,因此能將重量平均分布在凹凸不平的地面上。他們偶爾會發出深沉的嘆氣聲(這種嘆氣聲是喜馬拉雅一帶人們的背上負荷沉重的象徵),為整個畫面提供了一個和諧的背景音樂。一個訓練有素的苦力能承擔很大的重量,即使是一個外表看來瘦不拉幾的苦力也能夠揹起你我所無法想像的重量。這其實是和他們呼吸運氣的方式有著很密切的關係,反而與個人的肌肉發展無關。他們會根據山坡的傾斜度、高度和貨物的重量調整呼吸,再配合腳步前進,就像是在很自在地換檔一樣。不過他們需要有人先幫忙將貨物搬到背上,再由額前的帶子捆紮支撐後便能一氣呵成。他們有的人甚至能夠揹起一個很重又患了高山症的西方人,不過一天僅是能走數英哩的路程,而且最好是下山的路線。在一些較多人經過的路線,陡峭的高地下會建有一些小木屋。挑夫或苦力們會到那裡喝上幾口烈酒,在抵達目的地之前先暫時舒緩疲憊不堪的身軀上那隱約的痛楚。他們有許多只不過才三十來歲,看起來卻像六十歲般蒼老。

挑夫們負責揹那三個較重的行囊,為了更快將東西安排妥當,加上我也不習慣支使別人幹活自己卻不動手,於是便揹起了漢娜較輕的那個背包。這些挑夫教會我們的不僅是搬運的技術,他們所做的事,我們曾經也只是從一些書本上看過而已。午休時間時,他們會去收集稻草,接著便從身上穿著那種雪巴人開放式的外套拔出一些鬆散的羊毛線。他們身上就只是穿著這種開放式的外套和圍著一塊纏腰布而已。他們從口袋裡掏出一小片鋼鐵,鋼鐵的一面被磨得光亮。只見他們利用鐵片和隨地撿起的半透明石頭進行磨擦,產生火花後用來點燃石頭上的羊毛線,再用它來點燃稻草和一些乾的樹枝生火來加熱玉米稀飯。這些玉米稀飯是他們這一輩子最主要的糧食。我們幾個人站在他們的小屋旁,某隻只剩三條腿,看似缺鹽的牛走過來舔了舔我們。看著眼前正在幹活的人們,我不斷思度:,他們究竟是如何在這種環境下活過來的?

近傍晚時分,我們來到一間用樹枝搭建的廢棄小屋。從那裡可以看見整片山谷的壯麗景色,因此我們決定在此地留宿。一開始時,挑夫們都不敢靠近這個地方。他們說,以前曾經有一個法力高強的法師住在那裡,因此很危險。我們進入小屋後沒有發生任何意外事件,加上他們記得我和漢娜老是在持誦咒語,因此認定我和漢娜的法力更高,於是在請求我們給予他們保護後也進到屋裡來了。他們沒有攜帶任何墊子或毯子,只是隨意睡在地上的一些竹簡上。漫漫長夜,他們不斷翻動身子向生火的地方取暖,就這樣翻來覆去熬過了寒冷的夜晚。這樣的生活很不容易,他們卻沒有絲毫怨言。

翌日早晨,我們路經某處,那裡剛下過雪。儘管挑夫們走到哪兒都幾乎是赤腳而行,這一次他們似乎決定要自我提升。我們剛換上軍靴方便在雪地上行走,他們說想試穿我們的網球鞋。對他們來說這可是一次重大的體驗。我們把鞋借給他們試穿看看,另一方面在心裡卻暗自希望他們穿得不舒服,從此再也不想穿鞋。然而這只不過是我們一廂情願的想法。穿鞋的想法似乎已經平穩、篤定的入侵他們的心靈,對他們原來的價值觀和自信心造成了極大的威脅。

隔天中午,我們抵達一片大樹參天的美麗高原,這裡的植被分布亦已開始改變。我們在高原上發現了一群綠色的小植物。這些綠色的小植物看起來有一點像蒲公英,可是卻有著比蒲公英更厚及矮生的葉子。挑夫們高興的不得了,不願意繼續上路。他們只是一心想著要採集這些綠色的植物,然後把它們帶回家。這些稱為蔓陀羅(Datura,又稱Jimson
weed,屬茄科毒草)的植物的根部擁有強勁的致幻作用。歐洲中世紀時期,巫婆將蔓陀羅涂在人類身體感受性強的部位上以誘導各種不同的迷幻狀態。在我們那個年代也缺少一些較為強勁的興奮劑,於是這些蔓陀羅的根部有時在我們的朋友圈中會相當受歡迎。他們會趁入夜之時摸黑到植物公園去「採集」蔓陀羅,然後再踏上逍遙的迷幻之旅。雖然不需花一毛錢,可是它對身體所造成的副作用卻相對更大。正如溴化物(bromides)和東莨菪碱(scopolamine)一樣,長期下來會對脊椎造成傷害。挑夫說焚燒這些植物有軀邪惡鬼的作用,現在他們想要趕緊回去淨化他們的家。我們猜想他們也許是想要嘗一嘗這種強勁的興奮劑爽快一下,於是便給了他們每人十二盧比新鈔,要他們速去速返。如果他們的速度快,天黑之前必定能夠趕回來,否則入夜之後,即使他們再熟悉整個區域,摸黑走起來也會很危險。他們答應會在之前的地方把網球鞋留下。下山時,我們會需要換上這些鞋子。

我們又揹起所有行囊繼續前進。由於我們現在處於海拔四千米以上的山上,這些行囊揹起來似乎比之前更重。我們很快便學會挑夫們的方式走起路來。以腳跟著地,將腳步縮小,即使是陡峭的上坡路,平穩前進才是道。我們不斷地走,只是在人有三急要方便時才會停下來。這一路上一定是高度的變化引起生理上的反應,所以老是覺得尿急。天黑之前,我們登上五百米更高處,來到一個景色壯麗的高原。高原右方是藍塘雪峰群(Langtang
Himal)和戈桑坤達湖(Gosinkund);左方則是西藏群山。藏民在高原上建造了一些佛塔和一個可以留宿的小石屋。隔天,泰瑞和里察動身去勘察地形,想要繪製地圖,幸好這次他們帶上完備的器具一起上路。而我和漢娜則希望爭取時間靜坐禪修。我們爬上山脊,從那裡遠眺可見西藏,近看則是我們卸下行囊的小石屋。人來到東方國家總是很快便學會無論身在何處都要注意看管個人的隨身物品。

高原上那些用粗石建造的佛塔就如其他建造有序、光滑明亮的佛塔一樣,皆表徵五種圓覺的智慧。我選擇了一個靠近石塔的地方坐下,打開那本關於「拙火」禪修法的書籍。「拙火」禪修法非常有名,而且效果顯著。雖然此書籍和我在監獄裡時所翻閱的版本不同,可是皆同屬噶舉實修傳承的修法。我才剛開始觀想、做深呼吸,內心裡一種潛在的能量便驟然衝向身體的中心。這顯然是過去世的福德所感。當時我尚未領受相關修法的指導,或具足其他修持條件,對於體格較弱者而言很可能會產生一種類似帕金森症狀般的效果。由於內在的種種阻塞已被破除,一種無法言喻的光明、能量和喜悅將我拉開,沒有留下任何疑問和疑慮的餘地。

當年密勒日巴尊者就是在此地修持拙火定而證得了圓滿的證量。今天尊者的能量遍照著整個噶舉傳承。我們所感受的便是尊者超越時空的證悟力量。離開此地不遠之處,山谷的另一方,就在西藏境內有一座吉榮寺。吉榮寺的領導就是傑珠仁波切。仁波切曾經在附近的某個洞穴裡和杜巴喇嘛(Dukpa
Lama)一起修行。當時他每日僅靠三勺水過活,這樣的生活一直維持了數年。這是我人生有史以來第一次感覺到有一股能量從雙手湧出,於是緊緊地抓住漢娜的手,想要將能量傳遞給她。

傍晚時分,我們回到小石屋。泰瑞和里察仍不見蹤影。天空開始飄起雪花來了,翌日清晨醒來時,四周圍已經是白茫茫的一片,積雪的厚度大約有半米高。泰瑞和里察雖然無法及時趕回來,不過我相信他們一定會平安無事。他們已帶上最好的羽絨裝備隨行,而且遍滿整個區域的加持能量必定也會保佑他們無恙。當夜幕再一次降臨,他們終於拖著沉重的步履回來了。

屋內溫暖的爐火讓他們感到非常高興,甫一進門便迫不及待拿起食物要填飽肚子。那套羽絨裝備和睡墊讓他們在雪地上舒舒服服地睡了一晚,他們對此讚不絕口。不過這兩天在外頭曾碰到過幾次緊急的狀況,有一次里察更因為雪盲的關係還差點兒踩了個空從岩石上掉下去。

隔天早晨,我們準備打道回府。新雪幾乎讓我們寸步難行,後來索性仰天躺著順坡滑下,這回可是又刺激又省力,一會兒功夫便向前邁進了一大段距離。我們急著想把腳上濕漉漉的靴子脫掉,然後換上先前借給挑夫們穿的網球鞋。可是當我們來到先前約定好的地點時卻不見鞋子的蹤影。雖然心裡有點兒不太高興,我們還是前往法師住的小屋,心想他們可能會把鞋子留在那裡,結果還是撲了個空。最後,我們又再次回到阿嬤的村莊。

阿嬤很沒誠意。她說那些把鞋子穿走的挑夫到山谷的另一頭去了,不確定甚麼時候才會回來。如今這真的把我們給惹惱了。這分明就是他們不想把鞋子還回來才使的詭計嘛!結果我們鐵了心,皮笑肉不笑地告訴他們說我們會邊吃邊等,ㄧ個小時後若鞋子還未被送回來,那麼我們就會把他們家的屋頂給燒了。ㄧ個小時後,當我們掏出火柴...(我們當然只是裝腔作勢嚇唬他們而已,不是真的要把這些窮困人家的財產給燒掉)就在千鈞一髮之際,當中某個年紀最大又最英勇的挑夫挺身而出,很神奇地「找到」了我們的鞋子,然後乖乖把它們歸還。他們現在才開始懂得尊敬我們。他們想要跟隨我們一起前往赤蘇里,但先前因為向我們要求太高的工資,覺得欺騙了我們,再加上網球鞋的鬧劇,於是現在希望能將「大事化小,小事化無」,所以願意免費替我們搬運行李。這次我們只是讓他們幫忙搬ㄧ兩個行囊。由於我們已學會正確的運氣方式,所以一路自己揹著行囊也走得不亦樂乎,再加上途中已消耗了不少食物的關係,這些行囊比一開始時已輕了不少。

我們一回到當初停泊福特汽車的地點後便付錢給那一位負責看守的老翁。連同開頂的車尾箱上的三名苦力,我們一行人浩浩蕩蕩出發,沿著曲折蜿蜒的道路駛往赤蘇里。這三名苦力是第一次進城。對他們來說,開車進城是一件非常拉風的事情。他們堅持要一路鳴響車笛,好讓所有人都看得見他們。我們尷尬地投其所好,一路鳴響著車笛駛進城裡,替他們為其族人的歷史寫下了光輝的一頁。

