\chapter{印度南部的西藏難民營}

我們選擇廉價的阿拉伯航空,經由敘利亞(Syria)飛往孟買(Bombay);當時人們對自己所支持的文化並不會想太多。由於在過去幾裡天忙著慶祝各種節日而不怎麼睡覺,漢娜和我在飛機上基本上就是不斷在補眠,間中醒過來幾次吃了一點東西果腹,看了看高空下那一片廣袤的沙漠幾眼,轉眼間飛機便已在印度降落。孟買是一個很難讓人振奮起來的地方。如果說德里給我的印像是「憤怒」,加爾各答給我的印象是「混亂」,那麼孟買便是以「傲慢」稱王。我們在抵達孟買後的當天傍晚,搭火車前往邁索爾(Mysore),再從邁索爾轉搭巴士前往西藏難民營。我們在擁擠的車廂內顛簸了一夜,隔天又換了幾次火車,所以根本沒辦法好好休息。再加上這一趟行程,我們帶了不少伴手禮隨行。我們千辛萬苦才順利通過海關的檢查,因此實在不希望因為一時的疏忽而讓人有機會順手牽羊。當第二天早晨又要換火車時,我們看了看地圖才發現自己壓根還未離開孟買的範圍。這時我們才知道原來在印度西部,搭巴士和汽船會比坐火車更有效率。沿途上不斷改變的地勢與風景讓我們大開眼界,相較於印度北部或東部,這裡能看見更多不同的族群。火車每到一站就會有不同相貌特徵,給人感覺迴然不同的人群登上火車,喧鬧之間「盧比」和「派薩」(印度的貨幣單位,約百分之一盧比)二字不斷響於耳邊,姑且無論各族群的人們說的是甚麼語言,字字句句之間,彷彿這兩個字才是不變的真言。

我們在火車上顛簸了一天一夜後,終於在隔天中午抵達邁索爾,一個位於芒加羅與邦加羅爾之間的小城鎮。我們坐上馬車,負責拉車的馬兒非常健壯,這種情景除了在阿富汗和佛教國家之外,在其它亞洲地區並不多見。正如所料,伊朗的情況最為糟糕,那些負責拉車的馬兒都瘦得只剩皮包骨,在拉車前往聖城瑪什哈德(Mashad)途中不斷被馬夫猛烈地鞭打。我們帶上一些橘子果腹,及時趕到巴士總站,擠上了一輛已經非常擁擠的巴士。邁索爾的感覺很不好。沿途有好幾個地方都在播放一些反嬉皮士的影片,而這些印度人總是一廂情願地對眼前所見深信不疑。這些影片一定是把嬉皮士描繪得非常不堪。他們顯然是對我們很反感,有時我甚至必須奮力去守住我們的座位。巴士穿越過無數的小巷與胡同,一排排的樹開滿了花,幾個小時後,那些新舊不一的西藏五色經幡終於第一次出現在眼前的右方。我深受眼前的畫面所打動,無論如何也止不住奪眶而出的淚水。我試圖以咳嗽、打噴嚏,甚至是別過頭去佯裝欣賞窗外的風景來掩飾激動的自己其實是在哭泣。看見佛法的智慧與修持出現在像這樣的一個地方,我真是感到太開心了!我頓時生起了一個非常強烈的願心,誓要力保佛法在此地繼續發展下去。

印度人不要西方人出現在這些西藏難民營裡和他們有所掛勾,所以我們沒在難民營的大門口下車,反而多坐了幾公里路到另一個地方去,再從那兒按照幾個西藏人的指示,抄另一條羊腸小徑,穿越田野和樹叢,直抵我們的目的地——第四營。這一條小路也能讓我們避開駐守難民營大門口印度官員的盤問。白人對印度人來說是金錢與地位的象徵,儘管他們無法對我們做些甚麼,不過卻非常嫉妒這些西藏難民擁有如此完善、具體的組織。所以任何能讓他們用來抑制和控制這些西藏難民的客人的手段都將無盡歡迎。

如今正值二月初旬,藏暦新年的各種慶典尚未落幕。西藏人居住的庫格(Koorg)一區座落於海拔一千米處,這裡的天氣宜人,白天不會太熱,夜晚也不至於過於寒冷。沿途一片片的玉米田種滿了玉米,西藏人的井然有序與道地的印度人形成強烈的對比。這些精進的西藏人只要擁有一個平和的生活,即使離鄉背井來到印度也一樣能夠取得成功。

第四營並不難找。由樹枝搭建而成,泥土地面的二號小屋亦然。阿陽仁波切與他的三名兄弟一起住在小屋的左方。西藏人一如既往地消息靈通,未幾便來了一群彬彬有禮的小童帶我們到八號小屋去。小屋擁有堅實的牆壁,屋裡也做好了接待我們的準備。他們為我們準備了許多對歐洲人來說還算是可消化的小吃,我們邊吃邊聊,與他們有不錯的接觸。我們想向喇嘛請法,學習波瓦法門,不過喇嘛已出外遠行幾個星期,不過我們已經給他傳了電報,再過幾天喇嘛就會回來。直到喇嘛回來之前,我們將會和他的三位兄弟在一起。他們再過幾個小時就會抵達。這段期間,我們和招待我們的那戶人家相處和睦,也正好有機會讓我們練習已生疏不少的藏文。我真的很欣賞他們家的供養壇。供養壇佔了整面牆。雖然和我們的傳承傳統不同,(招待我們的主人家來自藏傳佛教的薩迦教派),不過我們仍很開心能再次被佛像和唐卡所包圍。

我們花了一整天的時間組裝帶來的禮物,教導招待我們的那戶人家如何使用膠帶和打字機。這時喇嘛回來了。喇嘛主修波瓦遷識法門。我們即將成為第一個學習此修法的西方人。那一片開闊的田野上只有一棵樹。隔天,我們就在樹下靜坐禪修,怎知一群好奇的人不斷湧上前來盯著我們看,讓我差點兒沒暈倒。後來我們到鄰近的村莊去,這一個村莊和佛陀選擇進入涅槃的地方擁有同樣的名字:「幸福的小鎮」拘尸那羅(Kushinagar)。我們在一個潮濕,尚未建成的洋灰屋裡找到了在印度難能可貴的靜謐,好讓我們專心靜修。

在孟各(Mundgod)的難民營距離嬉皮士的聖地果阿(Goa)只不過是幾個小時的車程。或許這正是印度政府嚴禁西方人出入的原因。我們搭夜車在抵達胡布利(Hubli)之前的一個小鎮下車,然後在清晨乘著一輛擠滿了工人的貨車直抵難民營。我們必須趕在那些印度官員醒來之前到達。我們躡手躡腳在幾個熟睡的官員身旁走過,沒被發現截查遏止。這個地方不小,從那個靠近合作商店的高地上遠眺可清楚看見整個地區的十二座村莊。這些村莊周圍是一片一片的田園,分布在一片廣袤的丘地上。這一片土地原是一座森林,直到最近才被剷平。

不知何故,這一個地方總讓我們覺得不凡。這不僅是因為那一座新蓋的寺院或那些新舊不一、隨風飄揚的五色經幡而已。我們當然也很開心能看見這些寺院與經幡,只是這一個地區就是有一種無法言喻的獨特。後來,我們才漸漸發現到這一個營地獨有的特質:原來我們來到了一個「完整」的世界。在孟各期間,我們仿若是在西藏生活,這其中有某種無法言喻的完整性與平衡性,我們只曾在不丹偏遠的山谷居民身上感受過這一種特質。當然,這裡無論是氣候或周遭的環境都和真正的西藏截然不同,然而古老的西藏那種不曾間斷的能量場卻被帶到此地,繼續發揮著其功用。我們將會在這一個難民營中生活六個星期,與當地居民融為一體,以他們的方式過生活。這些西藏難民融入文明的物質世界只不過是五十年的時間。他們那質樸、簡單的生活條件,人與人之間所展現的那種本能的合作精神,期間各種的體驗在我們心中留下了很深刻的印象,久久無法忘懷。我們終於開始明白在這幾千年來究竟是甚麼讓脆弱的人類在種種艱苦與磨難中生存下來。

在這一段期間,某個灌頂法也向我們展示了加持力的意義與力量。有一天,喇嘛正在傳授無量光佛的灌頂,此時有人將一個患了肺結核病的病人抬進來。我敢說他的體重一定不超過六十磅。我們讓出前排鋪有地毯的座位,讓病人能夠躺在柔軟的地毯上,自己則退到後方與眾人一起。他能活著真是奇蹟;他整個人已經瘦得只剩皮包骨,還不時朝杯子裡咳出血和痰。我們幫忙用紙張蓋著杯口,向眾人解釋那些漫天飛舞的蒼蠅會到處散播病菌,因此不得不慎。

當喇嘛手持著法器給予他加持時,我們從他的眼中看見了變化。他的雙眸變得較為明亮,臉上痛苦的神色亦漸漸舒緩開來。他顯然已經進入大樂漸次增長的境界中,感受萬法萬物最純粹的本質。

就在灌頂法會的兩個小時後,他已能以意生身自由進入任何想去的地方。他離開了這已被摧毀的臭皮囊,追尋大自在的境界。如今,他已經持有通往極樂世界的「門票」。其實醫生早就診斷出他所剩時日不多,他卻奇蹟式地撐了一年。

孟各人的政治性沒有像大部分中、西部的西藏人般強烈。他們倒是擁有不少文化上的「遊戲」。不過,他們很快便發現和我們玩那些遊戲是毫無意義的。有一些體格矮小、身材豐滿的婦女發現漢娜其實是個女的,那麼我很自然便不是出家人,因此在背後竊竊私語說了一些閒話。我根本就不在意,不過有時卻也會令那些玩弄我的人非常難堪。他們當中有許很人的態度自在、彬彬有禮,不難看出他們內心的圓熟。這應該是和他們在日常上的修持有關:每當日出日落,難民營裡處處充斥著誦經聲以及清脆悅耳的鈴聲和手鼓聲響。這些藏民的屋子大多是由洋灰或樹枝所搭建而成,小小的屋子總是過於擁擠。撇開屋內悶熱不通風的情況、肺結核病與貧困的生活不說,他們當中有許多人可說是典型的在家修行者與瑜伽行者。他們將生活上的體驗和金剛乘佛教的見地與方法相結合,他們的行為模式日後也給西方國家帶來了深刻的啟發。

這裡的難民大多來自西藏的西部和北部地區,靠近拉達克(Ladak)和蒙古(Mongolia)。他們是遠方草原上和平的遊牧民族,與驃悍的康巴戰士不同。一九五九年,康巴戰士帶領約八萬五千名同鄉和大部分喇嘛逃離西藏,而這群西、北方的遊牧民族則是直到入侵事件爆發八年後才成為紅軍的目標。文化大革命奪取了他們的自由,因此這些人民被迫摧毀靜修之地,約有八千人在身無分文的情況下經由拉達克逃到印度。在中國政府的壓力下,印度人將大部分的難民安置在運畜的小車上,然後將他們載到南方的這座森林中,途中約有百分之三十的難民因為無法適應驟變的天氣和疾病的侵害而丟了性命。葬火一直持續燃燒了幾個月。那些倖免存活者在這片土地上發展了農業方面的技術,如今更指導身為東地主的印度人。多虧這些來自北方的人,如今將近所有當地的原居民都擁有不錯的工作。

他們剛開始來到此地定居時,曾有一群野象闖入他們居住的地方踩死了好幾個人。這些野象顯然比較喜歡這一座森林原來的模樣。每經過一些地方,就會看見有人指說:「塔希就是在這裡被踩死!」或「多瑪就是在那裡被追!」有人請求噶瑪巴在此地區設下護佑的能量,那麼這些動物便不會再來侵犯。我們恰好聽說在拜拉庫(Bylakuppe)也發生過類似的事。那裡的營地也是不斷受到野象的侵害,直到達賴喇嘛蒞訪該地區後才恢復太平。

難民營就像是他們自己的一個小小世界,不過這一個純樸的世界似乎不會保持永恆不變。這裡已出現了兩個很危險的跡象:傳統文化已逐漸開始衰退。大部分的年輕人已經離開難民營,來到城鎮裡謀生。此外,電源供應亦開始入駐當地。只見桅杆已經建起,而且也接上了第一條電線;這等同是為無意義的電台噪聲開路。我們曾以輕描淡寫的方式提醒他們要注意提防這種改變,感覺像是對他們擺出屈遵俯就的樣子似的。密續行者會憑自己的能力去轉化與接受事物,不會去逃避。沒有其他方法在能像它般使心靈更為成熟與堅毅。然而,這一個方法雖能快速帶來效果,可是同時亦很危險。至於在孟各招待我們的那戶人家是否會受到這些新進的發展和改變所影響,未來會告訴我們答案。

我知道有一天我們必須離開這裡,就像早幾個月前在丹麥接獲指示要來到南部的難民營時一樣。我們再次感受到噶瑪巴的召喚,不過我亦知道去見噶瑪巴之前,我們應該先到尼泊爾拜訪喇嘛傑珠。我們向孟各的這一群朋友道別,再次揹上行囊踏上旅途。我們在南部的難民營裡生活了六個星期,當我們從大門口離去時,那一群印度警察簡直是嚇得目瞪口呆,彷彿我們是從天上掉下來的一樣。

