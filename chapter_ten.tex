\chapter{不丹之路}

每天總是有川流不息的人潮湧至寺院。他們都想要見噶瑪巴,領受加持,向他請益,通過黑寶冠儀軌感受其證悟之心流。噶瑪巴的出現帶來了加持力和淨化力,親見他的次數越多,我們所感受到的改變便越多。一開始時,它們只是一種純粹的具體能量,像受到雷擊般。我全身劇烈地顫抖,後來我們倆雖感到筋疲力竭,不過卻處於一種大樂與圓滿的狀態。漸漸地,這一種經驗變得更流暢、無限與持久。每一次的儀軌都將我們引領至超越了時空的境地,因此只要一有機會,我們便會好好把握,與眾一同分享噶瑪巴的智慧心。

既然再見噶瑪巴,我們當然不願意離開了。我突然冒出想當噶瑪巴的司機的主意。這些年來,我在高速公路上累積了不少經驗,不但開車的速度很快,也能不依照慣例駕駛,十分敏捷。其實我們心裡是希望這些印度人沒那個膽子驅逐噶瑪巴的司機而已。按規矩,外國人只能在錫金逗留幾天。錫金距離中國駐西藏的軍隊非常非常靠近。

當我們想盡辦法留在錫金的同時,那種無時無刻都想要見到噶瑪巴的渴求也逐漸消失。噶瑪巴的能量無所不在,我們不再堅持一定要一直貼近他身旁寸步不離了。太多人希望爭取他的關注,他所要兼顧的事務真的太多。慈悲的噶瑪巴有求必應,他總是耐心地回答每一個人的問題,替他們排憂解難,也滿足他們的願望。對於其他人會認為是無可救藥或敗局已定的對象,他也從不逃避。他似乎對任何狀況,無論好壞皆一視同仁。他不受過去和未來所限,平等關照一切眾生。他能和任何眾生結下深厚的法緣,待時機成熟之際,所種下的菩提種子就會開花結果。

僅通過日常的黑寶冠儀式,便能感受到噶瑪巴源源不絕的加持力,而且這一段期間,只是待在隆德寺便能啟發心性的潛能。不久後,漢娜開始認為我們是時候作出實際的修持,而非只是一昧沉醉於驚嘆之中不能自已。我們向噶瑪巴請法,他很高興地大聲喊出三個不同的咒語,接下來我們應依能力儘量持誦。他所傳授的第一個咒語乃為第二世噶瑪巴,偉大的瑜伽行者噶瑪巴希(Karma
Pakshi)的心咒。馬可波羅(Marco
Polo)聲稱曾在蒙古親見噶瑪巴希,並且描述了噶瑪巴希如何通過其法力教化了忽必烈(Kublai
Khan)。此心咒將能激活噶瑪巴希殊妙的能量;由於噶瑪巴希也是蓮花生大士的化現,因此心咒也帶來了舊譯派(又稱寧瑪或大圓滿)的加持。第二個咒語乃與金剛亥母(多傑帕嫫)相應的鑰匙。金剛亥母表徵一切諸佛的內在智慧,過去密勒日巴尊者也曾以金剛亥母作為觀修的本尊形相。金剛亥母乃現女相之本尊,身裸,呈紅色,舞立姿,全身發光透明。第三個咒語乃為「金剛黑袍」大黑天護法之心咒。大黑天乃噶瑪噶舉傳承之事業護法。我們很訝異噶嗎巴並未以耳語傳授的方式將這些密法傳給我們。不過噶瑪巴說這些殊勝的加持力與能量將會在我們的自心裡外不斷增長。

隆德寺裡眾僧侶每天的生活大致上是這樣的:喇嘛和出家人們會在清晨三點鐘左右起床,然後各自開始打坐和誦經。六點鐘左右,他們快樂地聚在院子裡供洗刷的地方,這裡的水龍頭也是唯一供水的水源。接著,他們會各自回到寮房吃早餐。早餐一般是藏式酥油茶拌壓碎的玉米或大米。當時隆德從慷慨的國王處獲得不丹紅糙米,因此他們會較其他吃當地白米的僧侶更為健康。儘管如此,他們當中仍有百分之八十的人患有肺結核病,而且全部人都營養不均,缺少維生素與蛋白質。當他們從家人或功德主處獲得金錢時,他們總是會去購買烘過的新鮮糌粑。糌粑可說是他們最喜愛的食物。

早上七時正,眾僧侶進入寺院大殿誦經共修。眾人集合共修的力量為一切眾生,為特定的家庭或個人祈福除障。他們通過專注持誦諸佛的經文和心咒與甚深的禪定來激活證悟的能量,而且所祈求之效果往往會以意想不到的速度彰顯。此過程稱為「法會」(梵:Puja,藏:Cheupa),意即「通過供養承事祈請」。供養可分為外、內、密三個層次和虛空供養。只有捨下我執才能領受佛力的加持。法會中所響起的每一個音符都是一種意念的表達,與身體中的覺性產生共鳴,讓我們的心從分別概念的思維中解脫出來。心的界限一旦瓦解,大樂便即刻現前,深層的淨化因此而產生。當一個人在法會上強烈的音樂中找到自己的生命力時,內在強烈的證量便會恆常現前。

到了午休時間,寺院裡傳來一陣陣的琅琅聲響。小童學習拼字,較年長者則在背誦各種經文。一般的現代教育著重於個人的獨立性、理解能力以及對資訊的迅速掌握。然而,西藏人卻知道更好的學習方法和更好的教導方式。他們在前三年用背誦的方式學習,先打好根基,讓未來的學習更有次第的增長,在生活上對他們有所幫助。儘管當地疾病猖獗,他們當中有的人又經常受到一些惡和尚的殘酷毆打,外在的生活環境條件悲慘,可是這些兒童卻比我們所見來自其他文化背景的孩童更顯得和睦。

中午時分,眾僧相續進行其他法會,或分擔寺院內外的雜務。他們以木刻版一頁一頁地印書;他們製做草藥、教學。當夕陽西下,長號角聲再次響起,眾人又分別繼續修持個人的晚課,回到寮房裡靜坐禪修,他們當中根基較好者甚至能將禪定延長至睡眠中。

當噶瑪巴蒞訪時,這些日常的規律便會完全亂了套。除了訪客川流不息,寺院裡從早到晚都有各種儀式和灌頂在進行著。不過,西藏人的應變能力實在是令人佩服。即便是針對一個很特別的儀式,一切已準備就緒,他們對於突如其來的發展和變化都能夠從容且毫不猶豫地作出改變和調整。他們總是能讓所眾人覺得是特別為他們所作出的改變。多年的禪修經驗使他們消除了慣性的思維,變得不受時空所限制。

漢娜與我希望能擁有更多自由,尤其是時間上的自由。我們希望和噶瑪巴在一起的時間能長一些(最好是能在他身邊待一輩子),只是外在的條件並不允許。印度官員達斯先生一天內打了好幾次電話過來。他要我們立即離開錫金。在此之前從未有外國人在錫金待上超過七天的時間。一般人在三天之後便要申請展延簽證。從寺院沿著蜿蜒的道路北上二十公里是乃度拉山口(Nathu-La)。從那裡可通往西藏的準比山谷(Chumbi
Valley)。那裡有許多中國軍人駐守,不久前才爆發了一些小衝突。邊界的印度軍人遭到中國軍人相當無禮的對待,導致他們非常急躁。個性開朗,操得一口流利英語的吉梅醫生嘗試利用自己的影響力和相關要員交涉過幾次。他告訴對方說噶瑪巴即將啟程前往不丹,屆時我們也會隨行。可是無論他說甚麼,始終無法改變結果。

漢娜和我才不屑去理會那些政府官員。如果我們繼續逗留下去,他們也只能將我們逐離錫金,但前提是他們必須先找到我們。不過,西藏難民在印度沒有法律權利,在印度人面前又處於相當不利的位置,作為這些難民的客人,我們也不得不違背意願遵從他們的意思。可是一想到必須與噶瑪巴分開,即使只是短短數天,我們仍感到萬分不捨。當時的我們非但沒有把握機會學習忍辱,觀察自心,反之卻不斷埋怨這些印度官員,認為他們的無能已經到達了一個全新的境界。當我們在寺院頂樓噶瑪巴的寢室內與他道別時,他所說的一番話激發了我們的慈悲心。他說:「因為無明,所以他們才會拆散上師與其弟子。這樣的舉動會在日後帶來麻煩。他們對此是毫不知情的。」噶瑪巴臉上的笑容更是直接觸動了我們的心弦。他繼續說道:「就以我為例好了。我每天都在為毛澤東祈福,他很需要幫助。」後來我們約定在卡林邦見面,然後我們將負責開他的其中一輛汽車或貨車一起前往不丹。我們會先在卡林邦申請不丹護照,因為我們一旦持有不丹護照,以後若前往印度或其他喜馬拉雅地區時便不再需要簽證,所有阻礙我們留在噶瑪巴身邊的障礙將會一一結束。

在位於錫金和印度之間的邊界小城朗博(Rangpo),我們逾期逗留讓當地的官員感到非常生氣。在那一個動盪不安的年代,戰爭可能隨時爆發,多疑的他們總以為處處都有間諜勾串之虞。在連綿不斷的雨中,我們並沒有渡過堤斯塔吊橋,離開此嚴受管制的禁區。反之就在抵達橋口之際,我們截停了一輛計程車,取左道開上了蜿蜒的山路前往卡林邦。聽人們談論卡林邦,口吻裡盡是唏噓。直到一九五九年西藏遭入侵之前,卡林邦一直是西藏、尼泊爾、錫金、不丹和印度在經濟與政治上交手的相匯之處。儘管如今中國已關閉通往西藏的乃度拉山口,卡林邦依然是喜馬拉雅東部地區最多事物看、聽、購買和體驗的一個地方。此地也住了一名在廿四來不曾離開其住所超過七步的華裔瑜伽行者。當他初抵卡林邦時,其居住的地方是卡林邦郊外一個環境清幽的住宅區。不過隨著這一個小鎮不斷開發,如今其居所周遭已多了成群吵鬧喧囂的兒童。這名瑜伽行者不時與世界各地的人保持聯繫,那一本我們在尼泊爾時所閱讀的度母法本正是他所贈送。由於噶瑪巴與隨行喇嘛至少還需要一天方會抵達卡林邦,因此我們仍有時間去拜訪這名華裔瑜伽行者。

走在鎮上,我們一路避開了辛苦傳教的基督教與印度教傳教士。試想想一個沒有傳教士的宗教!這名瑜伽行者在他的小黃屋前接引我們,甚至還向我們多次鞠躬。以前走私金條到印尼時曾經感受過類似的氛圍。這裡和新加坡的那種古早味中國風情如出一轍,到處都有小相框和手工藝品的擺設。這名華裔瑜伽士個子矮小、圓潤,充滿著活力,可是卻看不出他的實際年齡。後來,我們才聽說原來他已經七十歲。他告訴我們幾則非常奇怪的故事,關於他如何在一個道教風行的社會追求佛教。說故事的同時,他健步如飛地在工作室內來回走動,四處從箱子裡、從架子上掏出一些小本子。他顯然寫了不少作品,如今就疊擺在我們面前。當時,他所說所寫的東西,對於我們那「前嬉皮」,最近卻剛「矯正」過來的心而言甚是一種滋擾。這些小冊子將多層次的佛法在同一個時候全體展現,這其中包括道德主義者所誤會的部分,以及需要更多前行準備及額外口訣指示的部分。此外,直到我們對金剛乘佛教的開展負起全然的責任之前,我們會受他對幾位喇嘛的批判所影響。由於我們看不清這些僧袍與笑臉的背後所隱藏著的古老的政治問題,我們一廂情願地認為所有法教如此圓滿,上師也必然同樣完美。無可置否,西藏是一個中世紀社會。西藏社會裡沒有所謂民主、自由投票、透明制度、人權或新聞媒體。眼看西藏周邊國家如斯殘暴不仁,西藏境內又存在如此顯著的社會不平等現象,在任何情況下,我們其實不難想像在這個地方可會出現多殘暴的處罰,背後的政治紛爭又會有多齷齪。虔誠心的力量是如此強大,但是當我們在喜馬拉雅的時候,幾乎沒有想到可用業力的義理為眾生帶來改變的契機。幸好,我們從未聽說過任何關於噶瑪巴的閒話。有一次,這瑜伽行者陳氏被一條狗咬住了頸項,「噶瑪巴千諾」[註:,「噶瑪巴千諾」乃噶瑪巴心咒]讓他死裡逃生。

這些事情有很多是在後來的拜訪中才得知的。我們第一次見面時,瑜伽行者陳氏對我們而言只是牆上那一些精緻的小唐卡的主人而已。所有的唐卡都非常精緻,描繪著我們未曾見過的許多佛聖像,諸佛聖像的加持力更是將我們重重包圍。他也是一名說故事的好手,我們對他的了解也逐漸增長:當他出生時,母親的乳房上多長出了兩顆乳頭為他哺乳。這裡更深層的意義其實是指他領受宇宙四方的加持。

年輕時的他曾在中國當老師。當時,他經常害怕自己會遇上非時死亡而十分惶恐不安。他曾經花多年時間的修練道教的養生延壽之術,不過後來卻開始追隨佛教,稍後更前往西藏修行。他在康區(Kham)遇見了著名的法國婦女亞歷山大‧戴維德‧尼爾(Alexandra
David-Neel)。她可說是第一批入藏,而且又寫了不少關於修行上的神通等甚具啟發性的宗教書籍的西方人之一。陳氏沒提及當時在西藏梅毒肆虐一事,可是他卻不斷強調這名法國婦女不願意與她到山區裡拜訪的那些瑜伽行者們進行雙修。他覺得很可惜,否則所有人都能速證菩提也說不定。他也告訴我們有關自己在西藏某個山洞內閉關修行的故事以及前往拜訪他的婦女們,感慨現世「空行母」(覺悟的女性能量)真的是可遇不可求。

當我們仍在聊天的當兒(他的英語既古怪又逗趣),他喚了某個在窗外看熱鬧的小男孩進到屋裡來。外頭有一群人不斷盯著屋裡這奇怪的男人與他那奇怪的訪客狂看。陳氏給了小男孩一些錢,讓他去買些「莫莫」(momo)回來。

我們都知道「莫莫」。麵糰擀皮以肉碎包餡,再上鍋蒸熟了,就是「莫莫」(註:中國人的蒸餃)。西藏人非常喜歡吃。從進行療癒的那天開始迄今,我們已經茹素三年,此時並不想破戒吃葷,於是便告訴陳氏我們不吃肉餃子的原因。我把小男孩喚回,跟他說不必買了,不過又聽見陳氏跟他說了甚麼「莫莫」的,接著小男孩又跑了出去。我們以為他讓小男孩去買一些起司包餡的無肉餃子,不過小男孩拎著回來的很明顯是肉餡包餃。我們再次向陳氏強調說我們不吃肉,他卻回答說以前佛陀是施主供養甚麼,他就吃甚麼。絲毫沒有扭轉的餘地。我們心想這次一定會失去所有的加持力了,麻煩也必定接踵而來。當我們想到這些動物們在被屠宰前,因為極大的恐懼(它們怎麼看也不像會安祥地死去)所以大量分泌的荷爾蒙會殘留在肉裡,我十分擔心吃了這些肉後自己較具侵略性的一面會因此被喚醒。最重要的是,我們不禁猜想手上的銀製手鐲會否因此而失去了療癒的能量。

在哥本哈根第一次的療癒發生前的數小時,我們因為受到「先知」哈拉爾所啟發,因此將兩片火腿留在餐盤上不吃。當時我們嘗試要戒肉一個月。就在第一次的「療癒奇蹟」發生後,我們確定這一定是和戒肉有關,自此便決定轉為素食。此外,我們亦對白度母(我們殊妙的女護法)感到愧疚。我們正將祂所護佑的眾生吃進肚子裡呢!出乎意料的是在吃肉之後,或許除了晚間的睡眠變得更為深沉,自覺性較低以外,似乎對我們的身心沒有其他不利的影響。

接下來幾個星期,我們避免吃肉,甚至要求噶瑪巴授戒讓我們遠離肉類,奈何他怎麼也不肯。他顯然是希望我們能靈活變通。如今我們就如世界各地的佛教徒一樣:坦然接受所施予之物,私下儘量少買肉類食品。我們絕不允許任何動物因為我們的口腹之慾而遭宰割,絕不讓自己成為殘酷的殺戮背後的主因。我們可以通過對盤上食品祈願念咒「嗡阿彌疊瓦舍\textsubscript{利}」,幫助這些動物與屠夫,令之在未來世能投生善趣。如果念咒時距離動物往生該日未逾七周的話,這一個動作將對他們大有助益。

在卡林邦,我們下榻的貢普斯酒店不但地理位置優越,環境也想當優美,從那裡可以俯瞰所有進出鎮裡的公路。酒店的主人塔希與其妻子和母親和各區的人都有些交情,因此若有甚麼事情發生,他們必定會擁有第一手消息。塔希的上師是一名蒙古籍,年逾百歲的喇嘛。他對上師非常虔誠。這名蒙古籍的喇嘛就住在一座小山坡上,據說他擁有特殊的療癒能力。我們想要前往拜訪,順道去看看蓮花生大士閉關的洞穴,不過由於噶瑪巴的車子隨時會抵達,所以未能成行。從早晨開始,我們便帶著所有行囊,躲在酒店入口處附近,一來是要避開警察,二來只要車子一到達我們就可以馬上跳上車疾塵而去。一個下午晃晃悠悠又過去了,沒人出現。這次似乎不像是「西藏人慣性耽擱」這麼簡單。儘管要順利聯繫錫金的機會渺茫,但是我們還是嘗試撥了一通電話到錫金,而且出乎意料之外的是電話馬上就撥通了。電話的另一端傳來Jigmela嘹亮的聲音,他也確定了我們今早就已聽說可是卻不願意相信的一個謠言:連接錫金與卡林邦的公路又再次被大雨沖走了。噶瑪巴前一晚已從另一條地勢較高的軍用公路出發。按時間計算,他現在也許已經抵達不丹。

這意味著若我們想見噶瑪巴,那麼我們就必須非法入境不丹。不過似乎也沒什麼差別,不丹本來就不對外國人開放。若按當初的計畫充當噶瑪巴的司機,那麼過境時鐵定不會有人阻攔。只是現在計畫有變,這下我們也只能自己想辦法矇混過去。整個情況因此開始變得微妙地有趣刺激。

我們原本光禿禿的頭上又再次長出了一些頭髮,加上曬黑的膚色,我們和當地的康巴看起來沒兩樣。此外,我們也穿上了藏式傳統外袍「楚巴」(Chuba),巧妙地掩飾了我們的真正血統。如今,我只是要小心我的藍眼珠不被發現,否則麻煩就大了。當我們登上那一台即將從卡林邦開往彭措林(Phunchoeling)的巴士時,司機先生露出了奇怪的神情打量著我們。彭措林是我們進入不丹都城聽布(Thimphu)之前將會路過的第一個邊陲小鎮。幸好司機先生是西藏人,和他溝通好後,他答應不會向邊境守衛洩露我們的身份。

那一條地勢較高的軍用公路穿越一座座風景如畫、綿延起伏的山巒,路經破落的軍營後抵達山口。之後,道路蜿蜒下山路經一片一片濃郁蔥蘢的草木植被,儘管此區域非常潮濕,這些植物卻看起來像是典型的地中海型灌木叢。巴士沿路進入低地,一直順利地向東前進。我們穿越一片片無盡的茶園,茶園裡的平頂合歡樹綠蔭如傘,視野的左側是連綿起伏的山麓小丘。當巴士抵達第一個檢查站時,我們躲在椅背後。我們一上巴士就挑了較暗且又不易被發現的後座。結果,檢查站裡的印度官員只是在門邊探了探頭就讓巴士開走了。第二個檢查站的情況較緊張。幾名海關人員上了巴士,他們其中一人沿著狹窄的通道走向巴士末端。我們儘量卷縮著身子保持低調,佯裝暈車導致身體不適。當他走到我們的座位旁時,我彷彿能感覺到他的氣息就打在頸項上。不過對方似乎對架子上的行李較感興趣。在昏暗的燈光下,他並未注意到我們的膚色較淺。當他轉身離開後,巴士準備行駛,我們帶著成功矇混過關後的一陣狂喜,隨著巴士浩浩蕩蕩駛入了不丹境內。

抵達彭措林時天色已晚,開往都城聽布的巴士都已停駛,據說抵達聽布之前一路上還會經過七個檢查站。我們認為應先在彭措林待一晚,隔天再去辦理所需的入境文件。在這一個國家,人人都對噶瑪巴非常虔誠,所以我很願意奉公守法。我甚至還作了一個夢,夢見自己開著一台官用吉普車上山,看看多快能抵達都城聽布。

決定留宿不見得是個壞主意。打從我們越過邊境開始,某些較重的業報顯然已經成熟。漢娜突然病得很嚴重,我們找到留宿的地方後便卸下行李、掛上蚊帳酣然入睡。漢娜在半夜醒過來,迷糊之中看到有一個男人正在翻我們的背囊。漢娜顯然已將「勿以小人之心,度君子之腹」的教導深植在心裡。她沒有把我叫醒,迷糊間甚至責怪自己胡亂猜忌別人會居心不良想偷我們的東西,她以為那只是自己發燒腦袋不清楚亂做夢,結果不予以理會又再次沉沉睡去。

隔天早上醒來,只見行囊裡的東西有一半攤了在地板上,另一半則在窗戶外面。我們把東西收拾好後,發現隨行所攜帶的器材並未遺失。甚至那台舊佳能相機和彩色膠卷也沒被拿走。這些東西在此地可算是相當貴重的東西。只是有兩件物品不見了:噶瑪巴頂戴黑寶冠的照片,以及對我們護佑有加的白度母唐卡。噶瑪巴的照片是我們從卡林邦11英哩的「OM」照相館購得。我們經常購買這一類型的法照寄給朋友,或當成禮物送人。遺失了噶瑪巴的相片已經很糟,不見了唐卡更雪上加霜。這一幅唐卡活靈活現,栩栩如生。白度母(藏:Dolkar,
卓噶)乃為女性覺者,是觀世音菩薩慈悲眼淚的化現,祂的能量場在我們的時代非常地活躍和強大。

白度母以七眼凝視無垠虛空。七眼象徵佛母能夠觀一切眾生的苦難,並以無上智慧救度他們的能力。除了臉龐上的一雙眼睛,其餘五眼分別長在額頭、手心,以蓮花坐姿朝上的腳板上。度母的右手在膝前結施願印,左手當胸捻著蓮花。蓮花綻放於耳際,表徵純淨的本性。

遺失了這一幅唐卡尤其讓我們覺得心疼。雖然那只是度母聖像的簡單速寫,在我們西方人的眼裡看來,畫中度母的臉部比例是如此地完美。畫這一幅唐卡的人是隆德寺裡的一個老畫匠。老畫匠天賦異稟,只可惜已漸漸面臨失明。如今,老畫匠已經架著兩副眼鏡在頭上;不久後,也許他只能單憑「心眼」來作畫,或由其他弟子來秉承他的技術了。這一個奇怪的賊/密探(?),管他是誰,他的舉動可真讓人覺得匪夷所思。我們甚至覺得這些照片和畫像現在搞不好就在山上,在噶瑪巴那裡。依據金剛乘佛教的說法,「塞翁失馬,焉知非福」,若遺失此類型的聖物一般會被看作是消災擋劫,在佛力的加持下躲過了像是意外事故或疾病等磨難。

我們向人打聽了當地SDO(搞不好他是DSO也說不定)(註:SDO與DSO乃當地政府官員的職銜)的住所。我們總是搞不清楚這些前英殖民國所使用的職銜縮寫,聽起來是如此華而不實。不過,我們其實也能夠明白這些頭銜對當事者而言又有多重要。

第二天清晨,天色仍早,我們已來到這名官員的家門口。我們天真地以為早前成功矇騙邊界的印度官員混入境一事必定能惹得對方大笑,接著他會熱心地替我們辦理入境所需的文件,最後再請我們替他向噶瑪巴問好,然後送我們離開。我們甚至幻想自己會開著吉普車上山到聽布去。當這名官員睡臉惺忪出現在自家洋房的大門口時,他的第一個反應卻和我們所想像的相差十萬八千里。他臉上的表情彷彿寫著「我應該只是在作夢」般驚愕不已,回過神來後,他大聲囔道:「你們怎麼會在這兒啊?」這名SDO看起來就像典型的孟加拉印度人(這倒是出乎我們意料之外),不過如今我們確實是身處在不丹境內沒錯。我們向他解釋說其實我們是噶瑪巴的學生,因為在卡林邦錯過了,所以他邀請我們前來聽布會合,因此希望能夠獲取在檢查站所需的通行文件。這名官員漸漸從錯愕中恢復過來,熱心的他亦有意幫忙。只是上頭有令禁止白人上山。他說,若他們在更高處發現我們的話,大概已將我們關進牢裡了。他答應幫我們和聽布當局聯繫,只是眼下此事不易辦妥,因為政府部門裡的官員大概都已休假參加噶瑪巴的法會去了。他說,直至將所有事情弄清楚之前,我們將會以國賓的身份待在鎮邊上緣的賓館裡。他們為我們準備了一間大房,房外不僅是一片美麗的草坪,還能將延綿山坡的美景盡收眼底。他們也提供了一切物質上的享受。這裡有幾名尼泊爾籍的傭人將房裡上上下下擦拭得一塵不染,更費心為我們張羅一天三餐,這一切比起這些年來我們為自己準備的有過之而無不及。儘管我們過著像有錢人般的生活,可是心思卻始終無法落定。

我們的全副心思都在噶瑪巴的弘法事業上,心裡也清楚知道山上正在進行著很重要的事。正當我們每天受困此地動彈不得,只能焦灼地等待相關當局發下邀請函的同時,噶瑪巴在山上一定傳授了許多很有意義的灌頂。由於我們已經開始認識這些儀軌的意義與其所帶來的影響,因此錯過了更覺惋惜。以前我們純粹只是參加一個又一個的法會儀軌享受瞬間的快感,卻從未預想過它們會帶來長遠的影響。

所謂「灌頂」(藏:Wang),意指力量與力量的傳送。在使用或未使用法器的情況下,上師進入禪定,啟動無限佛性的某種功德和能量,然後傳授給受灌者。這種傳承加持力是代代相傳、不曾間斷的;一個清淨無間的傳承加持力是最大的禮物。受灌者與所處的世界將會有所提升,直至能達證「任運生起」,不假造作的證悟境界。屆時,我們將與佛合而為一,一切將會超越時空而圓滿無礙地自然現前。

通過灌頂,開悟的種子將會種入眾生的潛意識中,加上金剛乘見地上的修持以及有關禪修的訓練,就能夠完全開展心性的潛能。解脫之樂即是最大的安樂。這一世雖然無法具體地開展自心,可是所領受過的灌頂卻無論如何也不會白費。眾生在過去世通過灌頂所種下的菩提的種子將會在死亡之際或未來世一一從潛意識中再次浮現。這一種強大的力量能夠轉化業力,讓眾生在下一世與金剛乘結下法緣。這種能量場無論在任何狀況下都能運作,生生世世都能漸次增長,直到完全成佛為止。即使不明白其意義,只要有智慧,多參加這些佛教儀軌也是好的。然而,若無法(或無意)遵守某些灌頂的戒律,就不應該領受相關灌頂以免破戒。此外,把許多傳承的灌頂混亂的話也會帶來迷惑。所以在不看低其它傳承為前提,你必須一心專修與自己最具法緣的傳承。

當我們待在賓館裡享受著優質服務的同時,快樂的不丹人每天上山去見噶瑪巴時都會路過這裡。我們托人給噶瑪巴與吉梅醫生帶信,同時也嘗試打電話聯絡聽布的相關部委。當時整個不丹政府部門只有三部電話,各部門之間也沒設有分線。因此當我們終於撥通電話時,接電話的是西藏難民部。我們還未來得及說出我們的意願,電話就被掛斷了。

日復一日,我們越來越不耐煩。漢娜很不開心,而我也是極度焦躁,內心裡負氣到了極點。我們那狂妄的自尊心怎麼也不願意面對現實。我們貴為噶瑪巴「萬中選一」的弟子千里迢迢來到不丹就只能乾坐在一旁,白白浪費許多寶貴的時間。對此噶瑪巴似乎一點也不在意。我們心裡實在是咽不下這一個事實。雖然心裡從未停止這麼想過,可是我們也無法真的就這樣步行上山直抵聽布。我們曾向該名官員保證會乖乖在賓館裡等消息,我們也把護照交到了他手上。一個星期後,他告訴我們說,他無法留我們在彭措林。聽布方面遲遲未有消息,我們不得不返回印度繼續等待。

如今我們能做甚麼呢?我們身上只剩下約二十美元現金,又沒有錫金簽證,可是我們真的不想折返加德滿都。當我們拖著沉重的腳步,帶著極度鬱悶的心情離開彭措林時,錯愕的印度邊界的士兵不斷探頭過來。在官方程序上,他們壓根不知道我們的存在,即使看到他們的反應,我們一點也不覺得好笑。將這種過去曾幫助過我度過難關的力量付諸於行,我說:「我們直接回丹麥吧!這真的夠了!」其實我們誰都不願意回去。我們千里迢迢來到這兒,目標尚未達成又如何能離開呢?我們也尚未正式展開修行。更糟糕的是,我們並沒有一個清楚的方向,沒有任何東西可與一直等待著我們的朋友分享。儘管我們不時因為各種加持而感受到瞬間的快感,實際上我們心裡都清楚,我們在實質上根本就沒有任何改變。欲馴服自心的這個目標,我們尚未達成。

如今還剩下甚麼可能性呢?除了打道回府,再次回到哥本哈根賺錢,與人分享這一路上的種種經歷,我們還能做些甚麼呢?我們站在邊界旁灰塵滾滾的馬路上,此時有一輛豪華巴士開了過來。這裡實在不像是會出現這種豪華巴士的地方。旅巴裡坐滿了富有的法國遊客,他們每個人都穿戴著最時尚的服飾與珠寶。原來這一群遊客從加爾各答出發,前往盛產茶葉的阿薩姆旅行,如今他們也想見識見識「時髦」的不丹。印度邊界那裡的印度人先前因為我和漢娜的「突然」出現已經引起了騷動,現在又來了一巴士彷彿是來自另一個世界的入侵者的外國旅客,更是把他們給嚇壞了。這些駐守邊界的士兵除了大聲叫囂之外,同時也將槍頭瞄準了巴士,試圖驅逐他們離開,而我和漢娜則趁亂請求他們載我們一程。善良的他們讓我們上了巴士送我們一程。這是一段奇妙的旅程,我們就這樣莫名其妙地坐在了一輛舒適且引擎運作良好的旅巴內,和一群陌生人談論著文明卻毫無意義的話題。我們坐在他們之間,省思著這一輩子所奮鬥的種種事物。幾個小時後,我們在西里古里的火車站下車,如今我們有兩個選擇:從西里古里出發前往德里,再搭飛機回家,或前往大吉嶺。我們可以很肯定的一件事是:我們不回去歐洲。我們無法就這樣離開這群西藏人,無法就這樣離開噶瑪巴。

我們在大吉嶺等信,也打算給父母捎一封家書。由於我們不再靠走私金子或大麻賺錢,父母親知道我們留在東方國家純粹是為了學習,所以也很樂意給我們寄錢應急。如今我們要求他們每月給我們寄大約五十美元,這一個數目對他們來說不多,不過卻足以應付我們在印度的開銷,甚至有剩餘的可與他人分享。在等錢寄到之前,如果運氣好的話我們將能靠身上僅剩的這一點錢過活。其實吸引我們前往大吉嶺的最大原因是先前在斯瓦彥布某個露天派對中聽說的某個傳言,漢娜恰好在此時想了起來。在靠近大吉嶺有一個叫索納達(Sonada)的地方,那裡住著一名年老的高僧。這名高僧是在噶瑪巴的允准下,第一個向西方人教授傳統金剛乘法教的西藏上師。雖然只是在吞雲吐霧的一片迷茫中見過他的照片,卻也足以讓我們著迷不已。如今他的名字又再次在腦海中浮現。他就是卡盧仁波切(Kalu
Rinpoche)。

