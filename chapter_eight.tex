\chapter{最後的迷幻之旅}

壞消息正在加德滿都等待著我們。我們早在噶瑪巴離開加德滿都的一個月前就已開始申請錫金的簽證。德里內務部竟然在兩個月後的今天斷然拒絕了我們的申請。這是一個障礙。只要想到我們因為這些無能的印度官僚被迫與噶瑪巴分開,心裡就非常難受。除此之外,我們也不得不考慮一些很現實的問題。儘管布塔拉錫米一如往常的熱心,幫我們在喇嘛傑珠的住所附近找了一個很棒的出租房,只是現在我們不只是身上的錢已所剩無幾,就連尼泊爾的簽證也快要到期。我們尚有很多想要學習的東西,不想就這樣打包行李回歐洲去。

我在監獄裡第一次留了長髮。在多年的走私生涯中,我總是把頭髮剪短,這樣會顯得我比較平庸均可和低調。其實長髮搔撓著肩膀的感覺挺好,而且讓我看起來較為溫和,所以我就這樣留了一段時間的長髮。蓄長髮唯一的壞處就是頭髮很容易纏結。一天,當我正在洗髮時,因為突然覺得要花太多時間打理頭髮,於是便決定要將頭髮全部剃掉。我頂著光禿禿的頭從火警瞭望塔旁的理髮店走出來,頭皮上仍然可見早些年留下來的一道傷疤。泰瑞恰好騎著腳踏車路過,停在我的面前。他一如既往的冷靜,帶著淡淡地口吻對我說:「聽著,你是大學生吧?有意到美國領事館教英文嗎?」我的誠實終歸得報。這是我們繼續留在尼泊爾的大好機會。這一份工作不但會為我們提供簽證,所獲工資也可給我們解燃眉之急。我當然毫不猶豫便一口答應了。

漢娜和我就這樣成為了尼泊爾六個月的居民。我們是那少數非常幸運的人,竟然能夠在像天堂般的這一個地方生活和工作。如今那越發嚴格的簽證審批條例,或是銀行的錢幣匯率再也不會對我們構成任何影響。我任教的「學校」是位於新路(New
Road)的美國圖書館頂樓的語言課室。我每個星期只需教幾天課,而且每一天只是幾個小時的課程,遇上宗教或國定假日還可以休息。這份工作挺有趣的。我的學生們來學英語的目的是為了到第三世界的大學裡學習農業。他們充滿好奇心、聰明,也擅於作弊。

我們每個月的固定收入是八百盧比,當時折合美金差不多是四十元。這筆收入已經足夠用以應付生活上的開銷、幫助別人、購買一些藏傳佛教的法器,甚至還會有多餘的錢剩下。唯一讓我們擔心的事,就是漢娜的肚子不知道得了什麼怪病。當地的一家俄羅斯醫院一直無法揪出病因,不過我想即使漢娜身懷九個月身孕,他們大概也無法診斷出來!除了這一個問題之外,我們在尼泊爾的生活安好,甚至沒有受到雨季的影響。每當雨季來襲都會對許多人造成不少影響。俄羅斯領事館在不久後也聘請我去教導他們的某個大人物英語和德語,而且當時整個情況就很妙。俄羅斯方面付的工資較高,每次都會派出其中一輛ZIL禮科特夫豪華轎車到美國領事館的大門口接我。他們雖然長得牛高馬大,可是看起來總是一臉的不快樂。我在課堂上有一半的時間是用來鼓勵他們要對自己所曾受過的高等教育具有信心,幫助他們克服「害羞的心理」,大膽開口說話。

我們通常會在西藏邊境渡週末,除了為這一個已遭摧毀的國家祈福,我們也會去泡溫泉。儘管我們所專注的事,甚至是思維模式都仍未改變,可是我們卻出乎意料地成為了加德滿都裡外來自不同世界與世代的人們所依賴的對象。我們一直試圖避免受到這些資產階級所依賴,因此如今感覺非常微妙。然而更奇怪的是,我們卻開始不斷在他們身上發掘到很多「嬉皮世代」的我們所嚴缺的品質。

路比是我們的一個好同伴。它也是房東所飼養的一隻迷你黑色藏犬。有一天傍晚,它喝下拌有一些大麻小碎片的煉乳後就「茫」了。它斜著身子坐著,一直盯著我們看了幾個小時,似乎知道只有我們才會明白發生在它身上的事。當天我們徹夜未眠,試圖幫助路比渡過此次的迷幻經歷。它看起來似乎沒受到多大影響,臉上絲毫不見困惑或恐懼。雖然路比無法說話,可是它所經歷的過程和我們平時幫助渡過迷幻旅程的人們非常相似。那次以後,路比便對我們生起了極大的信心,決定要跟隨我們,讓我肩負起保護它不受到其他嫉妒的流浪狗所傷害的大任。這一段時間讓我更了解狗狗,也讓我開始了解人類心靈裡那一段灰色的地帶:我剛才究竟是因為慈悲,抑或是惱怒才會踹了這些流浪狗一腳呢?

我們每天都會向牆上噶瑪巴的聖像祈請他的加持,以金剛鈴專注所緣,一心不亂,然後誠心許願。我們無法接受在德里的那群草包竟然拒絕了我們的錫金簽證,因此私底下也同時在想辦法該如何在不需審批的情況下出入境。喇嘛傑珠勸我們繼續嘗試以合法的途經入境。後來我們還是乖乖聽取勸告,申請了為期幾天的遊客簽證,只是就連這種簽證也得要等上一個月的時間就讓我們覺得渾身不對勁。雖然我們從外表看起來不錯,可是其實內心裡已充斥著諸多不滿。我們每天傍晚都會去領受喇嘛傑珠的加持,而且總是在吸了大麻一頭茫然、雙眼通紅的情況下去見喇嘛。除了加持之外,喇嘛並未傳授我們任何教法。

一天傍晚,喇嘛把我們召到寺院裡去。我們興奮不已,心想:「哇嗚,終於等到這一天了!喇嘛終於要傳授我們灌頂,就像密勒日巴尊者生平故事裡寫的一樣。」結果,喇嘛只是請我們幫忙某個不丹喇嘛賣一些木製的杯子籌錢。我們走在回家那短短幾步的路程中,心裡除了感到莫名的空洞,亦深感自己受騙上當。

我們收到一封遲來的信,通知我們就在我三月份的生日那天,我們位於哥本哈根的寓所被某個德國嬉皮士給燒毀了。這一個地方曾經是我多次迷幻經歷的基地。對於此事,喇嘛傑珠只是微笑著說:「好啊!火具有淨化的作用。」不久後,我們在尼泊爾租房裡的供壇也在某天我們去參加宴會時莫名地燒了起來。除了噶瑪巴的照片和一些LSD,供壇上所有的東西都被燒成了灰燼。那些LSD是別人送我們的禮物,對方似乎對它的強勁藥效一無所知。

就在某個內心空洞茫然的夜晚,我們服用了這些LSD。此時此刻,我們至少可以確定會發生甚麼事。供壇雖被燒毀,這些致幻劑卻安然無恙。我們認為這是鼓勵我們去服用它的徵兆,而且非常期待前額內的那一股力道能夠轉化成明性。此LSD的藥性對我們來說也算強勁,它很快便開始發揮了藥效。我們感覺自己離開了軀殼在附近遊走,返回軀殼後又再次出離,然後繼續四處飄移。我們也有觀想蓮花生大士的聖相。當天亮在際我與漢娜交歡時,四周出現許多呈金屬色的菱形狀能量。我有一種感覺,就是這奇怪的能量欲使漢娜懷孕。我嘗試去驅逐它,迷幻的體驗不斷延續。即使幾個小時過去了,我仍然在語言課室裡看見那些菱形狀的能量。我的聲音變得非常沙啞,沙啞得甚至連話也說不清楚。當我在講解有關最近挪威沿海一帶所發現的石油會對經濟帶來的重要影響時,腦袋裡突然冒出一個想法:我想到加德滿都某個出口中國貨的商店去買一瓶人蔘口服液來喝。

我知道人蔘對身體非常有益。雖然指示寫說服用劑量只是幾滴,可是我卻一口氣把整瓶口服液喝光。當時我只感覺到有一種非常奇怪的能量進入喉間,而且那些菱形狀不見了。我突然覺得自己像回春似的年輕、充滿活力,彷彿回到了早些年前正在練拳擊時幾乎每周都會與人幹架的那一段時光,回到了那一個又急性子又魯莽的自己。可是這一個我卻魯莽地將某件事給鬧大了。換作是早幾天前,我應該會作出不同的反應。

是這樣的。我們乘著一輛全新的塔塔巴士在離開西藏邊境的路上。當時我們就坐在靠近前窗的位置。這輛巴士的引擎是在印度合法安裝的簡化版賓士引擎。在步伐緩慢的東方國家待著的時間長了,如今能夠感受到如此漂亮的一副引擎在腳底下嗡嗡作響,的確會讓我感到心醉神迷。只是那個長得牛高馬大,看起來像是頗具威嚴的退役廓爾喀士兵的司機似乎正在盡其所能在破壞它,這讓我非常反感。他就像當地其他的司機一樣,為了賺錢只要是站在路邊的東西都載,因此車廂裡很快就塞滿了許多人和動物,有時候甚至連車頂也擠滿了人。我有好幾次試圖告訴他,若超載的話,他必須輪流以不同檔駕駛,否則巴士的引擎就會被毀掉。他聽懂了我的話,不過卻完全沒有想要搭理我的意思。

果然不出我所料,這時問題出現了。由於引擎內的活塞被卡住了,司機只好讓巴士靠慣性滑行下坡到某個村口。當他下車檢查巴士的損傷時,我跑到他的面前,說:「是你害我們受困於此動彈不得的!現在我們需要搭另一輛巴士離開,你把車資退還給我們。」司機很惱火,用那雙油漬漬的手揪著我的襯衫,我一使力便直接把他扔到溝里去。他爬了起來試圖反擊,我又再次把他往溝裡扔去。當他第三次嘗試反擊時,手上還拿了一根鐵棍,不過巴士裡的其他乘客制止了他。我已把原本可用來擋拳頭的背囊丟下準備和他拼了,幸好事情就暫且在此打住。他要是拿著武器衝向我,我應該會把他傷得更重。

這一個狀況真的非常發人深思!其他人並不知道我和司機之前的對話。在他們眼裡,我這一個頂著光頭,頸項裡掛著一條念珠,襯衫上還別著一個印有噶瑪巴聖像小別針的白人,痛毆了一個倒大楣的司機。所有人的目光都停落在我的身上,我趕緊將念珠和小別針收進口袋裡。我尷尬得恨不得鑽進地洞裡,於是踉蹌地溜進當地的一家茶館,想要躲避眾人的目光。只是該負面的氣場卻一直如影隨形。平時的我對人可和藹了。可是現在每個人看到我都唯恐避之不及,就連狗狗也不太敢吃我給它們的食物。當天傍晚回到加德滿都後,我們便直奔喇嘛的住所,將所發生的事全都告訴了他。當我向他描繪那個金屬色的菱形狀能量時,他說:「啊!我知道那是甚麼。它總是到處惹事生非。別擔心,我會好好處理。」當天一直到深夜,喇嘛的搖鼓與金剛鈴的聲響和他祈請護法的陣陣念誦聲一直在耳邊迴盪。翌日早晨醒來時,我的雙眼洋溢著喜悅的淚水。那「東西」不見了。我感覺自己如釋重負。

一九七Ο年九月初一的前幾天,好福報和好運氣終於降臨。忽然之間,所有事情似乎開始各就其位,將我們引向錫金。簽證局裡有一名官員認得我和漢娜。那些年,除了簽證局裡的官員,要在尼泊爾遇見一個討厭鬼還真是不容易。我們和這名官員第一次見面時,因為他對漢娜無禮,所以我讓他吃了些苦頭。現在他升職了。在尼泊爾有很多高級官員會經常因為財富突增,與其工作收入不成正比而丟了工作,這也就是說他們會私賣簽證來賺錢。如今這位先生的職位正好可讓他一報當年之仇。他竟然勒令我們在一周之內離開尼泊爾。現在該怎麼辦呢?這時應該依靠美國人還是俄羅斯人好呢?還是要動用喇嘛的名義?我們知道他在尼泊爾的影響力甚至比領事館還大,因為當地皇族(儘管他們都是印度教徒)在往生時都必須依靠喇嘛的力量來轉移神識。還是我們乾脆自己動手,就利用經典的熟雞蛋來偽造簽證好了?當我們仍在權衡各種可能性時,德里方面傳來了消息說我們的簽證終於順利獲批。

幾天後,我們帶著許多藏族朋友交託要供養噶瑪巴與喇嘛們的禮物,踏上了前往錫金的旅程。按照西藏人的傳統,出門遠行者同時也要扮演郵遞員的角色。若先撇開遇到盜賊,或行李的負擔會將背脊壓斷不說,其好處就是無論你走到哪兒都會受到歡迎。丹麥的朋友們答應幫忙將我們這段期間在尼泊爾收藏的畫卷、佛像以及其他禪修用的法器用他們的福斯小巴一起載返歐洲。他們出現得正是時候!我們根本不可能帶著這些行李一起上路。出發之前,我們和幾個密友狂歡了一夜,而且還抽了不少大麻,害我隔天走起路來都顯得晃晃悠悠。在這種茫然的狀態下,我們壓根沒注意到裝著書本的那個背包在火車上被扒走了。當火車抵達某個印度北部的城市時,我為了避免踩到那些躺在門口的人而從窗戶爬出車廂外。我亦因為這樣而將某人所交託的熱水壺給打破了。熱水壺破了固然可惜,可是書本不見了可事大!我們把這視為是「僅限於知識層面的學習是時候要告一段落」的跡象,接下來應該是作出實修的時候了。

漢娜很肯定地認為抽大麻的習慣對於靈性方面的發展毫無益處可言,這也許是喇嘛傑珠遲遲未授予我們任何灌頂的原故。後來我們才發現原來是噶瑪巴先讓喇嘛「拖延」我們,聲稱我們是他(噶瑪巴)個人的弟子。當然漢娜也說得不錯。儘管喇嘛傑珠要將重要的程序留給噶瑪巴去進行,可是他之前也許曾嘗試讓我們明白吸大麻與靈性的發展(修行)之間所存在的衝突:大麻會導致我們的心變得散亂,修行卻能使之專注不二。就在前往覲見噶瑪巴的途中,我們毅然決定要放棄身上所有的毒品,一丁點也不留下。

我的九年毒海生涯(漢娜大概是我的一半)終於告一段落。這是我始料未及的事。這一種覺醒與能即時了斷的自由力量讓我感到非常震撼。儘管某些長期的後遺症仍會伴隨我,像我在甩頭時就會發出「卡嗒卡嗒」的聲響(這是開玩笑的!),至少我踏出了第一步。更令我們為之詫異的是,我們再也不想碰任何毒品了。

