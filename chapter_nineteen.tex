\chapter{正式開展弘法利生之事業}

我們的父母親早在機場等候多時,他們很高興我們這次沒有離開太久。在他們身邊站著的是我們的一群老朋友。他們當中好幾個人和我們在同一個時期脫離毒品,我們的第一本書《Teachings
on the Nature of
Mind》(暫譯:心之本質)更為他們帶來了深遠的影響。基於過去豐富的人生經驗和種種善因福報,如今他們已具有能力,並且很樂意去利益他人。他們今天會出現在機場,這其中的象徵意義已經很清楚,我們亦感到非常高興:我們在西方弘法利生的工作,將會從這一群形形色色、富有經驗以及關係親近的理想主義者開始。

隔天傍晚,噶瑪巴的口傳及加持力便已經在一群朋友中覺醒。雖然覺得不可思議,可是我們在當下便明白他最後那句話的意思:「無論身在何處,我將永遠與你們同在。」在丹麥的某個海岸旁,美麗的黃昏下,我第一次在歐洲大陸傳授噶瑪巴的教法。我能聽見自己在說法,也能看見自己用手握著噶瑪巴的舍利給大眾加持。這是一種比自我認同、苦苦掙扎的「我」的意識更為強烈和清楚的覺受,這本身就是一種大樂。

若以更廣的角度來看,這一個位置意味著漢娜和我作為一個被動、無憂無慮的佛教徒的日子將漸漸結束。若只是針對個人的人生,我也許不必對自己太嚴苛,可是倘若必須對他人的生命負責便不能如此。

這一本書的續集《Riding the Tiger》(註:暫無中譯)(亦是由藍海豚Blue
Dolphin
Publishing所出版),道出佛教來到西方國家後的發展健康與不健康的一面。不過,此書續集最主要仍是描繪噶瑪噶舉傳承在西方國家的獨特發展。最近,昆吉夏瑪巴在德里(Delhi)、多頓(Dordogne)、伊利斯塔(Elista
)和弗吉尼亞(Virginia)建立了噶瑪巴國際佛學院K.I.B.I.。這是在法國開展的一個寺院組織。目前為止,分布在全世界各個城市的兩百所在家與瑜伽禪修中心皆由我所負責管理。

從一九七二年迄今,多年來的弘法工作一直都很振奮人心。我很開心能夠看見西方國家能夠吸收傳統佛教的精髓,並且在過程中賦予以全新的生命。

一千兩百五十年後,蓮師的預言終於兌現:

「當火牛以輪奔馳(火車),鐵鳥騰空(飛機)之時,當藏民如蟻分散世界各地,吾之法教將會降陸白人之地。」

願有更多聰明獨立的朋友有機會接觸殊勝之法教,解脫輪迴之疾苦,利樂一切諸有情!

                                      ─ 全文完 ─
