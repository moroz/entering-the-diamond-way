\chapter{我們的蜜月之旅}
一九六八年的夏天,我們選擇到尼泊爾度蜜月,這顯然是一個還算不錯的主意。早期仍十分理想化的歐洲嬉皮文化與古老的藏傳佛教教派正好在此時相遇。這兩種迴然不同的文化的相遇,正好給予了前者一個指引的方向,也讓後者不至於淪落為在博物館內展覽的文化遺產,而得以作為實際的修持繼續被保存下來。爾後多年,我們不斷往返於青蔥翠綠的北歐低地平原與喜馬拉雅的冰峰雪脈之間,致力於推動在家或瑜伽式金剛乘佛教的修法,使之不斷繼續茁壯地成長。這也是西方唯心主義與亞洲唯物主義有史以來第一次相遇,開拓人類心性潛力廣大且全新的一面。

後來,我與我那漂亮的妻子漢娜,甚至是身邊的許多好朋友這才發現,其實我們所遇見的是一種能讓我們掌握生命的工具。金剛乘佛教將我們那狂亂不安的心識狀態,以善巧的方式轉化成一個澄淨與喜樂之境。它以一種非常實際與具體的方式,讓生活上的各個方面都成為是一種修行,使到解脫與證悟不再遙不可及。

那些年,我總是與其他走私販及一群敢作敢為、喜歡刺激冒險的朋友為伍。那個時候若想要待在一些來自西藏或不丹的得道高僧身邊承事學習也不是不可能的事情。這些高僧大德所展現的典範以及其善巧殊勝的教誨,皆深深地啟發了我們。爾後,經過慎密的觀察,他們也選擇我們作為法侶,將妙廣精深的傳法和禪修引介到西方國家。

我寫這一本書的目的是希望回顧當年當這兩種如此截然不同卻同樣珍貴的文化相遇時所發生的點點滴滴。這一本書的延續——《Riding the Tiger》\footnote{暫無中譯。}%
% TODO: Add details for the English edition
則著重描述之後在法務上所取得的發展與成果。這也許是這兩個高度文明的人有史以來第一次如此有意識地嘗試從彼此身上學習各自的優點。在過去數年間,由於有許多位持有完整內、密傳法的老喇嘛相繼圓寂,因此我想如今重新印刷此書,仔細回顧咀嚼,並與你分享當初相遇時的種種,也是一種難得的美好。

第一次遇見金剛乘佛教時,漢娜二十二歲,而我二十七歲。我們的父母是哥本哈根以北居民區學院裡的教師。他們都是難得可貴之人,我們在一個對「人性本善」充滿信心的世界裡成長。儘管自幼便在一個完全人性化的環境下被撫育長大,可是我總是不斷夢見某個不曾相識的山區裡戰亂流離的情境。夢境裡的我奮勇地將那些圓臉的士兵們一一擊退,拼了命去保護那一些當時的我僅能理解成是「裹著婦女紅色布段的那些男人們」。直到二十五年後第一次在尼泊爾看見藏族僧侶,我才開始明白夢境裡的這些人究竟是誰。又在隔了四十五年後,因為秘密前往已被中國所佔領的西藏東部旅行,才親眼目睹夢境裡那一些我拼了命想要保護的山區與村莊。

我這一輩子對於任何會限制我的自由的人事物(不管對方是否是甚麼大人物或是體制)很自然便會對抗到底。漢娜也是一個不受約束的人,只是她較為內斂,不像我會對外作出猛烈的反抗。

一九六一年的秋天,我服滿兵役退伍(我想當時所屬的部隊也是巴不得能儘快擺脫我)。我算是早期開始抽大麻的少數丹麥人之一。大麻(pot),當時候是這麼說的,後來相繼出現了不同的名稱,從大麻菸草葉(grass)到笑不停煙草(laughing
weed)都有。當時,我的哲學課成績很好,任何與心性有關的課題都能夠燃起我的強烈興趣。我從吸大麻的過程中獲得許多全新的見解,後來更藉由嗑迷幻藥踏入全新的體驗,接下來有好幾年都讓毒品主導了自己的生活。由於我認為它們為眾生帶來顯著的利益,因此決定要以奧爾德司‧赫胥黎\footnote{Aldous
Huxley}及其著作《知覺之門》(Doors of
Perception)作為撰寫博士論文的主題。

接下來幾年,當我在丹麥和德國求學時,除了任由毒品侵蝕我的心靈之外,我也相當熱衷於拳擊和賽摩托車,而且還曾釀出不少意外。而漢娜則努力完成最後幾年的高中教育。

漢娜與我就在哥本哈根大學裡的「食人族」(The
Cannibal)食堂再次相遇。我十歲的那年,漢娜只有五歲,我曾經在北方郊區的樹林裡教她利用樹枝造小木屋。因為女生都不擅長爬樹的關係,我對她們的評價頗差,認為她們煞是無趣,可是當時年幼的我卻漫步送漢娜回家。我想,那顯然是我這輩子第一次愛上了一個人。然而不久之後,漢娜舉家遷往更北部的地區生活,我們自此便失去了聯絡。如今事隔十五年,她又再次站在我的面前,不可思議的美麗。我甚至幾乎把身旁那位火辣的紅髮女郎給忘了。當時漢娜已訂婚四年,可是不久後,我們便開始一起生活。

一九六六年三月,二十五歲生日的那一天,我第一次嚐試了LSD\footnote{麥角二乙酰胺,香港人俗稱「弗得」的一種強烈致幻劑}。
% TODO: 找到臺灣的俗稱
當我走在中世紀哥本哈根式的老街道上(如今其面貌大概已經改變了不少),返回派對途中,某道窄門引起了我的注意。我穿越這一道窄門,庭院深深只有一盞燈的光亮。我知道有一些很特別的體驗正在等待著我。我駐足於某一棟房子前,其窗空廣而深邃。我聽見自己對著窗戶說:「就是現在了,向我展示一切吧!」就在此時,我能感覺到四周圍彷彿在迅速竄動著,有一股迴旋的白色光芒籠罩著我。這一種感覺彷彿就像是整個宇宙的能量猛然湧入了體內,然後在空中爆發似的。一股永恆、明亮的力量遍滿四周。回到派對上後,我試圖回想先前的種種,也只能說:「我無法確定那是甚麼,不過感覺真好。」

不久後,漢娜也加入我的行列,繼續探索意識裡的其他層次。她的經驗總是深邃而又溫暖。從二十名朋友開始,直到我們的小群體增至三十人(大概也是北歐首個內聚群體),我們嗑遍了當年所推出的迷幻藥,不過卻似乎沒有從中汲取任何教訓,天真地認為這些藥物能夠啟蒙我們,對我們有所助益,殊不知它們已經悄悄地開始損壞身邊好友們的身心。現在的我願意付出一切去交換他們,可惜他們當中大部分的人已經死了,再也回不來了。當初他們都是以非常痛苦的方式離開這一個世界。其他仍活著的人也相繼迷失在自己的私密小宇宙裡,此生恐怕再也無法回到現實世界中。今天,對於毒品我們只有一句忠告,那就是:遠離它們!毒品會侵害我們的潛意識,使我們的靈魂變得平板空洞。而且毒品具有潛伏性,對我們所造成的破壞並不會瞬即顯現,更不容易修復。若想要取得證悟,只要方法正確就可以回歸我們的自心本性。

當年,我就在一所夜間學院裡教英文。我們常趁著假期前往北非、黎巴嫩和阿富汗作短途旅行,順便給朋友們帶回一些哈希什大麻(Hash)。一九六七年,當我從孟買北部搭飛機飛往印尼的途中(當時因為不久之前肝炎發作,加上身上緊得讓人窒息的西裝背心裡裝帶著三十四公斤的黃金而感到反胃不適),我透過窗戶看見來自喜馬拉雅山的夜光雲。雲層美麗得讓人心動。我知道我一定要前往這些雲朵所來自的地方。那裡有甚麼東西正等待著我們。某種很重要的東西。

一九六八年五月,我和漢娜結婚了。這算是目前為止我們所做過最明智的一件事。當年漢娜二十二歲,我二十七歲,我們前往尼泊爾度蜜月。早在一九六六年,我曾經想要通過陸路的方式前往尼泊爾,不過由於當時印度和巴基斯坦爆發了一場戰爭,所有人滯留在阿富汗差不多長達三個星期(時間長得足以染上亞洲最可怕的痢疾)。當時我的體重大概掉了二十五公斤,
% TODO: 五十五磅 should we convert it to metric? check German original
在返回歐洲的途中,因為身無分文所以也只能沿路搭便車或徒步行走,我也曾為了掙路費到醫院賣血。現在回想起來,這一切也可說是一個相當有趣的經歷。

這一次我們並不想作任何中途停留。就在學生們考完試的當天,我們便鑽進了一台老舊的福斯小巴,一路開往漢堡市。有一台二手的福斯金龜車正在漢堡市等待著我們。取車後,我們開著小巴直驅南下。克雷斯、克里斯蒂安(我人生中的第一個大麻便是他提供的)及其太太也同行。當時,克里斯蒂安已經不良於行。他是在某次經歷「迷幻之旅」時忽地跳出窗外摔斷了背脊才導致要坐輪椅,因此這一趟旅程只剩漢娜、克雷斯和我負責開車。然而,克雷斯很快便受不了伊朗攝氏五十五度的高溫、境內漫天的沙塵飛揚以及穿越沙漠那一條未經鋪徹的破爛坡路。而我們則不惜一切代價要儘快趕往尼泊爾。打破了連續趕路六天六夜的紀錄後,我們終於入境阿富汗。克里斯蒂安和太太為了強勁的哈希什大麻想要留在喀布爾,因此在喀布爾放下他們後,我們便趕緊賣了小巴,將所賣得的價錢用作我們的旅費。才剛賣了小巴不久,漢娜便重蹈我兩年前的覆轍,染上痢疾病倒了,幸虧當時下榻的旅館設有洗手間。等到第二天,漢娜已經服食了許多抗生素,我帶著她搭巴士繼續往東前進。巴士將會一路開下開伯爾山口,進入印度和巴基斯坦以北的平原地區。

\placeholderFig{1968年,在喀布爾的貿易一景}{foobar.jpg}

在喀布爾,我們巧遇正在修復世界最大佛像的幾位丹麥人。此尊佛像高五十三米,矗立於喀布爾以北約一個小時巴士車程的巴米揚(Bamian)。這是一九九六年穆斯林塔利班假不符合伊斯蘭教之名誓要摧毀的文化瑰寶之一。若讓塔利班政權統治北阿富汗,同樣的情況仍然會發生。這幾個丹麥人是修復這些佛教文化遺跡的專家。雖然一千多年前,穆斯林為了剝奪佛教的影響力而將大佛的鼻子擊落,導致莊嚴的大佛遭到了相當程度的損壞,可是早在60年代後期,%
% 六十年代是大陸說法,要不要改成臺式的六零年代?
可觀看的東西仍不少。約公元前300年,就在亞歷山大大帝的時代結束後,犍陀羅文化在此地取得了興盛的發展。其實要辨識出犍陀羅文化的藝術並不難,佛像上的鬍子便展現了希臘與印度文化元素的混合,緊緊地追隨所征服之地區的文化元素。座落於巴基斯坦的白沙瓦博物館也擁有相當令人感到嘆為觀止的文物收藏。博物館中擁有眾多經過詳細考察和研究(甚至是可追溯至英格蘭時期)的收藏品,可是眼看他們將近期所奪得的印度槍械也以如此傲人的姿態展示於博物館中,便顯得極為突兀和格格不入。

來到印度後的感覺還不錯。我們在邊界時便已經感受到其中的差異了。我們將所有緊繃的情緒,回教國家性壓抑的氛圍都統統留在背後。身處回教國家時,我時常得把人從漢娜的身邊推開。現在終於可以鬆一口氣,如釋重負。費羅茲布爾(Ferozepur)是我們下一個即將前往的城鎮,我們會在那裡搭乘接續開往新德里的電車。在那個年代,若想要辦理尼泊爾簽證就必須前往新德里。現在他們在邊境就有發為期一周的簽證了。新德里是人們為了辦簽證才會去的地方,沒有多少遊客會想要在那裡多作逗留。在舊德里鄰近地區有那麼一丁點美麗的印度洋景象,也許有的西方人會喜歡,只是如今那裡也只是剩下一片喧囂。除非身上擁有多餘的金錢能夠自由地繼續移動,否則無可奈何要留在該地區的人,在面對那裡老盯著人看的居民,還真是少一點耐心也不行。

我們生平第一次接觸印度教大師是在新德里。那一次會面收穫豐富,真想向學校討回過去曾繳過的學費,只是他始終無法觸動我們的心靈。我們感受不到與他們之間的聯繫,心裡也完全沒有想要和他們產生任何聯繫的想法。他們有點過份熱情,讓我們無法適從。只是,他們所展現出的力量也頗讓人印象深刻。此事就發生在康諾特廣場最外圍的圓環區裡。這一個行政地區是英國人留給印度人最後離別的禮物。據說這使到印度人更糊塗了。

在這一個餐廳和紀念品商店林立的中心點,一條條的街道向外擴展,四周圍大使館和別墅林立。在新德里所發生的事也同樣在這裡上演。我們走在拱廊下,忽然有一名裹著頭巾的印度老人走到面前,嘴裡咕噥著說甚麼「幸運的額頭」,便往我手心裡塞了三張捲起的紙張。他不斷在我面前比手畫腳,讓我把注意力放在他的雙手上。他同時也直視我的雙眼,讓我說出一種水果的名稱。我在心裡想著各式的水果,當中有的或許他連聽也未曾聽過。當我還在心裡盤算著是否應該戲弄像他這樣一個老人家時,我已經衝口而出,說了「蘋果」。他看起來高興極了,從我手中拿出其中一張紙,打開一看,紙上寫著的正是「蘋果」。「現在請說出一種花的名字。」他堅持要我說,接著又發生了同樣的情況。當我心裡在想著木槿和其他的異國植物時,就在那種半恍惚的催眠狀態下,腦海裡突然冒出了玫瑰花,因此很自然地便說了「玫瑰」。

第二張紙裡寫的當然就是「玫瑰」,他老人家開心得直跳腳。在接下來的第三個測試中,我必須說出一到十之間的其中一個數字,我的自尊心只允許我默想著數字「1」,我堵住一切外來的干擾,堅持自己的想法。然而,他老人家還算是挺走運的。他試圖向我暗示數字「7」,然而由於最後的那一張紙上寫著的字看起來非常像是數字「1」,因此成為了他挽回局勢的籌碼,我們之間的遊戲才得以繼續。我們跟隨他來到某個不會被警察發現的角落,然後他打開大多數占卜師都會隨身攜帶的小本子,對我們說:「把錢放這兒吧!我給你占卜未來。」

在北歐,教會都是由稅收所支持,因此身為北歐人民的我們並不喜歡看見金錢和靈性如此公然的存在任何掛勾。不過正因為當天尼泊爾大使館關門了,也看在這老頭子如此賣力的份上,我們心裡生起了一絲悲憫,便在小本子夾了兩美元。漢娜在一旁不斷催促說我們必須趕緊離開。老人知道自己已經無法掌控整個局勢,或留住我們,於是也給我們贈送了一份臨別之禮。這似乎是正在向我們展示他的能耐不僅是能對人作出心理暗示而已。他說有一個名為簡森的女孩向老家那邊的警察提起了有關我們的事。當時,老人的一番話對我們來說一點兒意義也沒有,因為朋友們的姓氏,我們本來就不懂得多少。不過,後來我們是有聽說某個同名的女人向警方告密舉報我們。這一個老人原來還真的是具有某種程度的洞察力,能夠未卜先知啊!臨別之際,他還語重深長的叮囑我(正如其他智者經常會做的事一樣)要繼續留在漢娜身邊。

打從我們離開哥本哈根那天開始,那本綠色的書本便一直是我們在路上的良伴。書本的封面是魁偉莊嚴的佛像。我們一有時間就會努力鑽研書本上的內容。這一本書就是由喇嘛卡孜‧達瓦桑珠(Lama
Kazi Dawa
Samdup)所譯,英國學者艾文斯‧溫玆所編輯的《西藏瑜伽與秘密的教義》%
\footnote{Evans-Wentz, W. Y. \textit{Tibetan Yoga and Its Secret Doctrines}. London: Routlege, 2016.}
%(Tibetan Yoga and Secret Doctrines)
。書中包含了藏傳佛教噶舉傳承廣妙精深、極為有效的修法。當年,佛陀也只是將這些法門傳授給利根的及門弟子。每當我們翻開書本仔細閱讀,就會生起一種非常強烈的內在體驗,那是一種我們不曾認識的境界。內心裡是一陣特別的暖意、喜樂,一種心馳神往。當這種相當愉悅、強烈,又某種程度上刺麻疼痛的能量從體內中央向上移升時,內在響起了一把聲音,更趨強烈地不斷重複對我說著:「我們正在途中,現在真的在來著的途中了。」從漢堡市開始,這一把聲音便一直反覆在心中縈繞。如今我們與目標已如此接近,更是感到無比心焦如焚。於是就在一領到簽證後,我們便迫不及待地登上火車,直奔尼泊爾。

從德里至羅克索(Raxaul)的旅程是我們搭乘印度東北線火車的初體驗。羅克索是人們進入尼泊爾邊境城市比爾根傑(Birgunj),再從那裡搭巴士前往加德滿都的一個城鎮。不曾親身體驗過的人大概也無法想像。火車每開到一站,便會有數以百計身穿白衣的印度同胞高聲呼叫,然後奮力讓自己擠進車廂裡。車廂裡擁擠得人們都近乎跌落窗外,就連厢頂也同樣擠滿了人。然而神奇的是,在如此酷熱的天氣,人潮間不斷衝撞的情況下,竟然鮮少發生任何暴力行為。一開始時人們總是在高聲呼喊叫囂(這是標準程序的其中一部分),可是一旦火車開始行駛,人們的心情似乎又會緩和下來,直至火車抵達下一站,另一波的人潮推擠開始為止。

在印度眾多省份當中,比哈省(Bihar)算是受到自然災害的打擊最為嚴重的一個地方。年復一年,當地的莊稼都深受洪水或旱災所摧毀。就連當地居民的精神層面,其實也相當匱乏。他們甚至相信已過世的母親的靈魂也會處處和他們作對,因此他們擁有特別的護身符來抗衡這些「加害」他們的靈體。

我們的旅程漸漸從這一番喧囂的景象躍遷至尼泊爾美麗的山麓丘陵。我們彷彿跨入了一個居住著無數山民的「另一個」世界。早在巴特那(Patna)乘船橫渡恆河的時候,我們就看見了第一個體格瘦小卻扎實的尼泊爾人,他們通常是正在回家路上的廓爾喀軍人。他們就像是處於一片喧囂和騷擾的海洋中一座寧靜的島嶼。在他們身上,我們能夠感覺到那迷人的一體性。當印度平原與其困惑的人民逐漸消失在背後,四周圍開始由綠意盎然的小山所取代時,有一種寧靜的感覺漸漸滋長,隨即將我們緊緊包圍。這一個地方的人民讓我們感到安心、自在。即便是女性也可以自然又自由地獨自四處走動,不必擔心會陷入危險。這裡的女性會直視你的臉龐說笑嘻鬧。和之前我們途經穆斯林國家時所遇見那些裹著圍巾、矇著臉龐,活像是個「帳篷」般的婦女,又或者老是躲避任何形式上的接觸的印度婦女相比,這裡的女性多了一種難得的開放。儘管曾經有許多人告訴我們,與尼泊爾人進行交易的話要當心受騙,一開始時我們倒也不以為意。畢竟也只是幾毛錢的問題而已。不過說實在的,我們也不懂得如何去分辨那些小銅錢和鋁硬幣的價值。這些硬幣的兩面只刻有幸運符號,沒有數字。後來,當我們發現在當地售賣的火柴盒都是密封包裝以防遭盜時,才真的提醒我們應該更為審慎地處理身上的錢財。隨著我們漸漸懂得如何更聰明地和他們進行交易或打交道,我們和當地的人民才取得了更密切的接觸。這也是他們第一次如此認真地對待我們,不把我們當一般遊客耍得團團轉。

入境印度時,雨季已經開始了。從比爾根傑到加德滿都,中間有好幾個地方的道路都已經被雨水沖走了。在喜馬拉雅山區一帶,這些道路的年度修復工作就是當地人僅可確定的工作來源,所以這也是為甚麼他們不願意把道路的修補工作做得太好的原因。為保萬無一失,他們還會將大石留在山腰上的關鍵地點,然後選在「合適的時機」讓它們滾落下山,確保更多的收入。他們一天可賺取的最高收入只有半美元,因此對於像奢華的悠長假期這種想法並不太能夠接受。

我們所乘坐的巴士非常老舊,還會發出許多噪音。就像全世界任何一種交通工具一樣,這台老巴士也是滿員超載。我們乘著巴士,用心感受著旅途中的一切,顛簸搖晃地緩緩上山。有一件事我們挺欣賞,並且很快就適應,那就是適逢面對任何險境時(這一路上暗藏的危機還不少),無論是司機先生或者是乘客本身都會豁然大笑(或會心一笑)。我們都明白這是試圖消除內心裡的恐懼,卻又讓自己不失顏面的做法。這種智慧,確實讓我們留下了非常深刻的印象。就在黃昏降臨暮色蒼茫之際,四周圍的景色更顯超凡脫俗。巴士正緩緩駛入這一個受到諸佛菩薩所加被的美麗國度。

舊時的加德滿都非常美好。尼泊爾只是早幾年前才開始對外開放,因此遊客並不多見。嬉皮士潮流也尚未登陸此地。若在印度區不小心撞倒一頭牛,可會為自己招來危險。如果無法「證明」(一般是通過適地的捐錢)這一隻對印度教徒來說等同是神明的動物是自己決定在車前自殺的話,那麼肇事者就可要倒大霉了。我們在加德滿都看見兩、三輛車身非常堅固的老舊計程車,其中有一輛還是不必擔心維修問題的舊款沃爾沃(Volvo)。現在加德滿都街上那些車身被撞凹損毀的計程車都是日產的達特桑(Datsun)和豐田(Toyota)。

當地司機為了省汽油,所以常常會以高速檔開車,而且不時鳴響車笛,在狹窄的巷弄裡竄行。這種情形確實也展現了我們在文化上的差異。像是當地的苦力、手推車、騎著自行車或路上的行人都會非常有耐心的讓路,可是幾乎所有的遊客都會被這些噪音所激怒。即便到了今天仍然沒有改變的是,如果不想成為無數外來微生物的寄主,在當地還是要注意只喝煮過的水,只吃自己剝皮的水果。大多數的傍晚,文化豐富的尼瓦利人(Newari)都會聚集在寺廟裡,念誦著佛教古老的禪修法本。他們也以手搖手風琴的音聲,以及聲音尖細的敲擊鼓作為配樂。這一種音樂很直接地便能觸動人的心靈,隨即留下巨大的衝擊。走在一道一道具有歷史的街道上,人人的心靈似乎都能夠獲得寧靜,心裡所思所想都必然將會瞬即實現。就好比當我們想要去探望朋友時,總會在路上和他們碰個正著,然後聽他們說出像:「我正要去找你」這樣的話。有時甚至未有時間想清楚該如何明確地許下願望,所祈求的事物便已經掌握在手中。在這一個地方,心想即成的喜悅不曾間斷。這裡建有許多具有上百年歷史的佛教寺院,流傳著佛教瑜伽士們不間斷的傳承,這一切大概就是讓人們心想即成的箇中原因。祂們所創造的神聖能量場遍蓋加德滿都谷地,縮小了人們二元對立的習慣性與分別心,因此只要心想便靈。然而,品質最純、效果最強勁的哈希什大麻,也可在如此神聖的佛教國度,從授權商家處合法購得。我們悠閒地在店家樓上試貨,商家的女兒還會給我們奉上茶水招待,順便和我們聊聊對於下一次收成的期望。今天,人們所購得的大麻多為低檔貨,混雜了其他物質。再加上標榜著道德主義的美國人用錢打發尼泊爾政府,如今兜售大麻在尼泊爾可是受到禁止的。至少官方說法上是如此。

六十年代後期,加德滿都出現了許多有趣的怪人始祖,就像「八指艾迪」,那一個在小屋茶館創造了許多神秘語言的亞美尼亞人。他對加德滿都許多蓄著長髮,希望「有所作為」的人具有很大的影響力,而且不用多久屬於他的怪胎党便如雨後春筍般紛紛湧現,只要憑著他們的小鬍子和肢體動作便能夠輕易將他們認出來。當時也有一位吸大麻吸得有點亢奮,被稱為「Uncle」(安哥)的印度人。安哥四處分送加入了牽牛花提取物所改良過的哈希什大麻餅。許多「饞嘴的」在嚐過了之後都會飄然個好幾天。當然,就在拐角處還有一名醫生非常樂意提供應急注射的服務。每當他在尋找靜脈要進行注射時,他都會贊揚自己所使用的器具絕對是經過妥善的消毒,而且還不時建議其客戶在嚐試過中國的海洛因後,下次應該要試試他的緬甸可卡因。據說這一種可卡因的品質極為特殊。他從未給任何人在一天之內多過一次的注射,宣稱是為了不讓人上癮。然而,每每在清晨遇見他的客戶,他們總是看起來行色匆匆。我們試圖停下腳步和他們聊幾句話,可是他們卻總是迫不及待想要離開,看起來就是十足癮君子的模樣。也許醫生以為自己只是為他們提供了一點日常的小樂子,像去一趟電影院消遣般,殊不知他們都已經上了癮。要麼市場上提供這種服務的人不只是醫生一人,否則就是這些人其實遠比他所想像的更為脆弱。

回想當年,彼得‧賽樂(Peter
Seller)的電影在尼泊爾是受到禁止的,原因是他和尼泊爾國王長得太相像的關係。這裡時常舉辦各種各樣的宗教節日和遊行。有一次,整座城鎮發起罷工,某個德士司機遭到殺害的事件震驚了所有人。當地的居民對於日常生活上你瞞我詐的事情頗習以為常,甚至政治殺害事件也不是甚麼新鮮事兒,可是當地卻從未發生過任何持械搶劫(為了金錢公然殺人!)的事件。在大多數的日子裡,這裡的人們總是擁有不少歡慶的理由,在街上的某處總能夠發現有一個人人戴著花環,伴奏著音樂的遊行隊伍,在某個街角和另一群抬著擔架將死者送往河邊焚化廠的人擦肩而過。焚化廠就在屠場隔壁,在屠場裡工作的工人主要是穆斯林。他們在亞洲許多地方也是從事屠宰的工作,按照他們的宗教信仰,屠殺動物並沒錯。

加德滿都的房子都很小,身材中等的北歐人如漢娜和我來到尼泊爾後就成了巨人。我們在老城區大多數的房子內都無法站直身子,很多時候就連伸展一下身體也不可。在這裡房子的窗戶上都沒有裝上玻璃,只以漂亮的木雕裝飾;牆壁和天花板通常會被開放式壁爐的煙燻得一片漆黑,就只有泥土地板永遠看起來像是嶄新的一樣。那是因為地板一旦變髒或壞損,他們就會直接在舊泥土地板上重新鋪上一層薄薄的新地板。

在加德滿都舊區的每一處,甚至是毗鄰的城鎮如巴丹(Patan)和巴德崗(Bhatgaon),我們被許多美麗的景色和歷史所包圍,感覺就像是生活在一個令人感到震撼的藝術展覽之中一樣。當地幾乎處處可見的舍利佛塔更是出眾奪目。這些依據特別比例所建造的佛教舍利塔,如同寺廟裡所見種種圖像一般,都是為了開啟一個人內在的佛性而建。如果不畏懼街上撲鼻難聞的氣味、乞丐和病人,那麼這些舍利佛塔留在心靈裡的烙印,必定能夠在未來將你引領至證悟與喜樂的境界。

早在我們之前已有許多丹麥人來到了極具色彩的加德滿都,那些年有很多有趣的人和他們的事蹟都被寫進一些人的博士論文裡。我在這裡為大家舉一個例子。這一件事和毒販老手尼爾斯有關。奉行印度教的尼爾斯擁有和苦行僧無異的裝束打扮。他身穿淺褐色僧袍,帶著高過頭部的三叉鐵戟。有一天,就在帕斯帕提那(Pashupathinath)──尼泊爾最神聖的印度教聖地,有一群苦行僧發現了他這一個「舶來」的「同類」。他們大喊了一聲:「美國人!」,便衝過去將他大打一頓,然後就把他攆走了。隔天在加德滿都的市場內,出現了一套以低廉的價格出售的苦行僧袍、三叉戟和托缽。同時,有人瞧見尼爾斯穿著球鞋和牛仔褲,正準備著要體驗另一趟的「迷幻之旅」。

讀過《西藏瑜伽與秘密的教義》這一本書後,我們當然很渴望能夠親近喇嘛。而我們在加德滿都所見到的第一個「喇嘛」,倒是有一點兒讓我們大吃一驚。這位喇嘛叫做欽尼喇嘛(「欽尼」在當地方言即是「華人」的意思),就住在一棟粉紅色的房子裡,若從陽台上眺望,可見博德納大佛塔的入口處。由於他所提供的美元兌換率是銀行的兩倍,因此朋友老早就建議我們去找他換錢。我們注意到在他家門前停放著好幾輛賓士豪華轎車,不過當時對此倒也沒有多在意。直到後來,我們才聽說原來他的「喇嘛」頭銜是繼承而得,而他的職責就是作為博德納大佛塔的管理員。看來這一份工作他似乎是頗為得心應手,而且絕對是有賺沒虧。

欽尼喇嘛的個子不高,身材倒是頗為粗壯魁偉。他頂著光頭,五官面貌像極了華人。當我們進入房裡時,他看來似乎是剛睡醒。他慢慢坐直了身子,而我們詫異地發現他的兩隻手臂上各自戴了至少五隻手錶,從手腕倒手肘上部一字排開。雖然看起來很可笑,可是我們卻天真地在想,也許他是利用這些手錶來進行非常複雜的調息練習。他開口第一句話,就是問我們是否販賣槍械,我們也天真地心想,也許眼前這一個慈悲的男人正為康巴人(西藏的自由戰士)伸張正義呢!我們支持正義。結果我只能悔恨地告訴他,除了隨身攜帶的刀子以外,我們並沒有其他武器。他之後又想要知道我們是否擁有磁帶錄音機。那倒是有一台,而且是菲利浦的第一款型號。當時盒式磁帶錄音機在市面上才剛新鮮出爐,難得千里迢迢帶著這台錄音機踏上此趟向東之旅,而且迄今仍然運作正常,所以告訴欽尼喇嘛我們要先用它錄下藏傳寺院裡的一些音樂。當時我們相當肯定自己對佛教的這一份心意將會令他感到無比歡喜,不過若真為如此,他倒是隱藏得很好。

當他發現沒有甚麼東西可向我們購買後,反而想要賣鴉片給我們。我們對眼前這位「聖人」的單純感到十分訝異,執意向他解釋說鴉片會讓人上癮,十分不健康,他不應該隨便向人兜售。鴉片正慢慢地奪取了身邊許多朋友的性命。他此時所說的話卻給了我們一個錯覺,誤以為東方國度裡所有「聖人」都有服食某些毒品。這一個錯覺更是一直跟隨了我們許多年。欽尼喇嘛最後以特貴的價格把哈希什大麻賣給我們,不過也以極好的匯率兌換了我們的美金。

當天傍晚,我們和朋友們在城中某家茶館想試試新進的大麻,卻差點兒沒有笑破肚皮。欽尼喇嘛以三美元的高價賣給我們的「上等」哈希什,其實只是抹了黑鞋油的木炭。這一個教訓告訴我們,即便是最有名的商家,購買前也得先驗貨。

在加德滿都有很多可看、可做和可以體驗的事物。可是,城鎮外圍某座小山上的斯瓦彥布寺院(Swayambhu)卻尤其吸引我們。若要繞最短的捷徑而行,抵達斯瓦彥布之前會先經過一堆剛遭屠宰的動物的頭骨,以及一群看得出來吃得好可是卻又很兇猛的野狗。這一路上所途經的地區相當齷齪,而且沒有人會停在附近的茶館歇息(這些茶館大概是從附近的河流打水的吧?)。其實,不管是這些令人不快的因素,抑或是這一個城鎮本身的超凡魅力,對我們來說都不重要。我們每天只是一心想要往山上走去,那裡似乎有甚麼不斷地牽引我們前往。

從加德滿都放眼遠眺,斯瓦彥布就矗立在一座金字塔狀的山丘上。通往寺院的路有三條,都是一些陡峭的階梯,沿途有許多大佛像和無數的舍利佛塔。舍利佛塔的形相表徵成佛時所證得的五種智慧,這五種智慧乃一切有情眾生本俱的智慧。山上的中央圓塔傳說源自於飲光佛的時代(梵文:Mahakasha,迦葉)。祂是在釋迦牟尼佛之前降世的第三尊佛。這一棟外觀宏偉的建築是偉大的西藏瑜伽士嘉華噶瑪巴與其主要的傳承持有者昆吉夏瑪巴在尼泊爾的主要道場。後來噶瑪巴與夏瑪巴更成為了我們非常重要的兩位上師。

圓錐形的大佛塔周圍是一座環狀的院子。熙熙攘攘的訪客總是順時針繞塔而行,如果不嫌棄轉經筒油孳孳的人也可以伸手推動轉經筒。經過數座磚房後只見有一座灰白色的藏傳寺院,其莊嚴宏偉的中央入口直接通往佛殿裡的供養壇。

尼泊爾語「斯瓦彥布」意即「自生」之處。對於這一個地區的佛教徒而言,它就好比印度的菩提迦耶(當初釋迦牟尼佛就是在此地證得圓滿菩提)般神聖。由於在印度,當地人老是纏著遊客不放,因此大多數的人在尼泊爾會感到較為自在。此外,這一個地區的能量場非常殊勝和強烈,不僅利於禪修,亦能廣澤大眾,迅速滿一切願。

我們對此地的第一印象其實並不太好。我們既不喜歡此地的灰塵僕僕(就像在尼泊爾大多數地方一樣,想要找一個可以坐下來的地方也難),也不喜歡那一些假穿僧袍混進人群中,想要拉客兜售物品的騙子。四周也不時傳來猴子憤怒的尖叫聲,它們總是爭吵不休。眼看那些長得又大又壯的猴子欺侮較弱小的猴子,咬它們或搶食物時,心裡很不是滋味。生病的老狗會自行來到佛塔前嚥下最後一口氣,想必是受到此地的能量所牽引,這些可憐的小東西啊!儘管眼睛、耳朵、鼻子不斷遭到這種種景象所衝擊,我們卻感覺自己似乎非常了解這一個地方。雖然我們在尼泊爾擁有不少有趣的朋友,可是我們鮮少在天黑以前就離開斯瓦彥布。

寺院內的莊嚴報身佛像散發著一種巨大的能量。我們經常感覺佛像似乎會呼吸似的,而身穿紅袍的僧人們所唱誦的經文和法器交織而成的樂音,也會不知覺讓人著迷。直到很久以後,我們才了解箇中靈性與秘密的含意,不過即便不懂,也無礙於我們感受它的力量。

第一次來到斯瓦彥布時,有兩位僧人讓我們留下了深刻的印象。其中一位就是薩曲仁波切(Sabchu
Rinpoche),他後來成為了我們「寂靜」和「忿怒」出入息法的導師。當時,他是寺院裡的住持喇嘛,若干年後當仁波切圓寂時,他展示了其證量。據說仁波切圓寂後一個星期,他仍然保持著禪定的坐姿,他的體內及周圍的能量凝結成無數像珍珠和寶石般彩色的舍利子。

當時,我們對仁波切的成就一無所知,可是只要見到他,便會感到很振奮和高興。有一次,我很隨興地跟隨他繞著佛塔周匝而行,他突然轉過身,凝視著我的雙眼,然後用藏文對我說了一些話。當然,我是完全沒有聽懂他的意思。我能夠感受到他的慈愛,可是卻同時當著他的面,突然爆笑起來。就近距離來看,他和丹麥某個很有名氣卻非常搞笑的體育記者長得真像!

有一個叫彭厝的年輕出家人送了我們一串「瑪拉」。「瑪拉」即是串有一百零八顆珠子的念珠,表徵證悟之前的八識心法,以及證悟後所顯現的如來百德。他也送了我們一小張照片,照片上是一個魁梧莊嚴的男性,手持著黑寶冠頂戴在頭上。這位年輕的出家人彭厝還不斷重複地說:「噶瑪巴、噶瑪巴」,彷彿正試圖告訴我們照片中的人究竟有多重要似的。

在那個時候,若要前往尼泊爾第二大城市博克拉(Pokhara),就只能依賴那一台老舊的轟炸機。博克拉位於加德滿都以西的山谷地帶,地方機場就是一片草地。當地最常見的交通工具是租用的馬匹。由於我們無論如何也無法像他們那樣抽打這些馬匹(可是若不打,它們就真的是一動也不動),結果到最後我們無論走到哪兒都得硬拉著它們一塊走。瑞士人在這裡建立了一個規模頗大的西藏難民營。我們很開心見到他們是真的在幫助這些西藏難民學習不同的維生技能,像編織等,讓他們能夠憑著自己的力量養家餬口,而不是向他們兜售西方文化或基督教。然而,這些瑞士人最後還是因為受到中共的壓力而被迫離開。今天博克拉的西藏難民營變成主要靠旅遊業維生。

在尼泊爾待了幾個星期後,我們不只是能夠認出一些熟悉的面孔,就連他們當中幾種不同的民族和文化也能夠分辨出來。雖然對於一開始時所遇到的人們,他們親切的笑臉在我們的眼裡看起來都一樣,可是來自群山另一方的西藏民族卻擁有一種說不出的吸引力。我們總是會熱心地購買他們所兜售的物品,並且會試圖以肢體語言和他們溝通。

其實西藏人非常聰明,雖然這並無法從他們管理國家的方式中看出來。儘管世界已邁入二十世紀,他們卻仍然過著像中世紀時期歐洲般的生活:在中部和南部,他們擁有一個實行階級制度,並且沉湎於排外政策的中央政府;在北部和西部,他們是獨立自主,又溫馴善良的遊牧民族;國土內到處都有寺院林立,各自擁有不同的制度。

大家都害怕東部的康巴劫匪。康巴人樂天爽朗,數世紀以來一直竭力驅逐中國的入侵,是素以驍勇好戰聞名的部落。對康巴人來說,每個人都是自己的國王。儘管各地區的民俗風情、制度不同,他們也不奉行民主、透明制,也不講究人權,然而當一個人遇上了西藏人,清楚看見他們在人際間的交流,以及超乎於異常的彬彬有禮後,這一切種種的差異都顯得不重要了。除了佛教給與了他們深遠的影響之外,他們的幽默感往往也能夠讓所有事情都變得相當合理,讓人感覺自己似乎是置身於一個至善至美的地方似的。中部和西部的藏民性情溫和,擁有像愛斯基摩人般的臉龐;東部的部落則是剛毅堅韌,相貌酷似納瓦霍民族。他們看起來總是輕盈自在,流露出說不出的內在平靜。如此的氣韻,想必是和他們對於整個宇宙所持有的獨特見解有關。

就物質層面而言,他們當中大多數人仍屬世界上最貧困的難民之一。他們沒有聲稱自己遭到種族迫害,也沒有威脅要發起戰爭或革命,因此一直以來都沒有激起太多的國際注意或援助。今天,有許多西藏孩童死於痢疾,而大多數的成年人則患有肺結核。

莎瑪,是我們投宿的那一家旅館的主人。旅館坐落於城鎮的中心,一晚才五十美分。莎瑪說得一口流利的英語,常常提供我們當地活動的最新消息。他也是我們的乳麋供應商,也常為我們提供大量一般哈希什愛好者都饞嘴的甜奶茶。我們的隔壁房住了一個來自喀什米爾的商人,擁有該地區典型的英迪拉‧甘地型五官。有時候他看見我們喂他的狗兒吃焦糖會感到很吃驚。在這一個大部分父母親沒能力給孩子買糖果的國度,我們的動作確實是頗不尋常。和我們一起一路開車到喀布爾的克萊斯和他的關係特別要好。克萊斯在我們抵達數星期後也飛到了加德滿都。他花了一些時間才把他的甲蟲車賣掉,而且途經巴基斯坦時,有人幾乎把他的耳朵咬斷了一半,所以才姍姍來遲。當時他被迫要到當地的醫院求醫,住院期間也差點兒要了他的命。現在康復了,才能再次動身來到這兒。也許這一個喀什米爾人曾從克萊斯那兒聽說過我有多麼敬愛我的父母和弟弟。反正有一天他來到我們的房間,當時我們正在嘗試某些非常強勁的LSD,他不斷在我們的面前說起了有關我父親的好話而贏得了我的信心。在迷幻藥的作用下,我們聽他敘述那些從西藏來到喀什米爾的喇嘛們的本領,以及許多未曾聽說過的神秘故事,包括喇嘛們如何戰勝印度教祭司的精采故事。我們聽得出神,甚至因為太興奮差點兒從椅子上掉了下來。這些對於西方國家的人來說神秘不已的事蹟,現在能夠親耳聽說,感覺太美妙了!後來,他提及自己隔天會和某個擁有這種神秘力量的婦女見面,說如果我們能夠給她一些錢,她會在山區裡的聖地替我們點燃酥油燈祈福,保佑我們遠離一切障礙和險難。他也建議我們戴上銀製手鐲,說是對肝臟好。我剛經歷了兩次肝炎復發,漢娜則是首次受到肝炎的折磨。他的一番話正好戳中我們的死穴。不管怎樣,我們再也不願意受到肝炎的威脅。

隔天晚上,我們按照原定計畫飛返歐洲。為了躲避印度的海關,我們選擇了一條較迂迴可是卻相對便宜,途經巴基斯坦東、西部的路線。印度機場內的官員膽敢視尼泊爾為較落後的「內陸」地區,我們擔心他們會因為我們在尼泊爾所購買的東西而對我們諸多為難。當時,我們還以為自己所購得的西藏古畫卷和佛像很值錢,殊不知其實我們已經犯下了新手常犯的錯誤,也就是只會傻傻地去注重物品上的那一層銅綠(其實只要將物品埋在煙囪或地底下幾個星期就會產生銅綠的效果),反而沒去注意看手工是否精細。今天,我們擺放在禪修中心裡的各種飾品,沒有一件是第一次遠赴尼泊爾時所帶回來的物品。當時,這一雙因為時常吸大麻而通紅的眼睛,大概也很難挑出甚麼好東西。其實這些畫軸或銅像,非常講究比例和顏色的使用是否正確,否則無法帶來任何具有啟發性的效果。若想要對心性有所助益,正確的比例和顏色非常重要。關於這一點,經書裡都有非常精確的記載。我們唯一擁有上好品質的物品,就是「多噶」(Dolkar)這位女菩薩的畫卷和銅像。「多噶」,也就是打從一開始就以不同的形相顯現在我們面前的白度母。祂曾經在南非的某座山腰上救了我一命。現在,祂又以「度噶」(Dukar),又稱大白傘蓋佛母的形相顯現,護佑著遍布世界各地的我的學生們和愛飆車的我。

在尼泊爾的三個星期是一場美妙的經歷,所有的事情都比所願的來得圓滿。離開之前,我們見到了喀什米爾人所提及的那位婦女,那天她來到我們的旅館。我們和她交談了幾句,她的一雙「佛眼」,和身上所散發著的深沉、平靜的氣質,讓我們留下了一個非常深刻的印象。我們將一個裝有廿十美元的信封交給那位喀什米爾朋友,請他轉交該名婦女。反倒是克萊斯對這個喀什米爾人有所保留。他以為我瘋了,認為我們應該把錢留著到阿富汗多買一些哈希什大麻。那裡的哈希什至少要比尼泊爾的價格便宜一半。我們想趁離開之前購買那一個可用以對抗肝炎的銀製手鐲。我們幾乎搜遍了加德滿都的小商店,可是仍然一無所獲。後來,有人建議我們到西藏村裡去找。雖然時間有限,我們心想在前往機場途中,應該可以繞道去找找看。不到十分鐘的時間,這一個看起來頗具威嚴的西藏婦女拿出了兩隻重達170克,手工精美的純銀手鐲。她當時開價要\$10一隻。當我們用手輕輕觸碰手鐲時,似乎能夠感覺到裡頭一陣陣脈搏式的跳動,就像一種鮮活的能量。類似的經驗,在這麼多年來我們也只是碰過一次。離開尼泊爾之前,我們已經決定了,一定要再回來。

在坎達哈省(Kandahar)短暫的停留期間,我們前往阿里酒店(我們的毒品秘密供應處)欲領等候多時的大麻。之後,我們經由德黑蘭(Teheran)飛往羅馬,並且在當地找到了一輛要遞送至哥本哈根的可租用汽車。約二十個小時後,我們便來到了丹麥邊境。我們先試著空手過境探一探風聲,看看是否有人留訊給我們。儘管有個女孩曾經向警察提起過我們的事,慶幸的是整個過程仍然相當順利。

雖然這是我們第一次前往尼泊爾,而且逗留的時間並不超過一個月,可是內心裡的某個部分卻自此未曾離開。這是一次非常深刻的經歷,也令我們對生命的看法增添了不同的色彩。雖然從外在看來並沒有多大改變,可是我們清楚知道,內心裡已經經歷了一場重大的變化。

就在短短幾天之內,我們的鄰居喬真突然病得非常嚴重。他和我們這一群「毒友」一起住在充滿中世紀風情,風景如畫的哥本哈根克里斯欽港(Christianshavn)堤岸對面的一棟老房子內。喬真在樓梯間昏倒了,臉色黯沉臘黃,雙眼充血呈褐紅色澤,看起來可怕極了。當時他患上了嚴重的肝炎。我把他抬回寓所,在他尿尿的時候扶了他一把。他的一泡尿看起來不但呈棕褐色,而且很濃濁。其實他相當可憐,這是他第二次病情復發。回想起多年以前,我們可說是不打不相識。雖然打成平手,可是後來卻發展出一段深厚的情誼。摩托車和毒品是我們的共同興趣。他也練拳擊,而且在四十二場賽事中都幾乎是大獲全勝。由於相較於其他吸大麻的人,他會需要更多的體力,因此有時候他會為自己注射安非他命來增強體魄。他顯然是曾在某個地方和生了病的人共享了一支針筒,所以現在才會患上肝炎。我想起了喀什米爾人曾經說過,我們的銀製手鐲會對肝臟有利,因此我把自己手上的鐲子戴在了他的手腕上。曾有那麼一瞬間,我停駐在掛在門上的剃鬚鏡前,出神地凝視了一會兒。忽然之間,我感受到有一股巨大的能量向我襲侵而至,就像接通了一股高壓電流似的。我像柱子般佇立著,似乎有一道光芒吸收了所有的想法和體驗。我不確定像這樣的情況究竟持續了多久,不過當我回過神來之後,我只是覺得整個人疲憊不堪,茫然失措。這一切是在我沒有吸食任何毒品的情況下發生,可是它卻比我吞了大劑量的迷幻藥時的經驗還要來得強勁。我只是曾在在後院第一次吸大麻時,有過類似的強烈體驗。我下樓來到街角的某個公共澡堂,在流水下站了好長一段時間,思度著:「那究竟是甚麼?那一道光究竟是從何而來?」

翌日早晨,我和漢娜到喬真家去探望。只見他的臉色和眼睛已經恢復了原來的色澤,他說連排尿也是清澈的。他還告訴我們晚上睡覺時作了許多很深沉的夢,告訴他要遠離針管。這真的是奇蹟啊!一般上受到感染的肝臟需要數月的時間再生,根本不可能以這樣的方式就被治癒好。我們真的感到非常開心!也許能夠解釋喬真突然好轉的理由只有一個,就是那些銀製手鐲真的具有療癒的力量。

正當我們在房內試圖給整件事想出一個合理的解釋時,耳邊忽然傳來了一陣敲門聲。傑特(她大概是我們所認識最不快樂的一個人了)就站在門外。她一看見我們,就只是說了這一句話:「我得肝炎了!」

我們再次覺得不可思議。在這之前我們已經有好幾個月未曾見過傑特。她不可能會知道這裡究竟發生了甚麼事,而且她似乎是受到某種力量的牽引才來到我們這兒。雖然她的情況頗糟,可是就在她戴上了漢娜的手鐲一個星期後,肝炎就被治好了。後來,她還向我們描述自己所看見的「恩典之光」。傑特來自丹麥西岸的漁人區。這裡也是國內仍然保持強烈基督教文化的區域,至於「恩典之光」,當地的人都是這麼說的。

漢娜和我感到非常幸福。通過這種方式去幫助別人,真的為我們帶來了非常不可思議的快樂。之後陸陸續續也出現了幾個療癒的個案。一九六八年的秋天,我們身邊有超過二十名的毒友被完全治好。我們總是感覺到治療的能量已經被激活,偶爾也會浮現出一些莫名的想法,大概是吸收了那些戴過手鐲的人的念流所致。現在沒有甚麼事能夠和這些療癒的事件相提並論了,就連我們時常去的黎巴嫩也開始黯然失色。基於習慣,我們不斷重複做著這一件事,朋友們也對我們抱有不少期望。所有事情似乎不斷把我們推向尼泊爾。就在開始放寒假的第一天(整整一個月都不會有學生來上課!),我們收拾好行囊再次動身往東出發。
