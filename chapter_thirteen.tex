\chapter{家在索納達}

我們從堤斯塔(Tista)坐上一輛開往大吉嶺的吉普車,然後在古木(Ghoom)下車,再轉搭一輛貨車前往索納達。這是我們第一次在索納達找到歸屬感。既然噶瑪巴要我們留下來學習,那麼我們就遵從他的意思。我們儘量把握時間在索納達學習和修行,先從每天兩千次的大禮拜開始,然後再漸漸增加至三千次。除了大禮拜,我們其餘的時間就是在吃飯、參加開示、和歐洲的家人朋友聯繫。當時,寫信是最主要的溝通方式,我們會趁空閒的時間提筆給家人和朋友寫信。由於睡眠時間變短了,於是每天多出了一些時間。在噶瑪巴的加持力之下,再加上大禮拜疏通和矯正了體內的脈輪,我們每天只睡四至五個小時就會自動醒過來。由於室內很冷,所以從大清早開始做大禮拜感覺還不賴,身體會因此而變得暖和起來。修持大禮拜讓我們的身體變成一個可以運用的工具,一切苦痛的精進變成了樂意的精進,這是我們從未料想過的事。

漢娜在修持大禮拜時所展現的智慧讓我留下很深刻的印象。我們總是肩並著肩一起大禮拜,由漢娜負責設定節奏。她很能吃苦,面對如此吃力兼消耗體能的運動,她總是能夠連續進行而不中途喊停。我們會以她所做的大禮拜為計數標準,而我偶爾會加快速度以促使更強烈的體驗。

\genericFigure{figures/hannah-with-nuns.jpg}{漢娜與眾女尼在索納達}

結合了身(動作本身),口(所持誦的咒語)與意(禪定與保持正知)三門,大禮拜是一個全面並具有轉化能力的密乘修持。大禮拜能為行者帶來持久的改變;我這麼多年來一直飲酒、吸毒不斷,積累了不少混濁的能量,因此沒有其他方法能像大禮拜般更有效地消除這些負面的能量和我那自以為是的一套人生觀或怠惰的障礙了。只是我們應該相信古老的智慧,遵從他人所教導的修持方法。我曾經擅自修改修持的方法,在胸口會碰觸地面的位置擺放一個硬物想藉此釋放心輪的束縛,可是卻無其裨益。儘管心間處產生了一些奇妙的感受,我的自作聰明帶來最顯著的效果,便是斷了一根肋骨。我的胸口真的非常疼痛,在進行最後三萬次的大禮拜時,我只靠左手臂支撐著整個身體,若說我是在持誦咒語,其實更常是在痛苦呻吟。通過修持大禮拜,內心裡有許多業習將會圓熟、出現。在這一整個星期,大黑袍金剛護法擋住了一切,然而脆弱無知的我們卻不知那是一種加持。祂就懸浮在前方的虛空中,如如不動。

\genericFigure{figures/gompus-hotel.jpg}{在貢普斯酒店的入口處}

修持大禮拜消除了許多毒品帶給我的影響和多年來的謬見,我在過程中往往會不斷顫抖,其中的痛苦都寫在了臉上。幸好我知道這只是一種淨化的過程,它正在將負面的能量排出體外。我們在大吉嶺的布提亞布斯提寺(Bhutia
Basty
Monastery)完成最後一萬次的大禮拜,平均一天做四千次,就在那充滿加持力的古老蓮師聖像旁。我們非常感恩喇嘛拓殿(Thubten)與洛迪(Lody)的安排,讓我們擁有如此殊勝的機緣在布提亞布斯提寺內完成最後的大禮拜,我們亦非常感恩吉梅醫生及其家人為我們送飯。一九七一年新年當天早晨,我們完成了四加行中的第一加行,亦即111,111次大禮拜。當天我們還泡了很長的溫水澡作為慶祝。

回到索納達後,卡盧仁波切傳授了一個灌頂。那是我們從仁波切處所領受的第一個灌頂。當時仁波切所傳授的是諸佛菩薩慈悲的總集,身呈白色的四臂觀音菩薩灌頂。一五ΟΟ年代,印度人稱觀音菩薩為「阿瓦洛吉蝶濕伐羅」(Avalokiteshvara),西藏人將之翻譯成「千瑞吉」(Chenrezig)。我們擠到仁波切面前,他的臉龐散發著智慧的光芒。仁波切琅琅的誦經聲與手鼓聲無處不在。忽然之間,我能清楚看見一個透明,身呈白色,擁有四隻手臂的菩薩行者就在眼前。那是一種不可思議的加持力,我的雙腳像釘在原地般無法移動半步。我永遠無法忘記那一次的灌頂。

我試圖「改良」大禮拜的修持方式一事惹得這群西藏人捧腹大笑。儘管胸口上一直綁著繃帶,也塗上了傳統的藥酒,可是我的肋骨卻一直不見好轉。不管怎樣,我竟然會面對如此愚蠢的障礙,讓我覺得自己遜斃了!由於我們有意前往卡林邦的星期日市集去看一些西藏文物,所以亦打算藉此機會去嘗試當地某個醫療喇嘛的醫術。

貢普斯酒店的塔希很清楚我們上一次來到卡林邦時所面對的種種苦樂。他對我們曾下榻的地方亦是瞭如指掌,這意味著我們也是他們西藏人閒話家常的對象之一。如今聽說我們有意要完成四加行的修持,他感到非常開心。塔希帶我們上山,為我們引見一名蒙古籍的喇嘛。我們還在途中購買了一些新鮮的糌粑。以前我們老是對上坡路感到非常厭惡,如今每一步走起來卻是如此地輕鬆、歡喜。通過修持大禮拜,我們的身體找到了自己的節奏,開始能順暢自在地運作自如。大禮拜是我們所踏出的很重要的一步。如今我們的身體已是一個聽話的僕役,不再是執拗難纏的主人了。它們不再限制心的自由或讓它變得懈怠。

我們抵達山脊的頂端後便沿著一道石牆走,最後來到了某座寺院的大門;通過大門入內,我們來到一排小屋前。塔希從角落處的那扇門進小屋內,我們尾隨其後。外頭陽光刺眼,與屋內的一片漆黑形成很大的對比。由於我們已經習慣漆黑的環境,因此眼睛很快便適應過來,只見屋內放滿了箱子和袋子,有一位和善的老人示意讓我們進來。我們向他問候,他給與我們摸頂加持。他就像喇嘛傑珠給人加持時一樣,會先將雙手擺放在我們的頭部兩側,然後緩緩向上移動至頭頂的位置。我們以剛買的新鮮糌粑和一些金錢作為供養,然後開始聊起彼此的故事。

就像大部分的蒙古人一樣,這位老喇嘛屬於「善規」格魯派。儘管現代人也許很難相信,這一個名字其實是由他們自己所選。格魯派重視對佛法的如理聞思和周密的組織和制度。他們管理西藏的政權,並且創立了許多容有千名僧侶的大寺院,就像色拉(Sera)、哲蚌(Drepung)與甘丹(Ganden)三大寺院。許多僧侶就在這些大寺院裡學習並通過因明學和辯論藝術的考試。

佛陀留給世間最美好的禮物,是其教誨不僅是能對治問題本身,亦能對治問題的成因。佛法尤其針對造就一切煩惱的根本,亦即「無明」。不管你是選擇修持三大舊傳承(也有人稱之為紅帽傳承)的禪修法門,或是選擇追隨黃帽傳承研習教理,這一千年來它們都保存了佛教最精華的法教。無論你是對佛陀和上師具有強烈的虔誠心,還是處於較慢的聞思修階段,無論是任何根器者都適合修持佛法。

這名蒙古老喇嘛和一般的喇嘛看起來不太一樣。塔希說他差不多有一百歲,可是他看起來完全不像,而且他看起來像是一名瑜伽士多過班智達(學者)。老喇嘛主要修持「斷行法」(Choed),這是一個從根本上斷除我執的法門。行者在大手鼓、金剛鈴和脛骨號的伴奏下,持誦著醉人的樂音,然後通過禪定力將個人的身體佈施給餓鬼及其他有需要的眾生,同時將意念傳送給上師。藏傳佛教的舊傳承有在修持斷行法,其中的經驗更是讓人留下深刻的印象。首先,所祈請的能量將會在禪修者的想像中出現,之後便會具體地顯現出來。行者在掌握此修法後,對於他人所恐懼或會感到生氣的事必能一笑置之。

老喇嘛建議傳授我們此修法,這讓我們覺得有點兒受寵若驚。只是我們必須留下來六個月,和他一起住在當地的墓地裡。然而,我們想要在卡盧仁波切的指導下,先完成四加行的修持。我們在骨子裡也能感受到第一加行所帶來的效果,因此內心裡非常期待繼續修持接下來的三組加行。而且我們心裡清楚知道在向上師請益之前,不應該隨便更換目前所修持的法門,否則日後肯定會出問題。我們感謝他對我們的信心,同時也坦白地告訴他目前我們正在卡盧仁波切的指導下修持著四加行。我們答應說會儘快展開斷行法的修持。離開之前,我突然想起斷了的肋骨,老喇嘛聽我說了傷勢的由來後更是忍不住大笑。其實在胸口觸地的位置擺放硬物的這一個想法來自山下那位華裔瑜伽行者的小冊子,不過老喇嘛似乎未曾聽說過他。不過,塔希卻對這名華裔瑜伽士一點兒也不陌生。原來上次那些難吃的肉餡餃子就是從他的酒店買來的!

老喇嘛用長長的勺子從兩個不同的瓶子舀取了一些灰黃色的粉末,然後用米紙摺成三小包。老喇嘛對著三包粉末念了幾句祈願文後,對我說:「這兩包粉末是讓你分別在明天與後天一大早配以熱水服用,而這包粉末則是在明晚配上溫水服用。」當我服用了老喇嘛的粉末後,胸口便不再感覺疼痛。我們也不曉得把我治好的究竟是他的那些粉末,還是他的加持力。不過搞不好是兩者相輔相成也說不定!

不久之後,我們又再動身前往加爾各答(Calcutta)。大吉嶺的警員已無法再替我們延展簽證,所以現在我們必須拜訪他們的頂頭上司申辦新的簽證。辦事處位於作家大樓(英:Writers
Building,印度西孟加拉邦的政府大樓)。那個看起來受過良好教育的年輕人無疑是擁有很好的福報,不然就是有一個有權有勢的父親在背後撐腰才讓他坐上了這個職位。他很同情我們,我們也很同情他。他答應會為我們辦理大吉嶺區長達半年的有效簽證。我們如常以醫療理由提出居留申請,說漢娜在尼泊爾得了痢疾,如今需要山區清新的空氣來調養身體。這些公務員大概也懷疑我們是傳教士,不過對此似乎並不反感。我知道若我們向他們坦承留下來的真正目的,他們一定會立即取消我們的簽證。他們經常用這一個手段來對付我們其他經驗尚淺的朋友,後來他們也曾試圖要用同樣的手段來對付我們,不過我們很少會讓他們得逞。對他們而言,西藏人是低種姓,齷齪的難民,而且還很可能是中共政府派來的情報員。他們無法接受西方人(他們的秘密偶像)如今會來到他們的國家向這些貧窮至極的難民學習。後來我們的朋友很快便學會了在前往任何官方機構之前會先將身上念珠和金剛繩藏起來,不讓他們有機會刁難。

由於簽證需要十天才能辦好,我們也不想一直對他們施壓以免壞了好事,因此決定在乖乖等待簽證的同時到山區走走。那個位於孟加拉灣加爾各答南部的小鎮普里(Puri)聽說不錯,它是冬天的好去處。無論是遊客或嬉皮士對此地皆讚嘆有加,可是真正吸引我們的原因是因為約瑟夫住在普里。約瑟夫曾經是艾文斯‧溫茲(W.Y.
Evans-Wentz)的僕人,一九二Ο年代在錫金專司承侍。這意味著說當年艾文斯‧溫茲教授獲得某些很重要且只能秘密傳授的法本的第一手(大概也是最好的)翻譯時,約瑟夫也在場。第一次世界大戰結束後不久,喇嘛卡孜‧達瓦桑珠(Lama
Kazi Dawa
Samdub)的四本翻譯《西藏生死書》、《西藏大解脫書》、《西藏大瑜伽師密勒日巴》與《西藏瑜伽與秘密的教義》由牛津大學出版社出版。《西藏瑜伽與秘密的教義》一書更是我們在牢獄裡時的良伴。我們希望這兩個先進的文化能夠做到最好的結合,如今我們終於有機會見到當時在場的見證人。遍布於世界的北歐文明和藏傳佛教在各層面上可說是相輔相成。一直到今天,儘管書中有許多法門必須在修好加行法以及在上師的指示之下方能修持,我們依舊認為這四本書是第一流的翻譯,不過書中的注釋卻相當難以理解。艾文斯‧溫茲教授的博士論文專攻是凱爾特宗教中的德魯伊(Druids),他顯然並不理解佛教。如果將不同宗教的名詞或見地混淆在一起的話,就會令其失去原有的鮮活和能量,這樣對任何人都將會是無其裨益。這兩種宗教無論是在修持方式或目標上都存在著極大的差異。但更重要的是,現今社會必須檢驗哪一個宗教能利益及幫助眾生解脫,否則都應加以遏止。

我們與約瑟夫的會面並不像想像般達到交流的效果。他的孫女戰戰兢兢地打開門讓我們進到屋裡,年老的約瑟夫臥在床上,已經不太能說話。我們有看見艾文斯‧溫茲的大禮帽與拐杖,不過約瑟夫的家人只是一味在談論著他們的鄰居。只因為他們是基督徒,這些信奉印度教的鄰居老是向他們扔石頭,他們的孫女也因此找不到結婚的對象。否則若按照當地大鬍子的標準來看,其孫女的長相也算是相當標青。

我們在普里靠近海邊的一家旅舍渡過了一段美好的時光,我們享受著海水(雖然沒想過那裡的海水會如此渾濁),也吃了許多許多的食物,當作是犒賞自己完成了一萬遍大禮拜的修持。我們也趁此期間將會在第二加行中持誦的百字大明咒默記起來,和當地的嬉皮士打交道收集消息;我們也協助一群人渡過了惡劣的迷幻之旅。他們當中有人開始對佛教感興趣,有意進一步去發掘和了解,後來更來到索納達求法。如今每當我提及噶瑪巴便很自然會真情流露。其實我們能在印度這一個愚人的天堂,而非在家人或朋友們充滿批判的氛圍之下展開如此艱鉅的修行是不錯的。否則,我們開口閉口談論的都是和解脫或靈性有關的課題,可能會給他們造成很大的困擾。佛陀教法的力量已將心維持在一個單一的明覺境界中。此外,重複持誦咒語與佛力的加持亦非常有效。這種種的修行方便能以更快的速度將我們的業習和煩惱淨化,直至只剩下赤裸裸的明覺。而且我可以肯定的是,從現在開始,心靈的開展將會是最自然不過的事,但我亦知道這只是漫長旅程的一個開端而已。

