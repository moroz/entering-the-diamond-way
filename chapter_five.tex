\chapter{黑寶冠喇嘛}

一九六九年十二月廿二日,我們抵達加德滿都。當我們開著小巴大搖大擺地進城時,只見我們的「老朋友」們站滿了新路(New
Road)兩旁。他們向我們喊著:「噶瑪巴來了!噶瑪巴來了!」原來噶瑪巴也在幾分鐘前抵達加德滿都。噶瑪巴是西藏最偉大的禪修大師,也是第一個轉世的瑜伽士。十三年前,我們第一次在加德滿都親見噶瑪巴。

西元一一一Ο年,噶瑪巴是西藏第一個自在轉世的喇嘛,他能夠在每次的投生轉世中自我認證。他也為藏傳佛教的其他傳承尋找轉世者,其中第一世達賴喇嘛便是第四世噶瑪巴弟子的學生。一九五九年,噶瑪巴帶領數百人躲開中共士兵,翻越喜馬拉雅山口逃離西藏,途中沒有任何人傷亡。自此,噶瑪巴便一直竭力在錫金(Sikkim)和不丹(Bhutan)建立道場,當時候這些國家都仍未對外開放。多年後,如今我們終於有機會在這裡親見噶瑪巴。我們是直到最近才發現,長久以來因緣際會已使事情的發展巧妙地各就其位,一層層地串聯出整體的畫面。而我們和噶瑪巴多生多世的師生法緣,恰逢時空的湊合,又再次圓滿地匯合成一了。年幼時不斷出現的深層夢境,在夢裡起伏不平的山巒間奮力擊退中國的士兵,拼死命地保護普通百姓和僧人的性命等種種情境,甚至漢娜自幼很隨興地會像西藏人般輕輕吟唱和起舞,如今這一切都說得通了。

只是當時的我們,並不曉得此種種夢兆似乎與前生的記憶有關。我們只是一心想要找到喇嘛傑珠,對於其他事並不關心。我們每天開車往返喇嘛位於馬哈拉亞坤(Maharajgunj)的住家數次,可是每次都是失望而歸。他們總是說喇嘛和噶瑪巴在一起。我們有許多事要告訴喇嘛,也想要好好地感謝他,不過由於苦無機會私下拜訪,於是便決定前往斯瓦彥布大佛塔。這裡是噶瑪巴居住的地方,喇嘛傑珠當然也會在那裡。

我們來到山腳下看見潮湧的人群就知道山上一定有甚麼特別的事正在進行著。我們從未見過那麼多西藏人同時出動,他們穿上了自家最好的傳統服飾,站在沿路的梯階上。他們不約而同朝山頂的方向仰望,臉上難掩興奮喜悅之情。他們每個人雙手合十於胸前,感恩與虔誠之心流露無遺。山坡上古老的西藏號角響起,聲聲迴旋。忽然之間所有人開始沿著階梯拾級而上,路經剛粉刷過的大佛聖像,朝著山頂上寺院的方向移動。受到號角聲所牽引,我牽起漢娜的手,越過緩緩移動的當地人,直奔山頂。當我們登上頂端後,首先映入眼簾的是大佛塔和金剛杵(表徵如金剛般堅固的證悟),此時號角聲已改成了像雙簧管般較為尖細的聲響。佛塔右側的院子裡擠滿了藏民,他們熱切地朝寺院入口處的方向望去。在半明半暗之中,寺院門口有一名身材魁梧,身著紅、黃外袍的男子坐在一個盒狀的席位上,手持著某個黑色的物體在頭上。由於陽光刺眼,一開始時我們幾乎看不見是甚麼。過了幾分鐘,他將它放下,然後裝入某個盒子內,而此時入口處的鐵柵卻迅速被拉上。所有人像遭雷擊般站著不動片刻後又再次開始移動。他們全都擠向了左側的小門,想要往寶座上的男子靠近。現場的人們像失了控般開始推撞擠壓,小孩開始尖叫,而我則乍然發現自己已經開始著手幫忙維持秩序,攔著年輕體狀者,讓老人和小孩先行通過,保護那一些在蜂擁的人群中容易遭到踩踏,較為「弱勢」的一群。後來這種維持秩序的工作似乎也成為了我的「職責」。像這樣的工作真是一種體力活,對於大多數屬「輕量級」的尼泊爾人以及營養不良的西藏人來說確實不容易。

大概是一個小時後,大多數的人也已經領受過加持,我的任務也結束了。我和漢娜加入最後的隊伍,隨著人潮的推擠,沿著昏暗的短廊緩緩前進。突然之間,就在號角聲下,我們來到了噶瑪巴的面前。當他將手放在我們的頭上給予我們摸頂加持時,我們抬頭仰望。噶瑪巴突然變得像天空般不可思議的浩瀚巨大,整個人呈現金黃色,並且閃耀著光芒。人們推擠著我們移動腳步。噶瑪巴所示現的巨大威力給予了我們極大的震撼,身體不由自主地顫抖。不知覺間,我們來到了身穿紅袍的僧人面前,他們將一條繩子圍在我們的脖子上。我們回到院子前,身體緊靠著鐵柵,思緒遠了。我們只是看見眼前莊嚴的金身大佛,正在給予隊伍中的最後一個人加持,我們知道他是任誰也無法忘記的圓滿證量。噶瑪巴的加持,已經融入我們的生命裡。

在噶瑪巴的隨行喇嘛當中,就以不丹籍醫生吉梅澤旺(Jigme
Tsewang)的英語說得最好。他總是看起來很高興的樣子,是一個在處事上面面俱到的一個人。由於他經常嚼檳榔,因此嘴唇總是紅通通的。當時他替我們翻譯,成為了噶瑪巴和我們中間的橋梁。布塔拉錫米也非常照顧我們。我們那「不斷衰退的經濟狀況」總是讓她很擔心。對於那些欲向我們兜售物品的西藏老婦人,她知道我們總是毫無招架之力,永遠無法說「不」。對於急需錢的人,我們同樣無法和他們討價還價。為了替我們儘量省錢,權宜之下她在老城區的朋友家裡替我們找到了一間小小的廉價出租房。這樣一來至少幫我們每天省下了半美元的住宿費!

我們的行李幾乎是原封不動地擱在房裡。我們從清晨起就一直待在斯瓦彥布,直到深夜大家都睡了才甘願離去。我們必須多親近噶瑪巴,噶瑪巴也表示沒問題。我們是他的第一個西方學生。過了一段時間後,不丹醫生替我們安排與噶瑪巴私下會面,這也是我們向噶瑪巴作出「最殊勝的供養」的一個機會。為了表示我們和噶瑪巴之間的法緣,我們將一個強力的丹麥馬蹄形磁鐵和一小包一千微克、最純淨的LSD(註:香港人俗稱「弗得」的致幻劑)供養給噶瑪巴。直到那個時候,這些純淨的LSD是讓我們接入本性實相和喜樂最有力的工具。噶瑪巴先是端詳我們一會兒,然後請我們吃糖果,笑了。他還讓我們用藏文跟著他說一遍表徵五佛智慧的顏色。離去之前,他將手放在我們的頭頂上,給予我們似無止盡的加持。

當天,滿月皎潔光亮,高掛天際。不丹醫生給我們送來了一個小紙包,說:「裡面裝有所有轉世噶瑪巴的頭髮。」他還說:「我從來不知道有這種東西,法王說要送給你們。」我們心裡非常激動,也非常感動。我們收下了小紙包,回程時我將它放入軍綠色襯衫左邊的口袋裡。走著走著,我忽然感覺到口袋下方的皮膚開始變暖,胸口上有刺痛的感覺。後來,刺痛的感覺漸漸加重,就像有某種東西熾熱地燃燒入體內似的。我掏出小紙包,然後放入胸前右側的口袋裡。此時,右側口袋的皮膚下也出現了同樣燃燒的刺痛感,只是沒那麼強烈。

每天晚上,我們和屋主都會上演同樣的戲碼。這棟房子建成的時候,尼泊爾仍未有現代的門鎖。我們每晚回家的時間,就尼泊爾人的標準來看算是「夜歸」,回到家時大門已經從屋內閂住。不知道為何,我總是得要幾乎把門砸了才會有人來給我們開門。更奇怪的是,那戶人家似乎對這樣的騷動不以為意。可是若他們能夠忍受的話,我們這些寄人籬下的遊客又能說甚麼呢?當天晚上,屋主被我們吵醒了兩次。另一次是我在脫掉口袋裡裝有噶瑪巴頭髮小紙包的襯衫時,因為劇烈的痛楚而大叫了一聲。

翌日早晨,噶瑪巴在博德納大佛塔傳授灌頂。博德納大佛塔可說是世界最大的佛塔。早前在尼泊爾,我們只是見過欽尼喇嘛,不然就是和朋友們派對狂歡。如今我們知道原來也有好幾位轉世得道高僧居於此地,因此決定要前往拜訪。

當天,事前並沒有人公佈噶瑪巴會在甚麼地方;他們從來都不說。只是我們很快便學會了跟隨街上的藏民隊伍。只見他們誠心專注地持誦著咒語,顯然是正在準備自己要迎接某個很重要的東西。這一群民眾所到之處,就是可以找到噶瑪巴的地方。我們在人群輕輕地推擠下緩緩前進,也許是噶瑪巴的加持,又也許是因為我們實在是太渴望親近噶瑪巴的緣故,一抬頭才發現目的地就在前方。

這天,噶瑪巴將三世諸佛證悟的能量傳遞下來。他手持著黑寶冠頂戴在頭上,通過「見即解脫」的方式,讓眾生了悟自心的真實本性。同時,噶瑪巴通過其甚深的禪定力將我們未覺的佛性和其證量銜接在一起。我們和噶瑪巴第一次見面時,他正在進行黑寶冠儀式。儘管在金剛乘佛教裡有許多強而有力的法門,此修法卻特別殊勝不共。數世紀以來,它讓見者立刻明心見性,直至了悟成佛。這也是噶瑪噶舉傳承有別於其他宗派特殊與不共的特徵。

那些年在加德滿都,噶瑪巴每天都會進行黑寶冠儀式。這一種儀式讓見聞者漸次增上地進入甚深明覺與內觀的殊勝禪定。當噶瑪巴和我們的心合而為一時,他已在我們的八識田中種下了未來得證佛果的種子。

每一次的儀軌都會產生不同的效驗和感應,並沒有「特定」的反應模式可被期待。就我個人而言,我的心性體驗是全然卻又充滿戲劇性的:跳傘、賽摩托車、「愛」,包括之前服迷幻藥等等,其他人也許會選擇較細微的體驗方式。雖然這種極端的方式能夠讓人見證心性的圓滿無缺,可是不一定比一般人隨流而來的方式來得更為優越或殊勝。就像在「愛」和禪定能量傳遞的過程中,行者不可存有任何期待。對於結果的期待只會阻礙自心本性的全然展現。一切的顯現都是殊妙的,即使沒有直接的體悟也沒關係。銘印已存於心中,結果遲早不求自得。就我而言,我需要強劇的「藥彈」來粉碎我那堅固的我執,而噶瑪巴就擁有充足的「藥彈」。這也是為何噶瑪巴的存在至今仍然形影不離,為我們帶來源源不絕的意義和喜樂。有的時候在儀軌中,我那凡俗的世界瓦解了,橙光的明淨境界豁然顯現,腦海裡只留下黑寶冠明晰的淨觀。其他時候則有一股能量之流在身體中心往上直衝,強烈得讓我失去意識,處於茫然的狀態長達數個小時。佛陀曾在兩部經典內授記,凡見過黑寶冠者,證悟的烙印將永存不滅,在中陰階段成就。處於中陰階段時,心識不再受到感官所限制,因此便能夠與寶冠的能量場全然匯合。我們將能夠了悟自心的真如本性,證入超越時空的解脫境界。令人驚嘆的是,當曾與噶瑪巴非常親近的我的父母,以及其他曾經參加過黑寶冠儀軌的人往生時,這些不可思議的能量頓時活躍了起來。他們都是在伴有不少明確瑞相的情況下往生。

在古老西藏,許多大喇嘛在儀式結束後只是象徵式地給予少數的人摸頂加持。而今,以難民的身分處於他國,這種傳統也隨著改變了,不過顯然沒有人知道。正因為如此,再加上他們當中有的人認為能夠作為最前幾個領受加持的人就是最殊勝的,因此每當黑寶冠儀軌一結束,他們都會衝向噶瑪巴。雖然眾人並不是出於嫉妒或敵意才會如此擁擠推撞,不過有時候,情況確實是挺危險的。站在前面的人會嘗試阻攔以免被撞倒,而後面的人就只是一股勁兒地向前推擠。所有人都想要領受加持,在數百人,有時甚至是上千人的場面下,真的很可能會亂了套。

博德納的情況尤其混亂。群眾已經近乎失控地胡亂推撞和踩踏。有那麼一瞬間,噶瑪巴看似會示現出忿怒的本尊形相,即諸佛為降伏萬惡時所示現的慍怒形相。由於忿怒形相看起來甚具威脅性,而且四周烈燄如劫火熾盛,外行者常會把它誤以為是魔鬼。此時的我雙手不知何時突然多出了一根竹竿,然后以突如其來的一股力量(顯然是來自於噶瑪巴),用力往前一推──我就這樣擋住了數百蜂擁推擠的人群,引導他們按秩序徐徐向噶瑪巴的寶座移動。隔天早上,我才赫然發現向來堅固的涼鞋已經變了形,全身儼然被輾過似的──每一吋的肌肉都疼痛不已。回到斯瓦彥布後,似乎有甚麼重大的事情已悄悄改變。

我們甫抵寺院便立即被帶到陽台,然後有幾位老喇嘛過來問了我們一些問題,像是我們的背景,特別是和我的大力氣有關的問題。他們問起我們的生肖,以及其他許多當時我們並不明白的事,不過顯然我們如今已經被接受作為噶瑪巴核心團體中的一員。自此,我便被要求充當噶瑪巴的貼身保鑣,為他平定四周圍潮湧的人群。 

數天後,噶瑪巴乘直升機飛往納吉貢巴(Nage
Gompa)。納吉貢巴坐落於加德滿都山谷盡頭的山腰上。祖古鄔金仁波切(Lama
Urgyen
Tulku)與妻子──她也是一名喇嘛[或尼泊爾語稱為「喇米尼」(Lamini)]──以及兩個皆為轉世活佛的兒子住在那裡。仁波切與其子孫被譽為是加德滿都山谷的守護者,他們有的在隆德寺長大和接受訓練。隆德寺是噶瑪巴於一九六一年至一九六五年間,在喜馬拉雅東部一個非常殊勝的地點上所建立。它是噶瑪巴在西藏以外的主寺,亦是當時追隨他一起逃離中共壓迫的人們的棲身之地。

漢娜和我希望能夠親近噶瑪巴,於是我們便乘著巴士前往中國鞋廠,再從那裡步行穿過竹林來到山谷盡頭的西瓦普里山(Shivpuri)山腳,然後沿著狹窄的羊腸小徑上山。漢娜很快便在途中病倒了,發著高燒(這可是獲得淨化的好兆頭啊!)。我因此負責扛著兩個背包上山,後來漢娜的身體變得非常虛弱,我幾乎得連同背包把她一塊兒扛上山。

當我們抵達寺院時,噶瑪巴正在傳授灌頂。我們所坐的位置較以往更靠近噶瑪巴,差不多就在他的正前方。當噶瑪巴將諸佛加持的力量傳授與信眾時,我們能夠真實地感覺到其甚深的禪定力量。此時的他看起來比平時更為巨大,我們倆不約而同認為眼前的噶瑪巴和平時的不同。我們問吉梅醫生,這究竟是噶瑪巴的真實身,還是他的意生身。他認為是真實身,不過以噶瑪巴而言,其實都不重要。為了更具體地說清楚,他告訴我們在前往尼泊爾時途中所發生的一件事。

當他們在孟買時,有一些政府官員堅持要噶瑪巴接受體檢。印度人視西藏人為低種姓民族,他們要求噶瑪巴進行體檢的動機不善。在印度軍隊中,最好的突擊隊和傘兵部隊都是西藏人。不過由於他們總是忙著追隨喇嘛高僧,因此這些印度人都想看看這些轉世高僧究竟能活多久。

當時噶瑪巴有其他事要處理,可是又無法對這一個收留了那麼多西藏難民的國家說「不」。後來體檢的結果頗令人吃驚。X光片顯示噶瑪巴的肺部大得驚人,可是其心臟卻如核桃般大小。他們也在噶瑪巴的尿液裡發現有糖,唾液裡有斑疹傷寒菌。由於有關當局感到無法置信,因此當噶瑪巴在數天後抵達加爾各答時,他們堅持要他接受第二次的檢查。這一次X光片顯示一個如足球般大小的心臟,以及一個非常薄的肺部。他們在他的唾液裡發現了霍亂菌而非斑疹傷寒菌,尿液裡的糖分則消失了。

就在噶瑪巴前往納吉貢巴的前幾天,他接受德國籍醫生費希爾的檢查。這名醫生多年來一直在加德滿都盡心盡力地為窮人服務,真正做到了無私的奉獻。這一次的檢驗結果顯示,噶瑪巴的心臟、尿液和唾液都很正常。

我們正聽得入神,無暇注意山谷上壯麗的景色。此時來了幾位年輕的少女並排著隊準備上樓。她們低著頭,手上拿著一條白色的絲巾。噶瑪巴顯然又要與信眾會面,我們見狀趕便緊混入隊伍中,跟在她們背後,然後溜進室內。噶瑪巴一見到我們便哈哈大笑起來。原來當時他正要為這幾位少女剃度出家,我一個大男人出現在尼姑庵內,女孩兒們可要忙了。噶瑪巴招手示意,讓我們在他身旁坐下。這一次是我們第一次真正有機會和他聊比較久。

他問我們來自哪個國家,我以身邊的朋友、弟弟和我自己作為典型的例子,告訴他有關丹麥人的種種。我大概把我們說得仍像是英勇的維京海盜,而不是如今收斂了不少(也是大部分人已變成)的那種「文明人」。噶瑪巴笑說自己也是個硬漢。他是康巴人,東藏的勇士部落。他在說話的同時也握拳輕輕打在我的肩上。當時我也忘了顧慮他的身份,很自然地也打他的肩膀。不過由於太使力,噶瑪巴差點兒從座位上掉下來,然而他卻只是放聲大笑。我這才意識到自己幹下的好事,不禁覺得汗顏。

他突然問:「你們想從我這兒得到甚麼?」

我清楚聽見自己的回答。我說了一些當時就連我們自己也不理解,更是未曾說過的話。我說:「我們希望成為利益眾生的菩薩。」

在我們所閱讀過的典籍中都有描述大乘佛教以菩薩道作為主要的修行目標。當這些菩薩的內觀覺慧漸次增長時,他們都會竭盡所能去幫助其他眾生。這一種利他的發心確實是很實在、自然。我們最懇切的心願,就是希望能夠成為利樂眾生的菩薩行者。

我的回答似乎讓噶瑪巴感到很欣慰,他送了我們一人一個錫製的佛牌。佛牌背面是一個金剛杵,象徵如如不動的證悟。噶瑪巴將佛牌掛在我們的脖子上,說:「這不是甚麼特別的東西,不過是我的一番心意。」

我們一直戴著噶瑪巴送我們的佛牌,直到幾年前佛牌破了,我們才將其中一個放在嘎烏盒裡,用以給予信眾加持。

翌日早晨,喇嘛傑珠乘著軍用直升機前來接噶瑪巴。我們很開心能夠再次和喇嘛見面。我們一直憶念著喇嘛,也為了先前的事向他道謝。喇嘛總是四處奔波,能夠見到他的機會實在不多。另一方面,我們又覺得自己愧對他,認為自己時常跟隨著噶瑪巴可能會削弱了我們和喇嘛之間的關係。當時我們仍然以西方人這種「自以為是」的眼光來看待我們和上師之間的關係,認為自己很傑出,其他上師必定會嫉妒我們所追隨的上師。當時,漢娜大概也算是一名很優秀的學生,而我則不折不扣是一個好斗又自負的人,就像一個能量不平穩的蒸汽壓路機一樣,會不顧一切將擋路的「障礙」夷為平地。

喇嘛傑珠察覺到我們的心意,不斷鼓勵「面臨兩難」的我們視噶瑪巴為上師,後來內心裡的尷尬感覺就漸漸消退了。儘管我們都經過理性思維的訓練,然而在學會真正的生命功課之前,還是得要面對很多次夢幻破滅的洗鍊。噶瑪巴能量場的加持圓熟了我們,促使我們從自己的上師處,乃至其他的人事物學習到不少的生命功課,可是我們也花了好一段時間才明白到這一點。與生命中這種深切的體驗相比,通過理論去了解佛的全知境界就相對的較為容易。

由於當天無法跟隨噶瑪巴一起前往,因此我們便和其他人一起下山。在噶瑪巴的隨行喇嘛中有四名青少年,初次見面後便對他們產生好感。他們提出了很多問題,可是似乎早就已經知道答案。我們從一塊石頭跳到另一塊石頭,緩緩走下山。憑著他們所懂的幾個英文字,加上我們所學過的幾句尼泊爾話,讓我們一路上溝通得還不錯。

雖然這幾位年輕的僧侶都很特殊,可是當山區人民見到他們時所作出的反應,卻讓我們非常訝異。只見他們連忙將頭上的帽子摘下,十分恭敬地向這幾位年輕人鞠躬行禮。而這幾位年輕人在走過他們的身旁時也會一一伸出手給予他們摸頂加持。當太陽在閃耀之際,人們便看不見月亮與星辰的光輝。我們將心思全放在了噶瑪巴的身上,幾乎不怎麼注意到噶舉實修傳承的另外四塊瑰寶。

雖然封建的西藏社會和轉世制度會對這些珍貴的身世造成某個程度上的傷害,這些年輕的轉世祖古依然是不凡的。在過去的八百年,他們其中一人或數人在歷代噶瑪巴轉世之間,對於噶瑪噶舉傳承法脈的延續,扮演著舉足輕重的角色。最受人所尊崇的是夏瑪仁波切(西藏「善規派」,又稱黃教格魯巴政府曾禁止他的轉世長達兩百年),接下來便是司徒、蔣貢康楚和嘉察仁波切。

這一段期間,有許多來自哥本哈根的朋友前來探訪,他們無不因為噶瑪巴的加持力而感到震撼。雖然他們當中此後能夠在生活上作出重大改變的沒有幾個,他們大多數都情願參加不同的灌頂,卻從未開始進行實修,可是他們仍然從中受益。在他們的內心深處,生起了一種令人動容的大信心。這一種信心能夠吸引諸佛非常明顯的護佑與加持。

在斯瓦彥布,整片山谷陷入一片鬧哄哄的景象。來自世界各地的人潮絡繹不絕,噶瑪巴日以繼夜地為信眾開示法教、加持、治療、聆聽。他總是有不得不去的地方,人們對他總是有不同的請求。可是我們未曾見他拒絕過任何人的要求,或凡事先為自己的利益著想。

在眾多儀軌中,要數「法會」(Puja)最受西方人所歡迎,尤其是祈請護法的法會。在法會中,行者以樂音為所緣進入禪定;法會上所使用的法器,其振動有時深沉得會讓人覺得自己周遭的環境都在搖動。大鼓的節奏、僧眾琅琅的念誦聲與心臟的跳動是一種同步的共鳴,是一種能夠勾攝人心,震撼意識的強烈氛圍。一般在法會上,眾僧侶分成兩排面對面而坐,手持號角、喇叭,或類似雙簧管的法器、大小不一的鼓以及金剛鈴。在伴有樂音的情況下,他們依照前方小桌上的經文,一心不亂地念誦。默念的部份往往會被響亮的鼓、鈴和號角聲所劃破;眾喇嘛與僧人的誦經聲,仿若來自另一個世界,如此超然。整個體驗往往會有一種飄然生起,不確定聲音從何傳來,卻又無處不在。

噶瑪巴和各轉世祖古就坐在盒狀的寶座上,絳紅色的僧袍上另外披了一層金黃色的外袍。出乎意料之外的是,他們並沒有因為整個法會的氛圍而變得拘謹。當法會在進行時,他們聊天、開玩笑,大笑,環顧四周,時而打打哈欠,直到某個片刻,所有人又會變得非常專注,之後又再次恢復輕鬆與隨興的姿態。

我們都會盡量帶朋友去參加法會。當我們溶入法音,心裡的鬱結便會自動鬆綁。久遠的記憶突然浮現,古老的感受又重新生起了,卻又隨著號角的聲音漸漸消退。雖然我們沒人知道加持力、灌頂、誦經等是如何運作,不過全都被它們的一體性和所能感覺到的能量所吸引。

我們懂得一些咒語,而咒語是持誦中很重要的一個部分。我們主要是重複持誦「噶瑪巴千諾」,噶瑪巴曾親自傳授我們這一個咒語。此咒語的效用廣大深遠,意即:「諸佛之事業力,請在我身上盡速彰顯」。我們覺得此咒語讓我們和噶瑪巴匯合為一,令我們有利於眾生,迅速地滿足他人的願望。我們也持誦布塔拉錫米告訴我們的那一個咒語(此咒語是洛本傑珠仁波切所傳授的咒語,當年在監獄裡時,我們有不斷地持誦),以及家喻戶曉的六字大明咒「嗡嘛呢唄彌吽」。這是西藏人為利他故而主要持誦的咒語。此六字大明咒乃為「咒語之王」,持誦以祈請諸佛慈悲之化現觀世音菩薩[藏:千瑞吉(Cherezig),梵:阿瓦洛吉蝶濕伐羅(Avalokitesvara)]。六字大明咒能將六種煩惱轉為菩提,諸煩惱如貪執等,其真如本性其實就是智慧。

嗡(Om):轉化傲慢心和我慢心

嘛(Ma):轉化嫉妒心與敵意

呢(Ni):轉化我執和自利心

唄(Pe):轉化無明與困惑

彌(Me):轉化貪欲

吽(Hung):轉化嗔恨心

單單使用咒語就能讓人對它們的良好效果感到信服。就像法會,它們能夠使內心裡喋喋不休的妄念(內在的噪音)漸漸止息,真正的平靜就會現起。

我們所遇見的種種修持上的方便都可說是常識,除了非常有用,也極具吸引力。然而,有那麼一個本尊,我們卻無法立刻適應和明白,祂就是大黑天金剛護法「瑪哈嘎拉」。隨著念誦名號的聲音越來越大與強烈時,瑪哈嘎拉的加持力也就越來越凝聚。我知道祂就是西藏畫軸和塑像裡那一個呈單身或雙運身相,周圍火燄熾盛的黑藍色力量。祂怒目圓睜,獠牙外露,雙手、四隻或六隻手上持有武器,頸項上掛有人首項鍊,身披虎皮和象皮。瑪哈嘎拉的怖畏凶猛相,讓人很難無視祂的存在。

這種威猛的力量總是讓我難受得想要握緊拳頭。我一直誤以為祂們代表絕對負面的能量,我幾乎想要與之交戰。制約於西方人的典型思維和想法,我對瑪哈嘎拉生起了這一種錯誤的知見,可是與此同時祂的能量卻又如此熟悉,深深地震撼了我。我從不情願地感受到自己與祂的緣分漸漸增長,到後來對祂欲罷不能的喜愛。我在內心裡暗中希望:這不是性格軟弱的一種表現!我知道自己懂得這種純粹、原始的力量,後來噶瑪巴除了以「戰士「(康巴)和「佛法將軍」的名字之外,也經常以「瑪哈嘎拉」來稱呼我。

藏歷新年落在每年的二月或三月初一,那天噶瑪巴會蒞臨斯瓦彥布和博德納給予加持。斯瓦彥布將會是表演喇嘛舞的地點,而博德納則會舉行盛宴,當地舞者會穿上特別的服飾和面具進行表演。

斯瓦彥布的寺院前擺放著一個瑪哈嘎拉的巨首聖像,聖像上是一個色彩繽紛,條紋瑰麗莊嚴的網狀物。在此年度公開儀式中,過去所有負面的能量會被驅入此網狀物中,然後再將此巨首聖像搬到某一個特殊的地點焚燒,以表徵瑪哈嘎拉已經摧毀了所有負面的能量。一切邪惡的負面能量將會在此盛大的慶典中,以火元素的形式重新回歸,在新的一年為人們帶來浴火鳳凰般的新生和重生。我們幾個人負責將巨首聖像搬到焚燒的地點,當我們拾階而下時,感覺從未如此辛苦過。這一個瑪哈嘎拉巨首聖像真的是沉重極了。我們將聖像翻轉過來,紅紅烈火開始將之吞沒,感覺就像戰勝了一切對我們有害的能量,我對瑪哈嘎拉所表徵的意義似乎有了更深一層的認識。我明白到祂並不負面,反之祂是一股征服萬惡的力量,祂那無窮的力量將能夠摧毀一切煩惱,以及妨礙我們成長的一切外境障礙。雖然外表怖畏凶猛,其本質卻乃為一切諸佛慈悲的化現。今天,瑪哈嘎拉與其能量場是全世界金剛乘佛教背後的大加持力。

尼泊爾的佛塔都是以相似的結構模式建造。儘管這些年已經變得相當商業化,然而在那裡所能體驗的定境和大樂,依然令人心醉神怡。我們也會盡量到噶瑪巴的房裡進行禪修,他總是會允許我們在裡頭待上數個小時,讓我們體會與他同在的感覺。每次見到我們,他都會微笑著說:「很好。」

在他的能量場中,有時來到房裡有要事和他商討的人都不會注意到我們的存在。當我們覺得應該先行迴避的時候,我們就會跑到寺院頂樓的陽台上,即使是在那裡也能感受到噶瑪巴不斷上揚的加持力,如果置身其中,很自然便能直接進入禪定中。

當我們沒有和噶瑪巴在一起時,我們總是和福特、將軍、尼爾斯‧艾貝等老朋友在一些鄉下地區裡閒逛,鬧哄哄地來到山谷裡眾多的聖地。夜晚,我們遊走於吞雲吐霧後的迷幻境界;白天則沿著佛塔周匝繞行。我們很自然地學當地的西藏人伸手去推動轉經輪。有時我們亦會有機會親近喇嘛傑珠。喇嘛本身獨具一種深沉、寧靜的力量,我們很自然地就能領受他的加持。博德納的某個銀匠替我們打造了一個管狀的容器,我們將歷代噶瑪巴的頭髮裝入,漢娜和我輪流一人戴一天。將它戴上時,皮膚仍能感覺到灼熱,觸碰過的人都有同樣的感覺。

就在春天的某個美好的一天,噶瑪巴要離開了。由於前來覲見他的人潮不斷,他已經展延了出發的日期好幾次。那天我們起得特別早,穿過濛濛濃霧上山。我們繞過陽台來到噶瑪巴的寢室,讓我們感到很驚訝的是,欽尼喇嘛已經坐在裡頭了。他看起來一臉惶惑,穿著噶瑪巴剛給他的白袍。直覺告訴我們,那是噶瑪巴要他「棄邪歸正、改過自新」的意思。隔天,欽尼喇嘛的淨罪過程便已經全幅展開。沒過多久,整個尼泊爾都在談論他們。

當噶瑪巴要出發前往機場時,數百人以各種交通工具(從拖拉機到卡車都有)一路尾隨。所有人都因為他的離去而感到非常傷心,可是大部分的人都只是默默地將這種感覺埋藏在心裡,這讓我感到有點不愉快。對此我並沒有一個合乎邏輯的理由。事情就如我們所願般發生了。噶瑪巴在我們的堅持下接受了我們的請求(當時的我們有點兒自我高估),讓我們先跟隨喇嘛傑珠學習。畢竟,每個人都是噶瑪巴的弟子,可是卻沒見過幾個喇嘛傑珠的弟子。此外,我們熱愛在尼泊爾的生活,希望可以繼續留在那裡。儘管噶瑪巴說我們很快就會相聚,可是當下的離別卻猶如截肢般,讓人感到痛徹心扉。

我們在機場見證了自在轉世者深沉的定力和臨在感。藏文「祖古」(Tulku)意即「化身」,指一些擁有一個不是他本人軀體的人。雖然數世紀以來,有不少孩童因為政治原因被選為轉世靈童,然而對於能夠保任其自心本性的明覺者,他們往往會是一個非常優秀且極具啟發性的導師。若不是深受傳統西藏認證制度的影響,許多西方的所謂理想主義者都值得這一個稱號。

在這些小祖古(Tulku)當中,其中有一位看起來也許正值學齡。他披著紅黃色僧袍,一個人坐在機場貴賓室裡的桌上,和那些穿著灰色衣服四處亂跑喧嘩吵鬧的孩童們形成很鮮明的對比。當噶瑪巴和一些政府官員在後室時,我們注意到了一個很奇怪的現象。只見有幾名西藏人和尼泊爾人走到這小男孩面前,放下供養品,然後領受他的加持。他們看起來都很開心,不過卻沒有人給其他小孩任何東西。這讓我覺得很反感。出自於內心裡強烈的「民主意識」,我突然想過去看看這小男孩兒究竟有甚麼特別之處。我這一個西方人,體型本來就較西藏人來得高大,再加上臉上留了五天的鬍子,應該怪嚇人的。我擠著鬼臉,突然跳出來大聲地喊了一聲:「Boo!」想嚇嚇他。我想要看看他的反應。結果他只是一臉淡定若無其事地看著我。他的臉上看不出絲毫的畏懼。當我們的目光交觸時,彼此間生起了純粹的心心相印的默契。這太不可思議了!這孩子很嬌小,我拉漢娜過來,希望能夠領受他的加持。我們站在他的面前,他沒有將手放在我們的頭頂上。反之,他將額頭貼近我們的額頭。我們知道這是表示「接受」和「認可」的意思。此時腦海中盡是一片光明燦爛,卻又不知其所以然,我們找了個地方坐下,試圖消化剛才的體驗。後來我們才聽說,原來他就是本樂仁波切(Ponlop
Rinpoche)的轉世。本樂仁波切是第十五世噶瑪巴七個轉世的兄弟之一,他也是大圓滿(Dzogchen)心意傳承的持有者。

