\chapter{菩薩境地}

這時我們聽說早前申請的錫金簽證已發了下來。如今我們都已學乖,每次在離開錫金時就會重新申請簽證,那麼下一次若有任何重大的活動便能趕得及參加。當時在隆德,我們是唯一的外國人,那一段回憶因此而尤其難忘。

噶瑪巴問及較早前我們按照一些書所進行過的那些高級禪修,我們坐在他的面前,如實一一作答。我們向他描述了當下所感受到的巨大加持力與能量,而且當時的我們對皈依和灌頂是一無所知的。噶瑪巴解釋說即便是如此強烈的體驗,若缺少了他的加持,這些經驗並不會帶來持久的效果,所以此時我們所修持的加行法其實會更為有效。當時也恰好發生了一件事,只是我們在一開始時並無法明白。

且讓我稍微說一下。當時噶瑪巴要我在不看手錶的情況下告訴他時間。我的答案偏離了實際時間十五分鐘,噶瑪巴說若一個人的內在脈輪完全清淨的話,總是能作出完全準確的預測。就在他說話時,我們有留意到他突然翻白眼數次,之後他便一聲不響地站了起來回到房裡,隨即便傳來他的誦經聲,伴隨著金剛鈴和手鼓的聲響。噶瑪巴正在將某個往生者的神識遷轉至淨土。我們不禁在猜測:「究竟是誰往生了呢?」正在四處奔忙的僧侶們和我們一樣震驚。兩個小時後,噶瑪巴仍在房內,而我們從收音機廣播中得知了答案。此時有關不丹國王因心臟病發在奈洛比(Nairobi)去世的消息迅速地流傳開來。不丹國王是隆德寺的大功德主,也是噶瑪巴的弟子。噶瑪巴顯然已經將其神識遷移。

當時的不丹國王曾在英國留學。他採取了那個時代較世俗和物質的見地,忘了自己的根。當然這和他在國外所受的教育有關。當他回國接管不丹政權時,恰好噶瑪巴就在不丹。他以非常傳統的方式提醒國王不要忘記心識的潛能。當地的寺院尋找三名轉世靈童多年,當噶瑪巴列出了這三名轉世靈童身體上的瑞相、家庭狀況、誕生時辰以及其他詳細的資料後,如今這些重大的職責將得以填補。這些轉世靈童住在偏遠的不丹山谷裡。不丹國王根據噶瑪巴提供的線索立刻派人前往尋找,儘管噶瑪巴未曾到過該地區,可是他所提供的資料卻非常準確。此外,這幾名兒童亦經過傳統的甄別試驗,從眾多相似的物品中辨認出前世活佛的遺物。他們也能辨認出幾位舊侍者。不丹國王深受這種不可思議的力量所震撼,後來更歸順成為噶瑪巴的弟子。這是一個明智的選擇。這位不丹國王在生時的生活富裕也充滿意義,如今往生後也獲得解脫。其實有一個很重要的事業挽救了他,也就是成為一個大力護持佛法的大功德主。這位不丹國王利用了自己的影響力與財力支持藏傳佛教三大「舊」傳承的發展,這些教派一般上很少獲得官方方面的贊助。即使是在心靈豐裕的東方國家裡,這種財物上(或世俗)的資助是很重要的。就像在現今的西方國家一樣,若少了教會或社會團體的支持,任何事情都將難以推行。人們必須以世俗的方式獲得靈性上的解脫,然而缺乏布施的慷慨心卻是我們行佛事業最大的瓶頸。有些人因為貪圖免費的心態所以才選擇佛教,他們所得到的加持其實才是最廉價的!

隔天,噶瑪巴帶著五十名喇嘛前往不丹。他必須出席不丹國王的葬禮以及新任國王的登基典禮。噶瑪巴將頭髮完全剃光,當他正在凝聚諸佛殊妙的加持力量時,他的身體一如既往地會看起來較平常更為魁梧龐大,聲音亦非常響亮。當他穿過院子走向吉普車時,我們近乎能看見圍繞著他的那一層保護能量。當他經過身旁時,我不經意觸碰到他的僧袍,此時身體只感覺到一陣酥麻,像觸了電般。

噶瑪巴在離開之前囑咐我們與四名年輕的活佛,亦即昆吉夏瑪巴、大司徒仁波切、嘉察仁波切與蔣貢康楚仁波切一起留在隆德寺。他要我們從眾年輕活佛的大師兄夏瑪仁波切處領受菩薩戒。

這一次恰好除了能讓我們有機會更進一步了解他們之餘,我們也正好能解答他們對西方國家的種種疑問。那些印度人似乎暫時忘了我們的存在。就這麼一次,我們身在隆德卻沒有接到任何電話提及有關簽證的問題。夏瑪仁波切向我們解說菩薩戒,其餘三名仁波切則一起坐在一旁聆聽。憑著他們日漸進步的英語,再加上我們零零落落的藏文和尼泊爾文單字,我們設法將要點都搞清楚。在偌大的佛殿裡,我們能夠和他們的心融合在一起,伴隨著上百名僧侶琅琅的誦經聲,這其中殊妙的體驗,竟久久無法散去。在回程途中,坐在吉普車裡,我們仍能感覺到他們安住在心中,繼續為我們開示廣妙的佛法。這一趟車程真的太美妙了!

「外在」的戒律能幫助眾生防非止惡,釋放內在的能量,以便能在靈性上取得發展。而密戒則是在灌頂或像修持大手印等殊勝法門時所領受,將世俗世界轉化為清淨之佛土。菩薩戒結合了這兩種層面,以「自律」及「利他」為主。凡事若以廣利眾生的角度出發,各個善行之功德將無量增廣,對於痛苦的見地亦會有所改善。對於一切不好的事不再將之視為是一個痛苦的淨化的過程,反之會視之為學習的其中一部份。只有了解痛苦,然後有意識地去體驗這些痛苦才能令智慧增長,日後方會有力去幫助其他陷於類似困境中的眾生。

菩薩戒分為兩大部分。第一個部分和「發心」有關:為了利樂一切眾生而發願成佛;不生起憤怒或嗔恨心;對所有眾生不分親疏遠近,一心為他們著想。這相當於是為人寬容大度的意思。毫無怨言地忍受那些有害的行為只是意味著我們縱容某些特定的人或文化更快趨向於滅亡而已。我們必須探討事物的「成因」,而不是在其「結果」上費心。今天,這意即要幫助窮苦的國家或社會降低生育率,同時普遍提高當地兒童的教育程度。只有具有智慧的慈悲以及影響力廣泛的行動才能帶來真正的成績。這能有效地消除大部分促成爭鬥的原因,沒有鬥爭便不會有痛苦。以善巧的方式消除任何憤怒的成因非常重要,因為沒有其他事物像憤怒的煩惱般更能徹底的摧毀辛苦累積的福報了。若從慈悲和空性的至高動機中墜落,又若將所發生的事以假當真並且作出負面的回應,這並不意味著受戒者完全破功。受戒者仍能清淨內心裡的忿怒,只是應儘快著手以便消溶掉其破壞力。當妒忌心和嗔恨心慢慢消失的時候,你就因此會覺得鬆了一口氣。

第二部分與「實際的行持」有關。受戒者決定能利樂眾生以及能夠讓他們達至證悟的行持、言語和意念,也就是修持六度:

\begin{enumerate}
\item
  \textbf{布施}:這包括身、口、意、財物受用、教育與護持。布施將眾生連結在一起,而且能豐富心靈。這意即不把美好的事物與經歷占為己有,而是大方地分享出去,並視施者、受者和施物皆悉本空。
\item
  \textbf{正確的行持}:這裡我們不用「道德觀」這種說法,因為這是長久以來教會和政府用以控制人民,使一切合理化的一種說法。活著時若沒有被政府逮獲,死後就由教會定罪送入地獄。在此,正確的行持意即善巧地利用一個人的身、口、意,作出會為眾生帶來永久快樂的事,同時避免造成任何傷害和痛苦。
\item
  \textbf{忍辱}:這裡主要的意思是避免生氣。如此一來便能夠避免摧毀過去通過布施和正確的行持所累積的功德。
\item
  \textbf{精進},或樂精進:這有助於將一個人現有的可能性擴大,永遠不會為目前所獲得的小小成績而沾沾自喜,反而會不斷向前邁進。戀人、律師和藝術家都能了解這四種行持的力量。對於不了解證悟的人而言,它們一般上也會被認為是不錯的品質。
\end{enumerate}

\begin{quote}
在這一個系統中,(五)\textbf{禪定}並不意味著是證得自心本性的最高方法。禪定的目標反而是要製造一種距離感,以讓我們能夠抽身出離去選擇生命的喜劇而非悲劇。理論上的內觀會漸漸成為一種全然的體驗,讓人能夠實際利用禪修的體驗於生活中而受益。這種「止」(藏:Shiney,梵:Shamatha)的禪修其實是更多更高的禪修技巧的基礎。
\end{quote}

(六)\textbf{般若智慧}:這不是一種任運的內觀,而是仍然具有概念的思維。如果之前五個是基礎,此種內觀便像是明亮的眼睛,指引我們前進的方向。其本質乃為知道行善是最自然不過的本份,因為主體、客體及所作都是相互緣起的,皆為「一因一果」,「自作自受」。

這六種波羅密多只是一個開端,圓滿了此六種波羅密多後尚有其餘四種不常提及的波羅密多。依據藏文的名相,分別是「Thab」(方便善巧)、「Monlam」(願)、「Thob」(力)與「Yeshe」(智)。此瞬即自生的不二境界讓人能圓滿地進行利生之事業。

昆吉夏瑪巴在一個誦經聲低沉而悠婉的儀式中傳授了菩薩戒。在整個過程中,我和漢娜多次拭淚。ㄧ開始時,守戒一點兒也不容易。我們受戒不久後便碰上了第一個考驗。那些印度人又再次請我們離開錫金。有很長的一段時間,我們偶爾必須甚麼都不想。後來我們才漸漸地明白到菩薩戒是一個無止盡的加持。它使我逐漸退去好鬥、極具侵略性的一面。像不殺、盜或妄語等「外在」律儀,只要受戒者不破戒或希望復戒,那麼此戒只需領受一次即可。然而像「內在」律儀或菩薩戒,若親近擁有這種戒體的上師亦能強化自己的戒體。此外,受戒者在日常生活中亦須對戒律常念不忘。「利樂一切眾生」是殊勝的大願力,沒有甚麼比這更美好的事了!

