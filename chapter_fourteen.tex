\chapter{修持}

我們領了長期簽證後便回到索納達,那裡已有好消息正在等待著我們。兩個月前,我們申請錫金的簽證,就在我們快把這件事遺忘的當兒,我們通過了申請,而且簽證來得正是時候。噶瑪巴即將在隆德傳授瑜伽度母(Dolma
Naljorma)大灌頂。此重要的大灌頂,歷代各個噶瑪巴也只是傳授數次。這一次的灌頂將會延續幾天進行,期間噶瑪巴也會傳授其他灌頂。索納達的一行人大部分已經上山去了。我們尚未來得及卸下行李,又急忙趕往隆德,這一次與我們同行的還有一對在普里認識的澳洲籍夫婦。他們對噶瑪巴一無所知,對於佛教也僅是略為懂得一些。他們在喜馬拉雅西部待過一天後就毅然決定申請簽證留下。一切和噶瑪巴有關的事情再次圓滿展現,各就其位。如今這一對夫妻,男的在協助土著居民的相關組織裡任居高職,女的則支承悉尼附近的一間南傳上座部佛教的禪修中心。

當天傍晚,我們渡過最後一座橋進入錫金境內。我們在抵達岡托克(Gangtok)前的八公里處下車,打算步行上山。這裡的天氣乾燥漸涼,我們沿路抄捷徑前行。此時天色已暗,沿途不時遇見成群躺在路上的牛羊。大部分牛羊的頸項仍栓著繩索。寺院主樓前的院子中亮著幾盞燈,為四處亂轉的人群照明。許多西藏人與大德高僧千里迢迢來到此地參加這一次的大灌頂法會。鄰近的旅館也只剩下睡地板的空房。耶喜說這一次因為亦有許多高僧大德蒞訪的關係,所以吸引了大批人潮。眼看旅館已經客滿,耶喜想必會讓許多前往求宿的人失望了。

\genericFigure{figures/tenga-rinpoche.jpg}{天噶仁波切(Tenga Rinpoche)在隆德寺的「黑寶冠金剛舞」儀式中}
\fullPageFigure{figures/lama-dances.jpg}{隆德寺的喇嘛舞「瑪哈嘎拉護法舞」}
\genericFigure{figures/sangye-nyenpa-rinpoche.jpg}{桑傑年巴仁波切(Sangye Nyenpa Rinpoche)在隆德寺的院子裡表演金剛舞}
\fullPageFigure{figures/mahakala-mask.jpg}{噶瑪噶舉傳承的大護法,「大黑袍金剛護法」的面具}
\fullPageHorizontalFigure{figures/procession-of-protectors.jpg}{瑪哈嘎拉護法舞中的眾護法從隆德寺走出來列隊而行}


翌日早晨,噶瑪巴主法灌頂前行法會。身為西方人的我們很幸運。由於我們是客人,加上我們來自於遙遠的國度,因此他們安排我們坐在最前排的位置,能將噶瑪巴的一舉一動看得清清楚楚。人們可以很具體地感受到噶瑪巴的舉動。伴隨著雙手鼓、脛骨號和金剛鈴,噶瑪巴不斷持咒,進入禪定,他在佛殿周圍建立起一個能量場。昨晚在路上碰到的那些牛羊似乎也來湊熱鬧,緩緩走在會場的通道上,顯然是被正在進行中的事吸引而至。每當有大法會時,山谷裡經常發生動物們用盡辦法掙脫繫繩,然後來到會場共襄盛舉的奇事。這些動物們的主人也都已經放棄去牽它們,因為它們之後又會再次逃脫,反正當法會結束後大部分都會找到回家的路,所以就索性由得它們。

當天晚上,所有人都非常疲憊,然而卻能感受到滿滿的加持力。噶瑪巴給所有出席法會的人分派兩根吉祥草。晚間睡覺時,我們必須將較長的一根豎放在睡墊下,較短的另一根則橫放在枕頭下,形成一「T」字型,而且要注意夢境。當晚所顯現的任何夢境也許具有重大而深遠的意義。

我做了兩個非常深沉且清晰的夢。在第一個夢境中,我和某個丹麥的朋友正在看著一大叠非常漂亮的唐卡。當我看到大黑袍金剛護法的唐卡時,我將它拉到面前,說:「這幅唐卡我要了。」說畢,唐卡瞬即強行溶入心中,消失不見了。在第二個夢境中,某個相貌莊嚴的藏族老婦女不小心從山坡滑落,我及時抓住了她,後來還扶她上山。我記得當時內心裡生起了強烈的慈悲心,山上的風景不可思議的美麗。漢娜也做了一些甚有意思的夢境,隔天早晨醒來,我們都感到非常滿意。

大灌頂法會的正行部分較前行部分更讓人為之印象深刻。瑜伽度母具有一面八臂,身色綠色,呈半忿怒威猛女相。我們無法相信眼前所見,噶瑪巴開始顯化出本尊瑜伽度母的形象特徵和活力。雖然我沒有看見噶瑪巴的身色轉變成綠色,不過無疑能感覺到有一股女性的能量向我們湧來。當噶瑪巴成為一切能量的中心時,幾位轉世活佛亦開始在人群中走動。他們以具有噶瑪巴的禪定加持力量的法器給予每一個人加持,將他的能量與加持力傳授至我們的頭部及雙手。如此一來,證悟的種子便已被植入我們的心識中,我們圓滿地領受了灌頂。日後我們也將能證得與佛的身、口、意、功德與事業同等的證量。金剛乘的灌頂的力量能在眾生往生時將其帶往解脫的境界。當眾生不再受到感官的銘印所影響時,對自性潛能具信者將能消除對我相的執著。接著,各種現象都將具有解脫的法味,了知道一切皆為心的遊戲幻現。如此一來,我們就能免除墮落惡道的因緣。當內觀漸次增上時,能觀、觀和所觀就在剎那間融合為一體,我們便現證佛身。

人在死亡時,感官的銘印無法影響我們。受灌的智慧本尊身將會從我們潛藏的意識中以光和能量的方式湧現。噶瑪巴的黑寶冠就能達到這樣的效果。如果我們與這種超越時空的智慧顯現繼續保持聯繫的話,那麼我們便能進入智慧本尊超越時空的能量場。

在大乘佛教,人會在臨終時祈請無量光佛(阿彌陀佛)的淨土。人間一切的主要動力來自於貪念,若能當下轉化便能帶來立即的效果。擁有死亡意識的人都知道應該要多持誦無量光佛的心咒「嗡阿彌疊瓦舍\textsubscript{利}」。我們亦應從具德上師處領受蓮花部之佛的灌頂。

灌頂法會在隔天傍晚圓滿結束,人們陸陸續續回到居所品嚐法味。此時沒有人說話,大家似乎擁有太多感悟,什麼話語也說不上來。在返回旅館途中,我們巧遇一名年紀看起來和我差不多,相貌清秀的出家人。由於他懂得說英語,又希望能多接觸西方人,因此在接下來的幾天,我們對他的了解又多了一些。他是阿陽祖古(Ayang
Tulku),來自噶舉傳承的其中一個小支派。

儘管三大舊傳承[又有人稱之為「紅帽」傳承]具有臨終時遷移神識的修法,此傳承通常會向大眾傳授,同時檢視眾人是否具有開頂的徵兆。這一種禪修能使意識從肉體離開。意識就在頭頂(髮際八指處)出來,與無量光佛融合為一體。此修法外在的徵兆就是一個小傷口、小膿包或一滴血,在心的能量穿透皮膚之處就會有這些徵兆。臨終時的體驗比修法時更為強烈,但是過程同樣涉及我們的意識從中脈的光管中向上攝出。意識從肉體中飛出,進入不可思議的大樂和覺性的境界中。這種大樂及圓滿的體悟會漸次增長,直到所有二元對立的概念消溶為止。這裡就是無量光佛的能量場,祂的「淨土」。從一九八七至一九九七年間,我曾向世界各地兩萬六千名弟子傳授過這一項修法,那是一個極美妙的過程。如今,我每年大概會傳授破瓦修法十二次左右。

阿陽祖古的夢境亦很殊勝。他在夢境裡遇見無量光佛的化身昆吉夏瑪巴(Kunzig
Shamarpa),並且向他祈請加持和供養水果。他亦告訴我們西藏人詮釋夢境的方式。阿陽祖古所居住的地方位於邦加羅爾(Bangalore)和芒加羅(Mangalore)之間,他和我們談及該地區西藏難民營的情形。我們現在才知道原來有上千來自高原地區的西藏難民如今住在氣候炎熱的印度南部。他們大部分是西藏西部的遊牧民族,在其他藏民逃離西藏七年後才放棄他們的家園、牲畜、帳篷和大部分財物,於一九六六年文革時期來到印度。中共政府聲稱這些溫和的難民是罪犯,施壓印度將他們送遣回國。正因為如此,印度將他們從北部邊界送往南部一座無人居住的森林裡。他們當中有三分之一的人因此而丟了性命,這其中的衝擊實在太大了!生存下來的難民展現了他們的求生技能。他們以驚人的速度從一個寒帶地區的遊牧民族化身為森林裡的農民,他們亦向印度政府證明了一點:只要努力就能培養出優良的莊稼。阿陽祖古大力邀請我們前往拜訪,希望我們能助他一臂之力。我們答應他會保持聯絡,並且會將他的請求銘記於心。

在接下來幾天,噶瑪巴為弟子、信眾傳授了一系列的灌頂。我們也看見某個對學生很嚴格的加拿大籍喇嘛。這是我們第一次遇見西方人的喇嘛,我和漢娜都一致覺得他的行為甚為誇張。然而,噶瑪巴並沒有揭穿他。噶瑪巴將諸佛的喜樂、智慧、慈悲、事業和守護的能量種植在受灌者心中,這一系列能量將會帶來佛果的成就。當我們進入加行修法最後的階段時,這些灌頂其實是一種加持而非修法的開端,它們對後續的禪修會有很大的幫助。顯然這次是噶瑪巴動用了一點政府關係才讓所有人都能在錫金逗留十天參與大法會。我們心裡慶幸不必為了簽證的問題與印度當局爭持不下,省下了不少麻煩。

回到索納達後,我們也是時候展開111,111遍金剛心菩薩[又稱「金剛薩埵」(Vajrasatvva)或「多傑森巴」(Dorje
Sempa)]百字明咒的持誦及禪修。由於在持誦此百字真言時必須字字清晰,所以要完成此加行所需的時間會較大禮拜來得更長。金剛心菩薩乃為一切諸佛淨化能量的總聚,其修法具有巨大的力量。當有害的銘印離開我們的阿賴耶識時,大樂就能增長,最後證悟自心圓滿光明的本性。

世間有器的境界將無法和心的這一種圓滿狀態相比。尚未證悟就是表示有缺陷。怎麼說呢?佛陀解釋說這一種痛苦、遍在的現象是這樣生起的:二元對立的幻象(主體、客體和所作有所分別)從自心裡無始的無明中生起,就如烏雲蒙蔽晴朗的天空。當我們的心執取於空性(其「虛空」,虛偽的「我」),再去體驗其光明本性時,在此虛空中所發生的一切就會被視為是「你」或「他者」。從這一個分化的感受中,基本的煩惱便會生起:對於所喜歡的事物會產生貪執,對於不喜歡的事物會生起嗔怒。人總是很自然便會想要保留一切美好的事物,因此執著便會帶來貪染;嗔怒會帶來嫉妒,因為人人都不希望自己的敵人過得好。最後,無明會產生愚笨和貢高我慢。這都是從我們認為「這些短暫的現象都是真實不虛」的幻象中而生,認為自己比他人優秀,殊不知他者亦是虛妄的假象而已。

佛陀後來宣說貪嗔癡這三種根本煩惱各組合又將延伸出八萬四千種煩惱,導致種種惡行、惡語與惡念。它們會帶來有害的習慣,蒙蔽我們以至於無法體驗到證悟的自心本性。儘管這些障礙感覺起來是如此真實,可是它們基本上就像是鏡子上的塵埃而已,並無自性,所以這些障礙和痛苦都能被清除。為了促成這一點,一切諸佛將祂們淨化的能量化為金剛心菩薩的形式來顯現。金剛心菩薩(又稱金剛薩埵)呈晶瑩剔透之相、慈悲,其如如不動的覺性將能消溶人們潛意識裡的心結。即使沒有投入上述的修持中,單單在臨睡前念誦幾次百字明咒也會是一個睿智的習慣,給予新的一天一個美好的開始。

當我們了解因果、業力的運轉方式以及如何避免惡業所能帶來的苦果後,淨化的過程就會特別有效。以下是我們從喇嘛處所領受的眾多相關開示中的精華,若想更深入了解的話可以參考《The
Way Things Are》一書,同樣是由Blue Dolphin
Publishing(藍海豚出版社)所出版。因果關係的確立可分為「生起」、「結果」和「消溶」三大組四大條件。以下是簡要之概括:

「業力」若要植種於眾生的意識,帶來或樂或苦的結果,必須先符合以下四大條件中至少一項。

(一)我們必須要知道整個處境。

(二)我們必須擁有讓某件事情發生的念頭。

(三)自己作或教他人作。

(四)之後對所作感到滿足。

當其他因緣聚合時,也會增強業力的效果。同樣的,業果也將以四種方式顯現:

(一)死亡後,潛意識中的種子被喚醒。這是在缺乏新的感官體驗之下發生,其中的投射會被體驗為是一種真實的狀況。它們帶來了或樂或恐、或貪或惑、或妒或執的覺受。在不超過七周(四十九天)之內,最強烈的種子(銘印)將會引導我們投生六道中的其中一道。

也許是過了很長的一段時間後,當心識再次和人的肉體相結合時,接著就會產生以下三種情況。

(二)投生到心所吸引的軀體,意即適合它的基因,這也就決定一個人的健康狀況,是否長壽或反之。

(三)所誕生的環境——百分之十在富強的國家或百分之九十在貧困的國家,在市郊或貧民區裡,這裡是說明所誕生的地方狀況或財富差距。若對他人造成性方面的傷害會投生於沙漠地帶;若擁有好福報卻抱有邪見的話,就會投生於西北歐等地。

(四)最後,我們的習慣模式或習氣將會造作新業。比如說,有些人天生很友善,很容易親近;有的控制慾很強,不太友善。

對於捆綁著未證悟的眾生的因果業力具有大概的了解後,我們該如何脫離這一個混亂、無始的桎梏呢?

這裡也同樣涉及四大步驟:

(一)我們必須確信生命中有某些部分是不稱心如意的,譬如我們尚未成佛,我們不斷體驗痛苦,我們傷害了他人,為他人付出犧牲的不夠等等。

(二)我們利用最快速的金剛乗的法門來清除罪業,例如金剛薩埵的禪修。

(三)我們發願不再重犯惡行或惡語,若不小心重犯則再次發願。

(四)儘量多做善行。我們將能感受到喜樂,並從過去所造作的痛苦中獲得解放。

卡盧仁波切經常說:「經過這四大步驟後,不管多麼嚴重的業力都將會被淨化。」他又繼續說道:「只有當你運用它們時才會帶來效果。這就好比一塊香皂,若你只是把它收藏在口袋裡,即使過了一百年後,它也無法將你洗滌乾淨。唯有實修才能實證。」

我們從仁波切處領受了這方面的口訣後,便開始111,111遍百字明咒的修持。我們儘量觀想本尊金剛薩埵就在頭頂上,有如水晶般晶瑩剔透,祈願本尊能將潛藏在意識中的障礙統統清除乾淨。我們重複持誦了幾遍咒語後便已能感受到本尊金剛薩埵的能量開始凝聚,向我們湧來。雖然在過程中生起各種令人不安的感覺,特別是煩躁感,不過我們知道這些都是自心獲得淨化的徵兆。它們只是我們所面對的障礙中的冰山一角而已。如果現在不通過修持將之瓦解,日後恐怕會有更大的問題顯現。我們每天在克里斯大宅的臥床上禪修,窗戶外總是會有流雲路過,靜坐的時間往往一晃便已過了六個小時。雖然知道這一個過程能夠燒毀一切惡果的種子,可是經歷過了十萬遍大禮拜後,我們也花了相當久的時間才讓膝蓋習慣長時間的靜坐。

此具有清淨作用的咒語有效地清除了過去所造作的惡業。甘露的流注使到過去因為無數暴力和毒癮所組塞的脈輪獲得清淨和消溶,所有機械化的反應也有所改變,進而產生ㄧ種像行雲流水般順暢,任運自在的行為模式。在修法進入尾聲時,我們作了很深沉、強烈的夢境。這些夢境就和文本中所提及的如出一轍——與黑色的巨獸搏鬥,喝下白乳,吐出黑色的穢物和液體——這些都是修法成就的徵兆。我們猶如漫步雲端般生起了一種輕安的感覺。
