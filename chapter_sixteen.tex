\chapter{第一次回國}

回到索納達後,我們開始了第三不共加行之獻曼達。在此修持中,行者將七撮彩色米粒放在一金屬盤上,與之前第一、二加行一樣重複修持111,111
遍。觀想這些彩色米粒乃為宇宙中的一切珍寶,行者在前方置放一簡化的皈依境的表徵,然後無有執著誠心作供養。接著,行者再將米撮拭入方巾中,再建立起另一個完美的世界供養出去。在這一個修持中,行者的身口意三門合一,一起運作。其成果並非是概念的遊戲,反而是自我完整性的一種豐化。重複上百次此簡短的供養後,行者將亞洲社會認為最為珍貴的三十七種物品作觀想供養出去。這其中包括大象寶、舞女寶和將軍寶等等。沉醉於這種傳統供養的歡喜樂受中,行者也可將任何現代社會中會讓他人感到快樂的物品,如寶馬摩托車、跑車、ROLO的時尚服飾、藍綠藻、摩登女郎或夜生活等供養出去。

獻曼達是四加行中需時最短的修持,大概二十天便可完成。我們在布提亞布斯提寺的蓮師聖像前完成最後一萬遍的修持。有時候,當我們在巴士上被一些美國人傳教士纏著不放時,便會和他們說起這一座寺院的歷史。我們不希望對方浪費時間,於是會很有禮貌地告訴對方,宗教應被視為是一種良藥,它們唯一的功能就是幫助眾生。因此,對宗教感興趣者應該自行檢視最適合自己的宗教信仰,也應給予他人作出選擇的自由。基本上只要該宗教不會促使其信徒壓制女性或發起所謂的「聖戰」,信仰宗教應是很個人的事情。然而,經驗尚淺的傳教士會相當堅持與急切,怎麼說也不肯退一步,非得要迫使我們使出非常手段來反擊,把他們惹惱才善罷甘休。因此每當我們對他們說出以下這一個有關迫使他人接受某種宗教的故事時,耳根就會頓時清靜下來。

布提亞布斯提寺是為了佛教的護法所興建的一座寺院,最初的建設地點就在目前的地點隔鄰的一座山丘上,也就是在朝拉薩廣場(Chowrastra
Square)和溫德米爾酒店後方,前往該寺時沿途還會看見許多乞丐。大黑天金剛護法瑪哈嘎拉曾在此山頭示現數次,因此人們才將寺院建在那裡。不過後來英國人把寺院拆了,改建在目前的這一個地點上。原因是他們想要在這個能俯瞰尼泊爾、錫金和不丹的絕佳地點上建一座教堂。然而,教堂在竣工不久後就因不明原因塌了。他們再重建一次,接著又塌了。該教堂就這樣屢建屢塌三次。如今這一個地方又再次成為瑪哈嘎拉之地,不過其左邊的那一塊地區已由印度教和一些當地宗教所佔據。後來英國人又在大吉嶺的大街上興建教堂,此昔日的教堂如今亦已變成供人消遣的戲院,由始至終根本就沒有人真正從這座教堂中受益。

噶瑪巴這次將會在不丹逗留很長的時間。前任國王去世後,他要辦的事情不少,而且他剛找到新的轉世活佛,也要到不丹全國各地去傳授灌頂。就在我們剛完成獻曼達的修持時,卡盧仁波切和其侄子亦剛好抵步,我們非常感恩。他們在寺院裡主持了一些儀軌,正好也使用我們在修持曼達時所使用過的米粒。

卡盧仁波切離開之前,就像其他上師般捉弄了他的某個弟子。有些雪巴酒鬼經常來到寺院裡向仁波切投訴。這次他們要投訴天氣:天空遲遲不下雨,莊稼都旱得蔫頭耷腦了。卡盧仁波切非常了解這些人的脾性。這些雪巴酒鬼每到周末就會搖身一變成為共產黨,醉醺醺地在鎮內吵鬧地遊行示威。當他們在街上遇見仁波切時,他們會脫隊跑向仁波切請求他的加持,之後又再回到隊伍中繼續高喊示威口號。波卡祖古(Bokhar
Tulku)天性靈敏,十分善解人意,他原是藏北草原的遊牧民族。波卡祖古自幼便跟隨著卡盧仁波切,由仁波切一手撫育長大。他有大部分的時間都是在閉關禪修中渡過。每當那些雪巴人邀請他去主持法會時,他總是會驚愕失色地一臉慘白。而卡魯仁波切每次卻甚麼都不說,只是掛著一臉牽強的笑容一次又一次派他前往。

大吉嶺地區和雪巴人的家園只不過是隔了幾個山谷的距離。這些雪巴人來到大吉嶺後都變成了一些廉價的政治口號的犧牲者。我們在尼泊爾所遇見的雪巴人個個性情平穩和諧,然而這裡的雪巴人卻個個情緒沮喪,也盡顯壞脾氣。在缺乏了佛教傳統為基礎的情況下,宗教和酒精之間的關係便受到打擾,而他們就傾向於較容易取捨的一方。這一次他們來到寺院向仁波切求雨,仁波切說:「嗯,你們將會得償所願。」通常當仁波切想要給人教訓時,臉上就會出現異乎尋常的平靜表情,所以我們便等著看好戲。果然不出所料!就在幾個小時內,山上忽然烏雲滿布,隨即便下起了滂沱大雨,山上到處開始淹水,就連梯田邊緣多處亦被水沖塌。這些雪巴人冒著傾盆大雨正努力地修補損壞的梯田,實在不難想像在宿醉下與洪水搏鬥該會是多麼折騰的一件事。

噶瑪巴總是有辦法鼓勵和強化人,使人不斷成長。他會把美好的體驗帶給大家,直到愚昧的習氣和業力使他們從喜樂的頂端墮下。之後,他又會另有辦法使他們再次振作起來,直到弟子們了悟到這一切的起伏不外是心性的顯現而已。當我們能跳脫出對自我的執取,專注於為他人帶來永恆的覺悟時,我們自然而然就成為了噶瑪巴的法侶,而非學生了。另一方面,卡盧仁波切則以較個人化的方式趨近學生。他會以較強烈的方式打擊學生的自我。他甚至會催化一些狀況,讓學生們能迅速了解到當下其修行的進展。就如他的學生當中若有人試圖以表面的佛教道德觀來審判或約束其他人時(漢娜和我是極度厭惡這種事,可是有些學生很輕易便會墮入這種陷阱),卡盧仁波切對這件事是了了分明的。當他們尚未真正體驗某些事卻擅自發展出自己較頑固的意見或主張時,卡盧仁波切心裡也是一清二楚的。這時仁波切會以一貫的語調,從一般的傳統教法轉為告訴我們有關「狂慧」或瘋行者竹巴袞列(Drukpa
Kunle)等人的故事。

竹巴袞列(DruKpa
Kunle)是不丹的一名瑜伽行者,也是噶瑪巴的弟子。他曾與不少女弟子發展出不凡的親密關係。有關他的故事已經出現英文譯本,儘管他這種強而有力的說法風格和人格特質難有人可媲比,他可說是證悟事業最具體的一種展現。他的每一個行動都將眾生引向解脫,以一種強悍的方式將人們從有限的妄想和謬見中搖醒。不管人們的修行之道是小乘的逃避、大乘的發心或者是金剛乘的純淨見地,他們都能從他向拉薩大昭寺等身佛的喊話中學習:「你有今天的證悟,因為你心裡只想別人;而我會在這兒,是因為我只想到自己。」

卡盧仁波切亦知道該如何趨近我和漢娜。我們(尤其是我)總是自我感覺超好。對於一些不好的事,我們也能從中找到趣味。在困難的情境中,我們總會有轉身的餘地,並且將一切當作笑話來看待。漢娜的個性和諧,她通常只會因看見他人受苦而感到痛心,而我總是覺得所有的經驗都充滿刺激。對我們倆而言只有一方面是不堪一擊的弱點:也就是愛的這一個部分,我們對彼此的執取。這是我們的家族遺傳,全家人都是這個樣兒!無論是我的父母親抑或是祖父母,他們的關係都非常親近。就像是具有兩個頭的眾生,配偶之間的運作方式就是男人成為女人的力量和安樂的泉源;而女人則是男人的智慧與開放性的搭配。我們能夠感覺彼此一體同心,每一次的分離都會讓我們覺得十分不快。卡盧仁波切當然知道這一點,然而當我們在四月份的結婚三週年紀念日當天開開心心請求他給與加持時,他微笑著說道:「真不錯。接下來的三年,你們可以分別出家當和尚和女尼。這樣對你們都好。」如果他的目的是要激怒我們,那麼他可說是成功了。接下來的幾個小時,我們不得不努力去克制心裡許多不好的想法,否則之後又要花不少工夫去淨化內心。

最後一個不共加行(或稱第四加行)之上師相應法需要上師進一步的指示。這一個修持具有很長的前行準備階段,引介行者逐漸向傳承開放。之後行者將會領受有關智慧方面較深的教導,發成佛之願心,最後祈求諸佛菩薩賜於加持和護念。由於卡盧仁波切不在索納達,我們不希望從其他人處領受這方面的口訣與指示,因此唯有等待仁波切歸來。一切事物就像以往般都是最好的安排。時間並沒有被白白浪費。在等待的這段期間,我們的朋友來到索納達。金‧文斯、尼克和伊華、歐蕾與涵、克勞斯、力奇與荷拉的因緣成熟了,為領受噶瑪巴的加持而來到這一個地方。之前卡盧仁波切要我們去宣揚他的教法,如今我們能明白他的意思了。這時也恰好能和他們一起去拜訪之前一直未有時間前往的一些當地的佛教聖地。當仁波切和嘉千抵達索納達時,所有人還特地跑到小型火車的鐵道那裡去迎接他。同時我和漢娜也發現,我們真的非常敬愛這名老戰士。在領受了仁波切的加持後,我能感覺到他的雙手在頭上良久。

我們這幾個西方人在克里斯大宅像尼泊爾人般生活了太久,狹小的空間內總是擠滿許多人,亦作出太多無意義的談話。當身邊的朋友們即將開始實修時,我和漢娜鬆了一口氣,因為這意味著我們又將能繼續各自的修行。卡盧仁波切在傳授我們觀音菩薩的灌頂後,第一次讓我負責講解觀音菩薩的修持法門,朋友們亦正式踏上修行之路。這對我來說是莫大的榮耀。若一個人要教授金剛乘修法,他必須非常確定其中的細節,同時對於他人而言是一個良好的楷模方可。若他人因為教授者的行為怪異而對修法失去信心,那麼即使是再殊勝的法門也沒有用。

當我和漢娜領受了最後一個加行的口訣和指示後,我們終於可以展開相關的修持。在這一段間歇期間,我們的發心其實變得更為鞏固和強烈。與這群老朋友相伴的日子,我們的心所開展出來的警覺性已被一些慣性的思想、業力和言語所消溶和擊退。我們顯然需要額外的距離,以便不被這些妄習所左右。我們必須這麼做,這是毋容質疑的事。我們深深地了知這些習氣與煩惱的根本就是卡盧仁波切所講的執、嗔、貪、慢種種心毒。這些洞見讓我們覺得「修心」其實就是刻不容緩的一件樂事。

就在我們即將完成四加行的修持之際,索納達周圍的環境也出現了變化。巴基斯坦東部爆發了一場戰爭,導致民不聊生,許多難民因此紛紛湧入山裡。爆生戰爭的地區距離我們的住所只不過是十公里路,儘管雨季時節索納達的天空總是雷聲隆隆難以分辨,不過我可以肯定自己有時確實是聽到了戰場上的砲火聲響。在前往大吉嶺的路上,沿途擠滿了身穿粉紅色長袍的巴基斯坦難民。身處異鄉的他們在寒冷的天氣裡凍得慌,ㄧ雙眼眸也凍得通紅。在前往加爾各答途中,這些難民的身影亦處處可見。鐵道旁已經廢棄多年的水泥造儲水設備,如今已然成為這些難民一家大小的「家」。

我們當中有人主動提議要向這些難民伸出援手,可是似乎不得要領。此時,在我們內心裡也有一種非常強烈的感覺,逗留在索納達的日子即將結束,這一種傳統的學習是時候告一段落。我們亦漸漸意識到一個能讓自己專注於修行而不受打擾的外在環境條件可遇不可求,因此更積極把握在索納達所剩無多的日子。

外事警察已經通知說要找我們談話。在德里的官員已發現我們假借學生之名申請簽證留在限制區裡生活,並未入讀任何印度大學。如今,我們必須選擇離開山區,或離開這一個國家。我們寫了一封友好卻甚無意義的信件,心想按照當地官員的辦事速度,我們仍可享受幾個月「自由的空氣」,不過這一招再也不管用了。大吉嶺那裡有幾名官員也是佛教徒,曾試圖幫忙我們拖延,然而由於在德里的長官已經緊盯著我們,他們也無可奈何。不久後,我們又再次接到要我們選擇離開山區或直接離開印度的通知。這一次我們給他們寄了某醫生開的醫藥證明書,不過我們也心知肚明這拖不了多久。就在我們加緊步伐,希望能趕在離開索納達之前完成加行的修持時發生了一件事,讓我們覺得像是被雷電擊般備受衝擊。某個曾從越戰逃離、神經質的美國人匆匆忙忙跑到我們房裡來,然後大喊說:「卡盧仁波切要去美國。他們已經在收拾行李了!」那是一九七一年的秋天,我們一直以為自己很冷靜,隨時準備好迎接一切,然而這突如其來的消息對我來說真的太震驚了,我直接就跑到洗手間去。這是這的嗎?我們開始感覺自己全身發燙。

消息是真的。現在正是將佛法事業弘揚至西方國家的時機。這位老喇嘛將會前往西方國家傳授密勒日巴尊者和噶瑪巴的傳統法教。他與嘉千以及阿尼措加剛從噶瑪巴那兒聽說他們的不丹護照已經辦好。

在這最後一段紛擾的日子裡,事實證明佛法也能攝化動物眾生。我們在前往大吉嶺途中,看見有一隻大黃狗被一輛吉普車輾過,躺在馬路的中央。我們連忙跑到它身邊,不斷重複為它念誦「嗡阿彌疊瓦舍\textsubscript{利}」,然後將念珠放在它的頭上。大黃狗沒有驚慌,鮮血從它的嘴裡湧出,然後它就斷氣了。我們知道它已解脫,將會投生更好的地方。看見了念咒對小動物的功效,我們的內心除了是喜悅,也感恩。

卡盧仁波切在出發前給弟子們傳授了長壽佛灌頂作為離別的禮物。這一次有許多西藏人前來參與,仁波切以手中的法器給予每一個人加持,為眾人消除種種逆緣與障礙,增福延壽。在整個過程中,我一直感覺右手很燙、很痛。當我攤開手一看,乍然發現手掌上多出了一條非常深刻的手紋。由於不知道這一道手紋意味著甚麼,我們因此將它視為是諸佛大德鍛鍊我們的象徵,為日後將教法弘揚至西方國家作準備。

看著這些藏民依依不捨地與仁波切道別,實在是令人為之動容。他們對上師非常執著,每一次的離別就像是截肢般讓他們感到痛心。法教中不時強調行者應視上師為一面鏡子,反映眾生真實圓滿的佛心。然而,許多人仍然非常執著於上師外在的形相。雖然上師外在顯現的形相會加深我們對加持的覺受,可是同時亦會令我們變得脆弱。當我們的藏族朋友們為了仁波切的離開而感到痛心的同時,我們這幾個西方人卻感到非常振奮。卡盧仁波切即將前往西方國家,我們的朋友將會有機會見到仁波切。這真是太棒了!

這一次卡盧仁波切的幾名西方學生將會與他同行,一起從索納達出發。喜樂塔欽負責支付所有人的費用。兩名加拿大人阿建與英格麗德亦一同隨行。他們剛完成了加行修法。另一對法國夫婦丹尼斯和露斯瑪莉則飛往巴黎為仁波切在當地的活動打點一切。

卡盧仁波切在飛往美國途中拜訪了以色列,也在羅馬與教皇會面,最後抵達巴黎。他在巴黎逗留約三個星期,我們有幾位朋友亦前往拜訪。我的弟弟伯恩在瑞士的農地裡收到我們的來信後,與一群朋友開著他的沃爾沃轎車南下。正當他要開始尋找法會的地點時才愕然自己已經來到大門口。當時仁波切正在傳授觀音菩薩灌頂,他們正恰好趕上。之後,仁波切從巴黎直飛北美,而且還在美國滯留了一年之久。由於仁波切平白無故給了喜樂塔欽性生活方面的建議,後者老羞成怒,拒絕支付仁波切回國的費用。

我們在索納達的日子結束了。德里方面傳來的消息越來越具威脅性,而且我們覺得老是讓當地的警察替我們掩護對他們也很不公平。然而由於各個部門都不知道其他部門的情況,因此我們學會周旋於這些不同部門之間,然後從中取巧。我們下山前往西里古里,多攢了額外兩個星期的時間,恰好可以完成最後的加行。當這兩周的期限快結束時,另一個部門給我們發下了錫金的簽證,我們又開開心心地上山去見噶瑪巴。

這次我們能夠再次回到隆德與朋友們相聚,感覺非常美好。這一個地方已然成為我們的「淨土」。噶瑪巴再次為眾人傳授一系列的灌頂,其中包括勝樂金剛(梵:Chakrasamvara,藏:Pal
Khorlo
Demchok),以及傳承中其他廣被修持的法門。整個過程中,我能感覺有一股強烈的能量湧入,那些西藏人對我的面部表情甚為讚嘆。在他們眼裡看來,這是強烈虔誠心的一種展現。但是往深一層想,這其實意味著我的身體上部的脈輪仍有淨化的空間。不過,我的身體已慢慢停止抖動。在格隆姆帕嫫(Gelongma
Palmo,又稱帕嫫比丘尼)灌頂期間,該名嚴肅的英籍女尼為我們傳授了下一個修法的口訣和指示。那是以第八世噶瑪巴及其殊勝加持力為對象的修法。由於她的藏文不好,再加上她具有很強烈的基督教、印度教和格魯巴學派的背景,後來我們也不得不放棄其中的一些指示。

那些印度官員又再次被我們嚇壞了,尤其是岡托克的達斯先生。他認為我們這一次真的太過分。我們要申請延長簽證,負責的官員必須先經由加爾各答辦事處去處理。我們給對方撥電,不過不用說也知道,對方當然是不肯幫我們延長任何簽證。他甚至出言威脅,我們要是再敢賴在印度不走,他可要派士兵來將我們押走。我在整個通話中的態度非常友善,我這是想要讓對方知道他已經無法激怒我們,不過這一招並無法為我們爭取更多的時間,所以我們只好藉噶瑪巴的幫助,迅速決定去向。我們心裡盤算著這次應該會回到尼泊爾,回到和藹友善的喇嘛傑珠身邊,不過噶瑪巴似乎另有安排。他說:「你們回家去。」我們錯愕不已,問道:「回家?哪兒的家?」噶瑪巴說:「當然是歐洲的家了。」

對我們而言,這是一個巨大的打擊。我們在心中盤算了各種可能性,心想若不是尼泊爾,那麼也許會前往錫蘭(Ceylon),又或者在噶瑪巴的加持下,我們也許還能到不丹去。可是歐洲,我們卻連想也未曾想過。噶瑪巴的字字句句就如冰水從頭頂澆下。當我們的心情平復下來後,回到歐洲的想法其實也頗讓人振奮。我們已準備就緒,如今已獲准在接下來的日子必須自力更生,這將會是我們的一大挑戰。作為第一批學習這些廣妙精深的法教的西方人,如今我們要將這些法教傳遞至一個擁有截然不同的文化的西方國家去,繼續將其發揚光大,這其中必須付出堅持不懈的努力。

噶瑪巴在離開之前送了我們一幅漂亮的唐卡。唐卡上所描繪的是象徵佛心的大悲、大智與大力的聖像。他答應會時常護佑我們,並且喚了一台貨車將我們送往岡托克路,在那裡已有一輛吉普車等候多時。突然之間,我們已踏上歸途,重返歐洲,噶瑪巴的話語仍在耳邊縈繞。他說:「我會一直與你們同在。」我們特地繞道大吉嶺,讓那些警察們能「正式」驅逐我們出境以便洗清「縱容」我們延長逗留的嫌疑,再從那裡乘火車前往德里。我們在一名友善的婦女的家(她只收留斯堪的納維亞人)留宿了一、兩天後,終於收到父母親寄來的電報和一百美元。我們手上仍有五十美元,加上這些錢就足夠我們回程所需的旅費,也就是差不多七十五美元左右。我們趕緊去找雅克,一個魁梧強壯的法國人。他打算開那一台老舊的貝德福巴士回歐洲,只是旅費比我們所預算的超出了許多,一個人需花一百美元左右。然而,雅克卻不介意收下我們僅有的錢作為車資,還為我們提供一日三餐。他其實可以不必虧錢就能找到其他旅客將巴士坐滿,可見他真是一個善良的人。

我們就這樣離開了印度。我和漢娜坐在巴士前窗的位置打坐,也不時和雅克聊天使他保持清醒。雅克的駕駛技術可不是蓋的,他一路鳴響著車笛,載著約五十名嬉皮士在印度和巴基斯坦的大街小巷上,穿越過人群、動物、拉車。阿富汗的夜晚凍得讓人發慌。我們那廉價的睡袋使用了兩年後已經變薄,現在我們終於明白當地人為何會如此喜歡陽光了。某個夜晚特別寒冷,漢娜搬出其餘的衣服蓋在身上,而我差不多每半個小時就會醒過來,凍得直跺腳。正當這寒冷的天氣越來越難熬之際,突然有一群看起來非常乾淨的狗兒從沙漠的邊緣出現。這些狗狗的頭像我們的頭般大,短毛,奶油色,它們來到我們身邊安靜地躺下,我們那虛弱的身體漸漸地暖和起來。我們也為它們念誦咒語和給予它們加持作為報答。雅克仿若不知疲倦地一路開車,每晚就只是睡幾個小時,不過他的食量倒是很驚人。這輛巴士的駕駛盤已經非常老舊,我們看著雅克操控著它,每到轉彎處就似乎更接近歐洲一步。

我們在伊斯坦堡(Istanbul)停留約一、兩天,換不同的乘客,也拜訪了一些曾經熟悉的地方。如今在沒有迷幻藥的作用下,這些地方不再擁有醉人的魅力了。其中最糟糕的不外是古漢酒店(Gulhane
Hotel)。這一個地方曾經是歐亞之間吸毒者的聚合地點。如今這一個地方可不是鬧著玩兒的糟糕。人們簡直是可以直接把它當作是一家醫院,立即開始治療。這裡頭的人幾乎沒有一個不是生病的,最糟糕的肝炎患者,其身色已成暗金綠色。此外,這裡的氛圍也已有所改變。早兩年前的那一種開放與理想的同志之誼已不復存在。如今在土耳其若被發現持有大麻會被抓去坐牢,判刑三十年。西方人幾乎無法在那里的監獄裡生存。因此,所有人都慌張了。最近就有一個美國人被抓。他似乎知道美國政府不會幫他,於是搶了某個警員的配槍,然後瘋狂掃射整個警局裡的人後才中槍斃命。

土耳其和其他回教相關的事物都屬於亞洲的一部分。可是我們卻很難定義保加利亞(Bulgaria)與南斯拉夫(Yugoslavia)。格拉茨(Graz)是一個美麗、充滿文化氣息、自由的地方,歐洲的各種風貌和氛圍都在此地聚合,完美有力地展現。看見這種種景象,我們便知道自己已經抵達。早在一九七一年,我已經有想法要在這一個城鎮裡建立佛教中心。這也是我第一次對一個地方抱有這樣的想法。這一定是一個美好的願望!今天在格拉茨,一些與我關係非常親近的學生正在管理和領導兩間很棒的禪修中心。鎮內的那一間較小型,鎮外的則較大型。我們很開心能夠再次感受歐洲房子的寬敞與潔淨。儘管我們仍乘著那一台老舊的巴士走在前往荷蘭的路上,可是看見高速公路上的汽車都以每小時超過100英哩的速度行駛著,那種感覺實在是太美妙了!內心裡有一種說不出的自由與暢快!在這裡,計程車司機即便是在倒車也比印度大部分正在向前行駛的汽車來得快。感受到一個地方無限的可能性以及其充沛的活力是很美妙的一件事。在一個步伐緩慢的國家待了這麼久的一段時間後,美麗的歐洲又再次瑰麗登場。

我們和特地開車南下的父母親在阿姆斯特丹重逢。我們彼此激動地擁抱,雙眼無法止住喜悅的淚水。我們一家人開開心心地一起開車回到丹麥。漢娜的父母亦看起來氣色不錯;我們大部分的老朋友在幾天內一個接一個出現。消息很快傳了開來。他們認為這金剛乘一定是甚麼非常有效用的東西,才能讓他們這個在過去總是惹麻煩的老朋友歐雷脫胎換骨。儘管我們自己並沒有察覺,如今我們已擁有全然不同的視野,顯然和兩年前離開哥本哈根時已經大不相同。

我們回到歐洲後的首要任務,就是要讓人們認識到過去這兩年來我們在喜馬拉雅山區裡所學習到的寶貴知識。金剛乗之道需以一種超越文化區隔的方式向西方人引介。許多善良的人們由於被一些奇怪的書籍所誤導而嚴重感到混亂,因此一本針對藏傳佛教作出清楚解說的書籍必然能讓他們從中受益。他們必須知道,藏傳佛教不僅僅是局限於一些冗長沉悶的儀軌或短暫的神通力而已;反之,它是結合了實際的心理學和哲理,並且是擁有兩千五百年經驗的一套方法。這一套方法教導我們如何活著、面對死亡、作出更好的投生、創造一個具有意義的生活,甚至是如何利益他人。這些教法必然有它的用處。若要讓人重新認識藏傳佛教,我們手上就擁有很好的資源幫助我們達成目標,那就是卡盧仁波切在索納達的指示和教法,我們一共寫滿了厚厚五本的筆記本。筆記本裡出現很多不斷重複的內容。每當有新加入的學生,卡盧仁波切就會從頭開始教起。不過即便如此,那也是一個很好的例子。雖然我們應該發願追求更高深的教理,但是先從基本功做起,穩固現有的基礎,這會為日後的修行作出良好的準備。我們手頭上的資料足以用來編輯一小本書。在接下來的日子,我們寸步不離樹林中的小屋,專心投入書本的編寫工作。

我們平均每天會花十至十二個小時實修以第八世噶瑪巴為對象的上師相應法,其餘的時間不是在撰寫小本子的內容就是在睡覺,小木屋內也逐漸形成一種能量場。這一個能量場防止一切干擾,我們能清楚地感覺到它的存在。我們家的廁所很「鄉下」,外建在距離小木屋約二十呎的不遠處,每次在前往廁所途中,心裡總會冒出許多奇怪的想法。可是每當我們回到小木屋或能量場所籠罩的範圍時,內心又會再次變得明亮澄淨,繼而又能專心地投入工作。即使是朋友們騎著摩托車從哥本哈根遠道而來,他們也無意留下來找我們閒話家常。他們每次只是將食物放在窗戶上後便會自行離開。我們在小木屋裡寫書寫了將近一個月,從初一一直寫到大黑天金剛護法日(下一個初一的前一天),終於完成了《Teachings
on the Nature of
Mind》(心之本質)一書。這是我所寫過關於佛法的六本書中的第一本。今天這些書籍已被結集整合為《The
Way Things Are》一書,並且被翻譯成多種歐洲語言。

接下來,我們最緊迫的問題就是資金。我的父親曾經撰寫過約五十本德文課本,因此「尼達爾」一名在丹麥學術界頗吃得開。我們很快便找到工作,在哥本哈根以西約六十公里的一所學校任教。我們每天必須冒著風雪開車去上班,而且運氣似乎還不賴。儘管我們所開的那台福斯小巴的輪胎花紋已被磨光,有時甚至得在結冰的路上推車發動引擎,可是總是能及時趕到學校。除了全職教課,晚間我們也在哥本哈根的另一所學校當保潔人員。朋友們有任何疑問時就會來到學校找我們,也順便幫忙。在早晚身兼兩職的情況下,我們賺了不少錢。如今再次投入那麼多體力活中,其實蠻多樂趣的。

我們其實也知道待在歐洲的這一段日子並不會持續太久,噶瑪巴一定會給我們指示該在何時回到印度。有一天,當漢娜仍在課堂裡教課,我來到學校裡的某一個小房靜坐。我們在這裡工作了幾個月,也存了一些錢,心裡能感覺到某些事情即將會發生。我在禪修中祈請噶瑪巴為我指示方向。我在靜坐中看見房門被打開,然後有三個小孩扛著偌大的白板走入房內,白板上是一幅亞洲的描略圖。地圖上除了印度南部,便沒有其他邊界、城鎮或地點的標示。地圖上印度南部的位置畫有一個圓形,圓形內有一偌大的笨拙字跡寫著「ANGALORE」一字。這幾個字如雷電般擊中了我,它可以是「Bangalore」(邦加羅爾),也可以是「Mangalore」(芒加羅),而在這兩個地方的中間恰好是南部西藏難民營的所在地。這一個地方就是我們的下一個目的地。翌日早晨,我們收到阿陽祖古的來信,更加確定了這一點。阿陽祖古就是在隆德的八臂瑜伽度母灌頂法會中告訴我們有關南營的那一個喇嘛。他在信中再次邀請我們前往拜訪。指示已經很明確。我們和家人朋友一起渡過溫馨的聖誕節和新年後,便收拾好行李,再次動身往東出發。
